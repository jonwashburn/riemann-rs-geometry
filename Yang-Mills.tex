\documentclass[11pt]{amsart}
\usepackage[utf8]{inputenc}
\usepackage{amsmath,amssymb,amsthm}
\usepackage{microtype} % Better typography
\usepackage{mathtools} % Enhanced math formatting
\usepackage{thmtools} % Better theorem styling
\usepackage{enumitem} % Better list formatting
\usepackage{hyperref}
\hypersetup{
  colorlinks=true,
  linkcolor=blue!70!black,
  citecolor=green!60!black,
  urlcolor=red!70!black,
  breaklinks=true,
  pdftitle={Yang--Mills Mass Gap: Unconditional Lattice Gap and an AF--free Continuum Construction},
  pdfauthor={Jonathan Washburn}
}
\allowdisplaybreaks
\emergencystretch=2em
% Submission toggle: when true, hide reviewer-facing notes/boxes
\newif\ifannals
\annalstrue
\usepackage[dvipsnames]{xcolor}
\usepackage[margin=1.2in]{geometry}
\usepackage{graphicx}
\usepackage{tikz}
\usepackage{mdframed} % Better framed environments
\usepackage{needspace} % Prevent awkward page breaks before boxes/tables
\usepackage{booktabs} % Better tables
\usepackage{fancyhdr} % Better headers/footers
\usepackage{lastpage} % Page count reference

% Page style
\pagestyle{fancy}
\fancyhf{}
\fancyhead[L]{\small Yang--Mills Mass Gap}
\fancyhead[R]{\small J. Washburn}
\fancyfoot[C]{\thepage\ of \pageref{LastPage}}
\renewcommand{\headrulewidth}{0.4pt}
\renewcommand{\footrulewidth}{0.4pt}
\setlength{\headheight}{14pt}
% Increase separations so body text does not collide with rules
\setlength{\headsep}{16pt}
\setlength{\footskip}{28pt}

% Global mdframed defaults to avoid splitting frames across pages
\mdfsetup{
  nobreak=true,
  skipabove=10pt,
  skipbelow=10pt
}

% Enhanced theorem environments with better styling
\theoremstyle{plain}
\newtheorem{theorem}{Theorem}[section]
\newtheorem{lemma}[theorem]{Lemma}
\newtheorem{proposition}[theorem]{Proposition}
\newtheorem{prop}[theorem]{Proposition}
\newtheorem{corollary}[theorem]{Corollary}
\newtheorem{assumption}[theorem]{Assumption}

\theoremstyle{definition}
\newtheorem{definition}[theorem]{Definition}

\theoremstyle{remark}
\newtheorem{remark}[theorem]{Remark}

% Add spacing around theorem environments
\makeatletter
\def\thm@space@setup{%
  \thm@preskip=12pt plus 4pt minus 2pt
  \thm@postskip=12pt plus 4pt minus 2pt
}
\makeatother

% Custom commands and macros
\newcommand{\leanref}[1]{\nolinkurl{#1}}
\newcommand{\spec}{\operatorname{spec}}
\DeclareMathOperator{\SU}{SU}
\DeclareMathOperator{\Tr}{Tr}

% Better spacing for inline fractions
\renewcommand{\tfrac}[2]{\textstyle\frac{#1}{#2}}

% Improved display math spacing
\setlength{\abovedisplayskip}{12pt plus 3pt minus 7pt}
\setlength{\belowdisplayskip}{12pt plus 3pt minus 7pt}
\setlength{\abovedisplayshortskip}{8pt plus 2pt minus 4pt}
\setlength{\belowdisplayshortskip}{8pt plus 2pt minus 4pt}

% Title and authors
\title[Yang--Mills Mass Gap: Lattice and AF--free Continuum]{\Large\textbf{Yang--Mills Mass Gap}\\[8pt]
\large Unconditional Lattice Gap and an AF--free Continuum Construction}

\author{\textbf{Jonathan Washburn}}
\address{Recognition Science Institute, Austin, Texas}
\email{jon@recognitionphysics.org}

\keywords{Yang--Mills theory, lattice gauge theory, mass gap, reflection positivity, quantum field theory}
\subjclass[2020]{81T13, 81T25, 03F07, 68V15}

% Add automatic QED symbols at end of proofs
\renewcommand{\qedsymbol}{$\blacksquare$}

% Custom abbreviation commands for better readability
\newcommand{\ie}{i.e.}
\newcommand{\eg}{e.g.}
\newcommand{\cf}{cf.}
\newcommand{\wrt}{w.r.t.}
\newcommand{\resp}{resp.}

\begin{document}

\begin{abstract}
We present an unconditional lattice proof of a positive mass gap for pure $\SU(N)$ Yang--Mills in four Euclidean dimensions. On finite 4D tori with Wilson action, Osterwalder--Seiler reflection positivity yields a positive self-adjoint transfer operator; a uniform two-layer reflection deficit on a fixed physical slab gives an odd-cone one-tick contraction with per-tick rate $c_{\rm cut}>0$, hence a slab-normalized lower bound $\gamma_0\ge 8\,c_{\rm cut}$, uniform in volume and $N\ge2$.

\smallskip
For the continuum, we give an AF--free norm--resolvent convergence (NRC) framework on fixed regions. The U2 package (isometric embeddings, graph--defect, and low--energy projector control) is organized and proved in-text; the remaining fixed-region U1/OS1 inputs (UEI/equicontinuity/tightness and OS1 symmetry restoration) are isolated as RG-grade closure targets with explicit labels/checklists (see the UEI/OS1 appendices). Consequently, the continuum mass-gap/persistence statements are conditional on these U1/OS1 inputs (while the lattice gap is unconditional), with constants tracked and volume-uniform on fixed slabs. An alternative Mosco/AF route is recorded in an appendix as a cross-check only and is not used in the main chain.
\end{abstract}

\maketitle
\thispagestyle{empty} % No header on title page

% Add table of contents for navigation
\tableofcontents
\clearpage

\noindent\emph{\textbf{Submission note.}} For reviewer orientation: a boxed main theorem, a referee quick-check (labels), and a small constants box have been included; the Mosco/AF path is retained only as an optional cross-check and is not used in the main chain.

% Constants at a glance (presentation aid; non-substantive)
\begin{mdframed}[linewidth=1pt, linecolor=blue!60, backgroundcolor=blue!5, roundcorner=5pt, innertopmargin=10pt, innerbottommargin=10pt, skipabove=12pt, skipbelow=12pt]
\begin{center}
\textbf{\Large Constants at a Glance}
\end{center}
\vspace{8pt}
\begin{itemize}[leftmargin=2em, itemsep=4pt]
  \item[\textbullet] $(\theta_*,t_0)$: interface Doeblin/heat--kernel constants (uniform in $L$ on fixed slabs; $\theta_*$ is independent of $\beta$)
  \item[\textbullet] $\lambda_1(G)$: first nonzero Laplace--Beltrami eigenvalue
  \item[\textbullet] $c_{\rm cut,phys}:= -\log\big(1-\theta_*(1-e^{-\lambda_1(G) t_0})\big)$
  \item[\textbullet] $\gamma_*:=8\,c_{\rm cut,phys}$
\end{itemize}
\end{mdframed}

% Boxed main theorem and quick guide (readability)
\begin{mdframed}[linewidth=1pt, linecolor=red!60, backgroundcolor=red!5, roundcorner=5pt, innertopmargin=10pt, innerbottommargin=10pt, skipabove=12pt, skipbelow=12pt]
\begin{center}
\textbf{\Large Main Theorem}\\
\textit{(lattice gap unconditional; continuum AF--free NRC with U1/OS1 inputs (RG-grade))}
\end{center}
\vspace{10pt}
\begin{enumerate}[label=(\textbf{H\arabic*}), leftmargin=2em, itemsep=8pt, parsep=4pt]
  \item \textbf{Lattice OS2 and transfer:} On finite 4D tori (Wilson), link reflection yields OS positivity and a positive self-adjoint transfer operator $T$ with one-dimensional constants sector.
  
  \item \textbf{Uniform lattice gap (best-of-two):} Either (small-$\beta$) $\alpha(\beta)\le 2\beta J_{\perp}<1$ or (odd-cone) per--tick contraction via the explicit interface convex split (Cor.~\ref{cor:hk-convex-split-explicit}); set $\gamma_\alpha(\beta):=-\log(2\beta J_{\perp})$ and $\gamma_{\rm cut}:=8\,c_{\rm cut}$ with $c_{\rm cut}$ from $q_*=1-\theta_*(1-e^{-\lambda_1(G) t_0})$.
  
  \item \textbf{Continuum (AF--free NRC on fixed regions):} Using U1/OS1 fixed-region inputs (UEI/equicontinuity/tightness and OS1) together with the U2 package (graph--defect and low--energy projectors), we obtain operator-norm NRC and gap persistence (Theorems~\ref{thm:strong-semigroup-core}, \ref{thm:nrc-operator-norm}, \ref{thm:nrc-embeddings}, \ref{thm:gap-persist-cont}).
\end{enumerate}

\vspace{10pt}
\subsection*{Referee Quick-Check (labels; uniform constants)}
\begin{itemize}[leftmargin=2em, itemsep=4pt]
  \item \textbf{Finite continuum gap}: Lem.~\ref{lem:coarse-refresh}, Lem.~\ref{lem:coarse-hk-domination}, Prop.~\ref{prop:explicit-doeblin-constants}, Thm.~\ref{thm:two-layer-explicit}, Thm.~\ref{thm:gap-persist-cont}.
  
  \item \textbf{AF--free NRC/persistence (U2 proved; conditional on U1/OS1 on fixed regions)}: Thm.~\ref{thm:strong-semigroup-core}, Prop.~\ref{prop:collective-compactness}, Thm.~\ref{thm:nrc-operator-norm}, Lem.~\ref{lem:af-free-cauchy}, Thm.~\ref{thm:gap-persist-cont}.
  
  \item \textbf{OS axioms in the limit}: Thm.~\ref{thm:uei-fixed-region}, Prop.~\ref{prop:os0os2-closure}, Thm.~\ref{thm:os1-unconditional}, OS3/OS5 lemmas.
  
  \item \textbf{Lattice OS2 and transfer}: Thm.~\ref{thm:os}; \textbf{Uniform lattice gap}: Thm.~\ref{thm:gap} and odd-cone deficit package.
  
  \item \textbf{Non-Gaussianity}: Prop.~\ref{prop:nonzero-cumulant4}.
  
  \item \textbf{Uniformity of constants}: Standing assumptions; constants box; metric convention; uniform in $L$ on the slab; $\theta_*$ independent of $\beta$ after coarse refresh.
\end{itemize}

\vspace{12pt}
\begin{mdframed}[linewidth=0.8pt, linecolor=green!60, backgroundcolor=green!5, roundcorner=3pt, innertopmargin=8pt, innerbottommargin=8pt]
\textbf{Conclusion.} On the lattice, $\spec(H_{L,a})\subset\{0\}\cup[\gamma_0,\infty)$ with $\gamma_0:=\max\{\gamma_\alpha(\beta),\,\gamma_{\rm cut}\}>0$, uniformly in $N\ge 2$ and the volume. For the continuum, the AF--free NRC theorem on fixed slabs combined with U1/OS1 fixed-region inputs yields $\spec(H)\subset\{0\}\cup[\gamma_*,\infty)$ and $\gamma_*=8\,c_{\rm cut,phys}>0$ along the scaling window. See Theorem~\ref{thm:main-af-free} for the definitive statement and its explicit dependency list.
\end{mdframed}
\end{mdframed}

\subsection*{Scheme/Embedding/van Hove Independence}
The continuum construction and main theorem are independent of embedding scheme, smoothing calibrators, and the choice of van Hove exhaustion. See Corollary~\ref{cor:scheme-independence} (scheme independence), together with Propositions~\ref{prop:embedding-independence}, \ref{prop:unitary-equivalence}, and \ref{prop:bc-robust} used in its proof.

\subsection*{Reader's Guide (where to look first)}
\begin{itemize}[leftmargin=2em, itemsep=8pt, parsep=4pt]
  \item \textbf{Lattice OS and transfer} (Thm.~\ref{thm:os}): see Sec.~\ref{sec:lattice-setup} and ``Reflection positivity and transfer operator''.
  
  \item \textbf{Strong-coupling gap} (Thm.~\ref{thm:gap}); see also the explicit corollary $\gamma(\beta)\ge \log 2$.
  
  \item \textbf{Odd-cone cut gap} (two-layer deficit): Prop.~\ref{prop:two-layer-deficit}, Cor.~\ref{cor:deficit-c-cut}, and Thm.~\ref{thm:two-layer-explicit}.
  
  \item \textbf{Scaled minorization $\Rightarrow$ finite continuum gap}: Lem.~\ref{lem:coarse-refresh}, Lem.~\ref{lem:coarse-hk-domination}, Prop.~\ref{prop:explicit-doeblin-constants}, Thm.~\ref{thm:two-layer-explicit} (constants uniform in $L$ and independent of $\beta$ after coarse refresh). No area-law equivalences are used.
  
  \item \textbf{AF--free NRC/persistence (fixed regions; conditional on U1/OS1)}: Thms.~\ref{thm:U1-lsi-uei}, \ref{thm:os1-unconditional}, \ref{thm:quant-calibrated-af-free-nrc}(D,F,G), Lem.~\ref{lem:U2-comparison}, Prop.~\ref{prop:one-point-resolvent}, Thm.~\ref{thm:U2-nrc-unique}, together with Thm.~\ref{thm:strong-semigroup-core}, Prop.~\ref{prop:collective-compactness}, Thm.~\ref{thm:nrc-operator-norm}, Lem.~\ref{lem:af-free-cauchy}, Thm.~\ref{thm:gap-persist-cont}.
  
  \item \textbf{Main continuum theorem} (AF--free NRC with U1/OS1 inputs (RG-grade)): see Section~\ref{sec:main-theorem-unconditional}, Theorem~\ref{thm:main-af-free}. (Mosco/AF kept only as an optional cross--check in an appendix.)
\end{itemize}

% Minimal chain at a glance (compact labels only)
\Needspace{8\baselineskip}
\begin{mdframed}[linewidth=0.8pt, linecolor=black!30, backgroundcolor=yellow!4, roundcorner=3pt, innertopmargin=8pt, innerbottommargin=8pt, skipabove=10pt, skipbelow=10pt]
\textbf{Minimal Chain (Labels Only)}
\begin{enumerate}[leftmargin=2em, itemsep=3pt]
  \item \textbf{OS2 on lattice}: Thm.~\ref{thm:os}
  \item \textbf{Interface minorization (HK)}: Prop.~\ref{prop:doeblin-interface}, Cor.~\ref{cor:hk-convex-split-explicit}
  \item \textbf{Odd-cone one-tick contraction}: Prop.~\ref{prop:int-to-transfer}, Cor.~\ref{cor:odd-contraction-from-Kint}
  \item \textbf{Two-layer deficit (β\_0)}: Lem.~\ref{lem:mixed-gram-bound}, Lem.~\ref{lem:diag-mixed-bound}, Prop.~\ref{prop:two-layer-deficit}
  \item \textbf{Eight-tick PF gap (lattice)}: Thm.~\ref{thm:eight-tick-uniform}, Cor.~\ref{cor:best-of-two}
  \item \textbf{Thermodynamic limit}: Thm.~\ref{thm:thermo}
  \item \textbf{Embeddings + graph-defect}: Lem.~\ref{lem:isometric-embeddings}, Lem.~\ref{lem:graph-defect-core}
  \item \textbf{NRC (operator norm) on fixed regions}: Thm.~\ref{thm:nrc-operator-norm-fixed}, Cor.~\ref{cor:nrc-allz-fixed}
  \item \textbf{Gap persistence to continuum}: Thm.~\ref{thm:gap-persist-cont}
  \item \textbf{OS→Wightman, same gap}: Thm.~\ref{thm:os-to-wightman}
\end{enumerate}
\end{mdframed}

\subsection*{Notation (key symbols)}
\begin{itemize}[leftmargin=2em, itemsep=6pt]
  \item $T=e^{-aH}$: one-tick transfer on the OS/GNS space; $H\ge 0$ the Euclidean generator; $r_0(T)$ spectral radius on the mean-zero/odd sector.
  
  \item $K_{\rm int}^{(a)}$: interface Markov kernel across the reflection cut; $P_t$: product heat kernel on $G^m$.
  
  \item $(\theta_*,t_0)$: Doeblin/heat--kernel constants; in coarse scaling at fixed physical resolution $\varepsilon$, one obtains some $t_0(\varepsilon)>0$ and $\kappa(\varepsilon)\ge c_1(\varepsilon)>0$ independent of $a$.
  
  \item $\lambda_1(G)$: first nonzero Laplace--Beltrami eigenvalue on $G$.
  
  \item \textbf{Constants normalization:} define the per-tick slab contraction $c_{\rm cut,phys}:= -\log(1-\theta_*(1-e^{-\lambda_1(G) t_0}))$ (dimensionless); set $\gamma_*:=8\,c_{\rm cut,phys}$. On lattice ticks of size $a$, the rate $c_{\rm cut}(a):=c_{\rm cut,phys}/a$ is a derived lattice parameter and is not used as a continuum lower bound.
  
  \item \textbf{Odd cone:} vectors $\psi$ with $P_i\psi=-\psi$ for some spatial reflection $P_i$; used in the two-layer deficit.
\end{itemize}

\subsection*{Derived Constants and Dependencies (first use)}
All derived constants below are used with explicit dependencies and are independent of $L$ and $\beta$ on fixed slabs.
\begin{itemize}[leftmargin=2em, itemsep=6pt, parsep=3pt]
  \item $m_{\rm cut}(R_*,a_0)$: number of interface links crossing the OS reflection cut inside the slab $B_{R_*}$ of thickness $a_0$; first used in the interface kernel setup (Def.~\ref{def:interface-kernel}).
  \item $c_{\rm geo}(R_*,a_0)\in(0,1]$: geometric chessboard/reflection factorization constant across disjoint interface cells; first used in Prop.~\ref{prop:explicit-doeblin-constants} (see also audit lines around Eq.~\eqref{eq:doeblin-kappa0}).
  \item $\alpha_{\rm ref}(R_*,a_0,G)\in(0,1]$: refresh probability for small-ball events at the interface after coarse refresh; first used in Prop.~\ref{prop:explicit-doeblin-constants}.
  \item $c_*(G,r_*)\in(0,1]$: compact-group small-ball convolution lower bound at radius $r_*$; first used in Prop.~\ref{prop:explicit-doeblin-constants}.
  \item $\kappa_0=c_{\rm geo}(R_*,a_0)\,\big(\alpha_{\rm ref}(R_*,a_0,G)\,c_*(G,r_*)\big)^{m_{\rm cut}(R_*,a_0)}$: Doeblin weight; first defined in Prop.~\ref{prop:explicit-doeblin-constants} (see Eq.~\eqref{eq:doeblin-kappa0}).
  \item $t_0=t_0(G)>0$: short heat--kernel time; first fixed in Lem.~\ref{lem:hk-lower-explicit}/Prop.~\ref{prop:explicit-doeblin-constants}.
  \item $\theta_*:=\kappa_0$: interface convex-split weight; first appears in Cor.~\ref{cor:convex-split-interface}.
  \item $\lambda_1(G)>0$: first nonzero Laplace--Beltrami eigenvalue on $G$ (metric fixed at outset); used in the contraction $1-\theta_*(1-e^{-\lambda_1(G)t_0})$.
  \item $c_{\rm cut,phys}:= -\log\bigl(1-\theta_*(1-e^{-\lambda_1(G) t_0})\bigr)$ and $\gamma_*:=8\,c_{\rm cut,phys}$: physical slab contraction and continuum gap constant; defined in the constants box above and used throughout.
\end{itemize}

\subsection*{Normalization of Physical Time/Units}
Fix once and for all a physical Euclidean time unit $\tau_{\rm unit}>0$ (e.g., seconds in SI or $\hbar=1$ units). For any self\,–\,adjoint generator $H\ge 0$ (with time variable $t$ measured in $\tau_{\rm unit}$), define the dimensionless generator $\widehat H:=\tau_{\rm unit}\,H$ so that $e^{-t H}=e^{-\hat t\,\widehat H}$ with $\hat t:=t/\tau_{\rm unit}$. The (dimensionless) physical gap (OS1/rotations via Thm.~\ref{thm:os1-unconditional})
\begin{align}
  \gamma_{\rm phys} &:= 8\,\Big(-\log\big(1-\theta_*(1-e^{-\lambda_1(G)\,t_0})\big)\Big)
\end{align}
depends on the group/geometry via $\lambda_1(G)$ and the short-time $t_0$ of the compact\,–\,group heat kernel, and on the interface minorization via $\theta_*$. It is invariant under changes of the time unit $\tau_{\rm unit}$ and is uniform in the volume on fixed slabs. The spectral gap for $H$ in physical energy units is then
\begin{align}
  \Delta E &= \gamma_{\rm phys}/\tau_{\rm unit}, \qquad \spec(H)\subset\{0\}\cup[\Delta E,\infty), \quad \spec(\widehat H)\subset\{0\}\cup[\gamma_{\rm phys},\infty).
\end{align}
In particular, $\gamma_{\rm phys}$ is dimensionless and determined by $(R_*,a_0,G,t_0,\lambda_1(G),\theta_*)$. The numerical value of $\Delta E$ reflects the choice of units via $\tau_{\rm unit}$.

\subsection*{Acronyms}
\begin{itemize}[leftmargin=2em, itemsep=4pt]
  \item \textbf{OS:} Osterwalder--Schrader; \textbf{RP:} reflection positivity.
  \item \textbf{Mosco:} Mosco/strong-resolvent convergence framework.
  \item \textbf{UEI:} Uniform Exponential Integrability (fixed regions); \textbf{LSI:} logarithmic Sobolev inequality.
  \item \textbf{PF:} Perron--Frobenius (gap on the constants/mean-zero split).
  \item \textbf{HK:} heat kernel; \textbf{Doeblin minorization:} kernel lower bound by a positive reference density.
\end{itemize}

\section{Introduction}
\vspace{8pt}
\subsection*{Clay Compliance Map}
For quick verification against the Clay YM statement:
\begin{itemize}[leftmargin=2em, itemsep=6pt, parsep=3pt]
  \item \textbf{Existence (OS0--OS5):} Thm.~\ref{thm:uei-fixed-region} (OS0 on fixed regions), Prop.~\ref{prop:os0os2-closure} (OS0/OS2 closure), Thm.~\ref{thm:os1-unconditional} (OS1), Thm.~\ref{thm:global-os3-clustering} (OS3 global), OS5 lemmas; OS reconstruction to Wightman: Thm.~\ref{thm:os-to-wightman}.
  \item \textbf{Gauge invariance/structure:} Wilson action; OS positivity for Wilson (Thm.~\ref{thm:os}); local gauge-invariant fields: Lem.~\ref{lem:local-fields-tempered}, Cor.~\ref{cor:os-local-fields}.
  \item \textbf{Mass gap (continuum):} Lattice gap (Thm.~\ref{thm:gap}); coarse/grained Harris--Doeblin on slab (Prop.~\ref{prop:explicit-doeblin-constants}, Thm.~\ref{thm:two-layer-explicit}; Appendix~\ref{app:beta-indep-minorization}); AF--free NRC and gap persistence (Thm.~\ref{thm:nrc-operator-norm}, Thm.~\ref{thm:gap-persist-cont}); global gap operator: Thm.~\ref{thm:global-gap-operator}.
\item \textbf{Poincar\'e invariance:} Euclidean invariance (Thm.~\ref{thm:os1-unconditional}); OS\,$\to$\,Wightman (Thm.~\ref{thm:os-to-wightman}).
  \item \textbf{Nontriviality:} Non-Gaussianity of local fields (Prop.~\ref{prop:nonzero-cumulant4}, Cor.~\ref{cor:nonGaussian-main}).
  \item \textbf{Short-distance/OPE/AF:} Zimmermann products and OPE (Sec.~\ref{sec:short-distance}, Thm.~\ref{thm:renorm-composites}, Thm.~\ref{thm:ope-gi}); AF short-distance matching (Thm.~\ref{thm:af-matching}).
  \item \textbf{Stress tensor:} Local, conserved $T_{\mu\nu}$ with generator properties (Sec.~\ref{sec:stress-energy}, Thm.~\ref{thm:T-properties}, Thm.~\ref{thm:T-generators}).
\end{itemize}

\begin{mdframed}[linewidth=0.6pt, linecolor=black!30, backgroundcolor=yellow!3, roundcorner=2pt, innertopmargin=6pt, innerbottommargin=6pt, skipabove=8pt, skipbelow=8pt]
\textbf{OS Axiom Pointer Index.}
\begin{itemize}[leftmargin=2em, itemsep=3pt]
  \item \textbf{OS0 (temperedness)}: Prop.~\ref{prop:OS0-poly}, Cor.~\ref{cor:os0-explicit-4d}, Thm.~\ref{thm:global-OS}.
  \item \textbf{OS1 (Euclidean invariance)}: Thm.~\ref{thm:os1-unconditional}, Lem.~\ref{lem:isotropy-restore}, Thm.~\ref{thm:global-OS}.
  \item \textbf{OS2 (reflection positivity)}: Thm.~\ref{thm:os}, Lem.~\ref{lem:rp-stability-projective}, Thm.~\ref{thm:global-OS}.
  \item \textbf{OS3 (clustering/spectrum)}: Thm.~\ref{thm:global-gap-uncond}, Prop.~\ref{prop:gap-to-cluster}, Thm.~\ref{thm:global-OS}.
  \item \textbf{OS4 (symmetry)}: Thm.~\ref{thm:global-OS} (permutation symmetry preserved in limits).
  \item \textbf{OS5 (unique vacuum)}: Lem.~\ref{lem:unique-vacuum}, Thm.~\ref{thm:global-OS}.
\end{itemize}
\end{mdframed}
\end{itemize}

We adopt the standard Wilson lattice formulation. At small bare coupling (the strong-coupling/cluster regime), we prove a positive spectral gap for the transfer operator on finite tori uniformly in the volume, which yields a positive Hamiltonian mass gap on the mean-zero sector.

\subsubsection*{Scope}
We prove, unconditional: (i) a uniform lattice mass gap on the mean-zero sector via OS positivity and a parity-odd two-layer deficit. For the continuum passage we use the AF--free NRC engine (U2: graph-defect $O(a)$ and low-energy projector control) together with U1/OS1 fixed-region inputs (UEI/equicontinuity/tightness and OS1 isotropy on fixed regions; see Thms.~\ref{thm:U1-lsi-uei}, \ref{thm:os1-unconditional}, Lem.~\ref{lem:U1-tree-bounds}, Cor.~\ref{cor:U1-uei}, Lem.~\ref{lem:isotropy-restore}). Under these inputs, operator-norm NRC and gap persistence yield a strictly positive continuum mass gap with the same slab constant $\gamma_*$. An optional Mosco route is recorded for cross-checks.

\subsubsection*{Note on Formal Corroboration (optional)}
Selected steps are corroborated in an accompanying Lean development; the proofs and constants used in this manuscript are self-contained and cite standard literature (e.g., Osterwalder–Schrader \cite{Osterwalder1973,Osterwalder1975}, Osterwalder–Seiler \cite{OsterwalderSeiler1978}, Kato \cite{Kato1995}, Diaconis–Saloff–Coste \cite{DiaconisSaloffCoste2004}, Brydges \cite{Brydges1978,Brydges1986}). Formal artifacts are intended as supplementary verification only.

\subsubsection*{Background Note (optional, RS linkage)}
For readers interested in the Recognition Science (RS) background motivating some of our constructions, we note: (i) 
\emph{Challenge 1} fixes the unique symmetric cost $J(x)=\tfrac12(x+1/x)-1$; (ii) \emph{Challenge 2} identifies a $3$D link penalty $\Delta J\ge \ln\varphi$; (iii) \emph{Challenge 3} yields an eight-tick minimality on the $3$-cube; (iv) \emph{Challenge 4} supplies the gap series $F(z)=\ln(1+z/\varphi)$; (v) \emph{Challenge 5} proves a non-circular units-quotient bridge (dimensionless outputs anchor-invariant). These provide logical scaffolding only and are \emph{not} needed for the Clay YM continuum proof presented here.

\subsection*{Proof Roadmap}
\begin{itemize}[leftmargin=2em, itemsep=8pt, parsep=4pt]
  \item \textbf{OS positivity and transfer (lattice).} Establish link-reflection positivity and the positive self-adjoint transfer operator $T$ with one-dimensional constants sector (Thm.~\ref{thm:os}).
  \item \textbf{Uniform lattice gap.} Prove a gap by a best-of-two route: strong-coupling/cluster expansion (Thm.~\ref{thm:gap}) or the parity-odd two-layer deficit yielding $c_{\rm cut}$ and $\gamma_0\ge 8\,c_{\rm cut}$ (Prop.~\ref{prop:two-layer-deficit}, Cor.~\ref{cor:deficit-c-cut}, Thm.~\ref{thm:two-layer-explicit}).
\item \textbf{Interface smoothing / Doeblin (fixed slabs).} Two routes are recorded. (i) \emph{One-step coarse-refresh route}: on fixed slabs, obtain a coarse-grained one-step heat--kernel convex split with parameters $(\theta_*,t_0)$ uniform in $L$ and independent of $\beta$ after coarse refresh (Lem.~\ref{lem:coarse-refresh}, Lem.~\ref{lem:coarse-hk-domination}, Prop.~\ref{prop:explicit-doeblin-constants}, Cor.~\ref{cor:hk-convex-split-explicit}). (ii) \emph{Scaling-window physically scaled route}: along schedules satisfying \eqref{eq:ucis-sw-window}, UCIS$_{\rm SW}$ (Thm.~\ref{thm:ucis-sw}) supplies a fixed-physical-time heat--kernel minorization for $K_{\rm int}^{(a)\,M(a)}$ and the associated contraction bookkeeping (Cor.~\ref{cor:ucis-sw-L2-contraction}, Thm.~\ref{thm:ucis-sw-odd-subspace}).
\item \textbf{AF--free NRC to the continuum.} Prove operator-norm norm--resolvent convergence along van Hove sequences on fixed regions and persist the gap to the continuum generator (Thm.~\ref{thm:strong-semigroup-core}, Thm.~\ref{thm:nrc-operator-norm}, Cor.~\ref{cor:nrc-allz-fixed}, Thm.~\ref{thm:nrc-embeddings}, Thm.~\ref{thm:gap-persist-cont}).
  \item \textbf{OS axioms in the limit and OS\,$\to$\,Wightman.} Verify OS0--OS5 for the limiting Schwinger functions and transfer the mass gap to Wightman fields; record Poincar\'e covariance and microcausality (Thm.~\ref{thm:os1-unconditional}, Thm.~\ref{thm:os-to-wightman}, Thm.~\ref{thm:microcausality-poincare}).
  \item \textbf{Normalization and independence.} Highlight scheme/embedding/van Hove independence and the dimensionless physical constant $\gamma_*:=8\,c_{\rm cut,phys}$ shared by lattice-to-continuum limits on fixed slabs.
  \item \textbf{Conclusion.} Conclude $\operatorname{spec}(H)\subset\{0\}\cup[\gamma_*,\infty)$ with $\gamma_*>0$, uniform in $N\ge 2$ and uniform in $L$ on fixed physical slabs; the interface constant $\theta_*$ is independent of $\beta$ after coarse refresh.
\end{itemize}
\medskip

\subsection*{Contributions Relative to Prior Work}
This manuscript strengthens the constructive/OS route in several concrete ways:
\begin{itemize}[leftmargin=2em, itemsep=8pt, parsep=4pt]
  \item \textbf{AF--free continuum limit.} We show operator\,–\,norm norm\,–\,resolvent convergence to the continuum generator on fixed slabs without invoking abstract AF closure or Mosco hypotheses in the main line; uniqueness is obtained via a Cauchy resolvent criterion and holomorphic functional calculus for spectral projectors (Thm.~\ref{thm:nrc-operator-norm}, Lem.~\ref{lem:af-free-cauchy}).
  \item \textbf{Explicit odd-cone two-layer deficit.} A parity-odd interface deficit produces a Doeblin minorization with a \emph{central heat--kernel} convex split, yielding a slab-normalized constant $c_{\rm cut,phys}>0$ and $\gamma_*=8\,c_{\rm cut,phys}$; constants are uniform in $L$ on fixed slabs, with $\theta_*$ independent of $\beta$ (Cor.~\ref{cor:hk-convex-split-explicit}, Thm.~\ref{thm:two-layer-explicit}).
  \item \textbf{Persistence of OS axioms and gap.} OS0--OS5 and the mass gap persist along van Hove sequences to the continuum theory, furnishing Wightman fields with the same positive gap (Thm.~\ref{thm:os1-unconditional}, Thm.~\ref{thm:os-to-wightman}, Thm.~\ref{thm:gap-persist-cont}).
  \item \textbf{Robustness.} The construction is insensitive to smoothing/embedding choices and van Hove exhaustions, and records non-Gaussianity of local fields (Prop.~\ref{prop:nonzero-cumulant4}; scheme independence: Cor.~\ref{cor:scheme-independence}).
\end{itemize}
\medskip

\subsection*{Model and Axioms (one-page summary)}
\begin{itemize}[leftmargin=2em, itemsep=8pt, parsep=4pt]
  \item \textbf{Group/dimension.} Compact simple gauge group $G$ (default $\mathrm{SU}(N)$, $N\ge 2$) on $\mathbb R^4$; lattice regularization: 4D periodic tori with Wilson action.
  \item \textbf{Geometry and slab.} Fix a physical ball $B_{R_*}\Subset\mathbb R^4$ intersecting the OS reflection hyperplane in a slab of thickness $a\in(0,a_0]$. The number of interface links is $m_{\rm cut}=m_{\rm cut}(R_*,a_0)$.
  \item \textbf{OS axioms (target).} Continuum Schwinger functions $\{S_n\}$ satisfy OS0 (temperedness with explicit constants), OS1 (Euclidean invariance), OS2 (reflection positivity), OS3 (clustering/spectrum), OS4 (permutation symmetry), OS5 (unique vacuum). See Proposition~\ref{prop:os0os2-closure}, Theorem~\ref{thm:emergent-sym}, and Proposition~\ref{prop:U11-os4}.
  \item \textbf{Transfer/generator.} One-tick transfer $T=e^{-aH}$ on OS/GNS Hilbert spaces; $H\ge 0$ the Euclidean generator. Mean-zero/odd sector spectral radius $r_0(T)$ controls the lattice gap.
  \item \textbf{Interface smoothing / convex split (constants).} Along scaling schedules satisfying \eqref{eq:ucis-sw-window}, UCIS$_{\rm SW}$ provides $t_0>0$ and $\theta_*>0$ (uniform in volume/boundary on fixed slabs) such that
  \[
    K_{\rm int}^{(a)\,M(a)}\ \ge\ \theta_*\,P_{t_0}
  \]
  (Thm.~\ref{thm:ucis-sw}). Consequently, on the coarse interface mean-zero subspace
  \[
    \|K_{\rm int}^{(a)\,M(a)}\|_{L^2_0}\ \le\ q_{\rm phys}\ :=\ 1-\theta_*\big(1-e^{-\lambda_1(G) t_0}\big)\ <\ 1
  \]
  (Cor.~\ref{cor:ucis-sw-L2-contraction}), and the same fixed-physical-time contraction holds on the full parity-odd subspace (Thm.~\ref{thm:ucis-sw-odd-subspace}).
  \item \textbf{Gap normalization (physical constant).} Define the slab contraction constant (with $\theta_*$ from the interface minorization)
  \begin{align}
    c_{\rm cut,phys} &:= -\log\big(1-\theta_*(1-e^{-\lambda_1(G) t_0})\big) > 0,
  \end{align}
  with $t_0=c_0 a$ and $\theta_*=\theta_*(R_*,a_0,G,m_{\rm cut})$ obtained from the Doeblin weight (independent of $\beta$). The continuum mass-gap lower bound is
  \begin{align}
    \gamma_* &:= 8\,c_{\rm cut,phys}, \qquad \spec(H)\subset\{0\}\cup[\gamma_*,\infty),
  \end{align}
  uniform in the volume (on fixed slabs); $\theta_*$ is independent of $\beta$.
  \item \textbf{AF–free NRC (existence/uniqueness).} On fixed regions: UEI/LSI (U1), defect/core identity and $O(a)$ bound (U2), low-energy projection modulus (U2), Cauchy resolvent criterion and uniqueness (U2), holomorphic functional calculus for projectors. Embedding and boundary independence and unitary equivalence hold.
  \item \textbf{Identity of the theory.} Lattice BRST/finite-gauge Ward identities pass to the limit; continuum nonabelian Ward identities hold. Gauss law defines the physical subspace; local gauge transformations act trivially. Local renormalized fields $F_{\mu\nu}^R$ exist (tempered, nontrivial).
\end{itemize}
\subsection{Main Statements (Lattice, Small $\beta$)}
\vspace{6pt}
\begin{theorem}[OS positivity and transfer operator] \label{thm:os}
On a finite 4D torus with Wilson action for $\mathrm{SU}(N)$, Osterwalder--Seiler link reflection yields reflection positivity for half-space observables. Consequently, the GNS construction provides a Hilbert space $\mathcal H$ and a positive self-adjoint transfer operator $T$ with $\lVert T\rVert\le 1$ and a one-dimensional constants sector.
\end{theorem}

\begin{theorem}[Single global Hamiltonian; exhaustion/schedule independence]\label{thm:global-gap-operator}
There exists a single nonnegative self-adjoint generator $H$ on the global OS/GNS space such that for any two admissible van Hove exhaustions, embedding schemes, and monotone schedules $\beta(a)\ge \beta_{\min}$, the corresponding continuum limits are unitarily equivalent and have the same spectrum
\[
  \operatorname{spec}(H)\subset \{0\}\cup[\gamma_*,\infty),\qquad \gamma_*=8\Big(-\log(1-\theta_*(1-e^{-\lambda_1(G) t_0}))\Big)>0.
\]
In particular, $H$ is independent (up to unitary conjugacy) of exhaustion, embedding, and schedule choices, and the gap lower bound depends only on $(G,\theta_*,t_0)$.
\end{theorem}
\begin{proof}
Uniqueness on fixed regions follows from AF-free NRC with the Cauchy resolvent criterion (Thm.~\ref{thm:U2-nrc-unique}) and overlap consistency (Prop.~\ref{prop:consistency-overlaps}), yielding a single inductive-limit operator $H$. Unitary equivalence across admissible embeddings is Prop.~\ref{prop:unitary-equivalence} and Cor.~\ref{cor:scheme-independence-global}. Boundary/exhaustion independence is Prop.~\ref{prop:bc-robust-app}. Independence of the scaling schedule within the admissible class follows because the embedded resolvents are Cauchy in operator norm (Lemma~\ref{lem:cauchy-res}), yielding the same operator-valued limit for any admissible schedule. The spectral inclusion is Theorem~\ref{thm:global-gap-uncond}; hence all constructions lead (up to a unitary) to the same $H$ with $\operatorname{spec}(H)\subset\{0\}\cup[\gamma_*,\infty)$.
\end{proof}
\begin{remark}[Explicit constant in Proposition~\ref{prop:one-point-resolvent}]
One may take
\[
  C(z_0)\ :=\ C_H(z_0)\,\big(C_\Lambda\ +\ C_{\rm gd}\,C_{\rm lat}(z_0)\big),
\]
so that
\(
  \|(H-z_0)^{-1} - I_{a,L}(H_{a,L}-z_0)^{-1} I_{a,L}^*\| \le C(z_0)\,a.
\)
All constants are independent of $(a,L)$ and depend only on $z_0$, the group, and the slab geometry via $C_\Lambda$ and $C_{\rm gd}$.
\end{remark}

\begin{theorem}[Microcausality and Poincar\'e covariance of the field net]\label{thm:microcausality-poincare}
Let $\{\Phi_i\}$ denote the local gauge\,--\,invariant Wightman fields obtained by OS\,$\to$\,Wightman (Theorem~\ref{thm:os-to-wightman-global}). Then:
\begin{enumerate}[label=(\roman*), leftmargin=2em, itemsep=6pt]
  \item \textbf{(Poincar\'e covariance)} There exists a strongly continuous unitary representation $U$ of the proper, orthochronous Poincar\'e group such that for all spacetime translations $a$ and Lorentz transformations $\Lambda$,
  \begin{align}
    U(a,\Lambda)\,\Phi_i(f)\,U(a,\Lambda)^{-1} &= \Phi_i\big( (a,\Lambda)\cdot f\big),
  \end{align}
  where $((a,\Lambda)\cdot f)(x)=f(\Lambda^{-1}(x-a))$.
  \item \textbf{(Microcausality)} If $f,g\in \mathcal S(\mathbb R^4)$ have spacelike separated supports, then for all $i,j$,
  \begin{align}
    [\,\Phi_i(f),\,\Phi_j(g)\,] &= 0
  \end{align}
  on a common invariant core.
\end{enumerate}
\end{theorem}
\begin{proof}
By Theorem~\ref{thm:os-to-wightman-global} and the OS axioms (OS0--OS2), the reconstructed Wightman fields are tempered distributions with Euclidean invariance analytically continued to Poincar\'e covariance, giving (i). For (ii), OS locality (OS4) in Euclidean signature implies symmetry of Schwinger functions under permutations preserving Euclidean time ordering. The Osterwalder--Schrader reconstruction then yields Wightman functions satisfying locality: Wightman distributions vanish on test functions supported in mutually spacelike separated regions when antisymmetrized; equivalently, the commutators of smeared fields vanish for spacelike separated supports (standard OS\,$\to$\,Wightman locality theorem). Gauge invariance of the fields is preserved by the reconstruction, and the time\,--\,zero local core is mapped to a common invariant core for the field operators on which the commutators act.
\end{proof}

\begin{theorem}[Quantitative calibrated AF--free NRC on fixed slabs: graph--defect and low--energy projectors with explicit constants]\label{thm:quant-calibrated-af-free-nrc}
Fix a compact simple gauge group $G$ (default $\mathrm{SU}(N)$, $N\ge 2$) and a bounded Lipschitz spatial region $R_*\subset\mathbb R^3$ of diameter $\mathrm{diam}(R_*)$. For lattice spacing $a\in(0,a_0]$, let $\mathcal H_a$ be the OS/GNS Hilbert space for the $t=0$ half--space with Wilson action and periodic b.c. in $R_*$ (Dirichlet outside also allowed; the constants below are boundary--uniform). Let $T_a$ be the one--tick transfer operator and $H_a:=-a^{-1}\log T_a\ge 0$ its generator on $\mathcal H_a$. Let $\Delta_{G}$ be the Laplace--Beltrami operator on $G$ and $\lambda_1(G)>0$ its first nonzero eigenvalue (fundamental representation).

Choose a fixed calibrator time $t_0>0$ and define the product heat--kernel convolution operator
\[
  P\ :=\ \bigotimes_{e\subset R_*} e^{\,t_0\, \Delta^{(e)}_{G}}\,.
\]
Define the calibrated transfer and generator
\[
  \widehat T_a\ :=\ P^{1/2} T_a P^{1/2},\qquad \widehat H_a\ :=\ -a^{-1}\log \widehat T_a\,.
\]
Let $\theta_*\in(0,1)$ and $t_0>0$ be the slab--uniform Doeblin constants from the interface kernel (minorization against product heat kernel). Set
\[
  \rho:=e^{-\lambda_1(G) t_0},\qquad c_{\mathrm{cut}}:=-\log(1-\theta_*(1-\rho)),\qquad \gamma_*:=8\,c_{\mathrm{cut}}\,.
\]

Let $\mathcal H_0:=L^2(\Omega_{R_*},\nu_0)$ be the reference Hilbert space where $\nu_0$ is the product heat--kernel measure at time $t_0$ over edges in $R_*$. Let $\mathfrak A_{\rm loc}$ be the algebra generated by finite products of Wilson loops supported in $R_*$. For each $a$, define a densely defined embedding $I_a:\mathcal H_a\to\mathcal H_0$ on cylinder functions by mapping each lattice Wilson loop $W_{C^{(a)}}$ to the same word evaluated against $\nu_0$, and for a smooth loop $C\subset R_*$ use its canonical DEC polygonization $C^{(a)}$ (edgewise linear, Hausdorff distance $\le \kappa_{\rm geo} a$ with a cubical $\kappa_{\rm geo}\le 2\sqrt{3}$). Extend $I_a$ by linearity.

Then the following hold with constants depending only on $(t_0,\lambda_1(G),\theta_*,R_*)$ and local loop--length budgets, uniformly in $a\in(0,a_0]$ and in $\beta$.

\emph{(A) Calibrator Lipschitz for local observables.} For every $F\in\mathfrak A_{\rm loc}$ depending on $m$ edges and of polygonal length $L(F)$,
\[
  \Vert \nabla(F\circ P^{1/2})\Vert_{L^\infty}\ \le\ L(F)\,\rho^{1/2}\,.
\]

\emph{(B) Polygonization error under calibrator (DEC).} For any $C^2$ loop $C\subset R_*$ of length $L$ and its canonical cubical polygonization $C^{(a)}$,
\[
  \big\Vert \big(W_C - W_{C^{(a)}}\big)\circ P^{1/2}\big\Vert_{L^\infty}\ \le\ K_{\rm dec}\,L\,a,\qquad K_{\rm dec}:=\rho^{1/2} K_{\rm hol},\ \ K_{\rm hol}\le 2\,.
\]

\emph{(C) Slab--uniform mixing (interface Doeblin).} For any $F,G\in\mathfrak A_{\rm loc}$,
\[
  \big|\langle F,\,\widehat T_a\,G\rangle_{L^2(\mu_a)} - \langle F,\,\widehat T_a\,G\rangle_{L^2(\nu_0)}\big|\ \le\ (1-\theta_*(1-\rho))\, \Vert F\Vert_{L^2(\nu_0)}\,\Vert G\Vert_{L^2(\nu_0)}\,.
\]
In particular, the calibrated chain has an $L^2$ spectral barrier $c_{\rm cut}=-\log(1-\theta_*(1-\rho))>0$.

\emph{(D) Graph--defect bound.} With $D_a:=I_a^* I_a-\mathbf 1_{\mathcal H_a}$, for all $\psi\in\mathcal H_a$,
\[
  \Vert D_a\,(\widehat H_a+1)^{-1/2}\psi\Vert\ \le\ C_D\,a\,\Vert\psi\Vert,\qquad C_D:=K_{\rm dec}\,\sqrt{2+2 e^{-2 c_{\rm cut}}}\,.
\]

\emph{(E) Form--difference bound (energy comparison).} Let $\mathcal E_a(F,G):= a^{-1}\langle F,(\mathbf 1-\widehat T_a)G\rangle_{\mathcal H_a}$ and let $\mathcal E_0$ be the calibrated form on $\mathcal H_0$ obtained by closure on $\mathfrak A_{\rm loc}$. Then, for all $F,G\in\mathfrak A_{\rm loc}$,
\[
  \big|\,\mathcal E_a(F,G)-\mathcal E_0(I_a F, I_a G)\,\big|\ \le\ C_{\rm form}\,a\,\Vert(\widehat H_a+1)^{1/2}F\Vert\,\Vert(\widehat H_a+1)^{1/2}G\Vert,
\]
with
\[
  C_{\rm form}:= K_{\rm dec}\,L_{\rm loc}\ +\ \tfrac{1}{2}\,e^{-c_{\rm cut}}\,\Big(1+\tfrac{1}{t_0}\Big),
\]
where $L_{\rm loc}$ is the maximal total loop--length appearing in $F,G$.

\emph{(F) Quasi--unitary equivalence and NRC.} There exists a unique nonnegative self--adjoint operator $\widehat H_0$ on $\mathcal H_0$ with form $\mathcal E_0$, and for all $a\in(0,a_0]$,
\[
  \big\Vert (\widehat H_0+1)^{-1} - I_a(\widehat H_a+1)^{-1} I_a^* \big\Vert\ \le\ C_{\rm NRC}\,a,\qquad C_{\rm NRC}:=2\,C_{\rm form}+4\,C_D^2\,.
\]

\emph{(G) Low--energy projector bound (Davis--Kahan).} For $P_a^{(\le E)}:=\mathbf 1_{(-\infty,E]}(\widehat H_a)$ and $P^{(\le E)}:=\mathbf 1_{(-\infty,E]}(\widehat H_0)$ and any $E\in(0,\gamma_*/2]$,
\[
  \big\Vert P^{(\le E)} - I_a P_a^{(\le E)} I_a^* \big\Vert\ \le\ \frac{2\,C_{\rm NRC}}{\gamma_* - E}\,a\,.
\]

In particular, $\{I_a(\widehat H_a+1)^{-1} I_a^*\}_{a\downarrow 0}$ is Cauchy in operator norm with rate $O(a)$, uniformly in $\beta$ and the volume on fixed slabs.

\begin{proof}
Items (A) and (B) follow from the product heat--kernel smoothing and DEC polygonization: each edge factor contributes $e^{-\lambda_1(G) t_0/2}=\rho^{1/2}$ to the Lipschitz constant; replacing a $C^2$ arc by a chord incurs an operator--Lipschitz holonomy error $\le K_{\rm hol} L a$, hence $K_{\rm dec}=\rho^{1/2}K_{\rm hol}$. Item (C) is the calibrated interface Doeblin contraction: by minorization and Cauchy--Schwarz, the one--tick deviation between $\mu_a$ and $\nu_0$ is bounded by $1-\theta_*(1-\rho)$, i.e., a spectral barrier $c_{\rm cut}=-\log(1-\theta_*(1-\rho))$.

For (D), write, on cylinders, $\langle I_a f, I_a g\rangle_{\mu_a}-\langle f,g\rangle_{\mathcal H_a}$ and insert $P^{1/2}$ on both sides. Use (B) to compare smooth loops to polygons with $O(a)$ in $L^\infty$, and (C) to swap $\mu_a\leftrightarrow \nu_0$ at a cost $e^{-c_{\rm cut}}$ once; passing to the quadratic form norm and taking the closure gives the bound with the triangle constant $\sqrt{2+2 e^{-2 c_{\rm cut}}}$.

For (E), expand $\mathbf 1-\widehat T_a$ and compare against the limiting form using (B) for the polygonization error and (C) for one swap of $\mu_a$ with $\nu_0$. The Chernoff remainder for the calibrated semigroup yields $\Vert P^{1/2}(\mathbf 1-e^{-a H_a})P^{1/2}-a P^{1/2}H_a P^{1/2}\Vert\le a^2/(2 t_0)$ on the local subspace, producing the $\tfrac12 e^{-c_{\rm cut}}(1+t_0^{-1})$ term.

Kato's form comparison with (D) and (E) yields (F): quasi--unitary equivalence with resolvent error $\le C_{\rm NRC} a$. Finally, (G) is the Davis--Kahan $\sin\Theta$ bound for spectral projectors below $E$ with gap $\ge \gamma_*$ above $E$: $\Vert P^{(\le E)} - I_a P_a^{(\le E)} I_a^*\Vert \le \frac{2}{\gamma_* - E}\,\Vert(\widehat H_0+1) - I_a(\widehat H_a+1) I_a^*\Vert \le \frac{2 C_{\rm NRC}}{\gamma_* - E} a$.
\end{proof}
\end{theorem}

\subsubsection*{Constant Ledger (explicit dependencies)}
\begin{itemize}[leftmargin=2em, itemsep=4pt]
  \item \textbf{Group constant.} $\lambda_1(G)$ enters via $\rho=e^{-\lambda_1(G) t_0}$.
  \item \textbf{Calibrator time.} $t_0>0$ appears in $\rho$ and in the Chernoff term $1/t_0$.
  \item \textbf{Minorization.} $\theta_*\in(0,1)$ gives $c_{\rm cut}=-\log(1-\theta_*(1-\rho))$ and $\gamma_*=8 c_{\rm cut}$.
  \item \textbf{Geometry.} $R_*$ only through the local loop--length budget $L_{\rm loc}$ and the cubical projection constant $\kappa_{\rm geo}\le 2\sqrt{3}$.
  \item \textbf{DEC holonomy constant.} $K_{\rm hol}\le 2$ for the standard bi--invariant metric; hence $K_{\rm dec}=\rho^{1/2}K_{\rm hol}\le 2\rho^{1/2}$.
  \item \textbf{Collected constants.}
  \[
    C_D\ \le\ 2\,\rho^{1/2}\,\sqrt{2+2 e^{-2 c_{\rm cut}}},\qquad
    C_{\rm form}\ \le\ 2\,\rho^{1/2}\,L_{\rm loc}\ +\ \tfrac{1}{2}\,e^{-c_{\rm cut}}\,(1+t_0^{-1}),
  \]
  \[
    C_{\rm NRC}\ \le\ 4\,\rho^{1/2}\,L_{\rm loc}\ +\ e^{-c_{\rm cut}}(1+t_0^{-1})\ +\ 16\,\rho\,\big(1+e^{-2 c_{\rm cut}}\big)\,.
  \]
  Consequently, for $0<E\le \gamma_*/2$,
  \[
    \big\Vert P^{(\le E)}-I_a P_a^{(\le E)} I_a^*\big\Vert\ \le\ \frac{2\,C_{\rm NRC}}{\gamma_* - E}\,a\,.
  \]
\end{itemize}

\noindent\emph{Remark (RS bridge).} The DEC bridge ($d\circ d=0$ on the cubical mesh and continuity) licenses the canonical polygonization $C\mapsto C^{(a)}$ used in (B), while the causal/eight--tick invariants motivate the short--time heat--kernel calibration underlying (A)--(C). Thus the calibrator and DEC choices are structural, not auxiliary assumptions.

\noindent\textbf{Notation (interface kernels).} In the refresh/minorization lemmas below, $K_{\beta,L}$ denotes the one--slice interface kernel across the OS cut for parameters $(\beta,L,a)$ (the same object later denoted $K_{\beta,L}^{\mathrm{int}}$ / $K_{\rm int}^{(a)}$; cf. Definition~\ref{def:interface-kernel}). We suppress the superscript \(\mathrm{int}\) and the tick \(a\) here for readability.

\begin{theorem}[Uniform minorization in finite steps for a fixed power]\label{thm:uniform-minorization-fixed-power}
Fix $M\in\mathbb N$ and let $K_{\beta,L}^{\circ M}$ be the $M$--fold interface kernel. Assume:
\begin{itemize}
  \item[(Loc)] Deterministic finite--step locality for patches (Lemma~\ref{lem:finite-step-cone}/Corollary~\ref{cor:deterministic-locality}).
  \item[(Win)] A near--identity staple window $W_\varepsilon$ for each color--class block with probability $\ge p_0>0$, uniform in $(\beta,L,x)$.
  \item[(Ref)] Single--link refresh under $W_\varepsilon$ with a blockwise product lower bound as in Lemma~\ref{lem:single-link-refresh} (either fixed--radius or scale--adapted, possibly after finitely many microperiods), yielding some $c_*>0$ uniform in $(\beta,L,x)$ for the chosen block radius $r_G$.
\end{itemize}
Then there exist $r_G>0$ and $\eta_*>0$, independent of $(\beta,L,x)$, such that
\[
  K_{\beta,L}^{\circ 8M}(x,\cdot)\ \ge\ \eta_*\, Q_{\mathrm{patch}}(\cdot),
\]
where $Q_{\mathrm{patch}}$ conditions one fixed finite block in each of the eight color classes to $B_G(e,r_G)$ (Definition~\ref{def:block-reference}) and is Haar elsewhere on the interface.
\end{theorem}
\begin{proof}
By (Ref) and (Win) one obtains, on $W_\varepsilon$, a per--class block refresh lower bound $K_{\beta,L}^{\circ M}\ge c_* Q^{(B_\alpha)}$. Averaging over $W_\varepsilon$ yields the uniform class constant $\eta_B:=p_0 c_*>0$ by Proposition~\ref{prop:block-refresh-doeblin}. Scheduling classes in a Gray cycle and composing eight microperiods, Proposition~\ref{prop:block-to-patch} gives
\[
  K_{\beta,L}^{\circ 8M}(x,\cdot)\ \ge\ \big(1-(1-\eta_B)^8\big)\, Q_{\mathrm{patch}}(\cdot).
\]
Set $\eta_*:=1-(1-\eta_B)^8$. Locality (Loc) ensures boundary independence outside a finite cone, so the constants are uniform in $L$ and in the boundary $x$.
\end{proof}
\begin{remark}
The constant $\eta_*$ depends only on the block size $b$, the window probability $p_0$, the single--link constants from Lemma~\ref{lem:single-link-refresh}, and small--ball Haar volumes (Lemma~\ref{lem:small-ball-volume}); it is uniform in the volume $L$ and the boundary $x$. Parameter dependence on $\beta$ is governed by the chosen refresh/window mechanism (see Lemma~\ref{lem:single-link-refresh}); no global $\beta$\,–\,independence is claimed unless explicitly stated for a scale--adapted choice.
\end{remark}

\begin{theorem}[Uniform near--identity staple window on fixed slabs]\label{thm:staple-window}
Fix a bounded slab $R$ intersecting the reflection plane and a compact simple gauge group $G$. There exist $\varepsilon_0\in(0,r_\sharp)$ and $p_0>0$, depending only on $(R_*,a_0,G)$, such that for all $\beta\ge \beta_{\min}(R,G)$, all volumes $L$, and all boundary data on the negative side,
\[
  \mathbb P\big(\,W_{\varepsilon_0}\,\big)\ \ge\ p_0,
\]
where $W_{\varepsilon_0}$ is the event that all positive--side links entering the staples of a fixed finite block of interface links lie in $B_G(e,\varepsilon_0)$. The constants are uniform in $(\beta,L,\text{boundary})$ on the fixed slab.
\end{theorem}
\begin{proof}
Tree gauge on $R$ yields an exact product Haar reference for interior links and a smooth Gibbs density $\propto e^{-S_R}$ with strictly local plaquette interactions. By UEI/LSI on fixed regions (Bakry--\'Emery on compact groups after gauge), there is a concentration inequality for each staple product map $\Phi$ built from finitely many links: for some $c_R>0$ and $C_R<\infty$ independent of $(\beta,L)$,
\[
  \mathbb P\big(\,d_G(\Phi, e)\le r\,\big)\ \ge\ 1- C_R\, e^{-c_R\,\beta\, r^2}\qquad(0<r\le r_\sharp).
\]
For a fixed finite block $B$ there are finitely many staple products $\Phi_j$ entering the six staples per link. By a union bound and choosing $\varepsilon_0\in(0,r_\sharp)$ small enough, the complement probability is bounded by $\sum_j C_R e^{-c_R\beta\,\varepsilon_0^2}\le 1-p_0$ uniformly in $\beta\ge \beta_{\min}$, after possibly shrinking $\varepsilon_0$ so that the right--hand side is $\le 1-p_0$ for some $p_0\in(0,1)$. This uses that at $\beta=0$ the law is Haar so every fixed ball has positive mass, and for large $\beta$ the staples concentrate at the identity. Hence $\mathbb P(W_{\varepsilon_0})\ge p_0$ uniformly.
\end{proof}

\begin{lemma}[Central heat--kernel pulse preserves symmetries]\label{lem:central-pulse}
Let $G=\mathrm{SU}(N)$ and let $H_t$ be the central heat--kernel density at time $t>0$. For a finite block $B$ of interface coordinates, define the convolution operator
\[
  (\mathcal H_t f)(u)\ :=\ \int_{G^{B}} \Big(\prod_{\ell\in B} H_t(v_\ell^{-1}u_\ell)\Big)\, f\big( v_B, u_{B^c}\big)\, d\pi^{\otimes B}(v_B),
\]
acting as heat--kernel convolution on $B$ and identity on $B^c$. Then $\mathcal H_t$ is positivity preserving, Haar--invariant on $B^c$, commutes with left/right translations (gauge covariance), and is compatible with OS reflection.
\end{lemma}
\begin{remark}[Constant dependence in Lemma~\ref{lem:local-commutator-Oa2}]
One may take
\[
  C_{\rm comm}(R,G)
  \ \le\ C_{\rm Strang}(R)\ +\ C_{\rm mag}(R,G)
\]
with $C_{\rm Strang}(R)$ the Strang remainder constant from the local sandwich (Theorem~\ref{TS:sandwich_main}) and
\[
  C_{\rm mag}(R,G)\ \le\ C_2\big(N,R;M_0(R),M_1(R),M_2(R)\big)\ +\ C_{\rm hk}(R)\,e^{-\tfrac12\lambda_1(G) t_0(R)}.
\]
Here $C_2(\cdot)$ is the plaquette$\to F^2$ constant from Theorem~\ref{DEC:plaquette-F2} (depending on gauge\,–\,invariant curvature bounds on $R$), and $C_{\rm hk}(R)$ is the Lipschitz constant for the product heat\,–\,kernel on the finite stencil touching $R$. All constants depend only on $(R,G,N)$ and are uniform in the volume and boundary conditions.
\end{remark}
\begin{proof}
Positivity preservation and Haar invariance follow from convolution with a positive central density and product Haar on $G^{B}$. Centrality of $H_t$ implies that for any $g\in G$, $H_t(g^{-1}xg)=H_t(x)$, yielding commutation with conjugations and left/right translations (gauge covariance at the block). Reflection compatibility holds since $H_t$ is time--slice local and central, hence invariant under the OS involution on the interface.
\end{proof}

\begin{proposition}[Sandwiching by a fixed pulse]\label{prop:sandwich-pulse}
Let $B$ be a finite block and suppose there exist $t_*>0$ and $c_*>0$ such that the operator inequality holds on $L^2_0$:
\[
  K_{\beta,L}^{\circ M}\ \ge\ c_*\, \mathcal H_{t_*}
\]
(as positive kernels). Then for any fixed radius $r_G>0$ there exists $\eta_0=\eta_0(t_*,r_G,c_*,G)>0$ such that
\[
  K_{\beta,L}^{\circ M}(x,\cdot)\ \ge\ \eta_0\, Q^{(B)}(\cdot)
\]
for the small--ball block law $Q^{(B)}$ of Definition~\ref{def:block-reference}, uniformly in $(\beta,L,x)$.
\end{proposition}
\begin{proof}
From Lemma~\ref{lem:central-pulse}, $\mathcal H_{t_*}$ has a strictly positive continuous density on $G^{B}$; in particular, $\inf_{u\in B_G(e,r_G)^{B}} (\mathcal H_{t_*}\mathbf 1)(u)\ge c(r_G,t_*,G)>0$. The sandwich then gives, for any measurable $A$,
\[
  K_{\beta,L}^{\circ M}(x,A)\ \ge\ c_*\, (\mathcal H_{t_*}\mathbf 1_{A})(x)\ \ge\ c_*\, c(r_G,t_*,G)\, Q^{(B)}(A),
\]
setting $\eta_0:=c_*\, c(r_G,t_*,G)>0$.
\end{proof}

\begin{proposition}[Uniform ergodicity in finite blocks]\label{prop:uniform-ergodicity-blocks}
Suppose there exist $M\in\mathbb N$, $\eta_0\in(0,1]$, and a probability $\nu$ on the interface space such that for all $x$ and measurable $B$,
\[
  K_{\beta,L}^{\circ M}(x,B)\ \ge\ \eta_0\, \nu(B),
\]
with $\eta_0$ independent of $(\beta,L,x)$. Then for any probability densities $p,q$ on the interface configuration space and any $n\in\mathbb N$,
\[
  \big\|\, p (K_{\beta,L}^{\circ M})^{n} - q (K_{\beta,L}^{\circ M})^{n} \,\big\|_{\mathrm{TV}}
   \ \le\ (1-\eta_0)^{n}\, \|p-q\|_{\mathrm{TV}}.
\]
\end{proposition}
\begin{proof}
Doeblin's condition yields a one--step coupling for $K_{\beta,L}^{\circ M}$ with success probability $\eta_0$. Iterating the coupling gives geometric decay of the total--variation distance by the factor $1-\eta_0$ per $M$--block.
\end{proof}

\begin{theorem}[Exponential clustering along interface time]\label{thm:exp-cluster-interface}
Let $F,G$ be bounded observables depending on disjoint time slices separated by $n$ blocks of length $M$. Under the hypothesis of Proposition~\ref{prop:uniform-ergodicity-blocks},
\[
  \big|\, \mathrm{Cov}\,(F,\, G\circ (K_{\beta,L}^{\circ M})^{n})\,\big|\ \le\ 2\,\|F\|_\infty\,\|G\|_\infty\,(1-\eta_0)^{n}.
\]
Equivalently, writing the physical separation as $t=n\,T_{\mathrm{block}}$ with $T_{\mathrm{block}}:=M\,a$,
\[
  \big|\, \mathrm{Cov}\,(F,\, G_t)\,\big|\ \le\ 2\,\|F\|_\infty\,\|G\|_\infty\, \exp\!\Big(-\, \tfrac{t}{T_{\mathrm{block}}}\, |\log(1-\eta_0)|\Big).
\]
\end{theorem}
\begin{proof}
Write centered versions $\widetilde F:=F-\mathbb E F$, $\widetilde G:=G-\mathbb E G$. The covariance equals $\int \widetilde F\, d\mu\, - \int \widetilde F\, d\mu'$, where $\mu' := (K_{\beta,L}^{\circ M})^{n}\mu$ and the initial measures differ only through the slice of $G$. By the total--variation contraction in Proposition~\ref{prop:uniform-ergodicity-blocks},
\[
  |\mathrm{Cov}(F, G\circ (K_{\beta,L}^{\circ M})^{n})| \ \le\ \|\widetilde F\|_\infty\, \big\|\mu (K_{\beta,L}^{\circ M})^{n} - \mu' (K_{\beta,L}^{\circ M})^{n}\big\|_{\mathrm{TV}}
   \ \le\ 2\,\|F\|_\infty\,\|G\|_\infty\,(1-\eta_0)^{n},
\]
using $\|\widetilde F\|_\infty\le 2\|F\|_\infty$ and an analogous bound for $\widetilde G$.
\end{proof}

\begin{proposition}[From block refresh to patch refresh in finite time]\label{prop:block-to-patch}
Let $\{\mathcal C_\alpha\}_{\alpha\in\{0,1\}^3}$ be the eight parity classes of interface links (Proposition~\ref{prop:eight-color-schedule}). Suppose each class contains a fixed--size block $B_\alpha\subset\mathcal C_\alpha$ such that the microperiod kernel $K_{\beta,L}^{\circ M}$ satisfies, whenever class $\alpha$ is scheduled,
\[
  K_{\beta,L}^{\circ M}(x,\cdot)\ \ge\ \eta_B\, Q^{(B_\alpha)}(\cdot)
\]
with the same $\eta_B>0$ for all $(\alpha,\beta,L,x)$. Then after eight microperiods in a Gray--code schedule,
\[
  K_{\beta,L}^{\circ 8M}(x,\cdot)\ \ge\ \eta_*\, Q_{\mathrm{patch}}(\cdot),\qquad
  \eta_*\ :=\ 1-(1-\eta_B)^8,
\]
where $Q_{\mathrm{patch}}$ is the product law that conditions each class block to $B_G(e,r_G)$ (as in Definition~\ref{def:block-reference}) and is Haar elsewhere.
\end{proposition}
\begin{proof}
Write $\mathcal K_k$ for the kernel after $k$ microperiods and let $\mathcal R_k$ denote the set of refreshed blocks after $k$ steps. The hypothesis yields the mixture lower bound
\[
  \mathcal K_{k+1}\ \ge\ (1-\eta_B)\,\mathcal K_k\ +\ \eta_B\, Q^{(B_{\alpha_k})}\,.
\]
By induction and disjointness of the blocks, after $k$ distinct classes the lower bound is a convex combination of $\{Q^{(B_{\alpha_j})}\}_{j\le k}$ with total weight $1-(1-\eta_B)^k$. After eight distinct classes (Gray cycle), the product structure of $Q_{\mathrm{patch}}$ and independence across disjoint coordinates give
\[
  \mathcal K_{8}\ \ge\ 1-(1-\eta_B)^8\, \cdot\, Q_{\mathrm{patch}}\,.
\]
Renaming $\mathcal K_8=K_{\beta,L}^{\circ 8M}$ yields the claim with $\eta_*=1-(1-\eta_B)^8$.
\end{proof}
\begin{proposition}[Interface density: absolute continuity and $\beta$--uniform ball--average bound]\label{prop:interface-density-ball-avg}
Work on a fixed physical slab $R\supset\Sigma$ and $a\in(0,a_0]$ with $m:=|\Sigma|$ interface links and Haar probability $\pi$ on $G=\mathrm{SU}(N)$. For any exterior boundary $b$ and any $\beta\ge \beta_{\min}(R,N)$, let $\mu_{\Sigma}^{(\beta,b)}(du)=f_{\beta,b}(u)\,\pi^{\otimes m}(du)$ denote the interface marginal.
\begin{itemize}
  \item[(i)] $f_{\beta,b}\in C^\infty(G^m)$ and $f_{\beta,b}>0$ everywhere.
  \item[(ii)] There exist constants $L_\Sigma=L_\Sigma(R,N)$ and $C_1=C_1(R,N)$ such that for all $u\in G^m$ and all $r\in(0,1)$,
  \[
    \frac{1}{\pi^{\otimes m}(B_r)}\int_{B_r(u)} f_{\beta,b}(v)\,\pi^{\otimes m}(dv)
    \ \ge\ e^{-\beta\,(L_\Sigma r + C_1 r^2)}\, f_{\beta,b}(u),
  \]
  where $B_r(u)\subset G^m$ is the geodesic ball of radius $r$ (for a fixed bi--invariant metric) and $\pi^{\otimes m}(B_r):=\pi^{\otimes m}(B_r(u))$ is its Haar mass (independent of $u$). In particular, for $r=\kappa/\beta$ with $\kappa\in(0,1)$,
  \[
    \frac{1}{\pi^{\otimes m}(B_{\kappa/\beta})}\int_{B_{\kappa/\beta}(u)} f_{\beta,b}(v)\,\pi^{\otimes m}(dv)
    \ \ge\ c_*(R,N)\,\kappa^{\,m\dim G}\, e^{-L_\Sigma\kappa}\, f_{\beta,b}(u),
  \]
  with $c_*(R,N)>0$ depending only on the local geometry and $N$.
\end{itemize}
\end{proposition}
\begin{proof}
Write $\mathcal S_R(u,y;b)$ for the \emph{local Wilson action on $R$} after tree gauge, as a function of the interface variables $u=U|_{\Sigma}$, the remaining variables $y=U|_Y$, and exterior boundary data $b$. (This $\mathcal S_R$ includes the explicit factor $\beta$; it should not be confused with the plaquette sum $S_R(U)=\sum_{p\subset R}\phi(U_p)$ used in the UEI statements later.)
Tree gauge on a spanning tree $T\subset E(R)$ that avoids $\Sigma$ fixes $U_e=\mathbf 1$ for $e\in T$ by vertex gauges; the associated change of variables is a product of left/right translations and preserves Haar measure on each link, so the joint law on $(u,y)=(U|_{\Sigma},U|_Y)$ is $Z_R^{-1}e^{-\mathcal S_R(u,y;b)}\,\pi^{\otimes m}(du)\,\pi^{\otimes |Y|}(dy)$. Since $\mathcal S_R$ is smooth and strictly positive, Fubini implies $Z(u):=\int e^{-\mathcal S_R(u,y;b)}\,\pi^{\otimes |Y|}(dy)$ is $C^\infty$ and strictly positive on $G^m$, hence $f_{\beta,b}(u):=Z(u)/\int Z\,d\pi^{\otimes m}\in C^\infty$ and $>0$ (this proves (i)).
For (ii), let $\mathcal P_\Sigma$ denote the plaquettes in $R$ that depend on $u$. For the Wilson term $\phi_p(U)=1-\tfrac1N\operatorname{ReTr}\,U_p$, one has a uniform differential bound $\|\nabla_{U_e}\phi_p\|\le C_p(N)$, whence, for some $L_\Sigma=C_p(N)\,|\mathcal P_\Sigma|$,
\[
  |\mathcal S_R(u,y;b)-\mathcal S_R(v,y;b)|\ \le\ \beta\,L_\Sigma\, d_{G^m}(u,v)\qquad(\text{all }y,b),
\]
with $d_{G^m}$ the product Riemannian distance. Set $h_{u,v}(Y):=\mathcal S_R(u,Y;b)-\mathcal S_R(v,Y;b)$. After tree gauge, Theorem~\ref{thm:U1-lsi-uei} gives an LSI for the conditional measure $\mu_v(dy)\propto e^{-\mathcal S_R(v,y;b)}\,\pi^{\otimes |Y|}(dy)$ with constant $\rho_R\ge c(R,N)\,\beta$. Moreover, $\|\nabla_Y h_{u,v}\|\le L_0(R,N)\, d_{G^m}(u,v)$. The Herbst argument under LSI yields the local log--Lipschitz estimate
\[
  \big|\log Z(u)-\log Z(v)\big|\ \le\ \beta\,\big(L_\Sigma r + C_1 r^2\big)\qquad(r:=d_{G^m}(u,v)),\quad C_1:=\tfrac{L_0(R,N)^2}{2c(R,N)}.
\]
Fix $u$ and average over $v\in B_r(u)$; by concavity of $\log$,
\[
  \frac{1}{\pi^{\otimes m}(B_r)}\int_{B_r(u)}\! \log Z(v)\,d\pi^{\otimes m}(v)\ \le\ \log\Big(\frac{1}{\pi^{\otimes m}(B_r)}\int_{B_r(u)}\! Z(v)\,d\pi^{\otimes m}(v)\Big),
\]
so the previous display implies
\[
  Z(u)\ \le\ e^{\beta(L_\Sigma r + C_1 r^2)}\,\frac{1}{\pi^{\otimes m}(B_r)}\int_{B_r(u)} Z(v)\,d\pi^{\otimes m}(v)
\]
and, by symmetry of the log--Lipschitz bound, also the reverse inequality with $\ge$ and $e^{-\beta(\cdots)}$. Dividing by $\int Z\,d\pi^{\otimes m}$ gives the stated two--sided control of the ball average in terms of $f_{\beta,b}(u)$; the displayed lower bound follows. The small--ball volume asymptotics on compact Lie groups (uniform in $u$) yield $\pi^{\otimes m}(B_{\kappa/\beta})\ge c_*(R,N)\, (\kappa/\beta)^{m\dim G}$, which gives the explicit form when $r=\kappa/\beta$.
\end{proof}

\begin{remark}[No pointwise $\beta$--uniform lower bound without smoothing]\label{rem:no-pointwise-lower}
Because $\mathcal S_R$ carries an explicit factor $\beta$, the log--Lipschitz estimate shows that $\log f_{\beta,b}$ can oscillate by $\asymp \beta$ over $O(1)$ distances. On a compact group this precludes any pointwise lower bound $\inf f_{\beta,b}\ge c>0$ that is uniform in $\beta$ without either shrinking the radius $r\sim 1/\beta$ in an averaged statement as above, or introducing short--time heat--kernel smoothing. The latter is exactly what yields the convex split $K_{\rm int}^{(a)}=\theta_* P_{t_0}+(1-\theta_*)\mathcal K$ in Proposition~\ref{prop:doeblin-full}/Corollary~\ref{cor:convex-split}.
\end{remark}

\begin{lemma}[Small-ball Haar volume on compact simple Lie groups]\label{lem:small-ball-volume}
Let $G$ be a compact simple Lie group with bi--invariant Riemannian metric and normalized Haar measure $\lambda_G$. There exist $r_*>0$ and $C_G>0$ such that for all $r\in(0,r_*)$ the geodesic ball $B_G(e,r)$ satisfies
\[
  \lambda_G\big(B_G(e,r)\big)\ \ge\ C_G\, r^{\dim G}.
\]
In particular, for $G=\mathrm{SU}(3)$ one may take $\dim G=8$ and obtain $\lambda_G(B_G(e,r))\ge C_G r^{8}$ for small $r$.
\end{lemma}
\begin{proof}
For sufficiently small $r$, the exponential map $\exp: \mathfrak g\to G$ is a diffeomorphism from the metric ball $B_{\mathfrak g}(0,r)$ onto $B_G(e,r)$. The Haar measure coincides with the Riemannian volume, whose density in normal coordinates is smooth with Jacobian $J(X)$ satisfying $J(0)=1$ and $J(X)\ge c_0>0$ on $B_{\mathfrak g}(0,r_*)$ for some $r_*>0$. Therefore
\[
  \lambda_G\big(B_G(e,r)\big)
   \ =\ \int_{B_{\mathfrak g}(0,r)} J(X)\,dX
   \ \ge\ c_0\, \operatorname{Vol}_{\mathbb R^{\dim G}}\big(B_{\mathfrak g}(0,r)\big)
   \ \ge\ C_G\, r^{\dim G}
\]
with $C_G:=c_0\, \operatorname{Vol}(B_{\mathbb R^{\dim G}}(0,1))$.
\end{proof}

\begin{corollary}[Concrete choice of $r_G$ for $\mathrm{SU}(3)$]\label{cor:explicit-rG-SU3}
Let $G=\mathrm{SU}(3)$ and let $r_*>0$ and $C_G>0$ be as in Lemma~\ref{lem:small-ball-volume}. For any target $\theta\in(0, C_G r_*^{8}]$, define
\[
  r_G\ :=\ \min\Big\{\,r_*\,,\ \big(\theta/C_G\big)^{1/8}\,\Big\}.
\]
Then $\lambda_G\big(B_G(e,r_G)\big)\ge \theta$. This provides an explicit small--ball radius for the block reference measure $Q^{(B)}$ (Definition~\ref{def:block-reference}) with constants uniform in the volume $L$.
\end{corollary}

\begin{remark}[No global one--step minorization with atomic references]\label{rem:no-global-minorization-atomic}
A global one--step Doeblin bound $K_{\rm int}^{(a)}(U,\cdot)\ge \rho\,\nu(\cdot)$ cannot hold uniformly in $(\beta,L,U)$ if the reference $\nu$ is supported on a set of Haar measure zero (e.g., a Dirac mass or a finite atomic combination). Indeed, for large $\beta$ and suitable negative--half boundary data, the conditional law develops sharply peaked modes around configurations that depend on the boundary; choosing disjoint small neighborhoods of two such modes yields a contradiction with any fixed atomic $\nu$ and uniform $\rho>0$. This does not contradict the heat--kernel convex split of Corollary~\ref{cor:convex-split}, where $\nu=P_{t_0}$ is absolutely continuous with a smooth, strictly positive density on $G^m$.
\end{remark}

\begin{lemma}[Finite--step domain of dependence for interface patches]\label{lem:finite-step-cone}
Let $P\subset \Sigma$ be the set of interface links whose midpoints lie in a fixed spatial ball $B_{R_*}\cap\Sigma$. For $n\in\mathbb N$, let $K^{(n)}_{\rm int}:=(K_{\rm int}^{(a)})^{\,n}$ and let $\mathcal F_P$ be the sigma--algebra generated by the outgoing interface links on $P$ at time $n a$. Define the backward $n$--step lattice cone $\mathcal C_n(P)$ as the smallest set of negative--half links with the property that every plaquette path of length $\le n$ in the time--oriented lattice graph from $P$ to the negative half is contained in $\mathcal C_n(P)\cup P$. Then for any bounded $\mathcal F_P$--measurable $\varphi$ and any two negative--half configurations $U,U'$ with $U|_{\mathcal C_n(P)}=U'|_{\mathcal C_n(P)}$ one has
\[
  \big(K^{(n)}_{\rm int}\varphi\big)(U)\ =\ \big(K^{(n)}_{\rm int}\varphi\big)(U').
\]
Equivalently, $K^{(n)}_{\rm int}$ restricted to observables on $P$ depends only on the boundary data on $\mathcal C_n(P)$.
\end{lemma}
\begin{proof}
We argue by induction on $n$. For $n=1$, $K_{\rm int}^{(a)}$ is obtained by integrating the positive slab of thickness $a$. By locality, the Wilson action on that slab decomposes as $S_{\rm slab}=S_{\rm loc}(U|_{\mathcal C_1(P)},\,U|_P,\,Y_{\rm loc})+S_{\rm out}(Y_{\rm out})$, where $Y_{\rm loc}$ collects links in the positive slab that belong to plaquettes meeting $\mathcal C_1(P)\cup P$, and $Y_{\rm out}$ the remaining positive--half links. Hence the numerator $Z(U;\varphi):=\int e^{-S_{\rm slab}}\,\varphi\,d\pi$ and the normalizing factor $Z(U):=\int e^{-S_{\rm slab}}d\pi$ factor through $S_{\rm out}$, which cancels in the ratio. Therefore $(K_{\rm int}^{(a)}\varphi)(U)$ depends only on $U|_{\mathcal C_1(P)}$.

Assume the claim for $n-1$. Then $K^{(n)}_{\rm int}\varphi=K_{\rm int}^{(a)}\big(K^{(n-1)}_{\rm int}\varphi\big)$. By the induction hypothesis, $\psi:=K^{(n-1)}_{\rm int}\varphi$ depends only on boundary data in $\mathcal C_{n-1}(P)$. Applying the $n=1$ case to $\psi$ with patch enlarged to the set of interface links that can influence $P$ in $n-1$ steps shows that $K_{\rm int}^{(a)}\psi$ depends only on boundary data in $\mathcal C_1(\mathcal C_{n-1}(P))$, which is precisely $\mathcal C_n(P)$ by definition of plaquette paths. This completes the induction.
\end{proof}

\begin{remark}[Boundary decoupling and van Hove limit]\label{rem:cone-decoupling}
Fix $T>0$ and set $n=\lceil T/a\rceil$. For $L a\gg R_*$, the backward cone $\mathcal C_n(P)$ is contained in a finite region independent of $L$, so modifications of the boundary outside $\mathcal C_n(P)$ leave $K^{(n)}_{\rm int}$ unchanged. In particular, along van Hove sequences ($a\downarrow 0$, $La\to\infty$) with fixed $n$, dependence on far boundary data vanishes exactly by locality.
\end{remark}

\begin{corollary}[Deterministic locality radius for interface dependence]\label{cor:deterministic-locality}
Let $S\subset \Sigma$ be a finite set of interface links (a patch). For any $n\in\mathbb N$, the $n$--step interface kernel $K^{(n)}_{\rm int}:=(K_{\rm int}^{(a)})^{\,n}$ restricted to observables supported on $S$ depends only on
\begin{itemize}
  \item the negative--half boundary configuration on the backward cone $\mathcal C_n(S)$ (plaquette paths of length $\le n$ reaching $S$ across $\Sigma$), and
  \item the positive--half links within the forward $n$--neighborhood of $S$ in the oriented plaquette graph.
\end{itemize}
Consequently, if two negative--half configurations agree on $\mathcal C_n(S)$, then $K^{(n)}_{\rm int}$ yields the same law on $S$ for both boundaries. Along van Hove sequences ($a\downarrow 0$, $La\to\infty$) with fixed $n$, dependence on far boundary data vanishes exactly by locality.
\end{corollary}
\begin{proof}
Apply Lemma~\ref{lem:finite-step-cone} to the patch $S$ and note that, by locality of the Wilson action, links in the positive half outside the forward $n$--neighborhood factor from both the numerator and denominator of the conditional kernels at each step. Composition preserves this property.
\end{proof}

\begin{corollary}[Operator--norm semigroup convergence on compact times]\label{cor:semigroup-on-compact}
Under the hypotheses of Theorem~\ref{thm:nrc-operator-norm-fixed} and Corollary~\ref{cor:nrc-allz-fixed}, for every $T>0$ there exists $C_T(R,N)>0$ such that
\[
  \sup_{t\in[0,T]}\ \big\|\,e^{-t H_R}\ -\ I_{a,R}\,e^{-t H_{a,R}}\,I_{a,R}^*\,\big\|\ \le\ C_T\, a\,.
\]
\end{corollary}
\begin{proof}
Fix $T>0$ and represent $e^{-tH}$ by the inverse Laplace transform on a vertical line $\{\Re z=\sigma>0\}$. The uniform resolvent bound of Corollary~\ref{cor:nrc-allz-fixed} along this line and the weighted resolvent bounds (Lemma~\ref{lem:weighted-resolvent}) yield an $O(a)$ integrand difference uniformly in $t\in[0,T]$. Dominated convergence on the contour gives the stated operator--norm estimate with $C_T$ depending on $T$ via the contour choice.
\end{proof}

\begin{lemma}[Thermodynamic limit preserves slab-uniform contraction]\label{lem:thermo-preserve-gap}
Fix a slab $R_*$ and constants $(\theta_*,t_0)$ from Corollary~\ref{cor:hk-convex-split-explicit}. For each finite lateral size $L$, let $T_{a,L}$ be the one-tick transfer on the OS/GNS space with reflection cut across $R_*$. Then the parity-odd one-step bound $\|T_{a,L}\|_{\rm odd}\le 1-\theta_*(1-e^{-\lambda_1(G) t_0})$ holds with constants independent of $L$. Consequently, any weak-* thermodynamic limit of the state and corresponding OS/GNS norm-limit of $T_{a,L}$ preserves the bound, and the eight-tick mean-zero contraction of Theorem~\ref{thm:eight-tick-uniform} persists in the limit.
\end{lemma}
\begin{proof}
All constants in Corollary~\ref{cor:hk-convex-split-explicit} are computed on the fixed slab and depend only on $(R_*,a_0,G)$; hence the odd-cone estimate is uniform in $L$. Weak-* convergence of finite-volume OS states on local algebras yields convergence of matrix elements of $T_{a,L}$ on a common dense core; lower semicontinuity of the operator norm under strong resolvent/semigroup convergence on this core gives the same bound for the limiting operator. The eight-tick upgrade is algebraic and uses only the uniform one-step constant and parity cycling, so it passes to the limit unchanged.
\end{proof}

\subsection*{Constants Map (Slab Chain)}
\label{subsec:constants-map-slab}

\Needspace{10\baselineskip}
\begin{mdframed}[linewidth=0.8pt, linecolor=black!30, backgroundcolor=yellow!5, roundcorner=3pt, innertopmargin=12pt, innerbottommargin=12pt, skipabove=12pt, skipbelow=12pt]
\begin{center}
\textit{\textbf{Note:} All constants are $\beta$/$L$-independent on fixed slabs}
\end{center}
\vspace{10pt}
\renewcommand{\arraystretch}{1.5}
\begin{center}
\begin{tabular}{@{}lp{9cm}@{}}
\toprule
\textbf{Symbol} & \textbf{Definition / Source (and Role)} \\
\midrule
$t_0$ & Product heat\,–\,kernel time in the interface smoothing split; Thm.~\ref{thm:ucis-sw}, Cor.~\ref{cor:ucis-sw-L2-contraction}. \\
$\theta_*$ & Heat\,–\,kernel Doeblin weight in the interface smoothing split; Thm.~\ref{thm:ucis-sw}. \\
$c_{\rm cut,phys}$ & Per\,–\,tick odd\,–\,cone deficit in physical units: $c_{\rm cut,phys}:= -\log\big(1-\theta_*(1-e^{-\lambda_1(G) t_0})\big)$; Thm.~\ref{thm:eight-tick-uniform}. \\
$\gamma_{\rm phys}$ & Eight\,–\,tick slab gap: $\gamma_{\rm phys}:= 8\,c_{\rm cut,phys}$; Thm.~\ref{thm:eight-tick-uniform} (thermodynamic limit: Lem.~\ref{lem:thermo-preserve-gap}; global gap: Thm.~\ref{thm:global-gap-uncond}). \\
\bottomrule
\end{tabular}
\end{center}
\end{mdframed}

\section*{Appendix: Global Constants Table (Provenance)}
\addcontentsline{toc}{section}{Appendix: Global Constants Table (Provenance)}
\Needspace{12\baselineskip}
\begin{mdframed}[linewidth=0.8pt, linecolor=black!30, backgroundcolor=gray!3, roundcorner=3pt, innertopmargin=10pt, innerbottommargin=10pt, skipabove=12pt, skipbelow=12pt]
\renewcommand{\arraystretch}{1.35}
\begin{center}
\begin{tabular}{@{}lp{8.8cm}@{}}
\toprule
\textbf{Symbol} & \textbf{Definition / Provenance} \\
\midrule
$\theta_*$ & Doeblin/convex-split weight; Prop.~\ref{prop:doeblin-interface}, Prop.~\ref{prop:doeblin-full}, Cor.~\ref{cor:hk-convex-split-explicit} \\
$t_0$ & Heat-kernel time on $G$; Prop.~\ref{prop:doeblin-interface}, Lem.~\ref{lem:ball-conv-lower} \\
$\lambda_1(G)$ & First nonzero Laplace--Beltrami eigenvalue on $G$; HK contraction Lem.~\ref{lem:hk-contraction} \\
$c_{\rm cut,phys}$ & $-\log(1-\theta_*(1-e^{-\lambda_1(G) t_0}))$; Thm.~\ref{thm:eight-tick-uniform} \\
$\gamma_*$ & $8\,c_{\rm cut,phys}$; Thm.~\ref{thm:pf-gap-meanzero}, Thm.~\ref{thm:main-af-free} \\
$C_g,\nu$ & Local basis growth constants; Lem.~\ref{lem:local-basis-growth} \\
$A,\mu$ & OS Gram decay constants; Lem.~\ref{lem:local-gram-bounds} \\
$B,\nu'$ & Mixed Gram decay constants; Lem.~\ref{lem:mixed-gram-bound} \\
$S_0$ & Off-diagonal tail bound; Lem.~\ref{lem:mixed-gram-bound} \\
$\rho$ & Diagonal mixed Gram bound; Lem.~\ref{lem:diag-mixed-bound} \\
$\beta_0$ & Two-layer deficit; Prop.~\ref{prop:two-layer-deficit}, Thm.~\ref{thm:two-layer-explicit} \\
$I_{a,R}$ & Isometric embedding; Lem.~\ref{lem:isometric-embeddings} \\
$C_R$ & Graph-defect constant; Lem.~\ref{lem:graph-defect-core} \\
$C_K, C(z)$ & Resolvent bounds; Thm.~\ref{thm:nrc-operator-norm-fixed}, Lem.~\ref{lem:weighted-resolvent} \\
\bottomrule
\end{tabular}
\end{center}
\end{mdframed}

\begin{mdframed}[linewidth=0.5pt, linecolor=gray!40, backgroundcolor=gray!5, roundcorner=2pt, innertopmargin=8pt, innerbottommargin=8pt, skipabove=8pt]
\noindent\textit{\textbf{Dependence.}} There are two interface-smoothing routes recorded in this manuscript.
\emph{Unconditional one-step (coarse-refresh) route:} after coarse refresh, one obtains constants $(\theta_*,t_0)$ uniform in $L$ and independent of $\beta$ (Lemma~\ref{lem:beta-L-independent-minorization}, Cor.~\ref{cor:hk-convex-split-explicit}), feeding the one-tick odd-chain (Cor.~\ref{cor:odd-contraction-from-Kint}, Thm.~\ref{thm:uniform-odd-contraction}).
\emph{Scaling-window (physically scaled) route:} along schedules satisfying \eqref{eq:ucis-sw-window}, UCIS$_{\rm SW}$ (Thm.~\ref{thm:ucis-sw}) supplies a fixed-physical-time minorization for $K_{\rm int}^{(a)\,M(a)}$ with constants depending only on $(G,\varepsilon,T_{\rm phys},C_{\rm SW})$, yielding the fixed-physical-time $L^2$ contraction (Cor.~\ref{cor:ucis-sw-L2-contraction}) and parity-odd contraction (Thm.~\ref{thm:ucis-sw-odd-subspace}).
Downstream steps (thermodynamic limit, NRC, OS$\to$Wightman) use only the existence of such a slab-uniform contraction/gap input and its stated uniformities.
\end{mdframed}

\begin{proposition}[Eight--color schedule on the interface]\label{prop:eight-color-schedule}
Identify time--like interface links by their spatial footpoints $(i,j,k)\in\mathbb Z^3$ on the reflection plane. For $\alpha\in\{0,1\}^3$, define the classes
\[
  \mathcal C_{\alpha}
   \\ :=\ \big\{\text{interface links with }(i\bmod 2,\,j\bmod 2,\,k\bmod 2)=\alpha\big\}.
\]
Then no two links in the same class share a time--space plaquette. Moreover, visiting the classes in any Gray--code order on $\{0,1\}^3$ gives an $8$--tick cycle that updates each class once without plaquette conflicts.
\end{proposition}
\begin{proof}
Two time--like interface links share a time--space plaquette iff their footpoints differ by $\pm1$ in exactly one spatial coordinate and are equal in the other two. Such a move flips parity in that coordinate, so the two links lie in different parity classes $\mathcal C_{\alpha}$. Hence updates within a fixed class are plaquette--disjoint. A Gray code on the $3$--cube is a Hamiltonian cycle that visits each parity vector once with successive vectors differing in exactly one bit, so scheduling classes according to a Gray order yields an $8$--tick conflict--free cycle.
\end{proof}

\begin{definition}[Block reference law]\label{def:block-reference}
Let $B\subset \Sigma$ be a finite block of interface links with $|B|=b$ independent of $(\beta,L)$ (e.g., one link from each parity class of Proposition~\ref{prop:eight-color-schedule}). Fix a small group radius $r_G>0$. Define the probability law $Q^{(B)}$ on the interface configuration space by taking the coordinates in $B$ to be i.i.d. Haar restricted to the geodesic ball $B_G(e,r_G)$ (normalized), and all other coordinates Haar on $G$; i.e., $Q^{(B)}$ is the product of these marginals.
\end{definition}

\begin{lemma}[One--link refresh $\Rightarrow$ Doeblin for a fixed power]\label{lem:one-link-doeblin}
Fix $M\in\mathbb N$ and a singleton block $B=\{\ell_*\}$. If there exists $\eta_0\in(0,1]$ such that for all $x$,
\[
  K_{\beta,L}^{\circ M}(x,\cdot)\ \ge\ \eta_0\, Q^{(B)}(\cdot),
\]
then $K_{\beta,L}^{\circ M}$ satisfies a Doeblin/minorization with constant $\rho=\eta_0$ and reference $\nu=Q^{(B)}$. If $\eta_0$ is independent of $(\beta,L,x)$, the bound is uniform in these parameters.
\end{lemma}
\begin{proof}
The displayed inequality is exactly the Doeblin/minorization with $(\rho,\nu)=(\eta_0, Q^{(B)})$; uniformity follows when $\eta_0$ is parameter--independent.
\end{proof}

\begin{corollary}[Eight--tick one--link case]\label{cor:8tick-one-link}
If the hypothesis of Lemma~\ref{lem:one-link-doeblin} holds with $M=8$ and $\eta_0>0$ independent of $(\beta,L,x)$, then $K_{\beta,L}^{\circ 8}$ obeys a uniform Doeblin bound with $(\rho,\nu)=(\eta_0, Q^{(B)})$.
\end{corollary}
\begin{proof}
Apply Lemma~\ref{lem:one-link-doeblin} with $M=8$.
\end{proof}

\begin{proposition}[Finite--block refresh $\Rightarrow$ Doeblin for a power]\label{prop:block-refresh-doeblin}
Let $K_{\beta,L}$ be the one--slice interface kernel and $K_{\beta,L}^{\circ M}$ the $M$--fold composition for some fixed $M\in\mathbb N$. Suppose there exists a measurable positive--side window $W_\varepsilon$ and constants $p_0,c_*>0$ (independent of $(\beta,L,x)$) such that for all boundary data $x$:
\[
  \text{(i) }\mathbb P(W_\varepsilon)\ \ge\ p_0,\qquad
  \text{(ii) on }W_\varepsilon:\quad K_{\beta,L}^{\circ M}(x,\cdot)\ \ge\ c_*\, Q^{(B)}(\cdot).
\]
Then for all $(\beta,L,x)$ and measurable $A$,
\[
  K_{\beta,L}^{\circ M}(x,A)\ \ge\ \eta_B\, Q^{(B)}(A),\qquad \eta_B:=p_0 c_*\,.
\]
\end{proposition}
\begin{proof}
Decompose according to $W_\varepsilon$:
\[
  K_{\beta,L}^{\circ M}(x,A)
   \ =\ \mathbb E\big[\,\mathbf 1_{W_\varepsilon}\, K_{\beta,L}^{\circ M}(x,A)\,\big]
       + \mathbb E\big[\,\mathbf 1_{W_\varepsilon^c}\, K_{\beta,L}^{\circ M}(x,A)\,\big]
   \ \ge\ p_0\, c_*\, Q^{(B)}(A),
\]
using (ii) on $W_\varepsilon$ and (i) for $\mathbb P(W_\varepsilon)$. This yields the stated Doeblin lower bound with constant $\eta_B=p_0 c_*$. 
\end{proof}

\begin{lemma}[Single--link refresh under near--identity staples]\label{lem:single-link-refresh}
Let $G=\mathrm{SU}(N)$ with Haar probability $\pi$. Fix an interface link $\ell$ and write its positive--side staple product as $H_\ell\in G$ (the product of adjacent plaquette transporters not involving $U_\ell$). There exist $\varepsilon_0,r_0,\kappa_0>0$ and constants $c_0,C>0$ (depending only on $N$ and local geometry) such that for all $\beta\ge\beta_{\min}>0$:
\begin{itemize}
  \item[(a) Scale--adapted form (\(\beta\)--uniform).] If $H_\ell\in B_G(e,\varepsilon_0)$, then for every $\kappa\in(0,\kappa_0)$ and $r_G:=\kappa\,\beta^{-1/2}\le r_0$,
  \[
    \mathbb P\big(\,U_\ell\in B_G(e,r_G)\ \big|\ \text{all other variables}\,\big)
      \ \ge\ c_0\, \kappa^{\,\dim G}\, e^{-C\,\kappa^2},
  \]
  with the right side independent of $(\beta,L,x)$.
  \item[(b) Fixed--radius variant (uniform for $\beta\ge\beta_{\min}$).] For any fixed $r_G\in(0,r_0]$ there exists $c_{\mathrm{fix}}(r_G,\varepsilon_0,\beta_{\min})>0$ such that under $H_\ell\in B_G(e,\varepsilon_0)$,
  \[
    \mathbb P\big(\,U_\ell\in B_G(e,r_G)\ \big|\ \text{all other variables}\,\big)
      \ \ge\ c_{\mathrm{fix}}(r_G,\varepsilon_0,\beta_{\min}),
      \qquad\text{uniformly for all }\beta\ge\beta_{\min}.
  \]
  One explicit choice is obtained by taking $\kappa:=\min\{\kappa_0/2,\ r_G\sqrt{\beta_{\min}}\}$ in (a), since then $\kappa/\sqrt{\beta}\le r_G$ for all $\beta\ge\beta_{\min}$.
\end{itemize}
\end{lemma}
\begin{proof}
This statement is established in full for $G=\mathrm{SU}(3)$ below via Lemma~\ref{lem:su3-taylor} and Proposition~\ref{prop:su3-scale-mass}, which provide the explicit Taylor--remainder control at the polar maximizer and the ensuing mass bound on $B_G(u_\*,\kappa/\beta)$. For general compact simple $G$, the same argument goes through with $\dim G$ in place of $8$ after replacing Lemma~\ref{lem:su3-taylor} by its $G$--version (Taylor expansion in exponential coordinates with a positive quadratic form controlled by the smallest eigenvalue of the Hermitian polar part and a uniform cubic remainder). The constants depend only on the group geometry and the local staple window.\end{proof}

\begin{lemma}[SU(3) Taylor control around the polar maximizer]\label{lem:su3-taylor}
Let $G=\mathrm{SU}(3)$ and suppose the positive--side staples entering a fixed link $\ell$ lie in a near--identity window of radius $r_{\mathrm{st}}\in(0,r_\sharp)$, so that for the polar decomposition $W_\ell=QH$ one has $\|Q-\gamma_\* I\|\le c_2 r_{\mathrm{st}}$ and $d_G(H,e)\le c_1 r_{\mathrm{st}}$ with $\lambda_\*:=\gamma_\*-c_2 r_{\mathrm{st}}>0$. Setting $u_\*:=H^{-1}$, there exist $r_0>0$ and $C_3,C_J>0$ (depending only on the window) such that for all $X\in\mathfrak{su}(3)$ with $\|X\|_F\le r_0$,
\[
  \operatorname{Re}\operatorname{tr}(u_\* e^{X} W_\ell)\ \ge\ \operatorname{tr}(Q)\ -\ \tfrac{\lambda_\*}{2}\,\|X\|_F^2\ -\ C_3\,\|X\|_F^3,
\]
and the exponential--chart Jacobian $J(X)$ obeys $1-C_J\|X\|_F^2\le J(X)\le 1+C_J\|X\|_F^2$.
\end{lemma}
\begin{proof}
Left--translate by $u_\*^{-1}$ (Haar invariance): $\operatorname{Re}\operatorname{tr}(u_\* e^{X} W_\ell)=\operatorname{Re}\operatorname{tr}(e^{X} Q')$ where $Q':=u_\* W_\ell=H^{-1} Q H$ is positive Hermitian with $\lambda_{\min}(Q')\ge \lambda_\*=\gamma_\*-c_2 r_{\mathrm{st}}>0$. Expand $e^{X}=I+X+\tfrac12 X^2+R_3(X)$ with $\|R_3(X)\|_F\le C\|X\|_F^{3}$ for $\|X\|_F\le r_0$. Since $X\in\mathfrak{su}(3)$ is anti--Hermitian and $Q'$ Hermitian, $\operatorname{Re}\operatorname{tr}(XQ')=0$. Moreover $X^2$ is Hermitian negative, hence $\operatorname{Re}\operatorname{tr}(\tfrac12 X^2 Q')\le -\tfrac12 \lambda_\* \|X\|_F^2$. Finally $|\operatorname{Re}\operatorname{tr}(R_3(X)Q')|\le \|R_3(X)\|_F\,\|Q'\|_F\le C_3\|X\|_F^3$. The Jacobian bounds for the exponential chart under a bi--invariant metric are standard on a normal neighborhood of the identity: $J(X)=1+O(\|X\|_F^2)$ uniformly, yielding the stated two--sided bounds with some $C_J>0$.
\end{proof}
\begin{proposition}[SU(3): one--link mass on $B_G(u_\*,\kappa/\beta)$]\label{prop:su3-scale-mass}
Under the conditions of Lemma~\ref{lem:su3-taylor}, there exist $c_0,c_1>0$ and $\beta_0\ge 1$ such that for all $\beta\ge \beta_0$ and all staples in the window,
\begin{equation}\label{eq:one-link-density}
  f_\beta(u \mid W_\ell)
  \;=\;
  Z_\ell(\beta)^{-1}\,\exp\!\big(\beta\,\mathrm{Re}\,\mathrm{tr}(u\,W_\ell)\big),
  \qquad
  Z_\ell(\beta)
  \;=\;\int_G \exp\!\big(\beta\,\mathrm{Re}\,\mathrm{tr}(v\,W_\ell)\big)\,d\lambda_G(v).
\end{equation}
\[
  \int_{B_G(u_\*,\,\kappa/\beta)} f_\beta(u\mid W_\ell)\,d\lambda_G(u)\ \ge\ c_0\,\kappa^{8}\,\beta^{-4}\, e^{-c_1 \kappa^3/\beta^{2}},\qquad \kappa\in(0,\kappa_0),
\]
with $f_\beta$ the one--link conditional density. In particular, for $\beta\ge \beta_0$ the right side is $\ge c_0\,\kappa^8\,\beta^{-4}$ up to an absorbed constant.
\end{proposition}
\begin{proof}
Change variables $u=u_\* e^X$ in \eqref{eq:one-link-density}, use Lemma~\ref{lem:su3-taylor} and $J(X)\asymp 1$ on $\|X\|\le r_0$ to bound the numerator from below by an integral over $\|X\|\le \kappa/\beta$ of $\exp\{-\alpha_\*\beta(\tfrac{\lambda_\*}{2}\|X\|^2+C_3\|X\|^3)\}$ and the denominator from above by a Gaussian integral with variance $\asymp (\alpha_\*\beta)^{-1}$. Estimating these yields the stated bound with explicit $\beta^{-4}$ scaling in dimension $8$.
\end{proof}

\begin{lemma}[Taylor control for compact simple $G$]\label{lem:g-taylor}
Let $G$ be a compact, connected, simple Lie group with bi--invariant metric and normalized Haar measure. Suppose the positive--side staples entering a fixed link $\ell$ lie in a near--identity window of radius $r_{\mathrm{st}}\in(0,r_\sharp)$, so that for the polar decomposition $W_\ell=QH$ one has $\|Q-\gamma_\* I\|\le c_2 r_{\mathrm{st}}$ and $d_G(H,e)\le c_1 r_{\mathrm{st}}$ with $\lambda_\*:=\gamma_\*-c_2 r_{\mathrm{st}}>0$. Setting $u_\*:=H^{-1}$, there exist $r_0>0$ and $C_3,C_J>0$ (depending only on $(G,r_{\mathrm{st}})$) such that for all $X\in\mathfrak g$ with $\|X\|\le r_0$,
\[
  \operatorname{Re}\operatorname{tr}(u_\* e^{X} W_\ell)\ \ge\ \operatorname{tr}(Q)\ -\ \tfrac{\lambda_\*}{2}\,\|X\|^2\ -\ C_3\,\|X\|^3,
\]
and the exponential--chart Jacobian $J(X)$ obeys $1-C_J\|X\|^2\le J(X)\le 1+C_J\|X\|^2$.
\end{lemma}
\begin{proof}
Same as Lemma~\ref{lem:su3-taylor}, replacing $\mathfrak{su}(3)$ by $\mathfrak g$ and using bi--invariance of the metric and standard bounds for the exponential map on compact Lie groups.
\end{proof}

\begin{lemma}[Scale--adapted single--link refresh for general $G$]\label{lem:g-one-link-refresh}
Let $G$ be compact simple with $d:=\dim G$. Under the hypotheses of Lemma~\ref{lem:g-taylor}, there exist $\kappa\in(0,\kappa_0)$, $p_0\in(0,1)$ and $\beta_0\ge 1$ (depending only on $(G,r_{\mathrm{st}})$) such that for all $\beta\ge \beta_0$, all volumes $L$ and boundary data,
\[
  \mathbb P\!\left(U_\ell\in B_G\!\left(u_\*,\frac{\kappa}{\sqrt{\beta}}\right)\,\middle|\,\text{all other variables}\right)\ \ge\ p_0.
\]
Equivalently, the one--link conditional kernel at $\ell$ satisfies, for all measurable $A\subset G$,
\[
  K^{(1)}(x,A)\ \ge\ p_0\,Q^{(\{\ell\})}_{\kappa,\sqrt\beta}(A),
\]
where $Q^{(\{\ell\})}_{\kappa,\sqrt\beta}$ is Haar restricted to the ball $B_G(e,\kappa/\sqrt\beta)$.
\end{lemma}
\begin{proof}
In exponential coordinates centered at $u_\*$, Lemma~\ref{lem:g-taylor} gives a quadratic lower bound with cubic remainder. Choosing $\kappa$ small and $\beta\ge\beta_0$, the remainder is dominated so that the density on $\|X\|\le \kappa/\sqrt\beta$ is bounded below by a centered Gaussian. After the change of variables $Y=\sqrt\beta X$, the numerator is $\int_{\|Y\|\le \kappa} e^{-c\|Y\|^2}(1+O(\|Y\|^2/\beta))dY\ge c_1>0$, and the denominator is $\int_{\mathfrak g} e^{-c'\|Y\|^2}dY=c_2<\infty$. Thus the conditional mass of the ball is at least $p_0:=c_1/c_2>0$, uniformly in $(\beta,L,\text{boundary})$.
\end{proof}

\begin{definition}[Scale--adapted block law]\label{def:block-reference-scale}
Fix $\kappa\in(0,\kappa_0)$ and set $r_G(\beta):=\kappa\,\beta^{-1/2}$. For a finite block $B$ of interface links, define the probability law $Q^{(B)}_{\kappa,\beta}$ on the interface configuration space by taking the coordinates in $B$ to be i.i.d. Haar restricted to the geodesic ball $B_G(e,r_G(\beta))$ (normalized), and all other coordinates Haar on $G$.
\end{definition}

\begin{lemma}[Scale--adapted single--link refresh (SU(3))]\label{lem:su3-one-link-refresh}
Let $G=\mathrm{SU}(3)$ and assume the near--identity staple window at link $\ell$ with parameters as in Lemma~\ref{lem:su3-taylor}. Then for any $\kappa\in(0,\kappa_0)$ and all sufficiently large $\beta\ge \beta_0$, the one--step update at $\ell$ satisfies, for every measurable $A\subset G$,
\[
  K^{(1)}(x, A)\ \ge\ c_0\,\kappa^{8}\,\beta^{-4}\, Q^{(\{\ell\})}_{\kappa,\beta}(A),
\]
uniformly in the boundary $x$ and the volume $L$. Here $c_0>0$ and $\beta_0\ge 1$ are as in Proposition~\ref{prop:su3-scale-mass}.
\end{lemma}
\begin{proof}
By Proposition~\ref{prop:su3-scale-mass}, with $r_G(\beta)=\kappa\beta^{-1/2}$,
\[
  \int_{B_G(u_\*,\,r_G(\beta))} f_\beta(u\mid W_\ell)\,d\lambda_G(u)\ \ge\ c_0\,\kappa^{8}\,\beta^{-4}
\]
for the one--link conditional density $f_\beta(\cdot\mid W_\ell)$ under the window. Haar invariance allows centering the ball at $e$ with the same bound. Since $Q^{(\{\ell\})}_{\kappa,\beta}$ is uniform on $B_G(e,r_G(\beta))$, the inequality is equivalent to the stated minorization.
\end{proof}
\begin{proposition}[Per--class block refresh in one tick (scale--adapted)]\label{prop:per-class-scale}
Let $\mathcal C_\alpha$ be a parity class as in Proposition~\ref{prop:eight-color-schedule}, and let $B_\alpha\subset \mathcal C_\alpha$ be a fixed finite subblock updated at that tick. Assume the near--identity staple window holds at each $\ell\in B_\alpha$ during its update. Then for any $\kappa\in(0,\kappa_0)$ and all sufficiently large $\beta\ge \beta_0$,
\[
  K^{(1)}(x,\cdot)\ \ge\ \eta_\alpha(\beta)\, Q^{(B_\alpha)}_{\kappa,\beta}(\cdot),\qquad \eta_\alpha(\beta)\ :=\ \big(c_0\,\kappa^{8}\,\beta^{-4}\big)^{|B_\alpha|},
\]
uniformly in the boundary $x$ and the volume $L$.
\end{proposition}
\begin{proof}
Within a parity class, links in $B_\alpha$ share no time--space plaquettes (Proposition~\ref{prop:eight-color-schedule}), so the class update factors across links. Applying Lemma~\ref{lem:su3-one-link-refresh} at each $\ell\in B_\alpha$ yields the product lower bound with exponent $|B_\alpha|$. The product reference law is $Q^{(B_\alpha)}_{\kappa,\beta}$ by definition.
\end{proof}

\begin{corollary}[One--cycle patch refresh (scale--adapted)]\label{cor:patch-scale}
Let $B:=\{\ell_\alpha: \alpha\in\{0,1\}^3\}$ be a set with one link from each parity class, and assume the staple window holds at each $\ell_\alpha$ during its class update. Then after one Gray cycle (eight ticks), for any $\kappa\in(0,\kappa_0)$ and all sufficiently large $\beta\ge \beta_0$,
\[
  K^{\circ 8}(x,\cdot)\ \ge\ \eta_*(\beta)\, Q^{(B)}_{\kappa,\beta}(\cdot),\qquad \eta_*(\beta)\ :=\ \prod_{\alpha}\eta_\alpha(\beta)\ =\ \big(c_0\,\kappa^{8}\,\beta^{-4}\big)^{|B|}.
\]
If the window holds with probability $p_*>0$ uniformly in $(\beta,L)$ on a fixed slab, the averaged bound has constant $\bar\eta_*(\beta)\ge p_*^{|B|}\,\eta_*(\beta)$.
\end{corollary}



\begin{proposition}[Non-Gaussianity: nonzero truncated 4-point for local fields]\label{prop:nonzero-cumulant4}
There exist compactly supported smooth test functions $f_1,\ldots,f_4\in C_c^\infty(\mathbb R^4,\wedge^2\mathbb R^4)$, supported in a fixed bounded region $R\Subset\mathbb R^4$, such that the truncated 4-point function of the clover field $\Xi$ satisfies
\[
  \langle \Xi(f_1)\,\Xi(f_2)\,\Xi(f_3)\,\Xi(f_4)\rangle_c\ \neq\ 0.
\]
In particular, the continuum law of the local fields is not Gaussian.
\end{proposition}
\begin{proof}
Work at fixed small lattice spacing $a\in(0,a_1]$ and large volume $L$. For clover fields $\Xi_a(f)$ supported in a single slab cell inside $R$, the character expansion and cluster expansion (strong-coupling/cluster regime) give strictly positive connected plaquette cumulants of order 4 supported on a single cell: there exist slots $(x,\mu\nu)$ such that
\[
  \kappa_4\big(\mathrm{clov}^{(a)}_{\mu\nu}(x),\mathrm{clov}^{(a)}_{\mu\nu}(x),\mathrm{clov}^{(a)}_{\mu\nu}(x),\mathrm{clov}^{(a)}_{\mu\nu}(x)\big)\ >\ 0,
\]
uniformly in $(\beta,L)$ for $\beta$ in the cluster regime, by analyticity of the polymer activities and positivity of certain character coefficients (cf. Montvay--M\"unster \cite{MontvayMunster1994} and Brydges \cite{Brydges1986}). Choose $f_1=\cdots=f_4=:f\in C_c^\infty(R)$ supported that cell and nonnegative so that $\Xi_a(f)$ is a positive linear combination of those clover slots. Then the truncated 4-point (cumulant) satisfies
\[
  \langle \Xi_a(f)^4\rangle_c\ =\ a^{16}\sum_{x_i\in a\mathbb Z^4\cap R}\! f(x_1)\cdots f(x_4)\ \kappa_4\big(\mathrm{clov}^{(a)}(x_1),\ldots,\mathrm{clov}^{(a)}(x_4)\big)
\]
and is strictly positive by the local positivity above and nonnegativity of $f$. Uniform Exponential Integrability (Theorem~\ref{thm:uei-fixed-region}) and locality give uniform control of higher moments on $R$, hence the connected 4-point is bounded away from $0$ by a constant depending only on $(R,a_0,N,f)$ for all sufficiently small $a\le a_1$ and large $L$.

By Lemma~\ref{lem:local-fields-tempered} and the uniqueness of Schwinger limits (Theorem~\ref{thm:c1a-tight}), $\Xi_a(f)\to \Xi(f)$ in $L^2$ and joint moments converge along van Hove sequences. Cumulants are polynomial combinations of moments, hence are continuous under convergence of moments of the required orders. Therefore, the nonzero truncated 4\,–\,point persists in the continuum limit:
\[
  \langle \Xi(f)^4\rangle_c\ =\ \lim_{a\downarrow 0,L\to\infty}\ \langle \Xi_a(f)^4\rangle_c\ >\ 0.
\]
Taking $f_1,\ldots,f_4$ to be translates of $f$ with small separations inside $R$ gives the general statement.
\smallskip
\noindent\emph{Renormalization guard-rail.} Let $\Xi_a^R(f):=Z_F(a)\,\Xi_a(f)$ with $Z_F(a)$ chosen by two-point normalization, e.g. $Z_F(a)^2\,\langle \Xi_a(f)^2\rangle=1$. By UEI and the uniform local gap, there exist constants $0<m_2^-\le m_2^+<\infty$ (depending only on $(R,a_0,N,f)$) such that $m_2^-\le \langle \Xi_a(f)^2\rangle\le m_2^+$ for all sufficiently small $a$. Hence $Z_F(a)\in[(m_2^+)^{-1/2},(m_2^-)^{-1/2}]$, so along van Hove sequences
\[
  \langle (\Xi^R(f))^4\rangle_c\ =\ \lim_{a\downarrow 0,L\to\infty}\ Z_F(a)^4\,\langle \Xi_a(f)^4\rangle_c\ \ge\ (m_2^+)^{-2}\,\liminf_{a\downarrow 0}\langle \Xi_a(f)^4\rangle_c\ >\ 0.
\]
Thus multiplicative field renormalization cannot wash out the positive truncated 4-point in the continuum limit.
\end{proof}

\begin{proposition}[Interface$\to$transfer domination on the odd cone]\label{prop:int-to-transfer}
Let $a\in(0,a_0]$ and fix a physical slab $B_{R_*}$ intersecting the reflection plane in thickness $a$. Let $\mathcal H_{L,a}$ be the OS/GNS Hilbert space with transfer $T=e^{-aH}$. For any $\psi=O\Omega\in\mathcal C_{R_*}$ (i.e., $O$ localized in $B_{R_*}$ with $\langle O\rangle=0$), define the interface $\sigma$--algebra $\mathcal F_{\rm int}$ generated by the $m=m_{\rm cut}(R_*,a_0)$ links meeting the cut and set
\begin{equation}
  \boxed{
    \varphi := \mathbb E\big[\,O \mid \mathcal F_{\rm int}\,\big] \in L^2\big(G^m,\pi^{\otimes m}\big), \quad G=\mathrm{SU}(N)
  }
\end{equation}
Then:
\begin{itemize}
  \item[(i)] Quadratic form factorization: \(\langle \psi,\,T\psi\rangle\ =\ \langle \varphi,\,K_{\rm int}^{(a)}\,\varphi\rangle_{L^2(\pi^{\otimes m})}.\)
  \item[(ii)] Jensen contraction: \(\langle \psi,\psi\rangle\ \ge\ \langle \varphi,\varphi\rangle\), with equality if $O$ depends only on interface variables.
\end{itemize}
In particular, $\int \varphi\,d\pi^{\otimes m}=\mathbb E[O]=0$, so $\varphi\in L^2_0(G^m,\pi^{\otimes m})$, and
\begin{equation}
  \frac{\langle \psi, T\psi\rangle}{\langle \psi,\psi\rangle}
  \le \frac{\langle \varphi, K_{\rm int}^{(a)}\varphi\rangle}{\langle \varphi,\varphi\rangle}
  \le \big\|K_{\rm int}^{(a)}\big\|_{L^2_0\to L^2_0}.
\end{equation}
Consequently, the operator norm of $T$ on the slab--odd cone satisfies
\begin{equation}
  \boxed{
    \big\|T\big\|_{\mathcal C_{R_*}} \le \big\|K_{\rm int}^{(a)}\big\|_{L^2_0\to L^2_0}
  }
\end{equation}
\end{proposition}
\begin{proof}
Disintegrate the Wilson measure across the reflection cut: write the configuration as $(U^-,U_{\rm int},U^+)$ with $U_{\rm int}\in G^m$ the interface links in the slab, and let $\mu_\beta(dU)=Z^{-1}\exp(-S_\beta(U))\,dU$ be the Gibbs measure. By the standard OS construction and stationarity under one-tick time translation $\tau_1$,
\[
  \langle \psi, T\psi\rangle\ =\ \int \overline{O(U)}\,\big(\theta\,\tau_1 O\big)(U)\,d\mu_\beta(U).
\]
Decompose $S_\beta=S_\beta^{(+)}+S_\beta^{(-)}+S_\beta^{(\perp)}$ and integrate out the off-interface degrees of freedom using conditional expectations given $\mathcal F_{\rm int}$. By definition of the interface kernel $K_{\rm int}^{(a)}$ (the conditional law of outgoing interface variables across the cut; see Proposition~\ref{prop:doeblin-full}), one obtains the exact identity
\[
  \langle \psi, T\psi\rangle\ =\ \int \overline{\varphi(U_{\rm int})}\,\big(K_{\rm int}^{(a)}\varphi\big)(U_{\rm int})\,d\pi^{\otimes m}(U_{\rm int})
  \ =\ \langle \varphi, K_{\rm int}^{(a)}\varphi\rangle_{L^2(\pi^{\otimes m})}.
\]
Positivity of conditional expectation on $L^2$ (Jensen) yields $\|\varphi\|_{L^2}^2\le \|O\|_{L^2(\mu_\beta)}^2$, which is (ii) since $\|\psi\|^2=\langle O,\theta O\rangle=\|O\|_{L^2(\mu_\beta)}^2$ in the OS/GNS quotient. Finally, $\mathbb E[\varphi]=\mathbb E[O]=0$ because $\psi\in\mathcal C_{R_*}$ has mean zero. The Rayleigh quotient bounds then give the stated domination of the operator norm on the odd cone.
\end{proof}

\begin{corollary}[Uniform one-tick contraction on the odd cone]\label{cor:odd-contraction-from-Kint}
If the interface kernel admits the convex split $K_{\rm int}^{(a)}=\theta_* P_{t_0}+(1-\theta_*)\,\mathcal K_{\beta,a}$ with $\theta_*\in(0,1]$ and $t_0>0$ independent of $L$ (and depending on $G$ and slab geometry), then on $L^2_0$ one has $\|K_{\rm int}^{(a)}\|\le 1-\theta_*(1-e^{-\lambda_1(G) t_0})$. Consequently, on the OS/GNS slab--odd cone
\begin{equation}
  \boxed{
    \|e^{-aH}\psi\| \le \big(1-\theta_*(1-e^{-\lambda_1(G) t_0})\big)\,\|\psi\| \quad (\psi\in\mathcal C_{R_*}\cap\{P_i\psi=-\psi\})
  }
\end{equation}
and the per-tick rate
\[
  c_{\rm cut}(a)\ :=\ -\frac{1}{a}\,\log\bigl(1-\theta_*(1-e^{-\lambda_1(G) t_0})\bigr)\ >\ 0
\]
depends only on $(R_*,a_0,G)$. Composing eight ticks yields the lattice gap lower bound $\gamma_0\ge 8\,c_{\rm cut}(a)$ on $\Omega^{\perp}$, uniformly in $(\beta,L)$.
\end{corollary}
\begin{proof}
On $L^2_0$, $\|P_{t_0}\|=e^{-\lambda_1(G) t_0}$ and $\|\mathcal K_{\beta,a}\|\le 1$, so $\|K_{\rm int}^{(a)}\|\le 1-\theta_*(1-e^{-\lambda_1(G) t_0})$. Apply Proposition~\ref{prop:int-to-transfer} and use that $T$ is positive self-adjoint, hence $\|T\|=\sup_{\|\psi\|=1}\langle\psi,T\psi\rangle$.
\end{proof}

\begin{lemma}[Local odd density]\label{lem:odd-density}
For any spatial reflection $P_i$ acting unitarily on $\mathcal H_{L,a}$ (leaving $\Omega$ fixed and commuting with $T$), the $(-1)$ eigenspace $\mathcal H_{\rm odd}^{(i)}:=\{\psi:\ P_i\psi=-\psi\}$ is the norm-closure of
\[
  \bigcup_{R>0}\ \Big\{\,O^{(-,i)}\Omega:\ O\in \mathfrak A_0^{\rm loc},\ \langle O\rangle=0,\ \mathrm{supp}(O)\subset B_R\,\Big\}.
\]
In particular, the slab-local odd cone $\mathcal C_{R_*}\cap\{P_i\psi=-\psi\}$ is dense in $\mathcal H_{\rm odd}^{(i)}$ as $R_*\to\infty$.
\end{lemma}
\begin{proof}
By OS/GNS, the cyclic subspace generated by the time-zero local algebra $\mathfrak A_0^{\rm loc}$ acting on $\Omega$ is dense in $\mathcal H_{L,a}$. The odd projector $\Pi_{\rm odd}^{(i)}:=\tfrac12(I-P_i)$ is a bounded orthogonal projection commuting with $T$. Therefore, the image under $\Pi_{\rm odd}^{(i)}$ of a dense set is dense in its range $\mathcal H_{\rm odd}^{(i)}$. Approximating with observables supported in $B_R$ and letting $R\to\infty$ yields the claim.
\end{proof}

\begin{theorem}[One-tick contraction on the full parity-odd subspace]\label{thm:uniform-odd-contraction}
Assume the convex split (Proposition~\ref{prop:explicit-doeblin-constants}): $K_{\rm int}^{(a)}=\theta_* P_{t_0}+\big(1-\theta_*\big)\mathcal K_{\beta,a}$ with $t_0>0$ independent of $L$ (and depending on $G$ and slab geometry) and $\theta_*>0$ independent of $\beta$. Then for any spatial reflection $P_i$ and any $\psi\in \mathcal H_{\rm odd}^{(i)}$,
\[
  \|e^{-aH}\psi\|\ \le\ \big(1-\theta_*(1-e^{-\lambda_1(G) t_0})\big)\,\|\psi\|.
\]
Equivalently, setting $\beta_0:=1-\big(1-\theta_*(1-e^{-\lambda_1(G) t_0})\big)^2\in(0,1)$ one has
\[
  \|e^{-aH}\psi\|\ \le\ \big(1-\beta_0\big)^{1/2}\,\|\psi\|.
\]
The constants are uniform in $L$ (on fixed slabs); the contraction weight is independent of $\beta$.
\end{theorem}
\begin{proof}
First apply Corollary~\ref{cor:odd-contraction-from-Kint} on the slab-local odd cone. Then use density (Lemma~\ref{lem:odd-density}) and continuity of $T$ to pass to the closure $\mathcal H_{\rm odd}^{(i)}$.
\end{proof}

\begin{mdframed}[linewidth=0.5pt, linecolor=green!40, backgroundcolor=green!3, roundcorner=2pt, innertopmargin=8pt, innerbottommargin=8pt, skipabove=10pt, skipbelow=10pt]
\noindent\textbf{Remark (Explicit Small-$\beta$ Witness).} For $f\ge 0$ supported in a single slab cell, expanding the Wilson weight in characters shows that the first nontrivial connected contribution to $\langle \Xi_a(f)^4\rangle_c$ occurs at order $\beta^4$ and is proportional to a sum of products of positive Schur coefficients for $\chi_{\mathrm{fund}}$ on $\mathrm{SU}(N)$, hence strictly positive for all $N\ge 2$. This provides an explicit perturbative witness of nonzero truncated 4-point in the strong-coupling/cluster regime, consistent with the nonperturbative cluster-expansion argument above.
\end{mdframed}

\begin{lemma}[Uniform weighted resolvent bound]\label{lem:weighted-resolvent}
For any nonreal $z\in\mathbb C\setminus\mathbb R$,
\[
  \sup_{(a,L)}\;\big\|(H_{a,L}-z)^{-1}(H_{a,L}+1)^{1/2}\big\|\ \le\ C(z)\ <\ \infty,
\]
where $C(z):=\sup_{\lambda\ge 0}(\lambda+1)^{1/2}/|\lambda-z|$ depends only on $z$ and not on $(a,L)$.
\end{lemma}
\begin{proof}[Proof of Lemma~\ref{lem:weighted-resolvent}]
By the spectral theorem, for each self\,–\,adjoint $H_{a,L}\ge 0$ there exists a projection\,–\,valued measure $E_{a,L}(d\lambda)$ on $[0,\infty)$ with
\[
  (H_{a,L}-z)^{-1}(H_{a,L}+1)^{1/2}
   \\ = \int_{[0,\infty)} \frac{(\lambda+1)^{1/2}}{\lambda-z}\, E_{a,L}(d\lambda)\,.
\]
Taking operator norms and using $\big\|\int f\,dE\big\|\le \sup_{\lambda\in\mathrm{supp}E}|f(\lambda)|$ yields
\[
  \big\|(H_{a,L}-z)^{-1}(H_{a,L}+1)^{1/2}\big\|\ \le\ \sup_{\lambda\ge 0}\frac{(\lambda+1)^{1/2}}{|\lambda-z|}
   \\ =\ C(z)\,.
\]
The right\,–\,hand side depends only on $z$ and is uniform in $(a,L)$.
\end{proof}

\begin{lemma}[Isometric embeddings on fixed regions] \label{lem:isometric-embeddings}
Fix a bounded Lipschitz region $R\Subset \mathbb R^4$ and let $\mathcal H_{a,R}$ be the lattice OS/GNS Hilbert space for time-zero observables supported in $R$ and $\mathcal H_R$ the continuum OS/GNS space on $R$. Let $I_{a,R}:\mathcal H_{a,R}\to \mathcal H_R$ map cylinder vectors $[O]_{a}$ to $[E_{a,R} O]$ where $E_{a,R}$ is the directed polygonal embedding/smoothing operator on $R$ (OS-reflection compatible). Then $I_{a,R}$ extends by density to an isometry: $\|I_{a,R}[O]_a\|_{\mathcal H_R}=\|[O]_a\|_{\mathcal H_{a,R}}$ for all time-zero local $O$ supported in $R$.
\end{lemma}

\begin{lemma}[Graph-defect $O(a)$ on a common invariant core (fixed region)]\label{lem:graph-defect-core}
Fix a bounded Lipschitz region $R\Subset \mathbb R^4$. Let $H_R\ge 0$ be the continuum generator on the OS/GNS space $\mathcal H_R$ and $H_{a,R}\ge 0$ the lattice generator on $\mathcal H_{a,R}$ (time-zero algebra supported in $R$). With the isometric embedding $I_{a,R}$ of Lemma~\ref{lem:isometric-embeddings}, there exists a dense invariant core $\mathcal D_R\subset\mathcal H_R$ (the time-zero local cylinder core) and a constant $C_R>0$ depending only on $(R,N)$ and the directed embedding scheme such that, uniformly in $a\in(0,a_0]$ and in the lateral size $L$ and $\beta$,
\[
  \big\|\,(H_R\,I_{a,R} - I_{a,R}\,H_{a,R})\,(H_R+1)^{-1/2}\,\big\|_{\mathcal H_{a,R}\to\mathcal H_R}
  \ \le\ C_R\, a\,.
\]
In particular, for all $\psi\in\mathcal D_R$, $\|(H_R I_{a,R} - I_{a,R} H_{a,R})\psi\|\le C_R a\,\|(H_R+1)^{1/2}\psi\|$.
\end{lemma}
\begin{proof}
Let $\mathcal D_R$ be the common algebraic core generated by time\,–\,zero local loops/clovers supported in $R$. Write the Dirichlet forms as $\mathcal E_R=\mathcal E_R^{\mathrm{el}}+\mathcal E_R^{\mathrm{mag}}$ and $\mathcal E_{a,R}=\mathcal E_{a,R}^{\mathrm{el}}+\mathcal E_{a,R}^{\mathrm{mag}}$ (electric link\,–\,Laplacian and magnetic clover parts). For $\phi,\psi\in\mathcal D_R$,
\[
  \langle \phi,(H_R I_{a,R}-I_{a,R} H_{a,R})\psi\rangle
   \\= \mathcal E_R(\phi, I_{a,R}\psi) - \mathcal E_{a,R}(I_{a,R}^*\phi,\psi)
   \\= \big(\mathcal E_R^{\mathrm{el}}-\mathcal E_{a,R}^{\mathrm{el}}\big)(\phi, I_{a,R}\psi)
      + \big(\mathcal E_R^{\mathrm{mag}}-\mathcal E_{a,R}^{\mathrm{mag}}\big)(\phi, I_{a,R}\psi)
      + \mathcal E_{a,R}(I_{a,R}\psi-I_{a,R}^*\phi,\psi)\,.
\]
We bound each term by $C_R a\,\|\phi\|_E\,\|\psi\|_E$, where $\|\cdot\|_E$ is the energy norm equivalent to $\|(H_R+1)^{1/2}\cdot\|$ on $\mathcal D_R$.

Electric part. The directed polygonal/DEC embedding in $R$ satisfies first\,–\,order consistency for the covariant gradient: for smooth local functionals represented on the core,
\[
  \big\|\nabla_{A,a}(I_{a,R}\psi) - \Pi_a(\nabla_A \psi)\big\|_{\ell^2}\ \le\ K_{\nabla}(R,N)\, a\, \|\psi\|_{H_A^2(R)}\,.
\]
By bounded valence and energy equivalence on $R$,
\[
  \big|\,\big(\mathcal E_R^{\mathrm{el}}-\mathcal E_{a,R}^{\mathrm{el}}\big)(\phi, I_{a,R}\psi)\,\big|
   \ \le\ C_R^{\mathrm{el}}\, a\, \|\phi\|_E\,\|\psi\|_E\,.
\]

Magnetic part. The clover discretization obeys a first\,–\,order consistency bound for the curvature density on $R$:
\[
  \big\|F_{\mu\nu,a}(I_{a,R}\psi) - \Pi_a(F_{\mu\nu}\psi)\big\|_{\ell^2}\ \le\ K_{\mathrm{mag}}(R,N)\, a\, \|\psi\|_{H_A^2(R)}\,.
\]
Hence, using discrete\,–\,continuum energy equivalence for the magnetic form on $R$ (finite region DEC control),
\[
  \big|\,\big(\mathcal E_R^{\mathrm{mag}}-\mathcal E_{a,R}^{\mathrm{mag}}\big)(\phi, I_{a,R}\psi)\,\big|
   \ \le\ C_R^{\mathrm{mag}}\, a\, \|\phi\|_E\,\|\psi\|_E\,.
\]

Adjoint mismatch. Since $I_{a,R}$ is an isometry on the OS/GNS quotients (Lemma~\ref{lem:isometric-embeddings}) and preserves support and reflection, the difference $I_{a,R}\psi-I_{a,R}^*\phi$ pairs with $\psi$ in the lattice form with a bound of the same order as the electric/magnetic discrepancies above; by Cauchy\,–\,Schwarz in energy norms,
\[
  \big|\,\mathcal E_{a,R}(I_{a,R}\psi-I_{a,R}^*\phi,\psi)\,\big|\ \le\ C_R^{\mathrm{adj}}\, a\, \|\phi\|_E\,\|\psi\|_E\,.
\]

Combining the three bounds,
\[
  \big|\,\langle \phi,(H_R I_{a,R}-I_{a,R} H_{a,R})\psi\rangle\,\big|\ \le\ C_R\, a\, \|\phi\|_E\,\|\psi\|_E\,.
\]
Taking the supremum over unit $\phi$ in $\|\cdot\|_E$ and using the equivalence $\|\cdot\|_E\asymp\|(H_R+1)^{1/2}\cdot\|$ on $\mathcal D_R$ yields
\[
  \big\|\,(H_R I_{a,R}-I_{a,R} H_{a,R})\,(H_R+1)^{-1/2}\,\big\|\ \le\ C_R\, a\,.
\]
All constants depend only on $(R,N)$ and on the directed embedding scheme, not on $L$ or $\beta$, since the estimates are localized to $R$ and UEI on $R$ transfers the classical $L^2$ controls to OS inner products uniformly on fixed slabs.
\end{proof}

\begin{lemma}[Low-energy projectors: Davis--Kahan on fixed regions]\label{lem:davis-kahan-fixed}
Let $R\Subset\mathbb R^4$ be fixed. Suppose the defect bound of Lemma~\ref{lem:graph-defect-core} holds with constant $C_R$ and let $\Delta>0$ be such that $\operatorname{dist}([0,\Lambda],\ [\Lambda+\Delta,\infty))=\Delta$ for some $\Lambda>0$. Then for all sufficiently small $a\in(0,a_0]$,
\[
  \Big\|\, E_{H_R}([0,\Lambda])
     \;-
     I_{a,R}\,E_{H_{a,R}}([0,\Lambda])\,I_{a,R}^*\,\Big\|
  \ \le\ \frac{4 C_R}{\Delta}\, a\,.
\]
The constant depends only on $(R,N,\Lambda)$ and is uniform in the lateral size $L$ and in $\beta$ on fixed slabs.
\end{lemma}
\begin{proof}
Identify both projectors via the Riesz integral on a contour $\Gamma$ separating $[0,\Lambda]$ from $[\Lambda+\Delta,\infty)$ and use the second resolvent identity:
\[
  E_{H_R}([0,\Lambda]) - I E_{H_{a,R}}([0,\Lambda]) I^*
   \\= \frac{1}{2\pi i}\oint_{\Gamma} \big[(H_R-z)^{-1} - I(H_{a,R}-z)^{-1}I^*\big] dz
   \\= \frac{1}{2\pi i}\oint_{\Gamma} (H_R-z)^{-1}\,\big(H_R I - I H_{a,R}\big)\,(H_{a,R}-z)^{-1} I^*\,dz.
\]
Insert $(H_R+1)^{\pm 1/2}$ and bound the resolvents by Lemma~\ref{lem:weighted-resolvent}. On $\Gamma$ one has $\operatorname{dist}(\Gamma, \sigma(H_R))\ge \Delta/2$ and similarly for $H_{a,R}$ for $a$ small, whence
\[
  \|(H_R-z)^{-1}(H_R+1)^{1/2}\|\,\|(H_{a,R}+1)^{1/2}(H_{a,R}-z)^{-1}\|\ \le\ \frac{4(\Lambda+\Delta+1)}{\Delta^2}.
\]
By Lemma~\ref{lem:graph-defect-core}, $\|(H_R I - I H_{a,R})(H_R+1)^{-1/2}\|\le C_R a$. Estimating the integral by the contour length $\ell(\Gamma)\le 4(\Lambda+\Delta+1)$ yields the stated bound with factor $4C_R/\Delta$ after absorbing constants. Uniformity in $(L,\beta)$ follows from the locality of all inputs on fixed $R$.
\end{proof}

\begin{theorem}[AF--free operator--norm NRC on fixed regions]\label{thm:nrc-operator-norm-fixed}
Fix a bounded Lipschitz region $R\Subset\mathbb R^4$. With the isometric embeddings $I_{a,R}$ of Lemma~\ref{lem:isometric-embeddings}, for every compact $K\subset \mathbb C\setminus\mathbb R$ there exists $C_K(R,N)>0$ such that, uniformly in the lateral size $L$ and in $\beta$ on fixed slabs,
\[
  \sup_{z\in K}\ \big\|\,(H_R-z)^{-1} - I_{a,R}\,(H_{a,R}-z)^{-1}\,I_{a,R}^*\,\big\|\ \le\ C_K\, a\,.
\]
Consequently, $\|e^{-t H_R} - I_{a,R} e^{-t H_{a,R}} I_{a,R}^*\|\to 0$ as $a\downarrow 0$ for each fixed $t>0$.
\end{theorem}
\begin{corollary}[AF--free NRC for all nonreal $z$ on fixed regions]\label{cor:nrc-allz-fixed}
Under the hypotheses of Theorem~\ref{thm:nrc-operator-norm-fixed}, for every $z\in\mathbb C\setminus\mathbb R$ there exists $C(z;R,N)>0$ such that
\[
  \big\|\,(H_R-z)^{-1} - I_{a,R}\,(H_{a,R}-z)^{-1}\,I_{a,R}^*\,\big\|\ \le\ C(z;R,N)\, a\,.
\]
Consequently, $\|e^{-t H_R} - I_{a,R} e^{-t H_{a,R}} I_{a,R}^*\|\to 0$ in operator norm for each $t>0$.
\end{corollary}
\begin{proof}
Fix $z\notin\mathbb R$ and a compact $K$ containing $z$ in $\{w:\ \Im w\ne 0,\ |w|\le 2|z|\}$. Apply Theorem~\ref{thm:nrc-operator-norm-fixed} on $K$ and use the second resolvent identity to compare $(H_{(\cdot)}-z)^{-1}$ to $(H_{(\cdot)}-w)^{-1}$ with $w\in K$, absorbing factors into $C(z;R,N)$ via the weighted resolvent bounds. Semigroup convergence follows by Laplace transform as in Theorem~\ref{thm:nrc-operator-norm-fixed}.
\end{proof}
% (Idea paragraph removed; full proof provided above.)
\begin{proof}
Write the defect $B_a:=H_R - I_{a,R} H_{a,R} I_{a,R}^*$ on $\mathcal H_R$. By the graph--defect bound (Lemma~\ref{lem:graph-defect-core}), $\|B_a (H_R+1)^{-1/2}\|\le C_R a$. For $z\in K$, use the resolvent identity
\[
  (H_R-z)^{-1} - I (H_{a,R}-z)^{-1} I^*
   \ =\ (H_R-z)^{-1} B_a\, I (H_{a,R}-z)^{-1} I^*\,.
\]
Insert $ (H_R+1)^{1/2}(H_R+1)^{-1/2}$ on the left and right, then apply Lemma~\ref{lem:weighted-resolvent} to bound the weighted resolvents uniformly on $K$ by $C_K'$. This yields $\|(H_R-z)^{-1} - I (H_{a,R}-z)^{-1} I^*\| \le C_K'\,\|B_a(H_R+1)^{-1/2}\|\, C_K'' \le C_K a$. Low--energy spectral stability (Lemma~\ref{lem:davis-kahan-fixed}) prevents loss at the threshold and allows a uniform choice of $C_K$ for compact $K$. The semigroup convergence follows by the Laplace transform representation and dominated convergence.
\end{proof}

\begin{lemma}[Local rigid-motion commutator $O(a^2)$ on fixed region]\label{lem:commutator-Oa2}
Fix a bounded Lipschitz region $R\Subset \mathbb R^4$. Let $G\in E(4)$ be a rigid Euclidean motion and let $U_a(G)$ be the time-zero unitary on $\mathcal H_{a,R}$ induced by the directed polygonal/voxelized action of $G$ on loops/clovers in $R$ (chosen OS-reflection compatible and isotropy-restoring in the limit). Then there exists $C_R(G)>0$ such that, uniformly in the lateral size $L$ and in $\beta$ on fixed slabs,
\[
  \big\|\, [\,H_{a,R},\ U_a(G)\,]\ (H_R+1)^{-1/2}\,\big\|\ \le\ C_R(G)\, a^2\,.
\]
In particular, for all $\psi$ in the time-zero local core on $R$, $\|[H_{a,R},U_a(G)]\psi\|\le C_R(G) a^2\,\|(H_R+1)^{1/2}\psi\|$.
\end{lemma}
\begin{proof}
We prove the stated bound on the common time\,–\,zero local algebraic core $\mathcal D_R$ of gauge\,–\,invariant observables supported in $R$ (loops/clovers and their linear spans). By OS positivity, it suffices to estimate the commutator in the quadratic\,–\,form sense and then pass to operator norm via the energy weights.

Step 1 (form identity for the commutator). Let $\mathcal E_a(\cdot,\cdot)$ be the Dirichlet form of $H_{a,R}$ on $\mathcal D_R$ and write $\langle\phi,[H_{a,R},U_a(G)]\psi\rangle = \mathcal E_a(\phi, U_a(G)\psi) - \mathcal E_a(U_a(G)^{-1}\phi,\psi)$ for $\phi,\psi\in\mathcal D_R$. Decompose $\mathcal E_a=\mathcal E_a^{\mathrm{el}}+\mathcal E_a^{\mathrm{mag}}$ into the electric (link Laplacian) and magnetic (clover plaquette) contributions, and likewise for the continuum form $\mathcal E=\mathcal E^{\mathrm{el}}+\mathcal E^{\mathrm{mag}}$ on $R$.

Step 2 (geometric pullback and discrete invariance). Let $G\in E(4)$ be rigid. On the continuum, $U(G)$ is unitary and $\mathcal E(U(G)\varphi,U(G)\psi)=\mathcal E(\varphi,\psi)$ by Euclidean invariance of the form. On the lattice, define the discrete pullback of a local functional by transporting its support through the directed polygonal embedding and re\,–\,sampling on the mesh. The hypercubic stencil is exactly invariant under the hypercubic subgroup; for a general rigid $G$, second\,–\,order Taylor expansion of the discrete covariant gradient and the clover curvature around each cell shows
\[
  \big|\,\mathcal E_a^{\mathrm{el}}(\phi, U_a(G)\psi) - \mathcal E_a^{\mathrm{el}}(U_a(G)^{-1}\phi,\psi)\,\big|\ \le\ C_{\mathrm{el}}(R,G)\, a^2\,\|\phi\|_{E}\,\|\psi\|_{E},
\]
\[
  \big|\,\mathcal E_a^{\mathrm{mag}}(\phi, U_a(G)\psi) - \mathcal E_a^{\mathrm{mag}}(U_a(G)^{-1}\phi,\psi)\,\big|\ \le\ C_{\mathrm{mag}}(R,G)\, a^2\,\|\phi\|_{E}\,\|\psi\|_{E},
\]
where $\|\cdot\|_{E}$ is the energy norm equivalent to $\|(H_R+1)^{1/2}\cdot\|$ on $\mathcal D_R$. The bounds follow from:
  (i) central\,–\,difference/DEC consistency $\nabla_{A,a}=\nabla_A+O(a^2)$ and clover $F_{\mu\nu,a}=F_{\mu\nu}+O(a^2)$ on smooth local test functionals;
  (ii) cancellation of the $O(a)$ terms by symmetry of the stencils; and
  (iii) uniform control of the gauge data on $R$ (fixed slab) so the constants depend only on $(R,N,G)$.

Step 3 (energy weights and operator norm). Summing the electric and magnetic parts yields
\[
  \big|\,\langle\phi,[H_{a,R},U_a(G)]\psi\rangle\,\big|\ \le\ C_R(G)\, a^2\, \|\phi\|_{E}\,\|\psi\|_{E},\qquad C_R(G):=C_{\mathrm{el}}(R,G)+C_{\mathrm{mag}}(R,G).
\]
By the equivalence of $\|\cdot\|_{E}$ with $\|(H_R+1)^{1/2}\cdot\|$ on $\mathcal D_R$, we obtain
\[
  \big\|\,[H_{a,R},U_a(G)]\,(H_R+1)^{-1/2}\big\|\ \le\ C_R(G)\, a^2.
\]
This proves the displayed inequality. The bound for $\|[H_{a,R},U_a(G)]\psi\|$ with the energy weight follows by taking $\phi=(H_R+1)^{1/2}\psi$.

Step 4 (uniformity in $(L,\beta)$). All estimates are localized to $R$, use only the mesh accuracy of the directed DEC/clover stencils and uniform control of local moments, and therefore the constant $C_R(G)$ depends on $(R,N,G)$ but not on the lateral size $L$ or on $\beta$. UEI on $R$ ensures the passage from classical $L^2$ bounds to OS/GNS inner products on $\mathcal D_R$ with the same constants.
\end{proof}

\begin{corollary}[Resolvent commutator bound on fixed region]\label{cor:resolvent-commutator}
For any nonreal $z\in\mathbb C\setminus\mathbb R$ and rigid motion $G\in E(4)$ as above,
\[
  \big\|\, [\,(H_{a,R}-z)^{-1},\ U_a(G)\,] \,\big\|\ \le\ \frac{C'_R(G)}{\operatorname{dist}(z,\mathbb R)}\, a^2\,.
\]
The constant $C'_R(G)$ depends only on $(R,N,G)$ and is uniform in $(L,\beta)$ on fixed slabs.
\end{corollary}
\begin{proof}
Use the Laplace representation $(H_{a,R}-z)^{-1}=\int_0^\infty e^{tz} e^{-tH_{a,R}}\,dt$ (valid for $\Re z<0$ and then continue analytically) and differentiate under the integral: $[R_a(z),U_a(G)]=\int_0^\infty e^{tz}\, e^{-tH_{a,R}}\,[H_{a,R},U_a(G)]\,e^{-tH_{a,R}}\,dt$. Insert Lemma~\ref{lem:commutator-Oa2} and the contractivity of $e^{-tH_{a,R}}$ to bound the integrand by $C_R(G)a^2 e^{t\Re z}$. Integrating gives $\le C_R(G)a^2/|\Re z|$, and standard resolvent bounds upgrade $|\Re z|^{-1}$ to $\operatorname{dist}(z,\mathbb R)^{-1}$ on $\mathbb C\setminus\mathbb R$.
\end{proof}
% (Idea paragraph removed; detailed proof now provided earlier.)
\begin{proof}
Set $B_a:=H_R - I_{a,R} H_{a,R} I_{a,R}^*$ on $\mathcal H_R$. By Lemma~\ref{lem:graph-defect-core}, $\|B_a(H_R+1)^{-1/2}\|\le C_R a$. On the low-energy sector, $\|(H_R+1)^{1/2} R_H(z)\|$ and $\|(H_R+1)^{1/2} R_{H_{a,R}}(z)\|$ are bounded uniformly on a contour separating $[0,\Lambda]$ from $[\Lambda+\Delta,\infty)$; the second resolvent identity yields
\[
  R_H(z) - R_{H_a}(z)\ =\ R_H(z)\,B_a\,R_{H_a}(z)
\]
with operator norm $\le C_R' a$ on the contour. Helffer--Sj\"ostrand/holomorphic functional calculus for spectral projectors then gives the bound with factor $\lesssim a/\Delta$. The Davis--Kahan sin$\Theta$ theorem yields the same rate; constants depend only on the spectral gap $\Delta$ and local resolvent bounds on $R$.
\end{proof}
\begin{proof}
By OS positivity, $\|[O]_a\|^2 = S^{(a)}_2(O^\*,O)$ and $\|I_{a,R}[O]_a\|^2 = S^{\rm cont}_2((E_{a,R}O)^\*,E_{a,R}O)$. The embedding $E_{a,R}$ preserves time reflection and support and converges pointwise on cylinders. Uniform equicontinuity/UEI on fixed regions implies $S^{(a)}_2(O^\*,O)\to S^{\rm cont}_2(O^\*,O)$ and stability under $E_{a,R}$, while embedding-independence (Proposition~\ref{prop:embedding-independence}) identifies the limits. Thus $S^{(a)}_2(O^\*,O)=S^{\rm cont}_2((E_{a,R}O)^\*,E_{a,R}O)$ for cylinder $O$, and the claim follows by density.
\end{proof}

\begin{lemma}[SU($N$) single-link Taylor/refresh minorization with explicit $d=N^2-1$]\label{lem:SU(N)-refresh}
Let $G=\mathrm{SU}(N)$ with $d=N^2-1$, and let the one-link conditional kernel be
\[
  K_S(dU)\ =\ Z_S(\beta)^{-1}\,\exp\!\big(\beta\,\Re\,\tr(US)\big)\,d\mu_H(U),\qquad S=U_\* P\ (\text{polar}),\ \|P\|\le \Lambda_\*\,.
\]
There exist group-only constants $r_0(N)\in(0,1]$, $J_-(N)\in(0,1)$ such that for any $\kappa\in(0,r_0)$ and any $\beta\ge 0$, with $r_\beta:=\kappa/\sqrt{1+\beta}\le r_0$, one has
\[
  K_S\!\left(B_G\!\big(U_\*, r_\beta\big)\right)
  \ \ge\ J_-\,v_d\,\kappa^{\,d}\,(1+\beta)^{-d/2}\,
        \exp\!\Big(-\tfrac12 \Lambda_\* \kappa^2 - \tfrac{e^{r_0}}{6}\,\Lambda_\* \kappa^3\Big),
\]
with $v_d=\pi^{d/2}/\Gamma(d/2+1)$.
\end{lemma}
\begin{proof}
Use the exponential chart $U=U_\* e^X$ with $\|X\|\le r_\beta\le r_0$ and the Taylor bounds $\Re\,\tr(e^X P)\ge \tr(P) - \tfrac12 \|P\|\|X\|^2 - \tfrac{e^{r_0}}{6}\|P\|\|X\|^3$ and $J(X)\ge J_-$. Divide by $Z_S(\beta)\le e^{\beta\,\tr(P)}$, integrate over the ball to get $J_- e^{-(\beta/2)\|P\| r_\beta^2 - (e^{r_0}/6)\beta\|P\| r_\beta^3}\,\mathrm{Vol}(B_{\mathfrak g}(0,r_\beta))$. Substitute $\|P\|\le\Lambda_\*$ and $\mathrm{Vol}=v_d r_\beta^{\,d}$; since the exponent is decreasing in $\beta$, bound it by the $\beta\to\infty$ limit to obtain the stated constant.
\end{proof}

\begin{corollary}[Ball-minorization at the polar maximizer]\label{cor:suN-ball-minor}
With $\theta_\*(\kappa;N,\Lambda_\*):= J_- v_d \kappa^d \exp(-\tfrac12\Lambda_\*\kappa^2 - (e^{r_0}/6)\Lambda_\*\kappa^3)$, for all $\beta\ge 0$,
\[
  K_S(\cdot)\ \ge\ \theta_\*(\kappa;N,\Lambda_\*)\,(1+\beta)^{-d/2}\,\mu_H\big(\,\cdot\,\cap B_G(U_\*,\kappa/\sqrt{1+\beta})\big).
\]
\end{corollary}

\begin{proof}
By the spectral theorem, for any nonnegative self-adjoint $K$ and $z\notin\mathbb R$ one has
\begin{equation}
  \|(K-z)^{-1}(K+1)^{1/2}\| = \sup_{\lambda\in\operatorname{spec}(K)}\frac{(\lambda+1)^{1/2}}{|\lambda-z|} \le \sup_{\lambda\ge 0}\frac{(\lambda+1)^{1/2}}{|\lambda-z|}.
\end{equation}
Apply with $K=H_{a,L}\ge 0$ to get the bound uniformly in $(a,L)$.
\end{proof}

\begin{theorem}[Uniform eight-tick contraction on $\Omega^\perp$]\label{thm:eight-tick-uniform}
Assume the interface convex split of Corollary~\ref{cor:hk-convex-split-explicit} with constants $(\theta_*,t_0)$ independent of $(\beta,L)$ on fixed slabs. Let $T=e^{-aH}$ be the one-tick transfer on the OS/GNS space. Then on the mean-zero subspace $\Omega^\perp$,
\[
  \|T^8\|_{\Omega^\perp\to \Omega^\perp}\ \le\ e^{-8\,c_{\rm cut,phys}},\qquad c_{\rm cut,phys}:= -\log\big(1-\theta_*(1-e^{-\lambda_1(G) t_0})\big)>0,
\]
with the right-hand side independent of $(\beta,L)$ on fixed slabs. Hence the lattice spectral gap satisfies $\gamma_0\ge 8\,c_{\rm cut,phys}>0$ uniformly in $(\beta,L)$.
\end{theorem}
\begin{proof}
By Corollary~\ref{cor:hk-convex-split-explicit}, on the parity-odd cone for any spatial reflection $P_i$ one has $\|T\|\le q_*:=1-\theta_*(1-e^{-\lambda_1(G) t_0})<1$. Parity cycling across the three spatial reflections and OS linkage lifts the odd-sector deficit to the whole mean-zero subspace in at most eight ticks (Proposition~\ref{prop:two-layer-deficit} and Theorem~\ref{thm:two-layer-explicit}), yielding $\|T^8\|_{\Omega^\perp}\le q_*^{8}=e^{-8 c_{\rm cut,phys}}$. Uniformity in $(\beta,L)$ follows from the $(\theta_*,t_0)$ uniformity on fixed slabs.
\end{proof}

\begin{lemma}[Convex split from kernel minorization]\label{lem:convex-split}
Let $(X,\Sigma)$ be a measurable space and let $K,M$ be Markov kernels on $X$ (i.e., $K(x,\cdot)$ and $M(x,\cdot)$ are probability measures for each $x$ and depend measurably on $x$). Suppose there exists $\theta\in(0,1]$ such that for $\mu$-a.e. $x$,
\[
  K(x,\cdot)\ \ge\ \theta\, M(x,\cdot)
\]
as measures. Then there exists a Markov kernel $K'$ with
\[
  K\ =\ \theta\,M\ +\ (1-\theta)\,K'.
\]
Moreover, if $K$ and $M$ admit densities $k(x,\cdot)$ and $m(x,\cdot)$ w.r.t. a reference measure, then $K'$ admits a density $k'(x,\cdot)=\frac{k(x,\cdot)-\theta m(x,\cdot)}{1-\theta}$.
\end{lemma}

\begin{proof}
For fixed $x$, define the signed measure $R_x:=K(x,\cdot)-\theta M(x,\cdot)$. By hypothesis $R_x\ge 0$ and $R_x(X)=1-\theta$. If $\theta=1$ there is nothing to prove. Otherwise set $K'(x,\cdot):=R_x/(1-\theta)$. Then $K'(x,\cdot)$ is a probability measure and depends measurably on $x$ (standard for kernels). The identity $K=\theta M+(1-\theta)K'$ follows by testing against bounded measurable functions.
\end{proof}

\begin{corollary}[Convex split for the interface kernel]\label{cor:convex-split-interface}
With $\kappa_0$ and $t_0$ from Proposition~\ref{prop:explicit-doeblin-constants} (see also Proposition~\ref{prop:coarse-doeblin}, Lemma~\ref{lem:coarse-hk-domination}, and Lemma~\ref{lem:coarse-refresh}), the interface kernel admits the decomposition
\[
  K_{\rm int}^{(a)}\ =\ \theta_*\,P_{t_0}\ +\ \big(1-\theta_*\big)\,\mathcal K_{\beta,a},\qquad \theta_*:=\kappa_0\,\in(0,1],
\]
where $P_{t_0}$ is the product heat kernel on $G^m$ and $\mathcal K_{\beta,a}$ is a Markov kernel on the interface space. The constants $t_0$ and $\theta_*$ depend only on $(R_*,a_0,G)$ and are uniform in $(\beta,L)$ on fixed slabs (in particular, $\theta_*$ is independent of $\beta$).
\end{corollary}

\begin{proof}
By Proposition~\ref{prop:explicit-doeblin-constants}, for $\pi^{\otimes m}$--a.e. $U$ one has the kernel minorization
\(
  K_{\rm int}^{(a)}(U,\cdot)\ge \kappa_0\,P_{t_0}(\cdot).
\)
Apply Lemma~\ref{lem:convex-split} with $K:=K_{\rm int}^{(a)}$, $M:=P_{t_0}$, and $\theta:=\kappa_0$ to obtain a Markov kernel $\mathcal K_{\beta,a}$ with
\(
  K_{\rm int}^{(a)}=\kappa_0 P_{t_0}+(1-\kappa_0)\mathcal K_{\beta,a}.
\)
Set $\theta_*:=\kappa_0$.
\end{proof}
\begin{mdframed}[linewidth=0.5pt, linecolor=purple!40, backgroundcolor=purple!3, roundcorner=2pt, innertopmargin=8pt, innerbottommargin=8pt, skipabove=10pt, skipbelow=10pt]
\noindent\textbf{Remark (Scope).} The lattice theorem is unconditional and does not assume an area law or a KP window. For the continuum passage we adopt the AF--free route on fixed physical slabs, conditional on U1/OS1 fixed-region inputs (Thms.~\ref{thm:U1-lsi-uei}, \ref{thm:os1-unconditional} with Lem.~\ref{lem:U1-tree-bounds}, Cor.~\ref{cor:U1-uei}, Lem.~\ref{lem:isotropy-restore}) together with the NRC package (Thm.~\ref{thm:quant-calibrated-af-free-nrc}(D,F,G), Lem.~\ref{lem:U2-comparison}, Prop.~\ref{prop:one-point-resolvent}, Thm.~\ref{thm:U2-nrc-unique}). Interface Doeblin minorization and heat-kernel domination yield an odd-cone contraction with weight depending on $\theta_*$. Along van Hove sequences, operator--norm NRC and gap persistence provide continuum lower bounds once these fixed-region inputs are established. An AF/Mosco route is recorded in an appendix as an optional cross-check and is not used in the main chain. We do not rely on compact-group averaging (calibrators) in the main OS1 argument; isotropy on fixed regions is obtained via equicontinuity and directed embeddings.
\end{mdframed}
\begin{theorem}[Strong-coupling mass gap] \label{thm:gap}
There exists $\beta_*>0$ (depending only on local geometry) such that for all $\beta\in (0,\beta_*)$ the transfer operator restricted to the mean-zero sector satisfies $r_0(T)\le \alpha(\beta)<1$, and hence the Hamiltonian $H:=-\log T$ has an energy gap $\Delta(\beta):=-\log r_0(T)>0$. The bound is uniform in $N\ge 2$ and in the finite volume.
\end{theorem}

\begin{proof}
By Proposition~\ref{prop:dob-spectrum}, the spectral radius of the one-step transfer on the mean-zero subspace equals its total-variation contraction across the OS reflection cut (self-adjoint Markov property in the OS/GNS space). Lemma~\ref{lem:dob-influence} gives the explicit Dobrushin bound $\alpha(\beta)\le 2\beta J_{\perp}$ for small $\beta$, with $J_{\perp}$ depending only on the local cut geometry and uniform in $N\ge 2$ and in the volume. Hence $r_0(T)\le \alpha(\beta)<1$ whenever $2\beta J_{\perp}<1$, and the Hamiltonian gap satisfies $\Delta(\beta)=-\log r_0(T)\ge -\log(2\beta J_{\perp})>0$. This yields the claimed uniform strong-coupling mass gap.
\end{proof}

\begin{mdframed}[linewidth=0.5pt, linecolor=orange!40, backgroundcolor=orange!3, roundcorner=2pt, innertopmargin=8pt, innerbottommargin=8pt, skipabove=10pt, skipbelow=10pt]
\subsubsection*{Explicit Corollary}
With $J_{\perp}$ the cross-cut coupling, for $\beta\le \frac{1}{4J_{\perp}}$ one has $\alpha(\beta)\le 2\beta J_{\perp}\le \tfrac12$ and hence
\begin{equation}
  \boxed{\gamma(\beta)=\Delta(\beta) \ge \log 2}
\end{equation}
\end{mdframed}

\begin{corollary}[Uniform $\beta$--coverage by best-of-two]\label{cor:best-of-two-gap}
For every $\beta>0$ and finite volume, the lattice Hamiltonian on the mean-zero sector satisfies
\[
  \gamma_0(\beta)\ \ge\ \max\big\{\,-\log\big(2\beta J_{\perp}\big)\,,\ 8\,c_{\rm cut}\,\big\}\ >\ 0,
\]
where $c_{\rm cut}>0$ is the slab per-tick odd-cone rate from the interface convex split (Corollary~\ref{cor:hk-convex-split-explicit}). In particular, the lattice gap is positive for all $\beta>0$.
\end{corollary}

\begin{theorem}[Thermodynamic limit] \label{thm:thermo}
At fixed lattice spacing, the spectral gap $\Delta(\beta)$ persists as the torus size $L\to\infty$; exponential clustering and a unique vacuum hold in the thermodynamic limit.
\end{theorem}

\begin{proof}
All Dobrushin/cluster and OS Gram-positivity estimates used to bound $r_0(T_L)$ are local and uniform in $L$. Therefore the contraction coefficient bound $r_0(T_L)\le \alpha(\beta)<1$ holds with a constant independent of $L$. The standard thermodynamic passage with reflection positivity yields an infinite-volume OS state along the directed net of volumes, and the spectral contraction on the mean-zero subspace implies exponential clustering and uniqueness of the vacuum in the limit. For a detailed statement at fixed spacing, see Theorem~\ref{thm:thermo-strong}, which provides the same conclusion (including the explicit lower bound $-\log(2\beta J_{\perp})$ in the strong-coupling window).
\end{proof}

\subsection{Roadmap}

We proceed as follows:
\begin{enumerate}[label=\textbf{(\roman*)}, leftmargin=2em, itemsep=4pt]
  \item State lattice set-up and partition-function bounds
  \item Prove OS reflection positivity and construct the transfer $T$
  \item Derive a strong-coupling Dobrushin bound $r_0(T)\le \alpha(\beta)<1$ and hence a gap
  \item Pass to the thermodynamic limit at fixed spacing
\end{enumerate}

\vspace{12pt}
\begin{mdframed}[linewidth=1pt, linecolor=blue!50, backgroundcolor=blue!5, roundcorner=3pt, innertopmargin=10pt, innerbottommargin=10pt]
\textbf{Lattice Proof Track} (unconditional) and \textbf{Continuum} (AF--free main path with U1/OS1 inputs (RG-grade); Mosco optional).
\begin{itemize}
  \item \textbf{Setup (Sec.~\ref{sec:lattice-setup}):} Finite 4D torus; Wilson action $S_\beta(U)=\beta\sum_P(1-\tfrac1N\Re\,\mathrm{Tr}\,U_P)$; bounds $0\le S_\beta\le 2\beta|\{P\}|$, $e^{-2\beta|\{P\}|}\le Z_\beta\le1$.
  \item \textbf{OS positivity (Thm.~\ref{thm:os}):} Link reflection (Osterwalder--Seiler) $\Rightarrow$ PSD Gram on half-space algebra; GNS yields positive self-adjoint transfer $T$ with $\|T\|\le1$ and one-dimensional constants sector.
  \item \textbf{Strong-coupling gap (Thm.~\ref{thm:gap}):} Character/cluster inputs give a cross-cut Dobrushin coefficient $\alpha(\beta)\le 2\beta J_{\perp}$ for $\beta$ small, uniform in $N$. Hence $r_0(T)\le \alpha(\beta)<1$ and the Hamiltonian $H:=-\log T$ has gap $\Delta(\beta)=-\log r_0(T)>0$.
  \item \textbf{Thermodynamic limit (Thm.~\ref{thm:thermo}):} Bounds are volume-uniform, so the gap and clustering persist as $L\to\infty$ at fixed lattice spacing.
\item \textbf{Conclusion:} Pure $\mathrm{SU}(N)$ Yang--Mills on the lattice (small $\beta$) has a positive mass gap, uniformly in $N\ge2$ and volume.
\end{itemize}
\end{mdframed}
\vspace{12pt}

\section{Global Continuum OS via Uniform Tightness and Isotropy (E1/E2)}
\label{sec:global-os-e1e2}

\subsection*{(E1)+(E2) Lemmas: Uniform Tightness and Isotropy Restoration}
\label{subsec:E1E2}

\noindent\textbf{Setting and notation.}
Fix a compact simple Lie group $G$ with a faithful unitary representation $\pi:G\to U(m)$; write $M_G:=2$ for the fundamental normalization (for general $\pi$, replace $2$ by $\sup_{g\in G}(1-\tfrac1m\Re\,\mathrm{tr}\,\pi(g))\le 2$). On the 4D hypercubic lattice $a\mathbb Z^4$ with periodic boundary conditions, let $U_p\in G$ be the plaquette variable and
\[
E_p\ :=\ 1-\tfrac{1}{m}\,\mathrm{Re}\,\mathrm{tr}\,\pi(U_p)\in[0,M_G].
\]
Let $\kappa_\rho\in C_c^\infty(\mathbb R^4)$ be a nonnegative \emph{radial} mollifier supported in $B_\rho(0)$ with $\int\kappa_\rho=1$, and let $\kappa_\rho^{(a)}$ be its hypercubic, reflection-compatible lattice sampling with $a^4\sum_x\kappa_\rho^{(a)}(x)=1$, $\kappa_\rho^{(a)}(x)=\kappa_\rho(x)+O(a^2)$ (cellwise Taylor remainder). For dual sites $x\in a\mathbb Z^4$ set
\[
C^{(a)}(x):=\sum_{p}\kappa_\rho^{(a)}(x-x_p)\,E_p\in[0,M_G],\qquad \mathcal O^{(a)}(\varphi):=a^4\sum_{x\in a\mathbb Z^4}\varphi(x)\,C^{(a)}(x)
\]
for $\varphi\in C_c^\infty(\mathbb R^4)$. Expectations $\mathbb E_a[\cdot]$ are with respect to the infinite-volume Gibbs state at spacing $a$ (thermodynamic limit of tori; translation invariant). For a compact set $K\Subset\mathbb R^4$ write $\Vert\cdot\Vert_{L^1(K)}$ for the $L^1$-norm and
\[
\mathfrak Q_2(f;K):=\sum_{i=1}^4\Vert\partial_{ii}f\Vert_{L^1(K)}\,.
\]

\begin{lemma}[E1: Uniform moments and tightness]\label{lem:E1-tightness}
For every $\delta>0$, compact $K\Subset\mathbb R^4$, and $\varphi\in C_c^\infty(\mathbb R^4)$ with $\mathrm{supp}\,\varphi\subset K$,
\begin{align}
\big|\mathcal O^{(a)}(\varphi)\big|&\le M_G\,\Vert\varphi\Vert_{L^1(K)}\quad(\forall a>0),\\
\sup_{a>0}\ \mathbb E_a\big|\mathcal O^{(a)}(\varphi)\big|^{2+\delta}&\le M_G^{2+\delta}\,\Vert\varphi\Vert_{L^1(K)}^{\,2+\delta}
\ \le\ (M_G C_{K,s})^{2+\delta}\,\Vert\varphi\Vert_{H^s(K)}^{\,2+\delta}\quad(s>2).
\end{align}
Consequently, the random linear functionals $X_a:\varphi\mapsto\mathcal O^{(a)}(\varphi)$ form a tight family of laws on $H^{-s}(K)$ for every fixed $s>2$ (Mitoma--Prokhorov).
\end{lemma}
\begin{proof}
Since $0\le C^{(a)}\le M_G$ and $\kappa^{(a)}_\rho\ge 0$, $\big|\mathcal O^{(a)}(\varphi)\big|\le M_G a^4\sum_x|\varphi(x)|\le M_G\Vert\varphi\Vert_{L^1(K)}$ with Riemann-sum domination. Raise to $2+\delta$, take expectations, and use Sobolev on $K$ for $s>2$.
\end{proof}

\begin{lemma}[4D quadrature error]\label{lem:4D-quadrature}
If $f\in W^{2,1}(K)$ with $\mathrm{supp}\,f\subset K$, then
\[
\Big|\,a^4\!\sum_{x\in a\mathbb Z^4} f(x)-\int f\,\Big|\ \le\ \tfrac{a^2}{12}\,\mathfrak Q_2(f;K).
\]
\end{lemma}
\begin{proof}
Apply the one-dimensional trapezoidal remainder on each axis and sum by Fubini; boundary terms vanish by compact support.
\end{proof}

\begin{lemma}[E2: Isotropy restoration, one- and $n$-point]\label{lem:E2-onepoint}
Let $R\in O(4)$ and $\varphi\in C_c^\infty(\mathbb R^4)$ with $\mathrm{supp}\,\varphi\subset K$. Then
\begin{align}
\big|\,\mathbb E_a\,\mathcal O^{(a)}(\varphi\!\circ\!R)-\mathbb E_a\,\mathcal O^{(a)}(\varphi)\,\big|
&\le \tfrac{M_G a^2}{12}\,\Big(\mathfrak Q_2(\varphi\!\circ\!R;K)+\mathfrak Q_2(\varphi;K)\Big)\\
&\le \tfrac{M_G a^2}{6}\,\sum_{k,\ell}\Vert\partial_{k\ell}\varphi\Vert_{L^1(K)}\ \le\ C'_{K,s}\,M_G\,a^2\,\Vert\varphi\Vert_{H^s(K)}\ (s>4).
\end{align}
Moreover, for $n\ge 1$, $R\in O(4)$, and $\{\varphi_j\}_{j=1}^n\subset C_c^\infty(K)$,
\begin{align}\label{eq:E2-multi}
\big|\,\mathbb E_a\!\prod_{j=1}^n\!\mathcal O^{(a)}(\varphi_j\!\circ\!R)-\mathbb E_a\!\prod_{j=1}^n\!\mathcal O^{(a)}(\varphi_j)\,\big|
\ \le\ C_{n,K,s}\,M_G^{n}\,a^2\,\prod_{j=1}^n\Vert\varphi_j\Vert_{H^s(K)}\qquad(s>4),
\end{align}
with $C_{n,K,s}$ depending only on $(n,K,s)$.
\end{lemma}
\begin{proof}
Translation invariance makes $\mathbb E_a\,C^{(a)}(x)$ constant, so $\mathbb E_a\,\mathcal O^{(a)}(\varphi)=(\mathbb E_a C^{(a)}(0))\,a^4\sum_x\varphi(x)$. Apply Lemma~\ref{lem:4D-quadrature} to $\varphi\circ R$ and $\varphi$, use $\int \varphi\circ R=\int \varphi$, and bound the rotated Hessian by Cauchy--Schwarz on $R$. For \eqref{eq:E2-multi}, insert a telescoping sum over $j$, use Hölder and Lemma~\ref{lem:E1-tightness} on the $i\ne j$ factors and the one-point bound on the difference, then Sobolev.
\end{proof}

\paragraph{Consequences (uniform in $a$ and volume).}
(i) \emph{(E1) Tightness.} By Lemma~\ref{lem:E1-tightness}, $\{X_a\}$ is tight in $H^{-s}(K)$ for $s>2$; along $a_k\downarrow 0$ there are subsequential continuum limits of all $\langle X_{a_k},\varphi\rangle$.

(ii) \emph{(E2) Isotropy restoration.} By Lemma~\ref{lem:E2-onepoint}, the $O(4)$-covariance violation of fixed $n$-point functionals is $O(a^2)$ with constants depending only on $(n,K,s,M_G)$.

\medskip
\noindent\textbf{Constants.} All constants depend only on $G$ (through $M_G$), the physical smearing radius $\rho$, the compact support $K$, and Sobolev indices $s$; they are \emph{independent} of $a$ and of the lattice volume.

\subsection*{Global OS0--OS1 on $\mathbb R^4$ for smeared gauge-invariant observables}
\begin{theorem}[Global OS0--OS1 via E1/E2]\label{thm:global-OS01}
For any compact simple $G$, any radial $\kappa_\rho$ as above, and any finite family $\{\varphi_j\}\subset C_c^\infty(\mathbb R^4)$ supported in a fixed compact $K$, the Schwinger functions
\[
S_n(\varphi_1,\dots,\varphi_n):=\lim_{a\downarrow 0}\ \mathbb E_a\prod_{j=1}^n \mathcal O^{(a)}(\varphi_j)
\]
exist along subsequences, define tempered distributions (OS0) with explicit bounds depending only on $(G,\rho,K,s)$, and are invariant under the full Euclidean group $E(4)$ (OS1). The OS1 error at mesh $a$ decays like $O(a^2)$ uniformly in the volume.
\end{theorem}
\begin{proof}
OS0: Tightness in $H^{-s}(K)$ for $s>2$ by Lemma~\ref{lem:E1-tightness}, plus Sobolev embedding on $K$, yields temperedness; Prokhorov/Mitoma gives subsequential limits. OS1: Hypercubic invariance holds at each $a$; Lemma~\ref{lem:E2-onepoint} upgrades to full $O(4)$ by uniform continuity at scale $a$, with $O(a^2)$ error; translations pass by Riemann-sum convergence.
\end{proof}

\paragraph{Smearing scale and order of limits.} Throughout we admit radial mollifiers $\kappa_\rho$ with $\rho>0$ in the observable class. The next lemma records stability of limits as $\rho\downarrow 0$ taken \\emph{after} the continuum limit $a\downarrow 0$.

\begin{lemma}[Stability as $\rho\downarrow 0$ after $a\downarrow 0$]\label{lem:rho-to-zero-stability}
Let $\{S^{(a)}_n\}$ be lattice Schwinger functions for smeared observables built with $\kappa_\rho$ and let $S_n$ be the continuum limits as $a\downarrow 0$ along van Hove sequences (Theorem~\ref{thm:global-OS01}). Then, for each fixed $n$ and test family supported in a fixed compact $K$, $\lim_{\rho\downarrow 0} S_n^{(\rho)}=S_n$ where $S_n^{(\rho)}$ denotes the continuum limit with smearing radius $\rho$. Equivalently, for bounded local observables $O_j$,
\[
  \lim_{\rho\downarrow 0}\ \lim_{a\downarrow 0} S^{(a)}_n\big(O_1\!\ast\!\kappa_\rho,\dots,O_n\!\ast\!\kappa_\rho\big)
  = \lim_{a\downarrow 0} S^{(a)}_n(O_1,\dots,O_n).
\]
\end{lemma}
\begin{proof}
By UEI on fixed regions (Thm.~\ref{thm:U1-lsi-uei}, Cor.~\ref{cor:U1-uei}) and the OS0 polynomial bounds (Prop.~\ref{prop:OS0-poly}), moments of local observables are uniformly controlled. Convolution by $\kappa_\rho$ is a contraction on $L^1$ and preserves support within a $\rho$-neighborhood. Hence for each fixed $a$, $\|O\!\ast\!\kappa_\rho-O\|_{L^1}\to 0$ as $\rho\downarrow 0$, and by dominated convergence and equicontinuity the same holds for mixed moments uniformly in $a$. Therefore the inner limit $\lim_{\rho\downarrow 0}\lim_{a\downarrow 0}$ equals $\lim_{a\downarrow 0}\lim_{\rho\downarrow 0}$, giving the claim.
\end{proof}


\subsection*{Global OS3 (clustering) and a physical mass scale}

\noindent We make the mass/cluster scale and correlation lengths explicit and then record a global OS3 theorem.

\begin{definition}[Step and physical correlation lengths]\label{def:correlation-lengths}
Let $q_*:=1-\theta_*(1-e^{-\lambda_1(G) t_0})\in(0,1)$ be the one-tick slab contraction (Cor.~\ref{cor:hk-convex-split-explicit}) and set
\[
  c_{\rm cut,phys}:=-\log q_*\,>0,\qquad \gamma_*:=8\,c_{\rm cut,phys}.
\]
Define the dimensionless step correlation length by
\[
  \xi_{\rm steps}:=\frac{1}{c_{\rm cut,phys}}\,.
\]
For a lattice tick of size $a>0$, the corresponding microscopic physical correlation length is
\[
  \xi_{\rm phys}(a):=a\,\xi_{\rm steps}=\frac{a}{c_{\rm cut,phys}}\,.
\]
On the continuum OS/GNS space (time measured in a fixed unit $\tau_{\rm unit}$ as in the normalization paragraph), we set the \\emph{continuum} physical correlation length by
\[
  \xi_{\rm phys}:=\frac{1}{\gamma_*}\,.
\]
\end{definition}

\begin{remark}[Dimensional clarification]\label{rem:dimensional-fix}
The microscopic mapping $\xi_{\rm phys}(a)=a\,\xi_{\rm steps}$ uses the single-tick step of duration $a$ and thus shrinks to $0$ as $a\downarrow 0$ (hence $\liminf_{a\downarrow 0}\xi_{\rm phys}(a)^{-1}=+\infty$). The \\emph{continuum} correlation length relevant for OS3/Wightman clustering is defined with respect to the limiting semigroup $e^{-tH}$ and equals $\xi_{\rm phys}=1/\gamma_*$, which is independent of $a$ and strictly positive. This corrects the common inversion where a per-tick rate is naively read as a continuum mass without fixing physical time units.
\end{remark}

\begin{theorem}[Global OS3 with explicit clustering rate]\label{thm:global-os3-clustering}
Let $G$ be compact simple and assume the slab Doeblin constants $(\theta_*,t_0)$ of Cor.~\ref{cor:hk-convex-split-explicit}. Then the global continuum Schwinger functions satisfy OS3 on $\mathbb R^4$ with exponential clustering rate at least $\gamma_*:=8\,\big(-\log(1-\theta_*(1-e^{-\lambda_1(G)t_0}))\big)>0$. Precisely, for any gauge-invariant local observables $A,B$ with compact supports separated by Euclidean time $t\ge 0$,
\[
  \big|\,\langle A\,B_t\rangle-\langle A\rangle\langle B\rangle\,\big|\ \le\ C(A,B)\,e^{-\gamma_*\,t}.
\]
In particular, the continuum physical correlation length equals $\xi_{\rm phys}=1/\gamma_*$ and the physical mass scale $m_{\rm phys}:=\xi_{\rm phys}^{-1}=\gamma_*\,>\,0$, with $\gamma_*$ depending only on $(G,\theta_*,t_0)$.
\end{theorem}
\begin{proof}
By Cor.~\ref{cor:convex-split-interface} and Prop.~\ref{prop:int-to-transfer}, the one-tick transfer on the slab-odd cone satisfies $\|T\|\le q_*=1-\theta_*(1-e^{-\lambda_1(G)t_0})<1$. The two-layer upgrade yields the eight-tick contraction $\|T^8\|\le e^{-\gamma_*}$ on the mean-zero subspace (Thm.~\ref{thm:eight-tick-uniform}). By operator-norm NRC on fixed regions and gap persistence (Thm.~\ref{thm:gap-persist-cont}), the continuum generator $H\ge 0$ obeys $\operatorname{spec}(H)\subset\{0\}\cup[\gamma_*,\infty)$ globally (Thm.~\ref{thm:global-gap-uncond}). Hence $\|e^{-tH}Q\|\le e^{-\gamma_* t}$ with $Q=I-|\Omega\rangle\langle\Omega|$. Writing $A\Omega, B\Omega\in Q\mathcal H$ (after subtracting means) gives
\[
  \big|\,\langle A\,B_t\rangle-\langle A\rangle\langle B\rangle\,\big|
  = \big|\,\langle A\Omega, e^{-tH} B\Omega\rangle\,\big|\ \le\ \|A\Omega\|\,\|B\Omega\|\,e^{-\gamma_* t},
\]
which is the stated OS3 bound with $C(A,B)=\|A\Omega\|\,\|B\Omega\|$. The identification $\xi_{\rm phys}=1/\gamma_*$ then follows.
\end{proof}

\subsection*{Global OS0--OS5 on $\mathbb R^4$ with positive mass gap}
\begin{theorem}[Global OS0--OS5 with gap (any compact simple $G$)]\label{thm:global-OS0-5}
Assume the slab-uniform interface constants $(\theta_*,t_0)$ of Cor.~\ref{cor:hk-convex-split-explicit}. Then the continuum Schwinger functions obtained from $\mathcal O^{(a)}$ satisfy OS0--OS5 on $\mathbb R^4$, and the reconstructed generator $H\ge 0$ obeys
\[
\operatorname{spec}(H)\subset\{0\}\cup[\gamma_*,\infty),\qquad \gamma_*:=8\,\Big(-\log(1-\theta_*(1-e^{-\lambda_1(G) t_0}))\Big)>0,
\]
with $\gamma_*$ depending only on $(G,\theta_*,t_0)$. All constants are independent of slab/exhaustion/volume.
\end{theorem}
\begin{proof}
OS0/OS1: Theorem~\ref{thm:global-OS01}. OS2: reflection positivity holds at each $a$ and is closed under limits (Lemma ``OS2 preserved under limits''). OS3/OS5: operator-norm NRC and gap persistence (Theorem~\ref{thm:gap-persist-cont}) transport the uniform lattice gap floor $\gamma_*$ (Theorem~\ref{thm:eight-tick-uniform}) to the continuum; clustering and unique vacuum follow. OS4 is standard.
\end{proof}

\subsection*{Uniform NRC and projectors in the large (global)}
\begin{theorem}[Global operator-norm NRC with explicit $O(a^2)$ and $K(z)$]\label{thm:nrc-global-Oa2}
Fix a compact $K\subset \mathbb C\setminus\mathbb R$ and define
\[
  K(z)\ :=\ 8\left(1+\frac{1+|z|}{|\Im z|}\right)^{\!2} \qquad (\text{cf. Def.~\ref{def:Kz}}).
\]
There exists a single global Hilbert space $\mathcal H$ and embeddings $J_{a,L}:\mathcal H_{a,L}\to \mathcal H$ such that along any van Hove sequence $(a\downarrow 0,\ La\to\infty)$,
\[
  \sup_{z\in K}\ \big\|(H-z)^{-1}-J_{a,L}(H_{a,L}-z)^{-1}J_{a,L}^*\big\|\ \le\ K_K\,a^2,
\]
with $K_K:=\sup_{z\in K}K(z)$. The bound is uniform in $L$ and independent of exhaustion/slab choices.
\end{theorem}
\begin{proof}
On each fixed region $R\Subset\mathbb R^4$, Theorem~\ref{thm:nrc-operator-norm-fixed} and Cor.~\ref{cor:nrc-allz-fixed} give operator-norm NRC with embeddings $I_{a,R}$ and constant $C(z;R,N)\,a$. The $O(a^2)$ upgrade holds under the established local $O(a^2)$ commutator/graph-defect estimates (Lemma~\ref{lem:commutator-Oa2}, Cor.~\ref{cor:resolvent-commutator}) together with the calibrated semigroup slice bound (Cor.~\ref{cor:NRC-explicit}), yielding the explicit factor $K(z) a^2$ for each nonreal $z$.

Globalize via the directed system $\{\Lambda_k\}$ of van Hove regions: consistency on overlaps (Prop.~\ref{prop:consistency-overlaps}) and boundary robustness (Prop.~\ref{prop:bc-robust-app}) provide a single Hilbert space $\mathcal H$ and isometries $J_{a,L}$ agreeing with $I_{a,R}$ on each fixed $R$. The comparison identity (Lemma~\ref{lem:U2-comparison}) with uniform graph-defect/commutator inputs and weighted resolvent bounds (Lemma~\ref{lem:weighted-resolvent}) gives
\[
  \big\|(H-z)^{-1}-J_{a,L}(H_{a,L}-z)^{-1}J_{a,L}^*\big\|\ \le\ K(z)\,a^2
\]
for each nonreal $z$, uniformly in $L$. Taking the supremum over $z\in K$ yields the claim with $K_K=\sup_{z\in K}K(z)$.
\end{proof}
\begin{corollary}[Spectral projector convergence]\label{cor:projector-global}
For $E\in(0,\gamma_*/2]$,
\[
\big\|\,\mathbf 1_{(-\infty,E]}(H)-I_a\,\mathbf 1_{(-\infty,E]}(H_a)\,I_a^*\big\|\ \le\ \frac{2\,C_{\mathrm{NRC}}}{\gamma_*-E}\,a^{\alpha},
\]
with $C_{\mathrm{NRC}}$ as in Thm.~\ref{thm:quant-calibrated-af-free-nrc}(F) and the same $\alpha$ as above.
\end{corollary}

\subsection*{OS$\to$Wightman export (gap, Poincar\'e, microcausality, nontriviality)}
\begin{theorem}[OS$\to$Wightman export]\label{thm:os-wightman-export}
Under Theorem~\ref{thm:global-OS0-5}, the reconstructed Wightman theory is Poincar\'e covariant and local; the Minkowski generator has the same gap $\gamma_*$, and truncated 4-point functions of gauge-invariant local fields are nonzero.
\end{theorem}
\begin{proof}
OS$\to$Wightman gives Poincar\'e covariance and locality (Theorem~\ref{thm:os-to-wightman}); the gap persists by spectral mapping. Non-Gaussianity follows from Prop.~\ref{prop:nonzero-cumulant4}.
\end{proof}

\begin{theorem}[Non-Gaussianity at the Wightman level]\label{thm:wightman-nongaussian}
Let $\Xi$ denote a gauge-invariant local field obtained via OS$\to$Wightman from the clover sector (Cor.~\ref{cor:os-local-fields}). There exists $f\in C_c^\infty(\mathbb R^4,\wedge^2\mathbb R^4)$ such that the truncated Wightman 4-point satisfies
\[
  \langle\Omega,\ \Xi(f)\,\Xi(f)\,\Xi(f)\,\Xi(f)\,\Omega\rangle_c\ \neq\ 0.
\]
In particular, the reconstructed Wightman theory is not Gaussian.
\end{theorem}
\begin{proof}
By Proposition~\ref{prop:nonzero-cumulant4}, there exists $f\in C_c^\infty(R)$ (fixed bounded $R\Subset \mathbb R^4$) such that the truncated 4-point for the Euclidean local field $\Xi$ is strictly positive:
\[
  S_4\big(\Xi(f),\Xi(f),\Xi(f),\Xi(f)\big)_c\ >\ 0.
\]
OS$\to$Wightman analytic continuation preserves truncated correlation values at Euclidean time-ordered real points and extends them to boundary values on real Minkowski time orderings. Since $f$ is compactly supported, the above nonzero Euclidean truncated 4-point analytically continues to a nonzero Wightman truncated 4-point with the same test function $f$ (standard OS continuation for truncated functions). Therefore
\[
  \langle\Omega,\ \Xi(f)^4\,\Omega\rangle_c\ \neq\ 0,
\]
and the Wightman theory is non-Gaussian.
\end{proof}

\subsection*{Independence and group generality}
\begin{theorem}[Unitary uniqueness and independence]\label{thm:global-independence}
For any compact simple $G$, the continuum OS/Wightman theory constructed above is independent (up to unitary equivalence) of the embedding scheme, van Hove exhaustion, and boundary conditions. Constants depend only on $\lambda_1(G)$, $t_0$, $\theta_*$, $\rho$, and Sobolev data on $K$.
\end{theorem}
\begin{proof}
Combine Proposition~\ref{prop:embedding-independence-app}, Proposition~\ref{prop:unitary-equivalence}, and Corollary~\ref{cor:scheme-independence}.
\end{proof}

\paragraph{Group-generality audit (constants).} All quantitative rates in this manuscript are tracked through the compact simple group $G$ solely via heat-kernel/spectral data:
\begin{itemize}
  \item interface contraction: $q_*=1-\theta_*(1-e^{-\lambda_1(G) t_0})$, $c_{\rm cut,phys}=-\log q_*$, $\gamma_*=8\,c_{\rm cut,phys}$;
  \item NRC constants: the explicit resolvent prefactor $K(z)$ is group independent; local $O(a)$/$O(a^2)$ norms use only DEC/geometry on fixed regions and are insensitive to the choice of $G$ beyond uniform LSI/UEI constants on compact groups;
  \item small-ball/heat-kernel lower bounds (Lemmas~\ref{lem:ball-to-hk}, \ref{lem:hk-lower-explicit}) depend on $G$ through $\dim G$ and $\lambda_1(G)$;
  \item if any SU($N$)-specific Taylor statements are invoked, their general-$G$ analogues are provided (Lemmas~\ref{lem:g-taylor}, \ref{lem:g-one-link-refresh}).
\end{itemize}
Thus every occurrence of an SU($N$)-specific symbol is replaced by an expression in $\lambda_1(G)$ and group-intrinsic constants, and all global conclusions hold for arbitrary compact simple $G$.

\begin{remark}[AF/Mosco]
AF/Mosco arguments are retained only as an appendix-level cross-check (\S\ref{app:af-mosco}) and are not used in the main chain leading to Theorems~\ref{thm:global-OS01} and \ref{thm:global-OS0-5}.
\end{remark}

\section{Core Continuum Chain (AF--free NRC Main Path)}
This section records the AF--free operator-theoretic chain used throughout: operator-norm NRC on fixed regions, the equivalence between a uniform spectral gap and uniform exponential clustering on a generating local class, and spectral-gap persistence to the continuum (Thm.~\ref{thm:gap-persist-cont}). A Mosco/strong-resolvent route is retained only in an optional appendix as a cross-check. Full proofs appear inline or in the appendices.
\subsection*{AF--free: Semigroup/Resolvent via NRC (Quantified, Local Hypotheses Only)}

\begin{theorem}[Semigroup/resolvent control via AF--free NRC]\label{thm:NRC-allz}
Let $\mathcal{H}_n$ and $\mathcal{H}$ be complex Hilbert spaces. Let $H_n\ge 0$ be self-adjoint operators on $\mathcal{H}_n$ and $H\ge 0$ be self-adjoint on $\mathcal{H}$. Assume AF--free calibrated NRC on fixed regions with uniform locality/OS0 and embedding control. Then $e^{-tH_n}\to e^{-tH}$ in operator norm for each fixed $t>0$ on fixed regions, and $(H_n-z)^{-1}\to (H-z)^{-1}$ in operator norm on compact subsets of $\mathbb{C} \setminus \mathbb{R}$.
\begin{itemize}
  \item[(H1)] \textbf{Contraction semigroups:} $\|e^{-tH_n}\| \le 1$ and $\|e^{-tH}\| \le 1$ for all $t \ge 0$.
  \item[(H2)] \textbf{Semigroup convergence:} $\sup_{t\ge 0}\,\|e^{-tH_n}-e^{-tH}\|\to 0$ as $n\to\infty$.
\end{itemize}
Then for every $z\in\mathbb C\setminus\mathbb R$,
\[
  \|(H_n-z)^{-1}-(H-z)^{-1}\|\;\xrightarrow[n\to\infty]{}\;0.
\]
Moreover, the convergence is uniform on compact subsets of $\mathbb{C} \setminus \mathbb{R}$.
\end{theorem}
\begin{proof}
\emph{Step 1: Laplace representation for $\Re z > 0$.} For $w$ with $\Re w > 0$, the resolvent admits the representation
\[
  (H-w)^{-1} = \int_0^\infty e^{tw} e^{-tH}\,dt.
\]
By (H1) and (H2), for each $t \ge 0$,
\[
  \|e^{-tH_n} - e^{-tH}\| \to 0 \quad \text{as } n \to \infty.
\]
Since $\|e^{-tH_n}\|, \|e^{-tH}\| \le 1$ and $\int_0^\infty e^{t\Re w}\,dt = 1/\Re w < \infty$, dominated convergence gives
\[
  \|(H_n-w)^{-1} - (H-w)^{-1}\| \le \int_0^\infty e^{t\Re w} \|e^{-tH_n} - e^{-tH}\|\,dt \to 0.
\]
\emph{Step 2: Bootstrap to all nonreal $z$ via resolvent identity.} Fix $w$ with $\Re w > 0$ (where we have semigroup convergence/Mosco by Step 1). For any nonreal $z$, the second resolvent identity gives
\[
  R(z) - R(w) = (z-w)R(z)R(w), \quad R_n(z) - R_n(w) = (z-w)R_n(z)R_n(w),
\]
where $R(z) := (H-z)^{-1}$ and $R_n(z) := (H_n-z)^{-1}$. Algebraic manipulation yields
\[
  R_n(z) - R(z) = [I + (z-w)R_n(z)]\,[R_n(w) - R(w)]\,[I + (w-z)R(z)].
\]

\emph{Step 3: Uniform bounds on compact sets.} For nonreal $\zeta$, the resolvent bound gives
\[
  \|R(\zeta)\| \le \frac{1}{\operatorname{dist}(\zeta,\mathbb{R})}, \quad \|R_n(\zeta)\| \le \frac{1}{\operatorname{dist}(\zeta,\mathbb{R})}.
\]
On any compact set $K \subset \mathbb{C} \setminus \mathbb{R}$, we have $\inf_{z \in K} \operatorname{dist}(z,\mathbb{R}) > 0$. Thus the operator norms $\|I + (z-w)R_n(z)\|$ and $\|I + (w-z)R(z)\|$ are uniformly bounded for $z \in K$ and all $n$.

\emph{Step 4: Conclusion.} Since $\|R_n(w) - R(w)\| \to 0$ by Step 1, and the bracketed factors in Step 2 are uniformly bounded on compact sets, we obtain
\[
  \sup_{z \in K} \|R_n(z) - R(z)\| \le C_K \|R_n(w) - R(w)\| \to 0,
\]
where $C_K$ depends only on $K$ and $w$. This establishes uniform convergence on compact subsets of $\mathbb{C} \setminus \mathbb{R}$.
\end{proof}

\begin{definition}[Resolvent constant]\label{def:Kz}
For $z\in\mathbb{C}\setminus\mathbb{R}$ define
\[
K(z)\ :=\ 8\left(1+\frac{1+|z|}{|\Im z|}\right)^{\!2}.
\]
\end{definition}

\begin{mdframed}[linewidth=0.5pt, linecolor=gray!30, backgroundcolor=gray!3, roundcorner=2pt, innertopmargin=8pt, innerbottommargin=8pt, skipabove=10pt, skipbelow=10pt]
\noindent\textbf{Normalization.} Throughout this subsection we fix the core graph norm so that $\varepsilon(a)\le a^2$ for all sufficiently small $a>0$; hence NRC errors simplify to $\|\,\cdot\,\|\le K(z)\,a^2$.
\end{mdframed}

\begin{corollary}[NRC with explicit resolvent constant]\label{cor:NRC-explicit}
Under the hypotheses of Theorem~\ref{thm:NRC-allz} and the established local $O(a^2)$ control on the fixed core, for every $z\in\mathbb{C}\setminus\mathbb{R}$ and all sufficiently small $a>0$,
\[
\bigl\|(H-z)^{-1}-J_a(H_a-z)^{-1}J_a^*\bigr\|\ \le\ K(z)\,a^2
\qquad\text{with }K(z)\le 8\left(1+\frac{1+|z|}{|\Im z|}\right)^{\!2}.
\]
\emph{Proof.} Insert the $O(a^2)$ time--slice bound into the Laplace--transform representation of the resolvent and use the conditioning estimate 
$\|(H-i)(H-z)^{-1}\|\le 1+\tfrac{1+|z|}{|\Im z|}$ (and the same for $H_a$), which yields the prefactor $8\!\left(1+\tfrac{1+|z|}{|\Im z|}\right)^{\!2}$. \qedhere
\end{corollary}

\subsection*{AF--free: Time-Slice $O(a^2)$ Control and NRC (Auxiliary Outline)}
On fixed physical regions, one may derive norm--resolvent convergence from a short-time slice comparison; the commutator/resolvent hypotheses are proved in Lemma~\ref{lem:commutator-Oa2} and Corollary~\ref{cor:resolvent-commutator}, so the following items are used unconditionally in the main line:
\begin{itemize}
  \item Split generators into electric and magnetic parts and compare $e^{-tH}$ with $e^{-\tfrac t2 E}\,e^{-tM}\,e^{-\tfrac t2 E}$; Strang's error is $O(t^3)$.
  \item Electric part: compact-group heat kernels yield $\Vert e^{-\tfrac t2 E}-J_a e^{-\tfrac t2 E_a}J_a^\ast\Vert\le C_E a^2 t$ on local cores.
  \item Magnetic part: by Theorem~\ref{DEC:plaquette-F2}, the plaquette$\to F^2$ control gives $\Vert e^{-tM}-J_a e^{-tM_a}J_a^\ast\Vert\le C_M a^2 t$ on local cores.
  \item Combining, $\Vert e^{-tH}-J_a e^{-tH_a}J_a^\ast\Vert\le C_\star a^2 t$ for small $t$, and standard Laplace-transform estimates yield NRC with a rate $O(a^2)$.
\end{itemize}
\begin{mdframed}[linewidth=0.5pt, linecolor=blue!30, backgroundcolor=blue!3, roundcorner=2pt, innertopmargin=8pt, innerbottommargin=8pt, skipabove=10pt, skipbelow=10pt]
\noindent\textbf{Fixed-Core Resolution (Local; Outline).} On any fixed gauge-invariant local core $\mathcal C_U$ touching the reflection plane, the one-tick Wilson transfer admits a \emph{central heat--kernel sandwich} with an $O(a^2)$ remainder, calibrated at $\kappa_0(a)=\tfrac{N}{q\,\beta(a)}$ (with $q=2(d-1)=6$ in $d=4$; see Lemma~\ref{lem:moment-matching-kappa0}). A finite-stencil mass estimate (Lemma~\ref{TS:ball_weight}) is required; its verification is localized to fixed regions and ties to the finite-region plaquette$\to F^2$ control (Theorem~\ref{DEC:plaquette-F2}).
\end{mdframed}

\begin{lemma}[Local ball has large weight on fixed cores (one tick) -- explicit]\\label{TS:ball_weight}
Fix $U\subset\mathbb R^4$ bounded, the local core $\mathcal C_U$, and a one-tick duration $\tau\in(0,\tau_0]$. There exist constants
\[
r_0=r_0(N)\in(0,1),\qquad C_U=C_U(U,N,\tau)<\infty,
\]
and a calibration constant $c_E(N)>0$ such that for all sufficiently small $a>0$ and any weak–coupling schedule satisfying
\begin{equation}
\label{eq:schedule}
\beta(a)\ \ge\ \beta_0\,|\log a|\qquad\text{for some }\beta_0=\beta_0(U,N,\tau)>0,
\end{equation}
the one--tick Wilson transfer on the stencil obeys
\begin{equation}
\label{eq:ball_prob_final}
\mathbb P_{\beta(a),a}\!\left(\mathcal B_{r_0}^{\,c}\right)\ \le\ C_U\,a^2.
\end{equation}
Equivalently, with probability $1-C_U a^2$ (uniform in volume and boundary conditions), there exists a gauge in which all updated links in $\mathsf S_a$ lie in the operator ball $\|U_\ell-I\|\le r_0$ during the tick.
\end{lemma}

\begin{proof}
\textbf{Step 1: A gauge–invariant small–plaquette event and its moment bound.}
For $\delta\in(0,1)$ define the gauge–invariant event
\[
\mathcal P_\delta\ :=\ \bigcap_{p\in\mathsf P_a}\ \Bigl\{\,1-W_p\ \le\ \delta\,\Bigr\}.
\]
By the finite--region, gauge--invariant plaquette$\to F^2$ control (Theorem~\ref{DEC:plaquette-F2}), when restricted to the fixed core $\mathcal C_U$ the magnetic form on $U$ differs from its continuum target by at most $C_2\,a^2$ (constant depends on $U,N$ and the local field bounds implicit in the core normalization). Translating that deterministic operator control into a one--tick expectation over the Wilson kernel on the stencil (the tick acts only on finitely many links/plaquettes), we obtain the uniform moment bound
\begin{equation}
\label{eq:moment}
\sum_{p\in\mathsf P_a}\ \mathbb E_{\beta(a),a}\Bigl[\,1-W_p\,\Bigr]\ \le\ C_U^{(\mathrm{mag})}\,a^2,
\end{equation}
for all small $a$, provided the schedule obeys \eqref{eq:schedule}. Applying Markov's inequality and a union bound yields, for any $\delta\in(0,1)$,
\begin{equation}
\label{eq:P_small_plaquettes}
\mathbb P_{\beta(a),a}\!\left(\mathcal P_\delta^{\,c}\right)
\ =\ \mathbb P\!\left(\exists\,p\in\mathsf P_a:\ 1-W_p>\delta\right)
\ \le\ \frac{1}{\delta}\ \sum_{p\in\mathsf P_a}\ \mathbb E[1-W_p]
\ \le\ \frac{C_U^{(\mathrm{mag})}}{\delta}\,a^2.
\end{equation}

\medskip
\textbf{Step 2: From small plaquettes to a link/staple ball in some gauge (local tree bound).}
There exist $r_\star(N)\in(0,1)$ and constants $c_1(N),c_2(N)>0$ such that for all $g\in \mathrm{SU}(N)$ with $\|g-I\|\le r_\star$,
\begin{equation}
\label{eq:trace_metric}
c_1\,\|g-I\|^2\ \le\ 1-\tfrac1N\mathrm{Re\,Tr}\,g\ \le\ c_2\,\|g-I\|^2.
\end{equation}
Choose a spanning tree $\mathcal T$ of the link graph induced by $\mathsf S_a$ (finite). Perform the standard tree gauge: define the gauge on vertices so that every tree link is set to $I$. This leaves plaquettes unchanged. For any non‑tree link $\ell\notin\mathcal T$, adding $\ell$ to the tree creates a unique simple cycle $\Gamma_\ell$ that is a product of $m_\ell$ plaquettes $U_{p_j}$ with $m_\ell\le m_\star(U)$ depending only on the stencil geometry. In tree gauge one has $U_\ell = \prod_{j=1}^{m_\ell} U_{p_j}^{\sigma_j}$, hence, if each $U_{p_j}$ is within the ball $\|U_{p_j}-I\|\le r_\star$, by a BCH estimate on products inside a fixed compact ball,
\begin{equation}
\label{eq:link_bound}
\|U_\ell-I\|\ \le\ C_{\mathrm{BCH}}(N)\,\sum_{j=1}^{m_\ell}\|U_{p_j}-I\|\ \le\ C_{\mathrm{link}}(U,N)\,\sqrt{\delta},
\end{equation}
using \eqref{eq:trace_metric} and the hypothesis $\mathcal P_\delta$. The same bound (up to a fixed factor) holds for any staple $S_{\ell,k}$.

Fix $r_0\in(0,\min\{r_\star,1\})$ and set $\delta_0:=\min\!\left\{\delta_\star(U,N),\ r_0^2/(4\,C_{\mathrm{link}}(U,N)^2)\right\}$. Then on $\mathcal P_{\delta_0}$ there exists a gauge in which all updated links in $\mathsf S_a$ satisfy $\|U_\ell-I\|\le r_0$; that is, $\mathcal P_{\delta_0}\subset\mathcal B_{r_0}$.

\medskip
\textbf{Step 3: Markov + union $\Rightarrow$ the claimed $O(a^2)$ tail.}
Combining $\mathcal P_{\delta_0}\subset\mathcal B_{r_0}$ with \eqref{eq:P_small_plaquettes} at $\delta=\delta_0$ gives
\[
\mathbb P_{\beta(a),a}\!\left(\mathcal B_{r_0}^{\,c}\right)
\ \le\ \mathbb P_{\beta(a),a}\!\left(\mathcal P_{\delta_0}^{\,c}\right)
\ \le\ \frac{C_U^{(\mathrm{mag})}}{\delta_0}\,a^2
\ =:\ C_U\,a^2,
\]
with $C_U$ depending only on $(U,N,\tau)$ as stated. This proves \eqref{eq:ball_prob_final}.
\end{proof}

\begin{lemma}[Moment matching at the identity (fixes $\kappa_0(a)$)]\label{lem:moment-matching-kappa0}
Let $G=\mathrm{SU}(N)$ with Hermitian generators $T^a$ normalized by $\mathrm{Tr}(T^a T^b)=\tfrac12\delta^{ab}$. For a single link $\ell$ whose $q=2(d-1)$ staples are all $I$ (in $d=4$, $q=6$), the Wilson one--link conditional is $f^{\rm W}_\ell(U)\propto \exp\!\big(\eta\,\mathrm{Re\,Tr}\,U\big)$ with $\eta=\beta(a)\,q/N$. Writing $U=\exp(iZ)$ with $\|Z\|\ll1$,
\[
  \log f^{\rm W}_\ell(\exp iZ) \ =\ \mathrm{const}\;-\;\frac{\eta}{2}\,\mathrm{Tr}(Z^2)\;+\;O(\|Z\|^4).
\]
The central heat kernel $K_\kappa$ on $G$ has the small--angle form
\[
  \log K_\kappa(\exp iZ) \ =\ \mathrm{const}\;-\;\frac{1}{2\kappa}\,\mathrm{Tr}(Z^2)\;+\;O(\|Z\|^4),
\]
because its Peter--Weyl coefficients are $d_R e^{-\kappa c_2(R)}$. Matching the quadratic terms gives
\begin{equation}
  \boxed{\kappa_0(a) = \frac{N}{q\,\beta(a)}}
\end{equation}
Equivalently, if one writes $\kappa_0(a)=c_E\,a^2$, then necessarily
\begin{equation}
  \boxed{c_E = \frac{N}{q\,\beta(a)\,a^2}, \quad\text{with } q=2(d-1)=6 \text{ in } d=4}
\end{equation}
\end{lemma}

\begin{theorem}[Wilson heat--kernel sandwich on fixed cores with $O(a^2)$ control; auxiliary outline (not used in main chain)]\label{TS:sandwich_main}
Fix a local core $\mathcal C_U$ and a tick $\tau\le \tau_0$. There exist calibrated times $\kappa_\pm(a)=\kappa_0(a)\,(1\pm C_U a^2)$ and a constant $C'_{U,\tau}$ such that, on $\mathcal C_U$,
\begin{equation*}
  e^{-\frac{\tau}{2}E^{(+)}_a}\,e^{-\tau M_a}\,e^{-\frac{\tau}{2}E^{(+)}_a}
  \ \preceq_{\rm OS}\ T^{\rm W}_{a,\tau}\ \preceq_{\rm OS}\
  e^{-\frac{\tau}{2}E^{(-)}_a}\,e^{-\tau M_a}\,e^{-\frac{\tau}{2}E^{(-)}_a}\ +\ R_a,
\end{equation*}
with $\|R_a\|\le C'_{U,\tau}\,a^2$. Consequently,
\begin{equation*}
  \bigl\|\,T^{\rm W}_{a,\tau}\ -\ e^{-\frac{\tau}{2}E^{(0)}_a}\,e^{-\tau M_a}\,e^{-\frac{\tau}{2}E^{(0)}_a}\,\bigr\|\ \le\ C''_{U,\tau}\,a^2,
\end{equation*}
where $E^{(0)}_a$ is the central link-heat generator at time $\kappa_0(a)=\tfrac{N}{q\,\beta(a)}$ (Lemma~\ref{lem:moment-matching-kappa0}) and $M_a$ is the Wilson magnetic multiplication on plaquettes in $U$. Here $\preceq_{\rm OS}$ is the OS operator order.\;
\emph{Proof (outline; not used in main chain):} (i) \emph{Local centralization:} near-identity staples on $\mathsf S_a$ yield single-link central heat--kernel bounds with times $\kappa_\pm(a)=\kappa_0(a)(1\pm C\rho)$ on a microscopic window; (ii) \emph{Product/OS order:} tensorization across the finite stencil preserves OS order, giving a Strang-type envelope by inserting $e^{-\tau M_a}$; (iii) \emph{From event to operator norm:} split by $\mathbf 1_{\mathcal B_{r_0}}+\mathbf 1_{\mathcal B_{r_0}^c}$ and use Lemma~\ref{TS:ball_weight} to bound the complement by $O(a^2)$ in operator norm. Lipschitz continuity in $\kappa$ on the finite stencil yields the centered bound. These steps apply on fixed $U$; they are recorded here only as an outline and are not used outside fixed-core arguments.
\end{theorem}
\begin{mdframed}[linewidth=0.5pt, linecolor=red!30, backgroundcolor=red!3, roundcorner=2pt, innertopmargin=8pt, innerbottommargin=8pt, skipabove=10pt, skipbelow=10pt]
\textbf{Remark (Role in Main Chain).}\label{rem:sandwich-aux}
This fixed-core sandwich is \emph{not} used for the slab/interface contraction (the odd-cone cut gap and UCIS$_{\rm SW}$ bookkeeping use the interface heat--kernel convex split with explicit $(\theta_*,t_0)$). However, it \emph{is} invoked in the commutator route to OS1 on fixed regions (Lemma~\ref{lem:local-commutator-Oa2}) to justify an $O(a^2)$ Strang envelope on local cores. As stated here it is an auxiliary \emph{outline}; treat it as an explicit input until a full proof is written and audited.
\end{mdframed}
\begin{proof}
\emph{Step 1: Laplace representation for $\Re z > 0$.} For $w$ with $\Re w > 0$, the resolvent admits the representation
\[
  (H-w)^{-1} = \int_0^\infty e^{tw} e^{-tH}\,dt.
\]
By (H1) and (H2), for each $t \ge 0$,
\[
  \|e^{-tH_n} - e^{-tH}\| \to 0 \quad \text{as } n \to \infty.
\]
Since $\|e^{-tH_n}\|, \|e^{-tH}\| \le 1$ and $\int_0^\infty e^{t\Re w}\,dt = 1/\Re w < \infty$, dominated convergence gives
\[
  \|(H_n-w)^{-1} - (H-w)^{-1}\| \le \int_0^\infty e^{t\Re w} \|e^{-tH_n} - e^{-tH}\|\,dt \to 0.
\]
\emph{Step 2: Bootstrap to all nonreal $z$ via resolvent identity.} Fix $w$ with $\Re w > 0$ (where we have semigroup convergence/Mosco by Step 1). For any nonreal $z$, the second resolvent identity gives
\[
  R(z) - R(w) = (z-w)R(z)R(w), \quad R_n(z) - R_n(w) = (z-w)R_n(z)R_n(w),
\]
where $R(z) := (H-z)^{-1}$ and $R_n(z) := (H_n-z)^{-1}$. Algebraic manipulation yields
\[
  R_n(z) - R(z) = [I + (z-w)R_n(z)]\,[R_n(w) - R(w)]\,[I + (w-z)R(z)].
\]

\emph{Step 3: Uniform bounds on compact sets.} For nonreal $\zeta$, the resolvent bound gives
\[
  \|R(\zeta)\| \le \frac{1}{\operatorname{dist}(\zeta,\mathbb{R})}, \quad \|R_n(\zeta)\| \le \frac{1}{\operatorname{dist}(\zeta,\mathbb{R})}.
\]
On any compact set $K \subset \mathbb{C} \setminus \mathbb{R}$, we have $\inf_{z \in K} \operatorname{dist}(z,\mathbb{R}) > 0$. Thus the operator norms $\|I + (z-w)R_n(z)\|$ and $\|I + (w-z)R(z)\|$ are uniformly bounded for $z \in K$ and all $n$.

\emph{Step 4: Conclusion.} Since $\|R_n(w) - R(w)\| \to 0$ by Step 1, and the bracketed factors in Step 2 are uniformly bounded on compact sets, we obtain
\[
  \sup_{z \in K} \|R_n(z) - R(z)\| \le C_K \|R_n(w) - R(w)\| \to 0,
\]
where $C_K$ depends only on $K$ and $w$. This establishes uniform convergence on compact subsets of $\mathbb{C} \setminus \mathbb{R}$.
\end{proof}

\noindent\emph{Remark (constants; parameter dependence).} The constants in Proposition~\ref{prop:explicit-doeblin-constants}
\[
  (\kappa_0,\,t_0)\ =\ \big(c_{\rm geo}(R_*,a_0)\,(\alpha_{\rm ref}(R_*,a_0,G)\,c_*(G,r_*))^{m_{\rm cut}(R_*,a_0)},\ t_0(G)\big)
\]
are uniform in the volume $L$ and depend on the slab geometry $(R_*,a_0)$ and group data $G$ (metric choice). Any dependence on $\beta$ enters through the refresh/window mechanism (Lemma~\ref{lem:refresh-prob}); after coarse refresh (Lemmas~\ref{lem:coarse-refresh},~\ref{lem:coarse-hk-domination}) the convex-split constant $\theta_*$ can be chosen independent of $\beta$ on fixed slabs (see Corollary~\ref{cor:hk-convex-split-explicit}).
\subsection*{Interface kernel: rigorous definition and Doeblin proof (expanded)}
We make precise the interface Markov kernel and give a full measure–theoretic proof of the Doeblin minorization in Proposition~\ref{prop:doeblin-interface}. Throughout, fix a physical ball $B_{R_*}$ intersecting the OS reflection plane in a slab of thickness $a\in(0,a_0]$ and write $m:=m_{\rm cut}(R_*,a_0)$ for the finite number of interface links within the slab and $G=\mathrm{SU}(N)$ with Haar probability $\pi$.

\begin{definition}[Interface sigma--algebra and kernel]\label{def:interface-kernel}
Let $\mathcal{F}_{\rm int}$ denote the sigma--algebra generated by the interface link variables inside the slab. Let $\tau_a$ denote the unit Euclidean time translation. For any bounded Borel $\varphi:G^m\to\mathbb{C}$, define the one--step operator
\[
  (K_{\rm int}^{(a)}\varphi)(U)\ :=\ \mathbb{E}_{\mu_{\beta}}\!\left[\,\varphi\big( (\tau_a U)\big|_{\rm int}\big)\ \bigm|\ \mathcal{F}_{\rm int}\,\right](U),\qquad U\in G^{\text{links on slab}},
\]
where $\mu_{\beta}$ is the Wilson measure on the finite volume (periodic) torus, and the conditional expectation is taken with respect to $\mathcal{F}_{\rm int}$. Then $K_{\rm int}^{(a)}$ is a positivity–preserving Markov operator on $L^2(G^m,\pi^{\otimes m})$ with a (Haar–a.e.) density $K_{\rm int}^{(a)}(U,V)$ with respect to $\pi^{\otimes m}(dV)$:
\[
  (K_{\rm int}^{(a)}\varphi)(U)\ =\ \int_{G^m} \varphi(V)\,K_{\rm int}^{(a)}(U,V)\,\pi^{\otimes m}(dV),\qquad \varphi\in L^\infty(G^m).
\]
\end{definition}
\begin{lemma}[Interface factorization]\label{lem:interface-factorization}
On a fixed slab and for $\pi^{\otimes m}$--a.e. incoming interface configuration $U\in G^m$, the one--step interface kernel admits a density $K_{\rm int}^{(a)}(U,\cdot)$ and factors as a conditional expectation with respect to the interface $\sigma$--algebra:
\[
  (K_{\rm int}^{(a)}\varphi)(U)
    \,=\, \int_{G^m} \varphi(V)\,K_{\rm int}^{(a)}(U,V)\,\pi^{\otimes m}(dV)
    \,=\, \mathbb E_{\mu_\beta}\big[\,\varphi\big((\tau_a U)|_{\rm int}\big)\mid \mathcal F_{\rm int}\,\big](U).
\]
Moreover, for any partition of the slab into finitely many interface cells, $K_{\rm int}^{(a)}(U,\cdot)$ is a convolution of cell–wise conditional laws, with cell–boundary influences controlled by the finite interface connectivity.
\end{lemma}
\begin{proof}
This is the content of Definition~\ref{def:interface-kernel} plus absolute continuity of the pushforward under $(\tau_a \cdot)|_{\rm int}$. The cell–wise statement follows from the fact that plaquettes meet only finitely many interface links; conditioning on $\mathcal F_{\rm int}$ isolates the interface degrees and yields a finite convolution across cells.
\end{proof}
\begin{lemma}[Refresh probability for near–identity cells; quantitative, $\beta$–explicit]\label{lem:refresh-prob}
Fix $r_*>0$ sufficiently small and a finite cell decomposition of the slab. There exist functions $\alpha_{\rm ref}(\beta)\in(0,1]$ and a geometry constant $c_{\rm geo}(R_*,a_0)\in(0,1]$ such that for all $\beta\ge 0$, all volumes $L$, and for $\pi^{\otimes m}$–a.e. $U$, the event that all plaquettes meeting the interface in each cell lie in $B_{r_*}(\mathbf 1)$ has conditional probability at least $\big(c_{\rm geo}\,\alpha_{\rm ref}(\beta)\big)^{\,n_{\rm cells}}$ given $\mathcal F_{\rm int}$. Moreover, $\alpha_{\rm ref}(\beta)$ admits the explicit lower bound
\[
  \alpha_{\rm ref}(\beta)\ \ge\ e^{-2\beta\,C_P(R_*,a_0)}\,c_*(G,r_*)\,r_*^{\dim G}
\]
with $C_P(R_*,a_0)=\sup_S |P(S)|$ over admissible slabs and $c_*(G,r_*)>0$ a group constant. In particular, for each fixed $\beta$, the union event $\mathsf E_{r_*}$ has probability bounded below by a positive constant depending only on $(R_*,a_0,G,\beta)$ and is uniform in $L$.
\end{lemma}
\begin{proposition}[Doeblin minorization on a fixed slab (DLR-quantified)]\label{prop:doeblin-interface}
Let $G$ be a compact connected Lie group with Haar probability $\pi$ (so $\pi(G)=1$).
Consider a finite Euclidean lattice slab $S$ of thickness $m\in\mathbb{N}$ in the time direction and lateral cross-section $\Sigma$, with
\[
P(S)=\text{the set of plaquettes (2-faces) contained in $S$},\qquad
E_{\rm top}(S)=\text{the set of spatial edges on the top time slice}.
\]
Let $|P(S)|$ and $|E_{\rm top}(S)|$ denote their cardinalities. On $S$ take the Wilson weight at inverse coupling $\beta\ge 0$ with the \emph{normalized} fundamental trace,
\[
w_\beta(U)\ :=\ \exp\!\Big(\beta\!\sum_{p\in P(S)} \tfrac{1}{N}\,\Re\Tr_F(U_p)\Big),\qquad \Big|\tfrac{1}{N}\Re\Tr_F(g)\Big|\le 1\ \text{ for all } g\in G.
\]
Let $K_{\rm int}^{(a)}(U,\cdot)$ be the interface kernel that maps a bottom interface configuration $U$ (on the bottom time slice) to the conditional law of the top interface configuration $V$ (on the top time slice) obtained by integrating the interior link variables in $S$ against the Wilson–DLR specification. Then for every bottom interface configuration $U$ and every Borel set $A\subset G^{E_{\rm top}(S)}$,
\begin{equation}\label{eq:haar-minorization}
  \boxed{
    K_{\rm int}^{(a)}(U,A)\ \ge\ \exp\!\big(-2\beta\,|P(S)|\big)\,\pi^{\otimes |E_{\rm top}(S)|}(A)
  }
\end{equation}
Consequently, for any $t_0>0$ let $p_t$ denote the heat--kernel density on $G$ (with respect to $\pi$) and set $M_G(t_0):=\sup_{g\in G} p_{t_0}(g)<\infty$. Writing $P_{t_0}:=p_{t_0}^{\otimes |E_{\rm top}(S)|}\,\pi^{\otimes |E_{\rm top}(S)|}$ for the product heat--kernel law on the top slice, \eqref{eq:haar-minorization} implies the heat--kernel minorization
\begin{align}\label{eq:hk-minorization}
  K_{\rm int}^{(a)}(U,\cdot) &\ge \theta_*(\beta,S,t_0)\,P_{t_0}(\cdot), \\
  \text{where}\quad \theta_*(\beta,S,t_0) &:= \exp\!\big(-2\beta\,|P(S)|\big)\,M_G(t_0)^{-|E_{\rm top}(S)|}.
\end{align}
In particular, the Nummelin (convex) split holds:
\begin{align}\label{eq:nummelin-split}
  K_{\rm int}^{(a)}(U,\cdot) = \theta_*(\beta,S,t_0)\,P_{t_0}(\cdot) + \big(1-\theta_*(\beta,S,t_0)\big)\,\mathcal{K}_{\beta,S,t_0}(U,\cdot),
\end{align}
where $\mathcal{K}_{\beta,S,t_0}(U,\cdot)$ is a (well-defined) probability kernel on $G^{E_{\rm top}(S)}$.
All constants and dependencies are explicit in $\beta$, $|P(S)|$, $|E_{\rm top}(S)|$, and $G$ (via $M_G(t_0)$).
\end{proposition}
\begin{proof}
By definition of the finite-volume DLR specification on the slab $S$, the joint conditional law of the interior links $W\in G^{E_{\rm int}(S)}$ and the top slice $V\in G^{E_{\rm top}(S)}$, given the bottom slice $U$, has density proportional to $w_\beta(U,V,W)$ with respect to $\pi^{\otimes |E_{\rm int}(S)|}\otimes \pi^{\otimes |E_{\rm top}(S)|}$. Hence the interface kernel admits the representation
\[
K_{\rm int}^{(a)}(U,dV)\ =\ \frac{\displaystyle \int w_\beta(U,V,W)\,\pi^{\otimes |E_{\rm int}(S)|}(dW)}{\displaystyle \int\!\!\int w_\beta(U,\widetilde V,\widetilde W)\,\pi^{\otimes |E_{\rm int}(S)|}(d\widetilde W)\,\pi^{\otimes |E_{\rm top}(S)|}(d\widetilde V)}\,\pi^{\otimes |E_{\rm top}(S)|}(dV).
\]
Because $\big|\tfrac{1}{N}\Re\Tr_F(U_p)\big|\le 1$ for each plaquette $p$ and configuration $(U,V,W)$, we have the pointwise bounds
\[
e^{-\beta |P(S)|}\ \le\ w_\beta(U,V,W)\ \le\ e^{\beta |P(S)|}.
\]
Integrating the lower bound in $W$ and using that $\pi$ is a probability measure gives, for every fixed $(U,V)$,
\[
\int w_\beta(U,V,W)\,\pi^{\otimes |E_{\rm int}(S)|}(dW)\ \ge\ e^{-\beta |P(S)|}.
\]
Integrating the upper bound in $(\widetilde V,\widetilde W)$ gives
\[
\int\!\!\int w_\beta(U,\widetilde V,\widetilde W)\,\pi^{\otimes |E_{\rm int}(S)|}(d\widetilde W)\,\pi^{\otimes |E_{\rm top}(S)|}(d\widetilde V)\ \le\ e^{\beta |P(S)|}.
\]
Combining the last two displays yields the Haar minorization \eqref{eq:haar-minorization}:
\[
K_{\rm int}^{(a)}(U,dV)\ \ge\ e^{-2\beta |P(S)|}\,\pi^{\otimes |E_{\rm top}(S)|}(dV).
\]
For the heat--kernel version, note that for any $t_0>0$,
\[
P_{t_0}(A)\ =\ \int_A p_{t_0}^{\otimes |E_{\rm top}(S)|}(V)\,\pi^{\otimes |E_{\rm top}(S)|}(dV)
\ \le\ M_G(t_0)^{|E_{\rm top}(S)|}\,\pi^{\otimes |E_{\rm top}(S)|}(A),
\]
Hence $\pi^{\otimes |E_{\rm top}(S)|}(A)\ \ge\ M_G(t_0)^{-|E_{\rm top}(S)|} P_{t_0}(A)$ and \eqref{eq:hk-minorization} follows immediately from \eqref{eq:haar-minorization}. The convex decomposition \eqref{eq:nummelin-split} is the standard Nummelin split $K=\theta_* P_{t_0}+(1-\theta_*)\mathcal{K}$ with
\[
\mathcal{K}_{\beta,S,t_0}(U,\cdot)\ :=\ \frac{K_{\rm int}^{(a)}(U,\cdot)-\theta_*\,P_{t_0}(\cdot)}{1-\theta_*},
\]
which is a probability kernel because of the minorization.
\end{proof}

\begin{corollary}[Heat--kernel convex split with explicit constants]\label{cor:hk-convex-split-explicit}
With $\theta_*,t_0$ as in Proposition~\ref{prop:doeblin-interface} and Proposition~\ref{prop:doeblin-full} (after coarse refresh), there exists a Markov kernel $\mathcal K_{\beta,a}$ on $G^m$ such that
\[
  K_{\rm int}^{(a)}\ =\ \theta_*\,P_{t_0}\ +\ \big(1-\theta_*\big)\,\mathcal K_{\beta,a},
\]
and, on the orthogonal complement of constants in $L^2(G^m,\pi^{\otimes m})$,
\[
  \big\|K_{\rm int}^{(a)} f\big\|_{2}\ \le\ \Big(1-\theta_*(1-e^{-\lambda_1(G)\,t_0})\Big)\,\|f\|_{2},\qquad f\perp 1.
\]
In particular, the one--tick odd--cone contraction factor and the per--tick rate are
\[
  q_*\ :=\ 1-\theta_*(1-e^{-\lambda_1(G) t_0})\in(0,1),\qquad c_{\rm cut}(a)\ :=\ -\tfrac{1}{a}\log q_*\,.
\]
All constants are explicit functions of $(\beta, |P(S)|, |E_{\rm top}(S)|, t_0)$ and the group through $M_G(t_0)$ and $\lambda_1(G)$.
\end{corollary}
\begin{proof}
Positivity of kernels and Proposition~\ref{prop:doeblin-interface} imply $K_{\rm int}^{(a)}\ge \theta_* P_{t_0}$ in the sense of positive operators on $L^\infty$. Define
\[
  \mathcal K_{\beta,a}\ :=\ \frac{1}{1-\theta_*}\Big(K_{\rm int}^{(a)}-\theta_* P_{t_0}\Big),
\]
which is again a Markov kernel. On $L^2_0:=\{f:\ \int f\,d\pi^{\otimes m}=0\}$, $\|P_{t_0}\|_{L^2_0\to L^2_0}=e^{-\lambda_1(G) t_0}$ (spectral gap of the heat semigroup), while $\|\mathcal K_{\beta,a}\|\le 1$. Therefore
\[
  \|K_{\rm int}^{(a)} f\|_2\ \le\ \theta_*\,\|P_{t_0} f\|_2\ +\ \big(1-\theta_*\big)\,\|f\|_2\ \le\ \Big(1-\theta_*(1-e^{-\lambda_1(G) t_0})\Big)\,\|f\|_2.
\]
The expressions for $q_*$ and $c_{\rm cut}(a)$ are immediate.
\end{proof}

\begin{proposition}[Iterated heat--kernel lower bound]\label{prop:iterated-minorization}
Let $K:=K_{\rm int}^{(a)}$ and suppose $K\ge \theta_* P_{t_0}$ with $\theta_*>0$, $t_0>0$ as in Proposition~\ref{prop:doeblin-interface}. Then for every $M\in\mathbb N$,
\[
  K^{\,M}\ \ge\ \theta_*^{\,M}\, P_{M t_0}
\]
as positive kernels on $G^m$. In particular, for any fixed $M$, $K^{\,M}(\mathbf x,\cdot)\ge c_*\, H_{t}(\mathbf x,\cdot)$ with $c_*:=\theta_*^{\,M}$ and $t:=M t_0$, uniformly in $(\beta,L)$.
\end{proposition}
\begin{proof}
We argue by induction on $M$. The case $M=1$ is Proposition~\ref{prop:doeblin-interface}. Assume $K^{M-1}\ge \theta_*^{M-1} P_{(M-1)t_0}$. For any nonnegative $f$,
\[
  K^{M} f\ =\ K\big( K^{M-1} f\big)\ \ge\ \theta_*\, P_{t_0}\big( K^{M-1} f\big).
\]
Using $K\ge \theta_* P_{t_0}$ again with $f$ replaced by $P_{t_0} g$ (positivity), we have $P_{t_0} K\ge \theta_* P_{2 t_0}$. Thus
\[
  K^{M} f\ \ge\ \theta_*\, P_{t_0}\big( K^{M-1} f\big)
   \ \ge\ \theta_*\, P_{t_0}\big( \theta_*^{M-1} P_{(M-1)t_0} f\big)
   \ =\ \theta_*^{M}\, P_{M t_0} f.
\]
Since this holds for all $f\ge 0$, the operator inequality follows. Uniformity in $(\beta,L)$ is inherited from $\theta_*,t_0$ in Proposition~\ref{prop:doeblin-interface}.
\end{proof}
\begin{proof}
Work at fixed $U$ and boundary outside the slab. The joint density of finitely many plaquettes is continuous and strictly positive with respect to the product Haar measure on $G^{|\mathcal P_{\rm int}|}$. Compactness and continuity imply a Haar lower bound for the event that each plaquette lies in $B_{r_*}(\mathbf 1)$; by absolute continuity this lower bound transfers to the Gibbs law with an explicit factor $e^{-2\beta |P(S)|}$ from Proposition~\ref{prop:doeblin-interface}. Independence across cells up to a finite geometry factor yields the stated product lower bound with $\beta$–explicit constants.
\end{proof}
\begin{lemma}[Absolute continuity on fixed regions; averaged and smoothed lower bounds]\label{lem:abs-cont}
Fix a physical slab $R\supset\Sigma$ and $a\in(0,a_0]$. For $\pi^{\otimes m}$–a.e. $U\in G^m$ the interface kernel $K_{\rm int}^{(a)}(U,\cdot)$ is absolutely continuous with respect to $\pi^{\otimes m}$ and has a continuous, strictly positive density on $G^m$. No $\beta$–uniform pointwise lower bound for this density holds in general (cf. Remark~\ref{rem:no-pointwise-lower}). Nevertheless, the following uniform controls hold:
\begin{itemize}
  \item[(i)] (Averaged small–ball lower bound) There exist constants $L_\Sigma(R,N)$, $C_1(R,N)$ and $c_\*(R,N)>0$ such that for all $\beta\ge 0$, all $U,V\in G^m$, and all $r\in\big(0, r_0/\beta\big]$ (with $r_0=r_0(R,N)>0$ sufficiently small),
  \[
    K_{\rm int}^{(a)}(U,\,B_r(V))\ \ge\ e^{-\,\beta\,(L_\Sigma r + C_1 r^2)}\, c_\*(R,N)\, r^{m\dim G}.
  \]
  \item[(ii)] (Smoothed positivity) For any fixed $\rho\in(0,\rho_\ast)$, the symmetrically smoothed kernel $\widetilde K_{\beta,L}^{\mathrm{int}}:= \mathsf S_\rho\circ K_{\beta,L}^{\mathrm{int}}\circ \mathsf S_\rho$ has a strictly positive continuous density with quantitative lower bounds depending only on $(G,\rho,|E_{\mathrm{int}}|)$, uniformly in $(\beta,L)$.
\end{itemize}
\end{lemma}
\begin{proof}
After tree gauge on a spanning tree, the joint interior law has a smooth, strictly positive density on a compact manifold; the outgoing interface is obtained by a smooth submersion (finite group multiplications), hence the push–forward has a continuous strictly positive density by Sard–Federer and compactness. The averaged bound in (i) is the ball–average control from the log–Lipschitz estimate for $\log Z(u)$ together with the small–ball volume lower bound (see the display preceding Remark~\ref{rem:no-pointwise-lower} and Lemma~\ref{lem:small-ball-volume}). Part (ii) is Lemma~\ref{lem:interface-smoothing}.
\end{proof}

\begin{proof}
On the finite slab, after tree gauge (fixing a spanning tree with fixed boundary outside), the joint law of the finitely many plaquettes intersecting the slab has a continuous, strictly positive density with respect to product Haar on $G^{|\mathcal P_{\rm int}|}$ of the form $Z^{-1} J_{\rm bnd}(U_{\mathcal P})\exp\big(\beta \sum_{p\in\mathcal P_{\rm int}} \mathrm{Re\,Tr}\,U_p\big)$ with $0<J_{\min}\le J_{\rm bnd}\le J_{\max}<\infty$ uniformly in $(\beta,\text{bnd})$ (cf. Lemma~\ref{lem:refresh-prob}). The interface configuration at time $a$ is obtained from these plaquettes by finitely many continuous group multiplications, hence its conditional law given the time–$0$ interface is the push–forward of a strictly positive continuous density on a compact manifold under a smooth submersion. Therefore it is absolutely continuous with respect to $\pi^{\otimes m}$ with a continuous and strictly positive density (Sard–Federer and compactness). Averaging over the boundary preserves these properties.
\end{proof}

\begin{proposition}[Doeblin minorization, multi--step refresh version]\label{prop:doeblin-full}
There exist an integer $M_0\ge 1$ and constants $t_0=t_0(G)>0$ and $\kappa_0=\kappa_0(R_*,a_0,G)>0$ such that for every $a\in(0,a_0]$, every volume $L$, every $\beta\ge 0$, and $\pi^{\otimes m}$–a.e. $U\in G^m$,
\[
  K_{\rm int}^{(a)\,M_0}(U,\cdot)\ \ge\ \kappa_0\, P_{t_0}(\cdot),
\]
where $P_{t_0}$ is the product heat kernel on $G^m$ at time $t_0$.
\end{proposition}
\begin{proof}
This follows from the $\beta$-- and $L$--independent one--step slab minorization after coarse refresh (Lemma~\ref{lem:beta-L-independent-minorization}), which provides constants $\theta_*\in(0,1]$ and $t_1>0$, independent of $(\beta,L,a)$ on fixed slabs, such that
\(
  K_{\rm int}^{(a)} \ge \theta_*\,P_{t_1}.
\)
Apply Proposition~\ref{prop:iterated-minorization} (with $M=1$) to obtain the stated conclusion with $M_0:=1$, $\kappa_0:=\theta_*$, and $t_0:=t_1$.
\end{proof}

\begin{proposition}[Embedding–independence of continuum Schwinger functions]\label{prop:embedding-independence-app}
Fix a bounded region $R\Subset\mathbb R^4$ and $n\ge 1$. Let $\{I_\varepsilon\}$ and $\{J_\varepsilon\}$ be two admissible directed voxel embeddings for loops in $R$, chosen equivariantly under the hypercubic symmetries and preserving the OS reflection setup. For any loop family $\{\Gamma_i\}_{i=1}^n\subset R$,
\[
  \lim_{\varepsilon\to 0}\ \Big|\, S_{n,\varepsilon}^{(I)}(\Gamma_1,\dots,\Gamma_n)\,-\,S_{n,\varepsilon}^{(J)}(\Gamma_1,\dots,\Gamma_n)\,\Big|\ =\ 0.
\]
In particular, the continuum Schwinger limits $\{S_n\}$ (when they exist) are independent of the admissible embedding choice.
\end{proposition}
\begin{proposition}[Unitary equivalence of continuum limits]\label{prop:unitary-equivalence}
Let $\{I_\varepsilon\}$, $\{J_\varepsilon\}$ be two admissible embedding schemes on a fixed region $R\Subset\mathbb R^4$ as in Proposition~\ref{prop:embedding-independence}. Suppose the continuum Schwinger functions obtained via each scheme exist and coincide on the algebra generated by loop cylinders in $R$. Then there exists a unitary $U: \mathcal H^{(I)}_R \to \mathcal H^{(J)}_R$ between the corresponding OS/GNS Hilbert spaces such that $U\,[O]^{(I)} = [O]^{(J)}$ for all gauge-invariant time-zero local observables $O$ supported in $R$, and $U e^{-tH^{(I)}} = e^{-tH^{(J)}} U$ for all $t\ge 0$.
\end{proposition}
\begin{proof}
Define $U$ on the dense subspace spanned by $[O]^{(I)}$ by $U\,[O]^{(I)} := [O]^{(J)}$. By embedding–independence, OS inner products agree on generators, so $U$ is isometric and extends by completion to a unitary. Semigroup covariance follows from equality of Schwinger functions and OS reconstruction of $e^{-tH}$ from time translations.
\end{proof}
\begin{proof}
Directedness and equivariance give $d_H(I_\varepsilon(\Gamma_i),J_\varepsilon(\Gamma_i))\le C(R)\,\varepsilon$. Apply Lemma~\ref{lem:eqc-modulus} to control the difference uniformly; sum over $i$ and let $\varepsilon\to 0$.
\end{proof}
\begin{proposition}[Boundary–condition robustness on van Hove boxes]\label{prop:bc-robust-app}
Let $R\Subset\mathbb R^4$ be fixed. For any two boundary conditions on the complement of $R$ within a van Hove box, the time-zero local Schwinger functions in $R$ differ by at most $o_{L\to\infty}(1)$ uniformly in $a\in(0,a_0]$. Consequently, continuum limits on $R$ are independent of the boundary condition within the van Hove class.
\end{proposition}
\begin{proof}
Use the interface contraction and locality to show exponential decay of boundary influences in $L$; combine with UEI to pass uniform bounds to the limit.
\end{proof}
\subsubsection*{Scheme Independence (Embeddings, Anisotropy, van Hove)}
\begin{corollary}[Scheme independence up to unitary equivalence]\label{cor:scheme-independence}
Fix a bounded region $R\Subset\mathbb R^4$. Let $\{I_\varepsilon\}$, $\{J_\varepsilon\}$ be two admissible embedding/interpolation schemes (polygonal/voxel, different smoothing kernels) that preserve the OS reflection setup and the hypercubic symmetries, and let the lattice aspect ratio satisfy a mild anisotropy $a_t/a_s\to 1$. Then the continuum limits of Schwinger functions on $R$ coincide, and the corresponding OS/GNS Hilbert spaces and semigroups are unitarily equivalent. The same limit on $R$ is obtained for any van Hove exhaustion boxes.
\end{corollary}
\begin{proof}
Embedding–independence on $R$ is Proposition~\ref{prop:embedding-independence}; unitary equivalence of the OS/GNS realizations and semigroups is Proposition~\ref{prop:unitary-equivalence}. Boundary–condition robustness for van Hove boxes is Proposition~\ref{prop:bc-robust}. Mild anisotropy $a_t/a_s\to 1$ yields the same Euclidean limit by the isotropy restoration arguments (Lemma~\ref{lem:isotropy-restore}). We do not rely on compact-group averaging in the main chain. Combining these gives equality of Schwinger limits on $R$ and unitary equivalence of the reconstructed OS/GNS data; the conclusion for any van Hove exhaustion follows from Proposition~\ref{prop:bc-robust}.
\end{proof}

\subsection*{Continuum chain}

\begin{theorem}[Spectral gap $\Rightarrow$ exponential clustering]\label{thm:gap-to-clustering}
Let $T=e^{-\tau H}$ be a positive self-adjoint contraction on an OS/GNS Hilbert space with $\|T\|\le 1$ and spectral gap $\Delta>0$ on the mean-zero subspace. Then for any fixed bounded region $R\Subset\mathbb R^4$ there exists $C(R)>0$ such that for any bounded local $f\in\mathfrak A_0^{\rm loc}(R)$ and any integer $n\ge 0$,
\[
  \big|\,\langle \Omega,\ f\, T^n f\, \Omega\rangle\,\big|\ \le\ C(R)\, e^{-n\Delta}\,\|f\|^2.
\]
\end{theorem}
\begin{proof}
Let $P$ be the vacuum projection. Since $\langle f\rangle=0$, $f\Omega\in \Omega^\perp$. Thus $\|T^n f\Omega\|\le e^{-n\Delta}\,\|f\Omega\|$. Hence $|\langle \Omega, f T^n f \Omega\rangle|\le \|f\|\,\|T^n f\Omega\|\le \|f\|^2 e^{-n\Delta}$; absorb local operator-norm bounds into $C(R)$.
\end{proof}

\begin{theorem}[Exponential clustering $\Rightarrow$ spectral gap]\label{thm:clustering-to-gap}
Let $T=e^{-\tau H}$ be a positive self-adjoint contraction on an OS/GNS Hilbert space with $\|T\|\le 1$. Suppose there exists a fixed bounded region $R\Subset\mathbb R^4$ and constants $C(R),\Delta>0$ such that for all bounded local $f\in\mathfrak A_0^{\rm loc}(R)$ with $\langle f\rangle=0$ and all $n\ge 0$,
\[
  \big|\,\langle \Omega,\ f\, T^n f\, \Omega\rangle\,\big|\ \le\ C(R)\, e^{-n\Delta}\,\|f\|^2.
\]
Then $T$ has a spectral gap at least $\Delta$ on $\Omega^\perp$.
\end{theorem}
\begin{proof}
Assume by contradiction that $\|T\|_{\Omega^\perp}>e^{-\Delta}$. Then there exists a unit vector $\psi\in\Omega^\perp$ with $\|T^n\psi\|\ge e^{-n(\Delta-\varepsilon)}$ for some $\varepsilon>0$ and all large $n$. Approximate $\psi$ by $f\Omega$ with $f\in\mathfrak A_0^{\rm loc}(R)$ (cyclicity of the local algebra). Then $\|\langle \Omega, f T^n f \Omega\rangle\|$ decays slower than $e^{-n\Delta}$, contradicting the hypothesis.
\end{proof}

\begin{theorem}[Spectral gap persistence (AF--free, non\,–\,circular)]\label{thm:gap-persist-cont}
Let $(\mathcal H_n,\langle\cdot,\cdot\rangle_n)$ be Hilbert spaces and $T_n:\mathcal H_n\to\mathcal H_n$ be positive self\,–\,adjoint contractions ($\|T_n\|\le 1$). Assume:
\begin{itemize}
  \item[(i)] (Vacuum and isolation) For each $n$, $1$ is a simple eigenvalue of $T_n$ with unit eigenvector $\Omega_n$, and there exists $q\in[0,1)$ such that on $\Omega_n^{\perp}$ one has $\|T_n\|\le q$ (equivalently, a lattice gap $\Delta_n\ge -\log q$ with $\inf_n\Delta_n\ge -\log q>0$).
  \item[(ii)] (AF\,–\,free embeddings and norm convergence) There exist isometries $U_n:\mathcal H_n\to\mathcal H$ into a Hilbert space $\mathcal H$ such that $U_n\Omega_n\to\Omega$ (unit) and
  \[
    \big\|\,U_n T_n U_n^*\,-\,T\,\big\|\ \xrightarrow[n\to\infty]{}\ 0
  \]
  for some positive self\,–\,adjoint contraction $T$ on $\mathcal H$ (this follows, e.g., from AF\,–\,free operator\,–\,norm resolvent convergence via the holomorphic functional calculus).
\end{itemize}
Then $1$ is a simple eigenvalue of $T$ with eigenvector $\Omega$, and on $\Omega^{\perp}$ one has $\|T\|\le q$. In particular, if $T=e^{-\tau H}$ with $\tau>0$ and $H\ge 0$ self\,–\,adjoint, then
\[
  \operatorname{spec}(H)\ \subset\ \{0\}\,\cup\,\big[\,-\tfrac{1}{\tau}\log q\,,\,\infty\big)\,.
\]
\end{theorem}
\begin{proof}
Fix $\eta\in(0,\tfrac{1}{2}(1-q))$. For each $n$, the spectrum of $T_n$ is contained in $\{1\}\cup (-\infty,1-2\eta]$ by (i). Let $\gamma:=\{z\in\mathbb C:\ |z-1|=\eta\}$ and define the Riesz projections
\[
  P_n\ :=\ \frac{1}{2\pi i}\oint_{\gamma} (z-T_n)^{-1}\,dz\,,\qquad Q_n:=I-P_n\,.
\]
Then $P_n$ is the rank\,–\,one projection onto $\mathbb C\Omega_n$ and $\|T_n Q_n\|\le q$. Set $S_n:=U_n T_n U_n^*$. By (ii) and the resolvent identity,
\[
  \big\|\,(z-S_n)^{-1}-(z-T)^{-1}\,\big\|\ \le\ \frac{\|S_n-T\|}{\mathrm{dist}(z,\sigma(S_n))\,\mathrm{dist}(z,\sigma(T))}\ \xrightarrow[n\to\infty]{}\ 0\qquad(z\in\gamma),
\]
for $n$ large (since $\mathrm{dist}(\gamma,\sigma(S_n))$ and $\mathrm{dist}(\gamma,\sigma(T))$ stay $>0$ by norm convergence). Hence the projections
\[
  P\ :=\ \frac{1}{2\pi i}\oint_{\gamma} (z-T)^{-1}\,dz\,,\qquad \widetilde P_n\ :=\ U_n P_n U_n^*
\]
converge in operator norm: $\|\widetilde P_n-P\|\to 0$. In particular, $\mathrm{rank}(P)=1$ and we set $Q:=I-P$, so that $\operatorname{Ran}P=\mathbb C\Omega$ and $\Omega$ is the vacuum of $T$.

Let $\psi\in\mathcal H$ with $\langle\psi,\Omega\rangle=0$ (i.e., $\psi=Q\psi$). Decompose
\[
  \|T\psi\|\ \le\ \|T\psi - S_n\psi\|\ +\ \|S_n\psi - S_n\widetilde Q_n\psi\|\ +\ \|S_n\widetilde Q_n\psi\|\,\quad \widetilde Q_n:=I-\widetilde P_n\,.
\]
The first term is $\le \|S_n-T\|\,\|\psi\|\xrightarrow{}0$ by (ii). For the second, $\|\psi-\widetilde Q_n\psi\|=\| (P-\widetilde P_n)\psi\|\le \|P-\widetilde P_n\|\,\|\psi\|\xrightarrow{}0$, and $\|S_n\|\le 1$, hence the second term $\xrightarrow{}0$. For the third term, note that $\widetilde Q_n\psi\in \operatorname{Ran}\widetilde Q_n=U_n\operatorname{Ran} Q_n$, so there exists $\phi_n\in \mathcal H_n$ with $\phi_n=Q_n\phi_n$ and $\widetilde Q_n\psi=U_n\phi_n$. Therefore
\[
  \|S_n\widetilde Q_n\psi\|\ =\ \|U_n T_n U_n^* U_n\phi_n\|\ =\ \|U_n T_n\phi_n\|\ =\ \|T_n\phi_n\|\ \le\ q\,\|\phi_n\|\ =\ q\,\|\widetilde Q_n\psi\|\ \le\ q\,\|\psi\|\,.
\]
Taking $\limsup_{n\to\infty}$ in the three\,–\,term bound gives $\|T\psi\|\le q\,\|\psi\|$. Since $\psi\in\Omega^{\perp}$ was arbitrary, $\|T\|_{\Omega^{\perp}}\le q$ as claimed. If $T=e^{-\tau H}$, then the spectral mapping theorem yields $\sigma(T)=e^{-\tau\sigma(H)}$, so $\|T\|_{\Omega^{\perp}}\le e^{-\tau\Delta}$ with $\Delta:=\inf\sigma(H|_{\Omega^{\perp}})$; hence $e^{-\tau\Delta}\le q$ and $\Delta\ge -\tau^{-1}\log q$.
\end{proof}

\begin{corollary}[Generator formulation]\label{cor:gap-persist-generator}
Let $H_n\ge 0$ be self\,–\,adjoint on $\mathcal H_n$ with transfers $T_n=e^{-\tau H_n}$ ($\tau>0$ fixed). Assume (i) and (ii) of Theorem~\ref{thm:gap-persist-cont} with $\|T_n\|_{\Omega_n^{\perp}}\le e^{-\tau\Delta_*}$ for some $\Delta_*>0$. Then the limit generator $H\ge 0$ on $\mathcal H$ obeys
\[
  \operatorname{spec}(H)\ \subset\ \{0\}\,\cup\,[\,\Delta_*\,,\,\infty)\,.
\]
\end{corollary}

\subsection*{Interface Smoothing and Uniform Sandwich}

\subsubsection*{Notation for Interface Smoothing}
Let $E_{\mathrm{int}}$ be the set of oriented interface links, $m_E$ product Haar on $G^{E_{\mathrm{int}}}$. For parameters $(\beta,L,a)$, let $K_{\beta,L}^{\mathrm{int}}$ denote the one--tick interface kernel across the OS cut acting on the interface link configuration space (cf. Definition~\ref{def:interface-kernel}); when the tick size $a$ is emphasized and $(\beta,L)$ are suppressed, we also write $K_{\rm int}^{(a)}$ for the same operator.

For $\rho>0$ (below the injectivity radius), define the ball–average smoothing $\mathsf S_\rho$ by convolution with the product uniform density on $\prod_{e\in E_{\mathrm{int}}} B_G(e,\rho)$. Define the symmetrically smoothed interface kernel
\[
  \widetilde K_{\beta,L}^{\mathrm{int}}\ :=\ \mathsf S_\rho\ \circ\ K_{\beta,L}^{\mathrm{int}}\ \circ\ \mathsf S_\rho\,.
\]
\begin{lemma}[Interface smoothing yields strictly positive continuous density]\label{lem:interface-smoothing}
For any fixed $\rho\in(0,\rho_\ast)$, $\widetilde K_{\beta,L}^{\mathrm{int}}$ is a Feller, positivity–preserving Markov kernel on $G^{E_{\mathrm{int}}}$ with a strictly positive continuous density, uniformly in $(\beta,L)$. The quantitative lower bounds depend only on $(G,\rho,|E_{\mathrm{int}}|)$.
\end{lemma}
\begin{proof}[Proof sketch]
Convolution by a continuous strictly positive density on a neighborhood of the identity preserves positivity and regularizes densities; composing on both sides ensures continuity and strict positivity everywhere by compactness and finite convolution power arguments on $G^{E_{\mathrm{int}}}$.
\end{proof}

\begin{lemma}[Small-ball convolution lower bounds the heat kernel]\label{lem:ball-to-hk}
Let $G$ be a compact simple Lie group of dimension $d$, endowed with the bi-invariant metric and Haar probability $m_G$. Fix $\rho\in(0,\rho_\ast)$ below the injectivity radius and let $U_\rho$ be the central probability density equal to the normalized indicator of the geodesic ball $B_G(e,\rho)$. Then there exist integers $n_\ast\ge 1$ and constants $c_\ast\in(0,1)$ and $t_\ast>0$, depending only on $(G,\rho)$, such that
\[
  U_\rho^{\ast n_\ast}(g)\ \ge\ c_\ast\, H_{t_\ast}(g)\qquad\text{for all }g\in G,
\]
where $H_t$ is the heat--kernel density at time $t$ on $G$.
\end{lemma}
\begin{proof}
Write $U_\rho$ as a central, symmetric probability density with support in a normal neighborhood of the identity; its convolution powers are continuous, strictly positive for all large enough $n$ by standard hypoellipticity and the fact that the support generates $G$. By the local central limit theorem on compact Lie groups (parametrix/Varadhan Gaussian lower bounds), there exist $c_1,c_2,c_3>0$ (depending only on $G$) such that for all $n\ge 1$ and all $g\in G$,
\[
  U_\rho^{\ast n}(g)\ \ge\ c_1\, n^{-d/2}\,\exp\!\Big(\!-\,\frac{d_G(e,g)^2}{c_2\, n\,\rho^2}\Big)\,\mathbf 1_{\{n\ge c_3\,\mathrm{diam}(G)^2/\rho^2\}}.
\]
On the other hand, the heat kernel obeys the global Gaussian upper/lower bounds on compact groups: there exist $a_1,a_2>0$ such that for all $t\in(0,1]$ and $g\in G$,
\[
  a_1\,t^{-d/2}\,\exp\!\Big(\!-\,\frac{d_G(e,g)^2}{a_2 t}\Big)\ \le\ H_t(g)\ \le\ a_1^{-1}\,t^{-d/2}\,\exp\!\Big(\!-\,\frac{d_G(e,g)^2}{(a_2/2) t}\Big).
\]
Choosing $n_\ast:=\big\lceil c_3\,\mathrm{diam}(G)^2/\rho^2\big\rceil$ and $t_\ast:= c_2\, n_\ast\,\rho^2$ yields
\[
  U_\rho^{\ast n_\ast}(g)\ \ge\ c_1\, n_\ast^{-d/2}\,\exp\!\Big(\!-\,\frac{d_G(e,g)^2}{c_2\, n_\ast\,\rho^2}\Big)
  \ \ge\ c_\ast\, H_{t_\ast}(g)
\]
with $c_\ast:= c_1\,a_1\,(t_\ast/n_\ast)^{d/2}\,\exp\!\big(\!-\frac{d_G(e,g)^2}{c_2 n_\ast\rho^2}+\frac{d_G(e,g)^2}{a_2 t_\ast}\big)$; the exponentials match since $t_\ast=c_2 n_\ast\rho^2$, and the prefactor depends only on $(G,\rho)$ after taking the infimum over $g\in G$. This gives the stated pointwise lower bound uniformly in $g$.
\end{proof}

\begin{corollary}[Product form on interface blocks]\label{cor:product-ball-to-hk}
Let $B$ be a finite set of interface links and consider the product group $G^B$ with product Haar measure and product metric. Let $U_\rho^{(B)}$ be the product of the small-ball densities on each coordinate. Then there exist $n_\ast, t_\ast, c_\ast$ depending only on $(G,\rho,|B|)$ such that
\[
  \big(U_\rho^{(B)}\big)^{\ast n_\ast}(u)\ \ge\ c_\ast\, H_{t_\ast}^{(B)}(u)\qquad(u\in G^B),
\]
where $H_{t}^{(B)}$ is the product heat kernel on $G^B$.
\end{corollary}
\begin{proof}
Apply Lemma~\ref{lem:ball-to-hk} on each coordinate and use that convolution and heat kernels tensorize on product groups; constants multiply accordingly.
\end{proof}

\begin{proposition}[Uniform sandwich after smoothing]\label{prop:sandwich}
There exist integers $M_\ast\ge 1$, $T_\ast>0$, and $c_\ast\in(0,1)$, depending only on $(G,\rho,|E_{\mathrm{int}}|)$, such that uniformly in $(\beta,L)$,
\[
  (\widetilde K_{\beta,L}^{\mathrm{int}})^{M_\ast}(\mathbf x,\mathbf y)\ \ge\ c_\ast\, H_{T_\ast}(\mathbf x,\mathbf y)\qquad(\mathbf x,\mathbf y\in G^{E_{\mathrm{int}}}).
\]
\end{proposition}
\begin{proof}[Proof sketch / external input]
The smoothed kernel $\widetilde K_{\beta,L}^{\mathrm{int}}=\mathsf S_\rho\circ K_{\beta,L}^{\mathrm{int}}\circ \mathsf S_\rho$ is strong Feller on the compact state space $G^{E_{\mathrm{int}}}$ and inherits irreducibility/aperiodicity from the support of the small--ball pulse $\mathsf S_\rho$ and the interface dynamics. Standard compact--state--space minorization theory for strong Feller kernels then yields that for some $M_\ast\ge 1$ (depending only on $(G,\rho,|E_{\mathrm{int}}|)$) the iterate $(\widetilde K_{\beta,L}^{\mathrm{int}})^{M_\ast}$ admits a strictly positive continuous density bounded below uniformly in $(\beta,L)$, hence dominates Haar and therefore dominates a fixed heat kernel $H_{T_\ast}$ up to a constant $c_\ast\in(0,1)$. We record this as a standard smoothing/minorization input on compact groups; see, e.g., \cite{DiaconisSaloffCoste2004,VaropoulosSaloffCosteCoulhon1992} for quantitative compact-group smoothing and minorization mechanisms.
\end{proof}

% (Area-law equivalence block removed; see one-way consequences in Theorem~\ref{thm:AL-linear} and Proposition~\ref{prop:AL-torelon}.)

\subsection*{Area Law: One-Way Consequences Only}

\begin{theorem}[Area law $\Rightarrow$ linear confinement (finite--$T$ and asymptotic)]\label{thm:AL-linear}
Assume a rectangular Wilson loop area law $\langle W(R,T)\rangle\le K e^{-\sigma R T}$ for all $R\ge R_\ast$, $T\ge T_\ast$. Then $V_T(R):=-(1/T)\log\langle W(R,T)\rangle\ge \sigma R - (\log K)/T$ for all admissible $(R,T)$, and $V(R):=\limsup_{T\to\infty}V_T(R)\ge \sigma R$ for all $R\ge R_\ast$.
\end{theorem}

\begin{proposition}[Area law $\Rightarrow$ torelon lower bound]\label{prop:AL-torelon}
Under the same hypothesis, in a periodic box of spatial period $L\ge R_\ast$, the lowest energy in the sector with one unit of winding electric 1--form charge obeys $E_{\rm tor}(L)\ge \sigma L$.
\end{proposition}

\begin{mdframed}[linewidth=0.5pt, linecolor=gray!40, backgroundcolor=gray!5, roundcorner=2pt, innertopmargin=8pt, innerbottommargin=8pt, skipabove=10pt, skipbelow=10pt]
\textbf{Remark.} We do not claim any equivalence between area laws, gaps, and clustering. In particular, a spectral gap does not imply an area law in general (abelian Gaussian counterexample with perimeter law; see Proposition~\ref{AL:gaussian-perimeter}). The manuscript uses only one--way consequences of an assumed area law and keeps them logically disjoint from NRC.
\end{mdframed}

\begin{proposition}[Gapped Gaussian abelian gauge field has perimeter law]\label{AL:gaussian-perimeter}
In a $4$D abelian (Gaussian) gauge theory with a massive propagator kernel (e.g. Proca/Stueckelberg), after the standard multiplicative renormalization of Wilson loops one has a \emph{perimeter} law
\[
  -\log\,\langle W(C)\rangle\ =\ c_m\,\mathrm{Perimeter}(C)\ +\ o(\mathrm{Perimeter}(C))\quad(C\ \text{large, smooth}),
\]
with $c_m>0$ depending on the mass and coupling, and with no positive area term. Thus the theory has a positive mass gap but not an area law.
\emph{Proof (outline).} In a Gaussian gauge theory $-\log\langle W(C)\rangle$ reduces to a quadratic form $\tfrac{g^2}{2}\,\langle J_C,\,K_m J_C\rangle$ of the $1$-current $J_C$ supported on $C$, with $K_m$ an exponentially decaying kernel. For $J_C$ supported on a $1$D curve, the dominant self-interaction is local along $C$ and scales with its arc length; exponentially small distant contributions are subleading. No surface-extensive contribution appears. \qed
\end{proposition}

\subsection*{Local Reflection Negativity for Surface Curvature (Checkable Sign and Decay)}
\label{SurfNeg:section}
Let $\Sigma$ be a nearly flat rectangle symmetric about the OS reflection plane $\{t=0\}$, and let $\mathcal E(\xi)$ denote the basepoint--parallel--transported chromo--electric component $\mathcal F_{01}(\xi)$ on $\Sigma$. For test $2$--forms $\phi$ supported on the upper half $\Sigma_+$ and transported to the basepoint, set $O(\phi):=\int_{\Sigma_+} \mathcal E(\xi):\phi(\xi)\,d\sigma_\xi$.

\begin{theorem}[Reflection negativity for T--odd surface curvature]\label{SurfNeg:neg-def}
With notation above, the connected two--point kernel $K(\xi,\eta):=\big\langle\!\big\langle\,\mathcal E(\theta\xi):\mathcal E(\eta)\,\big\rangle\!\big\rangle$ on $\Sigma_+\times\Sigma_+$ is negative semidefinite:
\[
  \iint_{\Sigma_+^2} \phi(\xi):K(\xi,\eta):\phi(\eta)\,d\sigma_\xi d\sigma_\eta\ =\ -\,\langle\theta O(\phi),O(\phi)\rangle_{\rm OS}\ \le\ 0.
\]
Equality holds iff $O(\phi)\,\vert 0\rangle=0$.
\end{theorem}

\begin{proposition}[Exponential bound beyond a microscopic cutoff]\label{SurfNeg:exp-bound}
Assume a mass gap $\gamma>0$ for the OS generator. If $\operatorname{dist}(\operatorname{supp}\phi,\,t=0)\ge \varepsilon>0$, then
\[
  \Big\lvert\,\iint_{\Sigma_+^2} \phi(\xi):K(\xi,\eta):\phi(\eta)\,d\sigma_\xi d\sigma_\eta\,\Big\rvert\ \le\ C_{\Sigma}\,e^{-2\varepsilon\gamma}\,\Vert\phi\Vert_{L^2(\Sigma_+)}^2.
\]
Moreover, if the spectral projector onto $[\gamma,\gamma+\Delta]$ of $H$ has a uniform local weight on $O(\phi)\,\vert 0\rangle$, then the left-hand side is bounded by $-\kappa_{\Sigma,\Delta}\,e^{-2\varepsilon\gamma}\,\Vert\phi\Vert_{L^2}^2$ for some $\kappa_{\Sigma,\Delta}>0$ (checkable on a fixed rectangle).
\end{proposition}
\subsubsection*{Group Generality}
All arguments extend to any compact simple Lie group $G$, with spectral constants (e.g., heat--kernel gap) expressed in terms of $\lambda_1(G)$, the first nonzero Laplace--Beltrami eigenvalue on $G$. Bounds and rates depending on $\lambda_1(N)$ for $\mathrm{SU}(N)$ carry over by replacing $\lambda_1(N)$ with $\lambda_1(G)$.
\begin{lemma}[Interface minorization uniform in $L$; $\beta$-uniform]\label{lem:interface-minorization-uniform}
With $t_0=t_0(G)>0$ and $\kappa_0=\kappa_0(R_*,a_0,G)>0$ as in Proposition~\ref{prop:doeblin-full}, define $\theta_*:=\kappa_0$. Then for every $a\in(0,a_0]$, every volume $L$, and every $\beta\ge 0$,
\[
  K_{\rm int}^{(a)}(U,\cdot)\ \ge\ \theta_*\, p_{t_0}(\cdot)\,\pi^{\otimes m}(d\cdot)\qquad\text{for $\pi^{\otimes m}$--a.e. $U\in G^m$,}
\]
where $p_{t_0}$ is the product heat--kernel density on $G^m$ at time $t_0$. Equivalently,
\[
  K_{\rm int}^{(a)}\ =\ \theta_*\,P_{t_0}\ +\ (1-\theta_*)\,\mathcal K_{\beta,a}
\]
for some Markov kernel $\mathcal K_{\beta,a}$ on $G^m$. The time $t_0$ depends only on $(R_*,a_0,G)$ and is independent of $(\beta,L)$; $\theta_*$ depends only on $(R_*,a_0,G)$ (independent of $\beta$) and is uniform in $L$.
\end{lemma}

\begin{proof}
This is immediate from Proposition~\ref{prop:doeblin-full}: by the proof of Proposition~\ref{prop:doeblin-full} (via Lemma~\ref{lem:beta-L-independent-minorization}), one may take $M_0=1$ so that
\(
  K_{\rm int}^{(a)}\ge \kappa_0\,P_{t_0}
\)
with $\kappa_0,t_0$ independent of $(\beta,L,a)$ on fixed slabs. Setting $\theta_*:=\kappa_0$ yields the displayed inequality. The convex-split form is the standard Nummelin decomposition: define $\mathcal K_{\beta,a}:=(K_{\rm int}^{(a)}-\theta_*P_{t_0})/(1-\theta_*)$.
\end{proof}

\begin{remark}[Alternative derivation (sketch; conditional)]
Assume a $\beta$-- and boundary--uniform cell refresh bound for the event that the relevant plaquettes lie in a fixed small ball (this can be supplied by the coarse-refresh route in Lemma~\ref{lem:beta-L-independent-minorization} and Appendix~\ref{app:beta-indep-minorization}). Then one may derive the minorization by conditioning on the refresh event, using tree gauge and a compact-group convolution lower bound (e.g. Lemma~\ref{lem:ball-conv-lower}) to upgrade small-ball increments to a heat-kernel component, and finally averaging over the refresh event. This argument is recorded only as an intuition/bridge; the main text uses the coarse-refresh reduction.
\end{remark}

\begin{corollary}[Convex split and contraction]\label{cor:convex-split}
With $\kappa_0$ and $t_0$ as above, one has the convex decomposition on $L^2_0(G^m,\pi^{\otimes m})$,
\[
  K_{\rm int}^{(a)}\ =\ \theta_* P_{t_0}\ +\ (1-\theta_*)\,\mathcal{K}_{\beta,a},\qquad \theta_*:=\kappa_0\in(0,1),
\]
where $P_{t_0}$ is the product heat--kernel operator and $\|P_{t_0}\|_{L^2_0\to L^2_0}=e^{-\lambda_1(G) t_0}$. Consequently,
\[
  \|K_{\rm int}^{(a)} f\|_{L^2}\ \le\ \big(1-\theta_*(1-e^{-\lambda_1(G) t_0})\big)\,\|f\|_{L^2},\qquad f\perp \text{constants},
\]
which is the one--step contraction used in Theorem~\ref{thm:two-layer-explicit} and the definition of $c_{\rm cut}$.
\end{corollary}
\begin{proof}
The minorization of Proposition~\ref{prop:doeblin-full} implies $K_{\rm int}^{(a)}\ge \theta_* P_{t_0}$ as positive kernels. Write $\mathcal{K}_{\beta,a}:=(K_{\rm int}^{(a)}-\theta_* P_{t_0})/(1-\theta_*)$, which is Markov. On the orthogonal complement of constants, $\|P_{t_0}\|=e^{-\lambda_1(G) t_0}$ while $\|\mathcal{K}_{\beta,a}\|\le 1$, hence the displayed bound.
\end{proof}

\begin{mdframed}[linewidth=0.5pt, linecolor=yellow!40, backgroundcolor=yellow!3, roundcorner=2pt, innertopmargin=8pt, innerbottommargin=8pt, skipabove=10pt, skipbelow=10pt]
\noindent\textbf{Remark (Boundary and $\beta$-Independence).} Lemma~\ref{lem:abs-cont} ensures the existence of densities and removes measurability issues. The $\beta$--uniform interface minorization used in the main chain is supplied by the coarse-refresh route (Lemma~\ref{lem:beta-L-independent-minorization}) and the explicit appendix proof (Appendix~\ref{app:beta-indep-minorization}, Proposition~\ref{prop:explicit-doeblin-constants-appendix}). In particular, the resulting Doeblin weight $\kappa_0$ depends only on the fixed slab geometry $(R_*,a_0)$ and group data $G$ and is independent of $(\beta,L,a)$ on fixed slabs.
\end{mdframed}

\begin{proposition}[Explicit boundary–uniform Doeblin constants and short-time scaling]\label{prop:explicit-doeblin-constants}
Fix a physical slab radius $R_*>0$, maximal tick $a_0>0$, and $G=\mathrm{SU}(N)$. There exist constants
\[
  n_{\rm cells}=n_{\rm cells}(R_*),\quad r_*=r_*(R_*,a_0,N)>0,\quad \alpha_{\rm ref}=\alpha_{\rm ref}(R_*,a_0,N)\in(0,1],
\]
and group–intrinsic constants $m_*(N)\in\mathbb N$, $t_0(N)>0$, $c_*(N,r_*)>0$, together with a geometry factor $c_{\rm geo}(R_*,a_0)\in(0,1]$, such that for every $a\in(0,a_0]$, every torus size $L$, every $\beta\ge 0$, and $\pi^{\otimes m}$–a.e. $U\in G^m$,
\begin{equation}\label{eq:doeblin-kappa0}
  K_{\rm int}^{(a)}(U,\cdot)\ \ge\ \kappa_0\,\bigotimes_{\ell=1}^m p_{t_0}(\cdot),\qquad
  \kappa_0\ :=\ c_{\rm geo}(R_*,a_0)\,\big(\alpha_{\rm ref}(R_*,a_0,N)\,c_*(N,r_*)\big)^{m_{\rm cut}(R_*,a_0)}.
\end{equation}
In particular, $\kappa_0$ and $t_0$ are independent of $(\beta,L,a)$ and depend only on $(R_*,a_0,G)$. Moreover, one can choose short-time scalings $t_0(a)=c_0(G)\,a$ and $\kappa(a)\ge c_1(R_*,a_0,G)\,a$ so that $K_{\rm int}^{(a)}\ge \kappa(a)\,P_{t_0(a)}$ per slab tick $a$, with constants depending only on $m_{\rm cut}(R_*,a_0)$, $\lambda_1(G)$, and slab geometry (all independent of $\beta$).

\begin{proof}
This is the $\beta$--independent coarse-refresh minorization on a fixed slab, recorded here with explicit cell-factorization dependence in \eqref{eq:doeblin-kappa0}. One route is via the coarse interface Doeblin bound (Proposition~\ref{prop:coarse-doeblin}), which yields a one--step convex split on the fixed coarse interface state space (Lemma~\ref{lem:beta-L-independent-minorization}). An alternative derivation with explicit cellwise constants is given in Appendix~\ref{app:beta-indep-minorization}, Proposition~\ref{prop:explicit-doeblin-constants-appendix}.

For the short-time scalings, choose $t_0(a)=c_0(G)\,a$ in the small-time regime of Lemma~\ref{lem:hk-lower-explicit} and use the semigroup property to normalize the Doeblin component per tick, yielding $\kappa(a)\ge c_1(R_*,a_0,G)\,a$.
\end{proof}
\end{proposition}

\begin{lemma}[Short-time heat--kernel lower bound on compact groups]\label{lem:hk-lower-explicit}
Let $G=\mathrm{SU}(N)$ with bi-invariant Laplace--Beltrami operator and heat kernel $p_t$. There exist $c_0(N),c_*(N,r)>0$ and $t_*(N)>0$ such that for all $t\in(0,t_*)$ and all $g\in G$,
\[
  p_t(g)\ \ge\ c_*(N,r)\, t^{\,\dim G/2}\,\chi_{B_r(\mathbf 1)}(g)\,.
\]
In particular, for any $m\in\mathbb N$, the product kernel on $G^m$ satisfies $\prod_{\ell=1}^m p_t(g_\ell)\ge c_*(N,r)^{m}\,t^{m\dim G/2}\,\chi_{B_r(\mathbf 1)^m}(g)$.
\end{lemma}
\begin{proof}
Compactness and smoothness imply $p_t(\cdot)$ is strictly positive and near-identity admits a Gaussian lower bound for small $t$ (Varadhan/Minakshisundaram--Pleijel asymptotics). Choose $r$ below the injectivity radius and take $t_*$ small so that the coordinate chart and Jacobian variations are controlled; the bounds reduce to Euclidean heat kernel lower bounds times Jacobian and curvature constants.
\end{proof}

\begin{proposition}[Multi-step scale-adapted Doeblin with explicit constants]\label{prop:multistep-doeblin}
Fix $(R_*,a_0,N)$ and let $m=m_{\rm cut}(R_*,a_0)$. With constants from Proposition~\ref{prop:explicit-doeblin-constants} and Lemma~\ref{lem:hk-lower-explicit}, define
\[
  t_0\ :=\ t_0(G),\qquad \theta_*\ :=\ \kappa_0\,,\qquad \lambda_1\ :=\ \lambda_1(G)\,.
\]
Let $k\in\mathbb N$ and consider the $k$-fold interface transfer $K_{\rm int}^{(a),k}$. Then for any $k\ge 1$,
\[
  K_{\rm int}^{(a),k}\ \ge\ \kappa_k\, P_{k t_0}\,,\qquad \kappa_k\ :=\ 1 - \big(1-\theta_*\big)^{k}\,\Big(1- c_*(N,r_*)^{m}\Big)\,.
\]
In particular, choosing $k\asymp a^{-1}$ so that $k t_0\in[t_*,2t_*]$ for a fixed $t_*>0$, one gets a scale-adapted minorization
\[
  K_{\rm int}^{(a),k(a)}\ \ge\ \kappa_*\, P_{t_*}\,,\qquad \kappa_*\ =\ \kappa_*(R_*,a_0,N)\in(0,1] \,.
\]
Moreover, on $L^2_0$,
\[
  \big\|K_{\rm int}^{(a),k}\big\|\ \le\ \big(1-\theta_*(1-e^{-\lambda_1(G) t_0})\big)^{k}\,.
\]
\end{proposition}
\begin{proof}
Write the one-step convex split $K=\theta_* P_{t_0}+(1-\theta_*)\mathcal K$ with $\mathcal K$ Markov. Then
\[
  K^{k}\ =\ \sum_{j=0}^{k} \binom{k}{j} \theta_*^{\,j} (1-\theta_*)^{k-j}\, \underbrace{\mathcal K^{k-j} P_{t_0}^{\,j}}_{\ge 0}\ \ge\ \theta_*^{\,k} P_{k t_0}\,.
\]
Using Lemma~\ref{lem:hk-lower-explicit} at $t_0$ shows $P_{t_0}\ge c_*(N,r_*)^{m} \Pi_{B_{r_*}}$ (projection to densities supported in $B_{r_*}^m$). Hence every term with $j\ge 1$ contributes a strictly positive component, and summing gives the stated $\kappa_k$ (a crude but explicit bound suffices). The $L^2_0$-norm bound follows by functional calculus: on the orthogonal complement of constants, $\|P_{t_0}\|=e^{-\lambda_1(G) t_0}$ and $\|\mathcal K\|\le 1$, so $\|K\|\le 1-\theta_*(1-e^{-\lambda_1(G) t_0})$ and $\|K^k\|\le (1-\theta_*(1-e^{-\lambda_1(G) t_0}))^{k}$.
\end{proof}
\begin{corollary}[UEI with explicit constants (raw route A)]\label{cor:uei-explicit-constants}
In the raw small-field closure route A of Theorem~\ref{thm:uei-fixed-region}, let $M_R$ be as in Assumption~\ref{assump:uei-mean} and fix any $a\in(0,a_0]$ with $\beta=\beta(a)\ge \beta_{\min}(R,N)>0$. Let
\[
  \rho_{\min}(R,N)\ :=\ c_2(R,N)\,\beta_{\min}(R,N),\qquad
  G_R(R,N,a_0)\ :=\ C_1(R,N)\,a_0^4,
\]
where $c_2(R,N)$ is the LSI constant from Step 2 and $C_1(R,N)$ the Lipschitz constant from Step 3 of the proof of Theorem~\ref{thm:uei-fixed-region}. Set
\[
  \eta_R\ :=\ \min\Big\{\,t_*(R,N),\ \sqrt{\,\rho_{\min}(R,N)\big/ G_R(R,N,a_0)\,}\ \Big\},\qquad
  C_R\ :=\ \exp\big(\eta_R\,M_R\big)\,e^{1/2}.
\]
Then for all volumes $L$ and all boundary conditions outside $R$,
\[
  \mathbb{E}_{\mu_{L,a}}\big[e^{\eta_R S_R(U)}\big]\ \le\ C_R.
\]
All constants depend only on $(R,a_0,N,\beta_{\min})$ and are independent of $L$ and $\beta\ge \beta_{\min}$.
\end{corollary}
\begin{proof}
This is the consolidation of Steps 2--5 in the proof of Theorem~\ref{thm:uei-fixed-region} with $\rho_{\min}:=c_2\beta_{\min}$ and $G_R:=C_1 a_0^4$, choosing $\eta_R$ so that the Herbst bound yields a $\le e^{1/2}$ factor for the centered variable and then absorbing the (uniform) mean $M_R$.
\end{proof}

\subsection*{Uniform gap $\Rightarrow$ uniform clustering; converse}

\begin{proposition}[Gap $\Rightarrow$ clustering (uniform)]\label{prop:gap-to-cluster}
If $\mathrm{spec}(H_{L,a})\subset\{0\}\cup[\gamma_0,\infty)$ holds uniformly in $(L,a)$, then for any time-zero, gauge-invariant local $O$ with $\langle O\rangle=0$ and all $t\ge 0$,
\[
  |\langle\Omega, O(t)O(0)\Omega\rangle|\;\le\;\|O\Omega\|^2 e^{-\gamma_0 t},
\]
uniformly in $(L,a)$.
\end{proposition}

\begin{proposition}[OS0 polynomial bounds with explicit constants]\label{prop:OS0-poly}
Assume uniform exponential clustering of truncated correlations on fixed physical regions with parameters $(C_0,m)$ (independent of $(L,a)$). Fix any $q>d$ and set $p=d+1$. Then there exist explicit constants
\[
  C_n(C_0,m,q,d)\ :=\ C_0^n\,C_{\mathrm{tree}}(n)\,\Bigl(\frac{2^d\,\zeta(q-d)}{1-e^{-m}}\Bigr)^{n-1},\qquad C_{\mathrm{tree}}(n)\le n^{n-2},
\]
such that for all local loop families $\Gamma_1,\dots,\Gamma_n$,
\[
  |S_n(\Gamma_1,\dots,\Gamma_n)|\ \le\ C_n\,\prod_{i=1}^n (1+\operatorname{diam}\Gamma_i)^p\,\prod_{1\le i<j\le n} (1+\operatorname{dist}(\Gamma_i,\Gamma_j))^{-q},
\]
uniformly in $(L,a)$. In particular, the Schwinger functions are tempered (OS0).
\end{proposition}

\begin{proof}
Apply the Brydges tree-graph bound \cite{Brydges1978} to expand $S_n$ as a sum over labeled spanning trees $\tau$ on $n$ vertices of products of truncated correlators $\kappa_{|e|}$ over edges $e\in E(\tau)$, with signs and combinatorial factors bounded by $C_{\mathrm{tree}}(n)\le n^{n-2}$ (Cayley-Prüfer count). Insert the assumed exponential clustering: each edge contributes at most $C_0^{|e|} e^{-m \operatorname{dist}(e)}$. There are $n-1$ edges, yielding overall $C_0^n$ (overcounting the root).

For each edge, bound $e^{-m r} \le (1-e^{-m})^{-1} (1+r)^{-q}$ and sum over lattice positions using $\sum_{x\in\mathbb Z^d} (1+\|x\|)^{-q} \le 2^d \zeta(q-d)$ for $q>d$. Multiply the $(n-1)$ identical factors to get $\bigl(\frac{2^d \zeta(q-d)}{1-e^{-m}}\bigr)^{n-1}$.

The diameter factor arises from bounding the smearing over loop positions: each loop contributes a factor $(1+\operatorname{diam}\Gamma_i)^
{d+1}$ to account for the $d$-dimensional volume and an extra for boundary, setting $p=d+1$. All steps are uniform in $(L,a)$, completing the proof.
\end{proof}

\begin{corollary}[OS0 with explicit constants in $d=4$]\label{cor:os0-explicit-4d}
In $d=4$, fix any $q>4$ and set $p=5$. Under the clustering hypothesis of Proposition~\ref{prop:OS0-poly} with parameters $(C_0,m)$ independent of $(L,a)$, the constants
\[
  C_n\big(C_0,m,q\big)\ :=\ C_0^n\,C_{\mathrm{tree}}(n)\,\Big(\frac{16\,\zeta(q-4)}{1-e^{-m}}\Big)^{n-1},\qquad C_{\mathrm{tree}}(n)\le n^{n-2},
\]
yield for all loop families $\{\Gamma_i\}$ the bound
\[
  |S_n(\Gamma_1,\dots,\Gamma_n)|\ \le\ C_n\,\prod_{i=1}^n \bigl(1+\operatorname{diam}\Gamma_i\bigr)^5\,\prod_{1\le i<j\le n} \bigl(1+\operatorname{dist}(\Gamma_i,\Gamma_j)\bigr)^{-q}.
\]
Consequently, the Schwinger functions are tempered (OS0) with explicit constants.
\end{corollary}
\begin{proof}
Specialize Proposition~\ref{prop:OS0-poly} to $d=4$; $2^d=16$ and $p=d+1=5$.
\end{proof}

\begin{proposition}[Clustering on a generating local class $\Rightarrow$ gap]\label{prop:cluster-to-gap}
Suppose there exist $R_*>0$, $\gamma>0$, and $C_*<\infty$, independent of $(L,a)$, such that for all local $O$ with $\langle O\rangle=0$,
\[
  |\langle\Omega, O(t)O(0)\Omega\rangle|\;\le\; C_*\,\|O\Omega\|^2 e^{-\gamma t}\quad(\forall t\ge 0),
\]
and that the span of such $O\Omega$ is dense in $\Omega^\perp$. Then $\mathrm{spec}(H_{L,a})\subset\{0\}\cup[\gamma,\infty)$ uniformly in $(L,a)$.
\end{proposition}
\[
  \beta(a)\;=\; \frac{11N}{48\pi^2}\,\log\frac{1}{a\,\Lambda_{\mathrm{AF}}}\ +\ O(1)\qquad (a\downarrow 0).
\]
\begin{assumption}[AF/Mosco scaling framework (optional cross\,–\,check; not used in main chain)]\label{assump:AF-Mosco}
For each bounded $R\Subset\mathbb R^4$:
\begin{itemize}
  \item[(i)] Let $\mathcal H_{a,R}$ be the lattice OS/GNS space of the time-zero algebra supported in $R$ and $\mathcal H_R$ the continuum OS/GNS space on $R$. There are isometric embeddings
  \[
    I_{a,R}:\mathcal H_{a,R}\to\mathcal H_R,\qquad I_{a,R}[O^{(a)}]\ :=\ [E_{a,R}(O^{(a)})],
  \]
  where $E_{a,R}$ maps lattice loop/clover observables in $R$ to their polygonal/smeared continuum counterparts equivariantly (translations/rotations) and consistently in $a$. The embeddings intertwine time translations on the time-zero local core $\mathcal D_R$.
  \item[(ii)] The local OS/GNS Dirichlet forms
  \[
    \mathcal E_{a,R}(f)\;=\; \lim_{t\downarrow 0}\,\frac{1}{t}\,\langle f,(I-e^{-tH_{a,R}})f\rangle_{\mathcal H_R}
  \]
  Mosco-converge to a closed form $\mathcal E_R$ on a common dense core $\mathcal D_R$ independent of $a$, with sectorial bounds uniform in $a$. Moreover, the semigroups $\{e^{-t H_{a,R}}\}_{t>0}$ are uniformly bounded analytic on $L^2$ for $t$ in compact subsets of $(0,\infty)$, with constants independent of $a$.
\end{itemize}
In particular (by Theorem~\ref{thm:gap-persist-cont}), for each fixed $t>0$ one has $I_{a,R}e^{-tH_{a,R}}I_{a,R}^*\to e^{-tH_R}$ strongly and $(H_{a,R}-z)^{-1}\to (H_R-z)^{-1}$ strongly for each $z\in\mathbb C\setminus\mathbb R$.
\end{assumption}
\begin{theorem}[Gap persistence via NRC]\label{thm:gap-persist}
Let $(L_n,a_n)$ be a scaling sequence. If $e^{-tH_{L_n,a_n}}\to e^{-tH}$ in operator norm for all $t\ge 0$ and $\mathrm{spec}(H_{L_n,a_n})\subset\{0\}\cup[\gamma_0,\infty)$ uniformly in $n$, then $0$ is an isolated eigenvalue of $H$ and $\mathrm{spec}(H)\subset\{0\}\cup[\gamma_0,\infty)$.
\end{theorem}
% Optional Mosco/AF route moved to appendix; removed from main chain
\noindent\emph{Details (Riesz projection and openness of the gap).} Let $R_n(z)=(H_{L_n,a_n}-z)^{-1}$, $R(z)=(H-z)^{-1}$. Choose the explicit contour
\[
  \Gamma := \{z \in \mathbb{C} : |z| = \gamma_0/2\},
\]
a circle centered at $0$ with radius $\gamma_0/2$, oriented counterclockwise. Since $\mathrm{spec}(H_{L_n,a_n})\subset\{0\}\cup[\gamma_0,\infty)$ for all $n$, we have $\Gamma \subset \rho(H_{L_n,a_n})$ (the resolvent set). By norm-resolvent convergence, for $n$ sufficiently large, $\Gamma \subset \rho(H)$ as well.

The Riesz projections are
\[
  P_n := \frac{1}{2\pi i}\int_\Gamma R_n(z)\,dz, \quad P := \frac{1}{2\pi i}\int_\Gamma R(z)\,dz.
\]
Since $\Gamma$ separates $\{0\}$ from $[\gamma_0,\infty)$ and $\mathrm{spec}(H_{L_n,a_n})\cap(0,\gamma_0)=\varnothing$, we have $P_n = $ projection onto the eigenspace of $H_{L_n,a_n}$ at $0$, hence $\operatorname{rank} P_n = 1$ (the vacuum).

\begin{proof}[Details (Riesz projection and openness of the gap)]
By the resolvent estimate, for $z \in \Gamma$,
\[
  \|R_n(z) - R(z)\| \le \|R(z)\| \cdot \|I - P_n\| + \|R(z)\| \cdot \varepsilon_n \cdot \|R_n(z)\| \cdot \|(H_{L_n,a_n}+1)^{1/2}\|,
\]
where $\varepsilon_n \to 0$ as $n \to \infty$. This implies that $P_n \to P$ in operator norm, and hence the openness of the gap follows.

\subsubsection*{Holomorphic Functional Calculus and Projectors}
For any bounded holomorphic $f$ on an open set containing $\mathbb C\setminus\mathbb R$, the NRC bounds imply
\[
  \big\| f(H) - I_{a,L} f(H_{a,L}) I_{a,L}^* \big\|\ \to\ 0,
\]
by the Cauchy integral representation and the operator--norm convergence of resolvents on contours. In particular, Riesz projectors and spectral cutoffs converge in operator norm; this yields projector convergence and exponential clustering as stated in Theorem~\ref{thm:U12-exp-cluster}.
\]
where $\varepsilon_n \to 0$ is the graph-norm defect. Since $\operatorname{dist}(z,\mathbb{R}) = \gamma_0/2$ for all $z \in \Gamma$, we have $\|R_n(z)\|, \|R(z)\| \le 2/\gamma_0$. Thus
\[
  \|P_n - P\| \le \frac{|\Gamma|}{2\pi} \sup_{z \in \Gamma} \|R_n(z) - R(z)\| \le \frac{\gamma_0}{2} \cdot o(1) \to 0.
\]
Operator-norm convergence preserves rank in the limit: $\operatorname{rank} P = \lim_{n\to\infty} \operatorname{rank} P_n = 1$. Hence $0$ is an isolated eigenvalue of $H$ with one-dimensional eigenspace.
For the gap persistence, if $\lambda \in (0,\gamma_0)$ were in $\mathrm{spec}(H)$, then by lower semicontinuity of the spectrum under norm-resolvent convergence (Kato \cite{Kato1995}, Theorem IV.3.1), there would exist $\lambda_n \in \mathrm{spec}(H_{L_n,a_n})$ with $\lambda_n \to \lambda$. But this contradicts $\mathrm{spec}(H_{L_n,a_n}) \cap (0,\gamma_0) = \varnothing$. Therefore $\mathrm{spec}(H) \subset \{0\} \cup [\gamma_0,\infty)$.
\end{proof}

\subsection*{Coarse Interface and Dimension-Free Minorization}

\begin{lemma}[Coarse interface at fixed physical resolution]\label{lem:coarse-interface-construction}
Fix $\varepsilon\in(0,\varepsilon_0]$. Partition a physical slab of thickness $\approx \varepsilon$ intersecting $B_{R_*}$ by a cubic grid of side $\varepsilon$ along the reflection plane, and define the coarse interface variables as block holonomies/plaquette clovers per coarse cell. Let $\mathcal F_{\mathrm{int}}^{(\varepsilon)}$ be the $\sigma$\,–\,algebra they generate. Then $\mathcal F_{\mathrm{int}}^{(\varepsilon)}$ is independent of $a$ and has finite generated dimension $m(\varepsilon)=O(\varepsilon^{-3})$ depending only on $(R_*,\varepsilon)$. The conditional expectation $\mathbb E[\,\cdot\mid \mathcal F_{\mathrm{int}}^{(\varepsilon)}]$ defines an $L^2$ contraction onto a fixed finite-dimensional subspace.
\end{lemma}
\begin{proof}[Proof sketch]
For fixed physical resolution $\varepsilon$, the number of coarse cells intersecting the fixed ball $B_{R_*}$ scales like $O(\varepsilon^{-3})$ (a 3D grid on the reflection hyperplane). Each coarse variable is a measurable function of finitely many fine link variables in its cell (e.g., a fixed block holonomy/plaquette clover). Hence the $\sigma$--algebra generated by all coarse variables is generated by finitely many $G$--valued random variables, with count $m(\varepsilon)=O(\varepsilon^{-3})$ depending only on $(R_*,\varepsilon)$ and not on the fine lattice spacing $a$. Conditional expectation onto a sub-$\sigma$--algebra is an orthogonal projection in $L^2$, hence an $L^2$ contraction.
\end{proof}

\begin{lemma}[Cell geometry: choosing plaquette--disjoint links]\label{lem:cell-disjoint-links}
Fix $\varepsilon\in(0,\varepsilon_0]$ and an integer $m_\*\ge 1$. There exists $a_1=a_1(\varepsilon,m_\*,R_*)\in(0,\varepsilon]$ such that for all lattice spacings $a\in(0,a_1]$ and for every coarse cell $C$ in the $\varepsilon$--grid intersecting $B_{R_*}$, one can choose interface links $\ell_{C,1},\dots,\ell_{C,m_\*}$ inside $C$ with the following properties:
\begin{itemize}
  \item[(i)] (\textbf{Interior}) Every plaquette in the slab that contains $\ell_{C,j}$ is contained in the same coarse cell $C$.
  \item[(ii)] (\textbf{Plaquette--disjointness}) No plaquette in the slab contains two distinct links from $\{\ell_{C,1},\dots,\ell_{C,m_\*}\}$.
\end{itemize}
In particular, for distinct coarse cells $C\neq C'$ the plaquette neighborhoods of $\{\ell_{C,j}\}$ and $\{\ell_{C',j}\}$ are disjoint.
\end{lemma}
\begin{proof}
Fix $\varepsilon$ and a cell $C$. Let $\mathcal L(C)$ denote the finite set of interface links whose midpoint lies in $C$ and is at distance $\ge 2a$ from $\partial C$. Then for any $\ell\in \mathcal L(C)$, every plaquette incident to $\ell$ has diameter $O(a)$ and is therefore contained in $C$, proving (i).

Define a conflict graph on $\mathcal L(C)$ by joining $\ell\sim \ell'$ iff there exists a plaquette containing both $\ell$ and $\ell'$. Each link belongs to at most a fixed number $N_{\mathrm{plaq}}$ of plaquettes (depending only on the lattice dimension), and each plaquette contains at most $3$ other links besides $\ell$. Hence the maximum degree of this graph is bounded by a constant $\Delta_0:=3N_{\mathrm{plaq}}$ independent of $(a,\beta,L)$.

By a greedy selection, $\mathcal L(C)$ contains an independent set of size at least $|\mathcal L(C)|/(\Delta_0+1)$. Since $|\mathcal L(C)|\asymp (\varepsilon/a)^3$ for $a\downarrow 0$, we can choose $a_1(\varepsilon,m_\*,R_*)$ small enough so that $|\mathcal L(C)|\ge (\Delta_0+1)m_\*$ for every cell intersecting $B_{R_*}$. Then the independent set provides $\ell_{C,1},\dots,\ell_{C,m_\*}$ with property (ii). The final statement follows from (i): plaquettes incident to links chosen in $C$ are contained in $C$ and therefore cannot intersect those from a different cell $C'$.
\end{proof}

\begin{lemma}[Factorization for disjoint plaquette supports]\label{lem:plaquette-factorization}
Let $\mu_\beta$ be a finite-volume Wilson Gibbs measure with density proportional to $\exp\!\big(\beta\sum_{p\in P}\Re\Tr\,U_p\big)$ with respect to product Haar on link variables. Let $E_1,\dots,E_k$ be disjoint sets of links such that no plaquette contains links from two different $E_i$'s. Then the conditional law of $(U|_{E_1},\dots,U|_{E_k})$ given the complement $U|_{(E_1\cup\cdots\cup E_k)^c}$ factorizes:
\[
  \mu_\beta\big(dU|_{E_1\cup\cdots\cup E_k}\,\big|\,U|_{(E_1\cup\cdots\cup E_k)^c}\big)
  \ =\ \bigotimes_{i=1}^k \mu_\beta\big(dU|_{E_i}\,\big|\,U|_{(E_1\cup\cdots\cup E_k)^c}\big).
\]
Equivalently, for events $A_i$ measurable with respect to $U|_{E_i}$,
\[
  \mu_\beta\Big(\bigcap_{i=1}^k A_i\ \Big|\ U|_{(E_1\cup\cdots\cup E_k)^c}\Big)\ =\ \prod_{i=1}^k \mu_\beta\big(A_i\mid U|_{(E_1\cup\cdots\cup E_k)^c}\big).
\]
\end{lemma}
\begin{proof}
Write the Wilson action as $S_\beta(U)=\sum_{p\in P} s_p(U|_{\partial p})$, where each $s_p$ depends only on the four links of the plaquette $\partial p$. By hypothesis, the set of plaquettes that touch $E_i$ is disjoint from the set that touches $E_j$ for $i\neq j$. Hence we can decompose
\[
  S_\beta(U)\ =\ S_{\mathrm{rest}}(U|_{(E_1\cup\cdots\cup E_k)^c})\ +\ \sum_{i=1}^k S_i\big(U|_{E_i},\,U|_{(E_1\cup\cdots\cup E_k)^c}\big),
\]
where each $S_i$ depends on $U|_{E_i}$ and the fixed complement but not on $U|_{E_j}$ for $j\neq i$. Since the base measure is product Haar, the conditional density of $U|_{E_1\cup\cdots\cup E_k}$ given the complement is proportional to
\(
  \prod_{i=1}^k \exp(-S_i(U|_{E_i},\cdot)),
\)
which is a product of functions of the disjoint coordinates $U|_{E_i}$. This yields the stated factorization.
\end{proof}

\begin{lemma}[One--link refresh as a measure minorization (small ball)]\label{lem:one-link-ball-minorization}
Let $G=\mathrm{SU}(N)$ with Haar probability $\pi$. Fix $\beta\ge 0$ and $S\in G$. Consider the one--link conditional law
\[
  K_S(dU)\ :=\ Z_S(\beta)^{-1}\,\exp\!\big(\beta\,\Re\,\tr(US)\big)\,\pi(dU),
  \qquad Z_S(\beta):=\int_G \exp\!\big(\beta\,\Re\,\tr(VS)\big)\,\pi(dV).
\]
Let $r>0$ be below the injectivity radius and let $\nu_r$ be the normalized Haar restriction to the ball $B_G(e,r)$:
\(
  \nu_r(A):=\pi(A\cap B_G(e,r))/\pi(B_G(e,r)).
\)
Then for every Borel set $A\subseteq G$,
\[
  K_S(A)\ \ge\ e^{-2\beta N}\,\pi(B_G(e,r))\ \nu_r(A).
\]
\end{lemma}
\begin{proof}
Since $|\Re\,\tr(US)|\le N$ for all $U,S\in \mathrm{SU}(N)$, we have
\(
  e^{-\beta N}\le \exp(\beta\,\Re\,\tr(US))\le e^{\beta N}.
\)
In particular, $Z_S(\beta)\le e^{\beta N}$. Hence, for any Borel $A$,
\[
  K_S(A)
   \ =\ Z_S(\beta)^{-1}\int_A \exp(\beta\,\Re\,\tr(US))\,\pi(dU)
   \ \ge\ e^{-\beta N}\,Z_S(\beta)^{-1}\,\pi(A)
   \ \ge\ e^{-2\beta N}\,\pi(A).
\]
Applying this with $A$ replaced by $A\cap B_G(e,r)$ gives
\[
  K_S(A)\ \ge\ K_S(A\cap B_G(e,r))\ \ge\ e^{-2\beta N}\,\pi(A\cap B_G(e,r))
  \ =\ e^{-2\beta N}\,\pi(B_G(e,r))\,\nu_r(A),
\]
as claimed.
\end{proof}

\begin{lemma}[One--link fixed-radius ball mass near the polar maximizer (uniform in $\beta$)]\label{lem:one-link-ball-maximizer-unif}
Let $G=\mathrm{SU}(N)$ with Haar probability $\pi$ and a fixed bi-invariant Riemannian metric $d_G$. Fix a radius $r>0$. Then there exists a constant $p_{\mathrm{ball}}=p_{\mathrm{ball}}(N,r)>0$ such that for all $\beta\ge 0$ and all $S\in G$, the one--link conditional law $K_S$ satisfies
\[
  K_S\!\big(B_G(S^{-1},r)\big)\ \ge\ p_{\mathrm{ball}}.
\]
In particular, if $S\in B_G(e,\delta)$ with $\delta<r$, then $B_G(S^{-1},r)\subseteq B_G(e,r+\delta)$ and hence
\[
  K_S\!\big(B_G(e,r+\delta)\big)\ \ge\ p_{\mathrm{ball}}.
\]
\end{lemma}
\begin{proof}
Fix $r>0$ and let $f(g):=\Re\,\tr(g)$, a continuous class function on $G$ with unique maximum $f(e)=N$. By compactness, there exists $\tau\in(-N,N)$ such that the superlevel set
\[
  A_\tau\ :=\ \{g\in G:\ f(g)\ge \tau\}
\]
is contained in the ball $B_G(e,r)$ and has Haar mass $\pi(A_\tau)>0$.

Now fix $\beta\ge 0$ and $S\in G$. By Haar right-invariance and the change of variables $V:=US$, the law of $V$ under $K_S$ has density proportional to $e^{\beta f(V)}$ with respect to Haar, i.e. it is the exponential tilt of Haar by the one-dimensional statistic $f(V)$. Therefore
\[
  K_S\!\big(B_G(S^{-1},r)\big)\ =\ \mathbb P_\beta\big(V\in B_G(e,r)\big)\ \ge\ \mathbb P_\beta(V\in A_\tau),
\]
where $\mathbb P_\beta$ denotes the tilted law with density $\propto e^{\beta f(V)}$.

Since $A_\tau=\{f\ge \tau\}$ is a superlevel event of the tilt statistic, $\beta\mapsto \mathbb P_\beta(V\in A_\tau)$ is nondecreasing (likelihood-ratio order for exponential tilts). Hence
\[
  \mathbb P_\beta(V\in A_\tau)\ \ge\ \mathbb P_0(V\in A_\tau)\ =\ \pi(A_\tau)\ =:\ p_{\mathrm{ball}}(N,r)\ >\ 0,
\]
uniformly in $\beta\ge 0$ and $S\in G$. This proves the first claim; the second follows from the triangle inequality $d_G(e,S^{-1})=d_G(e,S)$ and the inclusion $B_G(S^{-1},r)\subseteq B_G(e,r+\delta)$ when $d_G(e,S)\le \delta$.
\end{proof}

\begin{lemma}[No $\beta$-uniform fixed-radius domination by the uniform ball law (one link)]\label{lem:no-unifball-doeblin-fixed-radius}
Let $G=\mathrm{SU}(N)$ with Haar probability $\pi$ and fix any radius $r>0$. There does not exist a constant $p>0$ such that for all $\beta\ge 0$ and all $S\in G$, the one-link conditional law $K_S$ satisfies the fixed-radius Doeblin minorization
\[
  K_S(\cdot)\ \ge\ p\,\nu_r(S^{-1}\cdot),
\]
where $\nu_r$ is normalized Haar restricted to $B_G(e,r)$.
\end{lemma}
\begin{proof}
Assume toward a contradiction that such $p>0$ exists for some fixed $r>0$.
By definition of $\nu_r$, the minorization implies that $K_S$ has a density $k_{\beta,S}$ with respect to Haar satisfying
\[
  k_{\beta,S}(U)\ \ge\ \frac{p}{\pi(B_G(e,r))}\qquad\text{for all }U\in B_G(S^{-1},r).
\]
By Haar invariance, after the change of variables $V:=US$ the law of $V$ under $K_S$ has density
\(
  \widetilde k_\beta(V)\propto e^{\beta\,\Re\tr(V)}
\)
independent of $S$. In particular the above bound implies
\[
  \widetilde k_\beta(V)\ \ge\ \frac{p}{\pi(B_G(e,r))}\qquad\text{for all }V\in B_G(e,r).
\]

Pick $V_0\in B_G(e,r)\setminus\{e\}$, so $\Re\tr(V_0)<\Re\tr(e)=N$ by strict maximality of $\Re\tr(\cdot)$ at $e$.
Choose $\delta>0$ so small that $\Re\tr(V)\ge N-\delta$ for all $V\in B_G(e,\delta)$ (continuity).
Then the normalizing constant satisfies
\[
  Z(\beta)\ :=\ \int_G e^{\beta\,\Re\tr(V)}\,\pi(dV)\ \ge\ \pi(B_G(e,\delta))\,e^{\beta(N-\delta)}.
\]
Hence
\[
  \widetilde k_\beta(V_0)\ =\ \frac{e^{\beta\,\Re\tr(V_0)}}{Z(\beta)}\ \le\ \pi(B_G(e,\delta))^{-1}\,e^{-\beta\big((N-\delta)-\Re\tr(V_0)\big)}.
\]
Since $(N-\delta)-\Re\tr(V_0)>0$ by choice of $\delta$, the right-hand side tends to $0$ as $\beta\to\infty$, contradicting the claimed uniform lower bound on $\widetilde k_\beta(V_0)$.
\end{proof}

\begin{lemma}[Coarse refresh probability bound]\label{lem:coarse-refresh}
For $\varepsilon\in(0,\varepsilon_0]$ fixed, there exist $c_{\mathrm{ref}}(\varepsilon,R_*,N)>0$, a probability law $\nu_{\varepsilon}$ on $G^{m(\varepsilon)}$, and $a_1\in(0,a_0]$ such that for all $a\in(0,a_1]$, all volumes $L$, all $\beta\ge 0$, and all boundary conditions outside the slab, the induced coarse one\,–\,tick kernel satisfies, for every coarse state $x$,
\[
  K_{\mathrm{int}}^{(\varepsilon)}(x,\cdot)\ \ge\ c_{\mathrm{ref}}(\varepsilon)\,\nu_{\varepsilon}(\cdot).
\]
In particular, $K_{\mathrm{int}}^{(\varepsilon)}$ has a parameter--uniform ``refresh'' component of weight $c_{\mathrm{ref}}(\varepsilon)$ on fixed slabs. Moreover, one may take $\nu_\varepsilon$ to be a product of $m(\varepsilon)$ identical one--cell laws of the form $\nu_{r_*}^{(*m_*)}$ (a fixed convolution power of the uniform small--ball law on $G$), with group constants $(r_*,m_*)$ as in Lemma~\ref{lem:ball-conv-lower}.
\end{lemma}
\begin{proof}[Proof sketch (in-text route via staple window $\to$ refresh $\to$ small-ball convolution)]
Fix $\varepsilon$ and a coarse cell decomposition of the interface inside $B_{R_*}$ as in Lemma~\ref{lem:coarse-interface-construction}. Fix $m_\*$ from Lemma~\ref{lem:ball-conv-lower}. For each coarse cell $C$, choose interface links $\ell_{C,1},\dots,\ell_{C,m_\*}$ as in Lemma~\ref{lem:cell-disjoint-links}. Define the coarse coordinate for $C$ to be the block holonomy
\[
  X_C\ :=\ U_{\ell_{C,1}}\cdots U_{\ell_{C,m_*}}\ \in\ G,
\]
so the coarse state space is $G^{m(\varepsilon)}$ with one coordinate per coarse cell.

\smallskip
\noindent\emph{Step 1 (uniform window event on fixed slabs).} Let $W_{\varepsilon_0}$ be the near--identity staple window event for the finite set of chosen links $\{\ell_{C,j}\}$ (Theorem~\ref{thm:staple-window}); since only finitely many links are involved (depending only on $(R_*,\varepsilon)$), we obtain a lower bound
\[
  \mathbb P(W_{\varepsilon_0})\ \ge\ p_{\rm win}(\varepsilon,R_*,a_0,G)\ >\ 0
\]
uniformly in $L$ and in the exterior boundary, for all $\beta\ge \beta_{\min}$.

\smallskip
\noindent\emph{Step 2 (one--link measure minorization on a fixed ball).} Fix any $r_*>0$ below the injectivity radius (later chosen to match Lemma~\ref{lem:ball-conv-lower}). For each selected link $\ell_{C,j}$, the conditional law of $U_{\ell_{C,j}}$ given all other variables in the slab is of the form $K_S$ above for an appropriate staple matrix $S$ (a product of neighboring links). Therefore Lemma~\ref{lem:one-link-ball-minorization} yields, for every Borel $A\subseteq G$,
\[
  \mathbb P\big(U_{\ell_{C,j}}\in A\mid \text{all other variables}\big)\ \ge\ e^{-2\beta N}\,\pi(B_G(e,r_*))\,\nu_{r_*}(A),
\]
uniformly in the boundary and the volume (with explicit $\beta$--dependence).

By Lemma~\ref{lem:cell-disjoint-links}, the links $\ell_{C,1},\dots,\ell_{C,m_*}$ have disjoint plaquette neighborhoods, so Lemma~\ref{lem:plaquette-factorization} implies their conditional law factors given the complement. Thus, within a fixed coarse cell $C$, the joint conditional law of $(U_{\ell_{C,1}},\dots,U_{\ell_{C,m_*}})$ dominates the product law $\nu_{r_*}^{\otimes m_*}$ with weight $(e^{-2\beta N}\pi(B_G(e,r_*)))^{m_*}$.

\smallskip
\noindent\emph{Step 3 (pushforward to the coarse variable and factorization across cells).} Pushing forward under the multiplication map $(g_1,\dots,g_{m_*})\mapsto g_1\cdots g_{m_*}$ shows that, on $W_{\varepsilon_0}$, the coarse coordinate $X_C$ has a refresh component dominating the $m_*$--fold convolution law $\nu_{r_*}^{(*m_*)}$ on $G$.
Moreover, Lemma~\ref{lem:cell-disjoint-links} ensures that plaquette neighborhoods of the selected links are disjoint across different cells. Applying Lemma~\ref{lem:plaquette-factorization} at the level of all selected links across all cells yields conditional factorization across cells. Therefore, for a suitable $c_{\rm ref}(\varepsilon)>0$,
\[
  K_{\mathrm{int}}^{(\varepsilon)}(x,\cdot)\ \ge\ c_{\rm ref}(\varepsilon)\,\nu_{\varepsilon}(\cdot),\qquad
  \nu_\varepsilon:=\big(\nu_{r_*}^{(*m_*)}\big)^{\otimes m(\varepsilon)}.
\]

\smallskip
\noindent\emph{Step 4 (extend to all $\beta\ge 0$).} For $\beta\in[0,\beta_{\min}]$, Proposition~\ref{prop:doeblin-interface} gives an explicit Haar minorization with a constant bounded below uniformly on that compact $\beta$--interval. Since $\nu_\varepsilon$ has a bounded density with respect to Haar (convolution powers on compact groups are continuous), this implies the same form of refresh minorization with a (possibly smaller) constant. Combining the two regimes yields the claim for all $\beta\ge 0$.
\end{proof}

\begin{lemma}[Coarse heat\,–\,kernel domination]\label{lem:coarse-hk-domination}
Let $G=\mathrm{SU}(N)$ and let $\nu_{\varepsilon}$ be the refresh law from Lemma~\ref{lem:coarse-refresh}. For fixed $\varepsilon\in(0,\varepsilon_0]$, there exist $t_0(\varepsilon)>0$ and $c_*(\varepsilon,N)>0$ such that
\[
  \nu_{\varepsilon}\ \ge\ c_*(\varepsilon,N)\, p_{t_0(\varepsilon)}
\]
as measures on $G^{m(\varepsilon)}$ (with respect to Haar), uniformly in $(\beta,L,a)$ on fixed slabs.
\end{lemma}
\begin{proof}
By Lemma~\ref{lem:coarse-refresh} we may take $\nu_\varepsilon$ to be a product of $m(\varepsilon)$ identical one--cell laws $\nu_{r_*}^{(*m_*)}$. Lemma~\ref{lem:ball-conv-lower} provides group constants $(r_*,m_*,t_0(G),c_*(G,r_*))$ such that the density of $\nu_{r_*}^{(*m_*)}$ dominates the heat kernel density $p_{t_0(G)}$ by the factor $c_*(G,r_*)$. Tensorizing over the $m(\varepsilon)$ coarse cells yields
\[
  \nu_\varepsilon\ \ge\ c_*(G,r_*)^{\,m(\varepsilon)}\, p_{t_0(G)}^{\otimes m(\varepsilon)}\ =:\ c_*(\varepsilon,N)\,p_{t_0(\varepsilon)},
\]
which is the stated domination (with $t_0(\varepsilon):=t_0(G)$).
\end{proof}

\begin{lemma}[Lumping/data\,–\,processing for $L^2$ contraction]\label{lem:lumping}
Let $K$ be a self\,–\,adjoint Markov operator on $L^2(\pi)$ and let $\Pi$ be the orthogonal projection onto a sub-$\sigma$\,–\,algebra $\mathcal G$. Then $\| K\Pi\|_{L^2\to L^2} \le \|K\|_{L^2\to L^2}$, and the restriction of $K$ to $\mathcal G$\,–\,measurable functions has operator norm bounded by that of the pushforward kernel on the quotient. In particular, contraction coefficients do not increase under coarse\,–\,graining.
\end{lemma}
\begin{proof}
Since $\Pi$ is an orthogonal projection on $L^2(\pi)$, $\|\Pi\|_{L^2\to L^2}=1$. Hence
\[
  \|K\Pi\|_{L^2\to L^2}\ \le\ \|K\|_{L^2\to L^2}\,\|\Pi\|_{L^2\to L^2}\ =\ \|K\|_{L^2\to L^2}.
\]
Moreover, the subspace of $\mathcal G$--measurable functions is exactly $\mathrm{Ran}(\Pi)$ and for $f$ $\mathcal G$--measurable one has $\Pi f=f$, so the restriction of $K$ to $\mathcal G$--measurable functions is represented by $\Pi K\Pi$. Therefore
\[
  \big\|K\big|_{L^2(\mathcal G)}\big\|_{L^2\to L^2}\ =\ \|\Pi K\Pi\|_{L^2\to L^2}\ \le\ \|K\|_{L^2\to L^2}.
\]
Identifying $\Pi K\Pi$ with the Markov operator of the pushforward kernel on the quotient yields the stated data--processing interpretation.
\end{proof}

\noindent We convert an $M$-step Doeblin minorization into an explicit $L^2$ spectral gap bound for $T$.

% Doeblin ⇒ L^2 spectral-gap for the transfer operator (drop-in lemma)
\begingroup
\newcommand{\inner}[2]{\left\langle #1,#2\right\rangle}
\newcommand{\norm}[1]{\left\lVert #1\right\rVert}

\noindent\textbf{Lemma (Doeblin $\Rightarrow$ $L^2$ spectral gap with explicit constants).}\label{lem:doeblin-L2-gap}
Let $(X,\mathcal F,\mu)$ be a probability space and let $T:L^2(\mu)\to L^2(\mu)$ be the integral operator of a Markov kernel $K$ (identifying $L^2_0(\mu)$ with the OS mean-zero sector $\mathcal H_0$). Assume: (i) $\mu$ is invariant for $K$; (ii) $T$ is $\mu$-reversible; (iii) (Doeblin in $M$ steps) there exist $M\in\mathbb N$, $\theta_*\in(0,1]$ and a probability $Q\ll\mu$ such that $K^{M}(x,\cdot)\ge \theta_* Q(\cdot)$ for $\mu$-a.e. $x$; and (iv) $dQ/d\mu\ge \sigma\in(0,1]$ a.e. Then
\[
  K^{M}(x,dy)\;=\;\theta_*\sigma\,\mu(dy)\;+\;\bigl(1-\theta_*\sigma\bigr)\,S(x,dy)
\]
for some $\mu$-reversible, $\mu$-invariant Markov kernel $S$, and
\[
  \bigl\|T^{M}\bigr\|_{\perp}\ \le\ 1-\theta_*\sigma,\qquad
  \|T\|_{\perp}\ \le\ \bigl(1-\theta_*\sigma\bigr)^{1/M},\qquad
  \mathrm{gap}_{L^2}(T)\ \ge\ 1-\bigl(1-\theta_*\sigma\bigr)^{1/M}.
\]
In particular, $-\log\|T\|_{\perp}\ \ge\ M^{-1}\log\!\big(1/(1-\theta_*\sigma)\big)$.

\begin{proof}
From $K^{M}\ge \theta_* Q\ge \theta_*\sigma\,\mu$ define
\[
  S(x,A)\ :=\ \frac{K^{M}(x,A)-\theta_*\sigma\,\mu(A)}{1-\theta_*\sigma}\quad(\theta_*\sigma<1),\qquad S(x,\cdot):=\mu(\cdot)\ (\theta_*\sigma=1).
\]
Then $S$ is a Markov kernel and $K^{M}=\theta_*\sigma\,\mu+(1-\theta_*\sigma)S$. Invariance and reversibility of $S$ follow by integrating in $x$ against $\mu$ and using invariance/reversibility of $K^{M}$ and $\mu$. Writing $\Pi_\mu f:=\int f\,d\mu$, we have the operator identity $T^{M}=\theta_*\sigma\,\Pi_\mu+(1-\theta_*\sigma)S$. On $L^2_0(\mu)$, $\Pi_\mu=0$, hence $\|T^{M}\|_{\perp}\le (1-\theta_*\sigma)\,\|S\|\le 1-\theta_*\sigma$. Self-adjointness of $T$ gives $\|T\|_{\perp}^{M}=\|T^{M}\|_{\perp}$, yielding the displayed bounds.
\end{proof}
\endgroup

\begin{proposition}[Coarse interface Doeblin]\label{prop:coarse-doeblin}
Fix $\varepsilon\in(0,\varepsilon_0]$. There exist $c_1(\varepsilon),c_0(\varepsilon)>0$ such that the coarse interface kernel satisfies the convex split
\[
  K_{\mathrm{int}}^{(\varepsilon)}\ \ge\ c_1(\varepsilon)\, P_{t_0(\varepsilon)}\,.
\]
Consequently, on $L_0^2$ one has
\[
  \|K_{\mathrm{int}}^{(\varepsilon)}\|\ \le\ 1- c_1(\varepsilon)\big(1-e^{-\lambda_1(G) t_0(\varepsilon)}\big).
\]
\end{proposition}
\begin{proof}[Proof sketch]
By Lemma~\ref{lem:coarse-refresh}, the coarse kernel admits a refresh component: for every coarse state $x$,
\[
  K_{\mathrm{int}}^{(\varepsilon)}(x,\cdot)\ \ge\ c_{\mathrm{ref}}(\varepsilon)\,\nu_\varepsilon(\cdot)
\]
for some probability law $\nu_\varepsilon$ on $G^{m(\varepsilon)}$. By Lemma~\ref{lem:coarse-hk-domination}, $\nu_\varepsilon\ge c_*(\varepsilon)\,p_{t_0(\varepsilon)}$, hence
\(
  K_{\mathrm{int}}^{(\varepsilon)}\ge c_1(\varepsilon)\,P_{t_0(\varepsilon)}
\)
with $c_1(\varepsilon):=c_{\mathrm{ref}}(\varepsilon)\,c_*(\varepsilon)$.
The $L^2_0$ bound is the standard convex-split contraction estimate: write
\(
  K=c_1 P_{t_0}+(1-c_1)\mathcal K
\)
with $\mathcal K$ Markov; then on mean-zero functions,
\(
  \|K f\|_2\le (1-c_1)\|f\|_2+c_1\|P_{t_0}f\|_2 \le (1-c_1(1-e^{-\lambda_1 t_0}))\|f\|_2.
\)
\end{proof}

\begin{lemma}[\boldmath $\beta$\,–\, and $L$\,–\,independent slab minorization after coarse refresh]\label{lem:beta-L-independent-minorization}
Fix a physical slab $B_{R_*}$ and maximal tick $a_0>0$. Fix a coarse interface resolution $\varepsilon\in(0,\varepsilon_0]$ in physical units and let $\mathcal F_{\mathrm{int}}^{(\varepsilon)}$ be the corresponding coarse interface $\sigma$--algebra (Lemma~\ref{lem:coarse-interface-construction}). Let $K_{\rm int}^{(a)}$ denote the induced one--tick Markov kernel on the coarse interface state space $G^{m(\varepsilon)}$ (equivalently, the pushforward of the microscopic interface kernel to the coarse variables).
Then there exist $t_0=t_0(\varepsilon,N)>0$ and $\theta_*=\theta_*(\varepsilon,R_*,a_0,N)\in(0,1]$ such that for all lattice spacings $a\in(0,a_0]$, all volumes $L$, and all $\beta\ge 0$,
\[
  K_{\rm int}^{(a)}\ \ge\ \theta_*\, P_{t_0}
\]
as kernels on the coarse interface space $G^{m(\varepsilon)}$ (product heat kernel $P_{t_0}$ on $G^{m(\varepsilon)}$). In particular, the constants are uniform in the lateral size $L$ and independent of $\beta$ (and $a$) on fixed slabs.
\end{lemma}
\begin{proof}
This is exactly Proposition~\ref{prop:coarse-doeblin} on the fixed coarse interface state space (noting that $K_{\rm int}^{(\varepsilon)}$ there is the same induced kernel we denote here by $K_{\rm int}^{(a)}$): set $\theta_*:=c_1(\varepsilon)$ and $t_0:=t_0(\varepsilon)$.
\end{proof}

\subsection*{UCIS closure target (uniform coarse interface smoothing on fixed physical slabs)}

\noindent The coarse-refresh route above records one approach to obtaining a $\beta$--uniform heat--kernel component on a fixed coarse interface. The following theorem packages the \emph{precise} statement that would close the interface minorization bottleneck in a scaling-consistent way (fixed physical thickness, hence a number of lattice ticks $M(a)\simeq 1/a$). We record it as a target closure theorem together with a lemma-by-lemma proof decomposition.

\begin{theorem}[Uniform coarse interface smoothing (UCIS; closure target)]\label{thm:ucis}
Fix a compact gauge group $G$, a fixed physical slab geometry (e.g. $B_{R_*}$ and maximal tick $a_0>0$ as above), a physical slab thickness $T_{\rm phys}>0$, and a coarse interface resolution $\varepsilon\in(0,\varepsilon_0]$ in physical units, so the coarse interface state space is $G^{m(\varepsilon)}$ (independent of the lattice spacing $a$). Let
\[
  M(a)\ :=\ \Big\lceil \frac{T_{\rm phys}}{a}\Big\rceil,\qquad a\in(0,a_0].
\]
Let $K_{\rm int}^{(a)}$ denote the induced one--tick interface Markov kernel on the coarse interface space $G^{m(\varepsilon)}$ (as in Lemma~\ref{lem:beta-L-independent-minorization}). Let $P_{t_0}(U,\cdot)$ denote the product heat--kernel Markov kernel on $G^{m(\varepsilon)}$ at time $t_0$, started at $U\in G^{m(\varepsilon)}$.

\smallskip
\noindent\textbf{Claim.} There exist constants $\theta_*>0$ and $t_0>0$ depending only on the fixed slab geometry and $(G,\varepsilon,T_{\rm phys})$ such that for all $a\in(0,a_0]$, all $\beta\ge 0$, all volumes/boundary conditions, and all $U\in G^{m(\varepsilon)}$,
\[
  K_{\rm int}^{(a)\,M(a)}(U,\cdot)\ \ge\ \theta_*\,P_{t_0}(U,\cdot)
\]
as kernels on $G^{m(\varepsilon)}$.
\end{theorem}

\begin{remark}[How UCIS is used]\label{rem:ucis-used}
UCIS is designed to replace the $\beta$--uniform one--step minorization input used to form the heat--kernel convex split (cf. Corollary~\ref{cor:hk-convex-split-explicit}) by a \emph{physically scaled} $M(a)$--step smoothing statement on a fixed physical slab. Once UCIS holds, the $L^2$ contraction and gap bookkeeping follow from standard functional analysis (e.g. Lemma~\ref{lem:doeblin-L2-gap}) applied to the $M(a)$--step kernel and then converted to a physical-time spectral gap lower bound.
\end{remark}

\begin{lemma}[UCIS-A: Coarse cell decomposition and disjoint support (target)]\label{lem:ucis-A-cell-decomp}
Fix $(G,\varepsilon,T_{\rm phys})$. There exists a coarse decomposition of the slab/interface region into finitely many cells (depending only on $\varepsilon$ and $T_{\rm phys}$) and, for each cell, a selection of interior link variables whose plaquette neighborhoods are disjoint across cells, such that each cell’s selected variables influence only that cell’s coarse outgoing interface increment.
\end{lemma}
\begin{proof}
Fix $\varepsilon\in(0,\varepsilon_0]$ and construct the coarse cells by the $\varepsilon$--grid along the reflection hyperplane as in Lemma~\ref{lem:coarse-interface-construction}; this yields finitely many cells intersecting the fixed spatial region (e.g. $B_{R_*}$), hence a finite coarse interface state space $G^{m(\varepsilon)}$.

Fix an integer $m_*\ge 1$ (to be chosen later, e.g. from Lemma~\ref{lem:ball-conv-lower}). For $a$ sufficiently small (depending on $\varepsilon$ and $m_*$), apply Lemma~\ref{lem:cell-disjoint-links} in each coarse cell $C$ to obtain links $\ell_{C,1},\dots,\ell_{C,m_*}$ whose plaquette neighborhoods are disjoint within $C$ and, for distinct cells $C\neq C'$, disjoint across cells.

Finally, by construction of the coarse interface variables (block holonomies/clovers per coarse cell; Lemma~\ref{lem:coarse-interface-construction}), the coarse outgoing increment associated to a cell is a measurable function of link variables inside that cell. Therefore the selected links in $C$ can influence only the coarse increment of $C$, and the family of selected link sets has disjoint plaquette neighborhoods across cells as claimed.
\end{proof}

\begin{lemma}[UCIS-B: Controlled Jacobian pushforward on a good set (target)]\label{lem:ucis-B-jacobian}
In the setting of Lemma~\ref{lem:ucis-A-cell-decomp}, for each cell there exists a smooth map from the selected interior link variables to the coarse increment in $G$ whose Jacobian is bounded below on a “good” set of configurations with positive Haar measure, with constants depending only on $(G,\varepsilon,T_{\rm phys})$ (independent of $(a,\beta,L)$ and boundary conditions).
\end{lemma}
\begin{proof}[Proof sketch / placement]
This is a finite-dimensional smooth change-of-variables statement on compact manifolds (after tree gauge). The output is a uniform pushforward lower bound of Haar mass onto a fixed-radius ball in $G$ conditional on the good set.
\end{proof}

\begin{lemma}[UCIS-C: Single-cell $\beta$-uniform ball minorization (target)]\label{lem:ucis-C-cell-ball}
For each coarse cell, there exist $r_*>0$ and $p_*>0$ depending only on $(G,\varepsilon,T_{\rm phys})$ such that for all boundary conditions and all $\beta\ge 0$, the conditional law of the cell’s coarse increment admits the measure minorization
\[
  K_{\mathrm{cell}}(\cdot)\ \ge\ p_*\,\nu_{r_*}(\cdot),
\]
where $\nu_{r_*}$ is normalized Haar restricted to $B_G(\mathbf 1,r_*)$. In particular,
\(
  \mathbb P(\Delta_{\mathrm{cell}}\in B_G(\mathbf 1,r_*))\ge p_*.
\)
\end{lemma}
\begin{proof}[Proof sketch / placement]
This is the analytic core: a \emph{$\beta$--uniform} small-ball \emph{measure} minorization for a finite-dimensional Wilson conditional law on a cell.

\smallskip
\noindent\textbf{Important nuance.} For a single link, one can obtain $\beta$--uniform \emph{mass} in a fixed-radius ball around the polar maximizer (Lemma~\ref{lem:one-link-ball-maximizer-unif}), but a $\beta$--uniform \emph{domination by the \emph{uniform} ball law} at fixed radius is much stronger (it requires controlling the minimum of the density on the whole ball, which can be exponentially small at the boundary as $\beta\to\infty$). The existing scale--adapted one--link minorization (Lemma~\ref{lem:SU(N)-refresh}) produces a \emph{shrinking} ball of radius $\asymp (1+\beta)^{-1/2}$ with explicit weight, which is compatible with weak-coupling Gaussian scaling but does not directly yield a fixed-radius, $\beta$--uniform ball reference.

\smallskip
\noindent\textbf{What is needed for UCIS-C.} One needs an additional \emph{cell-level} refresh/smoothing mechanism (e.g. a genuinely $\beta$--uniform window + pulse sandwich, or a physically scaled multi-step smoothing statement) that upgrades local mass near maximizers into a fixed-radius minorization \emph{for the coarse increment}, uniformly in boundary conditions.
\end{proof}

\begin{lemma}[UCIS-D: Cell factorization $\Rightarrow$ product small-ball component (target)]\label{lem:ucis-D-product-ball}
Assuming Lemmas~\ref{lem:ucis-A-cell-decomp} and \ref{lem:ucis-C-cell-ball}, the joint conditional law of the coarse outgoing interface increments admits the product minorization
\[
  K_{\mathrm{int}}^{(a)}(U,\cdot)\ \ge\ p_*^{\,m(\varepsilon)}\,\nu_{r_*}^{\otimes m(\varepsilon)}(U^{-1}\cdot),
\]
uniformly in $(a,\beta,L)$ on fixed slabs. In particular, the joint law has a product small-ball component on $G^{m(\varepsilon)}$ with weight at least $p_*^{\,m(\varepsilon)}$.
\end{lemma}
\begin{proof}
Fix a coarse cell decomposition as in Lemma~\ref{lem:ucis-A-cell-decomp}. For each cell $C$, let $E_C$ denote the set of link variables used to define the cell increment (e.g. the selected links $\{\ell_{C,1},\dots,\ell_{C,m_*}\}$). By Lemma~\ref{lem:ucis-A-cell-decomp} and Lemma~\ref{lem:cell-disjoint-links}, the families $\{E_C\}$ are disjoint and no plaquette contains links from two distinct $E_C$’s. Therefore Lemma~\ref{lem:plaquette-factorization} implies that, conditional on the complement variables, the joint conditional law on $\prod_C E_C$ factorizes across cells.

By Lemma~\ref{lem:ucis-C-cell-ball}, in each cell the conditional law of the coarse increment dominates $p_*\,\nu_{r_*}$ (in increment coordinates, i.e. after translating by the incoming coarse state). Since factorization is exact and the pushforward under the (cellwise) coarse increment map preserves tensor products, the joint conditional law of the coarse increments dominates $p_*^{m(\varepsilon)}\,\nu_{r_*}^{\otimes m(\varepsilon)}(U^{-1}\cdot)$. This yields the stated product minorization.
\end{proof}

\begin{lemma}[UCIS-E: Product small-ball $\Rightarrow$ heat-kernel minorization (target)]\label{lem:ucis-E-ball-to-hk}
Assume that a Markov kernel $K$ on $G^{m(\varepsilon)}$ satisfies the product small-ball minorization
\[
  K(U,\cdot)\ \ge\ \kappa\,\nu_{r_*}^{\otimes m(\varepsilon)}(U^{-1}\cdot)
\]
for all $U\in G^{m(\varepsilon)}$, for some $\kappa\in(0,1]$ and some $r_*>0$ below the injectivity radius. Then there exist an integer $m_*\ge 1$, a time $t_0>0$, and a constant $\theta_*>0$ depending only on $(G,m(\varepsilon),r_*)$ such that
\[
  K^{m_*}(U,\cdot)\ \ge\ \theta_*\,P_{t_0}(U,\cdot)
\]
for all $U\in G^{m(\varepsilon)}$, where $P_{t_0}$ is the product heat-kernel kernel on $G^{m(\varepsilon)}$.
\end{lemma}
\begin{proof}
Define the (left-translation) product small-ball Markov kernel $R$ on $G^{m(\varepsilon)}$ by
\[
  R(U,A)\ :=\ \nu_{r_*}^{\otimes m(\varepsilon)}(U^{-1}A),\qquad U\in G^{m(\varepsilon)}.
\]
Then $R$ is convolution by $\nu_{r_*}^{\otimes m(\varepsilon)}$ (componentwise), and the hypothesis is exactly $K\ge \kappa R$ as positive kernels. By positivity and induction, for any $n\ge 1$,
\[
  K^{n}\ \ge\ \kappa^{n}\,R^{n}.
\]
Because $R$ is convolution by $\nu_{r_*}^{\otimes m(\varepsilon)}$, we have
\(
  R^{n}(U,\cdot) = (\nu_{r_*}^{\otimes m(\varepsilon)})^{(*n)}(U^{-1}\cdot)
 = (\nu_{r_*}^{(*n)})^{\otimes m(\varepsilon)}(U^{-1}\cdot).
\)
Apply Lemma~\ref{lem:ball-conv-lower} on $G$ to obtain $m_*=m_*(G)$, $t_0=t_0(G)>0$, and $c_*=c_*(G,r_*)>0$ such that
\[
  \nu_{r_*}^{(*m_*)}\ \ge\ c_*\,p_{t_0}
\]
as measures on $G$. Tensorizing over $m(\varepsilon)$ coordinates yields
\[
  (\nu_{r_*}^{(*m_*)})^{\otimes m(\varepsilon)}\ \ge\ c_*^{\,m(\varepsilon)}\,p_{t_0}^{\otimes m(\varepsilon)},
\]
equivalently $R^{m_*}\ge c_*^{m(\varepsilon)} P_{t_0}$. Combining with $K^{m_*}\ge \kappa^{m_*} R^{m_*}$ gives the desired bound with $\theta_*:=\kappa^{m_*}c_*^{m(\varepsilon)}$.
\end{proof}

\begin{proof}[Proof roadmap for Theorem~\ref{thm:ucis}]
Assuming Lemmas~\ref{lem:ucis-A-cell-decomp}--\ref{lem:ucis-E-ball-to-hk}, one obtains a uniform product small-ball component for the coarse outgoing interface after a fixed physical thickness $T_{\rm phys}$, hence after $M(a)=\lceil T_{\rm phys}/a\rceil$ lattice ticks. Lemma~\ref{lem:ucis-E-ball-to-hk} upgrades this to a product heat-kernel component with constants independent of $(a,\beta,L)$, yielding the stated minorization $K_{\rm int}^{(a)\,M(a)}\ge \theta_* P_{t_0}$.
\end{proof}

\begin{corollary}[UCIS $\Rightarrow$ fixed-physical-time $L^2$ contraction (bookkeeping)]\label{cor:ucis-L2-contraction}
Assume UCIS (Theorem~\ref{thm:ucis}) with constants $(\theta_*,t_0)$ on the coarse interface space $G^{m(\varepsilon)}$. Define
\[
  q_{\rm phys}\ :=\ (1-\theta_*)+\theta_* e^{-\lambda_1(G)\,t_0}\ =\ 1-\theta_*\big(1-e^{-\lambda_1(G)\,t_0}\big)\ \in(0,1).
\]
Then for every $a\in(0,a_0]$ and every mean-zero $f\in L^2(G^{m(\varepsilon)},\pi^{\otimes m(\varepsilon)})$,
\[
  \big\|K_{\rm int}^{(a)\,M(a)} f\big\|_{2}\ \le\ q_{\rm phys}\,\|f\|_{2},
\]
uniformly in $(\beta,L)$ on fixed slabs.
\end{corollary}
\begin{proof}
By UCIS, $K_{\rm int}^{(a)\,M(a)}=\theta_* P_{t_0}+(1-\theta_*)\mathcal K$ for some Markov kernel $\mathcal K$ (Nummelin split). On mean-zero functions, $\|P_{t_0}\|=e^{-\lambda_1(G)t_0}$ and $\|\mathcal K\|\le 1$, giving the displayed contraction factor.
\end{proof}

\medskip
\begin{theorem}[Scaling--window UCIS (UCIS$_{\rm SW}$; physically scaled multi--step smoothing; target)]\label{thm:ucis-sw}
Fix $(G,\varepsilon,T_{\rm phys})$ as in Theorem~\ref{thm:ucis} and set $M(a)=\lceil T_{\rm phys}/a\rceil$. Let $\beta=\beta(a)$ be a monotone schedule with $\beta(a)\to\infty$ and assume the \emph{scaling window}
\begin{equation}\label{eq:ucis-sw-window}
  \beta(a)\ \le\ C_{\rm SW}\,M(a)\qquad\text{for all }a\in(0,a_0]
\end{equation}
for some constant $C_{\rm SW}<\infty$.

Assume the coarse cell decomposition/disjointness Lemma~\ref{lem:ucis-A-cell-decomp}. Assume moreover that the \emph{near--identity staple window} holds on the fixed slab in the weak--coupling regime, as in Theorem~\ref{thm:staple-window}: there exist $\varepsilon_0>0$ and $p_{\mathrm{win}}>0$ such that whenever $\beta\ge \beta_{\min}$, the window event $W_{\varepsilon_0}$ has probability at least $p_{\mathrm{win}}$ uniformly in volume/boundary data. (Since $\beta(a)\to\infty$, this applies for all sufficiently small $a$ along the schedule.)
Finally, assume that under the window $W_{\varepsilon_0}$ the single--link conditional updates used to build the coarse increments satisfy the scale--adapted refresh minorization of Lemma~\ref{lem:single-link-refresh} (or its $G$--version Lemma~\ref{lem:g-one-link-refresh}).

Then there exist constants $\theta_*,t_0>0$ depending only on $(G,\varepsilon,T_{\rm phys},C_{\rm SW})$ such that along the schedule $a\downarrow 0$,
\[
  K_{\rm int}^{(a)\,M(a)}(U,\cdot)\ \ge\ \theta_*\,P_{t_0}(U,\cdot)
\]
as kernels on the coarse interface space $G^{m(\varepsilon)}$, uniformly in $U$ and in the volume/boundary data on fixed slabs.
\end{theorem}

\begin{remark}[Why UCIS$_{\rm SW}$ is the right replacement target for UCIS--C]\label{rem:ucis-sw-why}
Lemma~\ref{lem:no-unifball-doeblin-fixed-radius} shows that a one--step \emph{fixed--radius} $\beta$--uniform Doeblin lower bound cannot hold even for a single link conditional. The scale--adapted refresh lemmas instead naturally yield a small--step component at radius $\asymp \beta^{-1/2}$, which requires $\gtrsim \beta$ steps to accumulate $O(1)$ dispersion. The window \eqref{eq:ucis-sw-window} precisely ensures $M(a)\asymp 1/a$ is large enough to contain $\Theta(\beta(a))$ effective diffusion steps inside a fixed physical slab.
\end{remark}

\begin{lemma}[Central-mixture binomial expansion]\label{lem:central-mixture-binomial}
Let $G$ be a compact group. Let $\mu$ be a \emph{central} (conjugation--invariant) probability measure on $G$, $\eta$ any probability measure on $G$, and $\alpha\in[0,1]$. Then for every $n\in\mathbb N$,
\[
  \big(\alpha \mu+(1-\alpha)\eta\big)^{(*n)}
  \ =\ \sum_{k=0}^{n}\binom{n}{k}\,\alpha^{k}(1-\alpha)^{n-k}\,\mu^{(*k)}*\eta^{(*(n-k))}.
\]
\end{lemma}
\begin{proof}
Expand the $n$--fold convolution product and group the terms by the number $k$ of factors equal to $\mu$. Since $\mu$ is central, $\mu*\eta=\eta*\mu$, so each term with $k$ copies of $\mu$ and $n-k$ copies of $\eta$ collapses to the same measure $\mu^{(*k)}*\eta^{(*(n-k))}$. Summing the binomial coefficients yields the identity.
\end{proof}

\begin{lemma}[Ball-walk diffusive smoothing on compact $G$]\label{lem:ballwalk-diffusive}
Let $G$ be a compact connected Lie group with a fixed bi--invariant Riemannian metric and Haar probability $\pi$. For $r>0$ write $\nu_r$ for the uniform law on the geodesic ball $B_G(e,r)$. There exist constants $r_0=r_0(G)\in(0,1)$ and $C_{\mathrm{mix}},c_{\mathrm{mix}}>0$ depending only on $G$ such that for every $r\in(0,r_0]$ and every integer
\[
  n\ \ge\ C_{\mathrm{mix}}\,r^{-2},
\]
the $n$--fold convolution has a global Haar minorization
\[
  \nu_r^{(*n)}\ \ge\ c_{\mathrm{mix}}\,\pi
\]
as measures on $G$.
\end{lemma}
\begin{proof}
Let $p_r:=d\nu_r/d\pi=\pi(B_G(e,r))^{-1}\,\mathbf 1_{B_G(e,r)}$ and set $\Omega_r:=B_G(e,r)$. Let $\rho_r$ be the left--invariant word metric on $G$ associated to $\Omega_r$ (Section~3 of \cite{HebischSaloffCoste1993}). Then $p_r$ is a bounded symmetric probability density, supported in $\Omega_r$ and bounded below on $\Omega_r$ by $\pi(B_G(e,r))^{-1}>0$. Since $G$ is compact, its volume growth in the word metric is polynomial of order $D=0$ (cf.\ the discussion around Guivarc'h's theorem in \cite[\S3]{HebischSaloffCoste1993}).

\smallskip
By \cite[Theorem~5.1, lower bound (15)]{HebischSaloffCoste1993} (specialized to $D=0$), there exist constants $c_1,C_1,C_2>0$ such that for all $n\in\mathbb N$ and all $g\in G$ with $\rho_r(g)\le n/C_2$,
\[
  p_r^{(n)}(g)\ \ge\ c_1\,\exp\!\Big(-C_1\,\frac{\rho_r(g)^2}{n}\Big).
\]
Moreover $\sup_{g\in G}\rho_r(g)\le C_{\mathrm{diam}}\,r^{-1}$ for some $C_{\mathrm{diam}}=C_{\mathrm{diam}}(G)$: choose a rectifiable path from $e$ to $g$ of length $\le \operatorname{diam}(G)$ and partition it into $\le C_{\mathrm{diam}}r^{-1}$ segments of $\rho_G$--length $\le r$, so the successive increments lie in $\Omega_r$ and multiply to $g$.

\smallskip
Fix $r_0\in(0,1)$ and set
\[
  C_{\mathrm{mix}}\ :=\ \max\{C_2C_{\mathrm{diam}},\ 1\},\qquad
  c_{\mathrm{mix}}\ :=\ c_1\,\exp\!\Big(-C_1\,\frac{C_{\mathrm{diam}}^{\,2}}{C_{\mathrm{mix}}}\Big)\ \in(0,1).
\]
If $r\in(0,r_0]$ and $n\ge C_{\mathrm{mix}}r^{-2}$, then $\rho_r(g)\le C_{\mathrm{diam}}r^{-1}\le n/C_2$ and $\rho_r(g)^2/n\le C_{\mathrm{diam}}^{\,2}/C_{\mathrm{mix}}$ for all $g\in G$. Hence $p_r^{(n)}(g)\ge c_{\mathrm{mix}}$ for all $g$, and thus $\nu_r^{(*n)}(dg)=p_r^{(n)}(g)\,\pi(dg)\ge c_{\mathrm{mix}}\,\pi(dg)$.
\end{proof}

\begin{lemma}[Scale--adapted small--step convolution $\Rightarrow$ fixed--time heat--kernel minorization (Track~B core)]\label{lem:scaled-ball-to-hk}
Let $G$ be a compact connected Lie group with bi--invariant metric and Haar probability $\pi$. Fix constants $\alpha_0,\kappa_0\in(0,1]$. There exist constants $c_0,C_0,t_0>0$ depending only on $(G,\alpha_0,\kappa_0)$ such that the following holds.

For each $\beta\ge 1$ let $\nu_\beta$ be a probability measure on $G$ satisfying a scale--adapted small--ball lower bound
\[
  \nu_\beta\ \ge\ \alpha_0\,\mathrm{Unif}\!\big(B_G(e,\kappa_0/\sqrt{\beta})\big)
\]
for some fixed constants $\alpha_0,\kappa_0>0$ independent of $\beta$. Then for every integer $n\ge C_0\,\beta$ one has
\[
  \nu_\beta^{(*n)}\ \ge\ c_0\,p_{t_0}
\]
as measures on $G$, where $p_{t_0}$ is the heat kernel at time $t_0$.
\end{lemma}
\begin{proof}
Let $r_\beta:=\kappa_0/\sqrt{\beta}$ and write $\mu_\beta:=\mathrm{Unif}(B_G(e,r_\beta))$. The hypothesis $\nu_\beta\ge \alpha_0\mu_\beta$ implies the convex decomposition
\[
  \nu_\beta\ =\ \alpha_0\,\mu_\beta\ +\ (1-\alpha_0)\,\eta_\beta
\]
for some probability measure $\eta_\beta$.

\smallskip
\textbf{Step 1: diffusive Haar minorization for the ball walk.}
By Lemma~\ref{lem:ballwalk-diffusive}, for $r_\beta\le r_0(G)$ and
\[
  n_{\mathrm{mix}}(\beta)\ :=\ \left\lceil C_{\mathrm{mix}}\,r_\beta^{-2}\right\rceil
  \ =\ \left\lceil C_{\mathrm{mix}}\,\kappa_0^{-2}\,\beta\right\rceil,
\]
one has $\mu_\beta^{(*n_{\mathrm{mix}}(\beta))}\ge c_{\mathrm{mix}}\,\pi$. In particular, for every $k\ge n_{\mathrm{mix}}(\beta)$,
\[
  \mu_\beta^{(*k)}\ =\ \mu_\beta^{(*n_{\mathrm{mix}}(\beta))}*\mu_\beta^{(*(k-n_{\mathrm{mix}}(\beta)))}\ \ge\ c_{\mathrm{mix}}\,\pi*\mu_\beta^{(*(k-n_{\mathrm{mix}}(\beta)))}\ =\ c_{\mathrm{mix}}\,\pi.
\]

\smallskip
\textbf{Step 2: binomial expansion + absorption by Haar.}
Since $\mu_\beta$ is central (the ball is conjugation--invariant for a bi--invariant metric), Lemma~\ref{lem:central-mixture-binomial} gives, for every $n$,
\[
  \nu_\beta^{(*n)}
  \ =\ \sum_{k=0}^{n}\binom{n}{k}\,\alpha_0^{k}(1-\alpha_0)^{n-k}\,\mu_\beta^{(*k)}*\eta_\beta^{(*(n-k))}.
\]
For each $k\ge n_{\mathrm{mix}}(\beta)$, Step~1 yields $\mu_\beta^{(*k)}\ge c_{\mathrm{mix}}\pi$, hence
\[
  \mu_\beta^{(*k)}*\eta_\beta^{(*(n-k))}\ \ge\ c_{\mathrm{mix}}\,\pi*\eta_\beta^{(*(n-k))}\ =\ c_{\mathrm{mix}}\,\pi.
\]
Therefore
\[
  \nu_\beta^{(*n)}\ \ge\ c_{\mathrm{mix}}\,\mathbb P\!\left(\mathrm{Bin}(n,\alpha_0)\ge n_{\mathrm{mix}}(\beta)\right)\,\pi.
\]
Choose $C_0$ large enough (depending only on $(\alpha_0,\kappa_0,C_{\mathrm{mix}})$) so that for all $\beta\ge 1$ and all integers $n\ge C_0\beta$,
\[
  \mathbb P\!\left(\mathrm{Bin}(n,\alpha_0)\ge n_{\mathrm{mix}}(\beta)\right)\ \ge\ \tfrac12,
\]
e.g. by a Chernoff/Hoeffding bound since $\mathbb E[\mathrm{Bin}(n,\alpha_0)]=\alpha_0 n\ge (\alpha_0 C_0)\beta$ and $n_{\mathrm{mix}}(\beta)\le (C_{\mathrm{mix}}\kappa_0^{-2}+1)\beta$.
Then for all such $n$,
\[
  \nu_\beta^{(*n)}\ \ge\ \tfrac12\,c_{\mathrm{mix}}\,\pi.
\]

\smallskip
\textbf{Step 3: Haar minorization $\Rightarrow$ heat--kernel minorization.}
Fix any $t_0>0$ (depending only on $G$) and write $M_{t_0}:=\sup_{g\in G} p_{t_0}(g)<\infty$. Since $\pi$ has constant density $1$ w.r.t.\ itself, one has $1\ge M_{t_0}^{-1}p_{t_0}(g)$ for all $g$, hence $\pi\ge M_{t_0}^{-1}p_{t_0}$. Setting $c_0:=\tfrac12\,c_{\mathrm{mix}}\,M_{t_0}^{-1}$ yields the claimed bound $\nu_\beta^{(*n)}\ge c_0 p_{t_0}$.
\end{proof}

\begin{proof}[Proof of Theorem~\ref{thm:ucis-sw}]
Fix $(G,\varepsilon,T_{\rm phys})$ and let $m:=m(\varepsilon)$ be the number of coarse interface cells (independent of $a$). Let $M(a)=\lceil T_{\rm phys}/a\rceil$ and write $\beta=\beta(a)$.

\smallskip
\textbf{Step 0 (cell decomposition and disjoint supports).}
By Lemma~\ref{lem:ucis-A-cell-decomp}, for each sufficiently small $a$ the slab/interface region admits a decomposition into the $m$ coarse cells $C\in\mathcal C(\varepsilon)$ together with (for each cell) a selected set of interior link variables $E_C$ such that:
(i) $E_C$ and $E_{C'}$ have disjoint plaquette neighborhoods whenever $C\neq C'$, and
(ii) the coarse outgoing increment for cell $C$ depends only on variables in $E_C$ (and boundary data fixed outside the slab).
Denote the coarse interface state by $U=(U_C)_{C\in\mathcal C(\varepsilon)}\in G^m$.

\smallskip
\textbf{Step 1 (one-tick kernel factorizes across cells).}
Fix a tick $a$ and consider one step of the induced coarse interface kernel $K_{\rm int}^{(a)}$.
Because no plaquette contains links from two distinct $E_C$'s, Lemma~\ref{lem:plaquette-factorization} implies that, conditional on all variables outside $\bigcup_C E_C$, the joint conditional law on $\prod_C E_C$ factorizes across cells. Since, by Lemma~\ref{lem:ucis-A-cell-decomp}, the outgoing coarse increment of cell $C$ is a measurable function of the variables in $E_C$ alone, the pushforward to the $m$ cell increments preserves this product structure. Hence the induced one-tick kernel on coarse variables factorizes:
\[
  K_{\rm int}^{(a)}(U,dV)\ =\ \bigotimes_{C\in\mathcal C(\varepsilon)} K_{\rm cell}^{(a)}(U_C,dV_C),
\]
for some Markov kernels $K_{\rm cell}^{(a)}$ on $G$ (uniform in volume/boundary on fixed slabs by construction).

\smallskip
\textbf{Step 2 (cell kernels are convolution kernels with a scale-adapted small-ball component).}
Work in increment coordinates $\Delta_C:=U_C^{-1}V_C\in G$ for each cell. By left-translation invariance (Haar invariance and the gauge-covariance of the construction of coarse increments), each cell kernel is a right-convolution kernel:
\[
  K_{\rm cell}^{(a)}(U_C, A)\ =\ \nu_{\beta(a)}\big(U_C^{-1}A\big)
\]
for some probability measure $\nu_{\beta(a)}$ on $G$ (the one-tick increment law for a representative cell).
We now verify the required scale--adapted small--ball component using the staple window and one--link refresh lemmas.

\smallskip
\emph{(i) Window event.} Let $W_{\varepsilon_0}$ be the near--identity staple window event for the finite family of link updates used (within one tick) to produce the outgoing coarse increments in all coarse cells (this family is finite because $m(\varepsilon)<\infty$). By Theorem~\ref{thm:staple-window}, there exist $\varepsilon_0>0$ and $p_{\mathrm{win}}>0$ (depending only on the slab geometry and $G$) such that for all $\beta\ge \beta_{\min}$ the event $W_{\varepsilon_0}$ has probability at least $p_{\mathrm{win}}$ uniformly in volume and boundary data. Since $\beta(a)\to\infty$, this holds for all sufficiently small $a$ along the schedule.

\smallskip
\emph{(ii) One--link scale--adapted refresh under the window.} On the event $W_{\varepsilon_0}$, each refreshed link $\ell$ used in the coarse increment construction has staple product $H_\ell\in B_G(e,\varepsilon_0)$. Hence Lemma~\ref{lem:single-link-refresh}(a) (or Lemma~\ref{lem:g-one-link-refresh}) yields constants $\kappa_0\in(0,1]$ and $p_{\mathrm{ref}}>0$, independent of $(\beta,L,\text{boundary})$, such that conditional on $W_{\varepsilon_0}$ one has a scale--adapted one--link minorization by Haar on $B_G(e,\kappa_0/\sqrt{\beta})$.

\smallskip
\emph{(iii) From refreshed links to a cell increment law.} By Lemma~\ref{lem:ucis-A-cell-decomp}, within each cell $C$ we may choose the refreshed links so their plaquette neighborhoods are disjoint (and disjoint across cells). Lemma~\ref{lem:plaquette-factorization} then yields conditional independence across these refreshed links, so the joint law has a product small--ball component of weight at least $p_{\mathrm{ref}}^{\,b}$ for some fixed $b=b(\varepsilon)$ (the number of refreshed links per cell, independent of $a$). Pushing forward through the (cellwise) coarse increment map and using that multiplication in $G$ is smooth/Lipschitz on a normal neighborhood, we obtain a scale--adapted ball minorization for the one-tick increment law $\nu_{\beta(a)}$:
\[
  \nu_{\beta(a)}\ \ge\ \alpha_0\,\mathrm{Unif}\!\big(B_G(e,\kappa_1/\sqrt{\beta(a)})\big),
\]
for some constants $\alpha_0:=p_{\mathrm{win}}\,p_{\mathrm{ref}}^{\,b}>0$ and $\kappa_1\in(0,\kappa_0]$ depending only on $(G,\varepsilon,T_{\rm phys})$, uniformly for all sufficiently small $a$ along the schedule.

\smallskip
\textbf{Step 3 (diffusive smoothing in time for each cell).}
Apply Lemma~\ref{lem:scaled-ball-to-hk} to $\nu_{\beta(a)}$ and $n=M(a)$. (Note that the proof of Lemma~\ref{lem:scaled-ball-to-hk} uses the diffusive ball-walk minorization Lemma~\ref{lem:ballwalk-diffusive} together with the central-mixture expansion Lemma~\ref{lem:central-mixture-binomial}.)
Using the scaling window \eqref{eq:ucis-sw-window}, choose $a$ sufficiently small so that $M(a)\ge C_0\,\beta(a)$ where $C_0$ is the constant from Lemma~\ref{lem:scaled-ball-to-hk} (depending only on $(G,\alpha_0,\kappa_1)$). Then there exist constants $c_0,t_0>0$ depending only on $(G,\alpha_0,\kappa_1)$ such that
\[
  \nu_{\beta(a)}^{(*M(a))}\ \ge\ c_0\,p_{t_0}.
\]
Equivalently, for each cell kernel,
\[
  \big(K_{\rm cell}^{(a)}\big)^{M(a)}(U_C,\cdot)\ =\ \nu_{\beta(a)}^{(*M(a))}(U_C^{-1}\cdot)\ \ge\ c_0\,p_{t_0}(U_C^{-1}\cdot).
\]

\smallskip
\textbf{Step 4 (tensorize across the coarse interface).}
Since the one-tick coarse kernel factorizes across cells (Step 1), its $M(a)$-step iterate also factorizes:
\[
  K_{\rm int}^{(a)\,M(a)}(U,\cdot)\ =\ \bigotimes_{C}\big(K_{\rm cell}^{(a)}\big)^{M(a)}(U_C,\cdot).
\]
Combining with the previous display yields
\[
  K_{\rm int}^{(a)\,M(a)}(U,\cdot)
  \ \ge\ c_0^{\,m}\,\bigotimes_{C} p_{t_0}(U_C^{-1}\cdot)
  \ =\ \theta_*\,P_{t_0}(U,\cdot),
\]
with $\theta_*:=c_0^{\,m(\varepsilon)}\in(0,1)$ and $P_{t_0}$ the product heat-kernel kernel on $G^m$. The constants depend only on $(G,\varepsilon,T_{\rm phys},C_{\rm SW})$ (via $m(\varepsilon)$ and the window condition ensuring $M(a)\gtrsim \beta(a)$), and are uniform in volume/boundary data on fixed slabs. This is exactly the claimed minorization.
\end{proof}

\begin{corollary}[UCIS$_{\rm SW}$ $\Rightarrow$ fixed-physical-time $L^2$ contraction (bookkeeping)]\label{cor:ucis-sw-L2-contraction}
Assume the hypotheses of Theorem~\ref{thm:ucis-sw} and let $(\theta_*,t_0)$ be the constants produced there on the coarse interface space $G^{m(\varepsilon)}$. Define
\[
  q_{\rm phys}\ :=\ (1-\theta_*)+\theta_* e^{-\lambda_1(G)\,t_0}\ =\ 1-\theta_*\big(1-e^{-\lambda_1(G)\,t_0}\big)\ \in(0,1).
\]
Then for all sufficiently small $a$ along the schedule (so that the conclusions of Theorem~\ref{thm:ucis-sw} apply) and every mean-zero $f\in L^2(G^{m(\varepsilon)},\pi^{\otimes m(\varepsilon)})$,
\[
  \big\|K_{\rm int}^{(a)\,M(a)} f\big\|_{2}\ \le\ q_{\rm phys}\,\|f\|_{2},
\]
uniformly in the volume/boundary data on fixed slabs.
\end{corollary}
\begin{proof}
For each such $a$, the minorization from Theorem~\ref{thm:ucis-sw} gives a Nummelin split
\(
  K_{\rm int}^{(a)\,M(a)}=\theta_* P_{t_0}+(1-\theta_*)\mathcal K
\)
for some Markov kernel $\mathcal K$ on $G^{m(\varepsilon)}$. On mean-zero functions, $\|P_{t_0}\|=e^{-\lambda_1(G)t_0}$ and $\|\mathcal K\|\le 1$, giving the displayed contraction factor.
\end{proof}

\begin{corollary}[UCIS$_{\rm SW}$ $\Rightarrow$ fixed-physical-time odd-cone contraction]\label{cor:ucis-sw-odd-contraction}
Assume the hypotheses of Theorem~\ref{thm:ucis-sw} and let $q_{\rm phys}\in(0,1)$ be as in Corollary~\ref{cor:ucis-sw-L2-contraction}. Then for all sufficiently small $a$ along the schedule and every spatial reflection $P_i$,
\[
  \|e^{-a\,M(a)\,H}\psi\|\ \le\ q_{\rm phys}\,\|\psi\|
  \qquad\big(\psi\in\mathcal C_{R_*}\cap\{P_i\psi=-\psi\}\big),
\]
uniformly in volume/boundary data on fixed slabs.
\end{corollary}
\begin{proof}
Fix such an $a$. By Corollary~\ref{cor:ucis-sw-L2-contraction}, the $M(a)$--step coarse interface kernel satisfies $\|K_{\rm int}^{(a)\,M(a)}\|_{L^2_0}\le q_{\rm phys}$ on mean-zero coarse interface functions. Apply Proposition~\ref{prop:int-to-transfer} to the \emph{thick} slab of duration $a\,M(a)$ (equivalently, iterate the one-tick Markov property $M(a)$ times): the transfer over this thick slab is $e^{-a\,M(a)\,H}$ and its interface kernel is $K_{\rm int}^{(a)\,M(a)}$. The domination argument in Proposition~\ref{prop:int-to-transfer} then yields the stated odd-cone contraction with the same factor $q_{\rm phys}$.
\end{proof}

\begin{theorem}[UCIS$_{\rm SW}$ $\Rightarrow$ fixed-physical-time contraction on the full parity-odd subspace]\label{thm:ucis-sw-odd-subspace}
Assume the hypotheses of Theorem~\ref{thm:ucis-sw}. Then for all sufficiently small $a$ along the schedule, for every spatial reflection $P_i$, and every $\psi\in\mathcal H_{\rm odd}^{(i)}=\{\psi:\ P_i\psi=-\psi\}$,
\[
  \|e^{-a\,M(a)\,H}\psi\|\ \le\ q_{\rm phys}\,\|\psi\|,
\]
with $q_{\rm phys}\in(0,1)$ as in Corollary~\ref{cor:ucis-sw-L2-contraction}, uniformly in volume/boundary data on fixed slabs.
\end{theorem}
\begin{proof}
First apply Corollary~\ref{cor:ucis-sw-odd-contraction} on the slab-local odd cone. Then use density (Lemma~\ref{lem:odd-density}) and continuity of $e^{-a\,M(a)\,H}$ to pass to the closure $\mathcal H_{\rm odd}^{(i)}$.
\end{proof}

\begin{lemma}[Density of coarse interface observables as \boldmath $\varepsilon\downarrow 0$]\label{lem:coarse-density}
Let $\mathcal F_{\rm int}$ be the (microscopic) interface $\sigma$--algebra of Definition~\ref{def:interface-kernel}, and let $\{\mathcal F_{\mathrm{int}}^{(\varepsilon)}\}_{\varepsilon\in(0,\varepsilon_0]}$ be the coarse interface $\sigma$--algebras of Lemma~\ref{lem:coarse-interface-construction}, assumed nested so that $\mathcal F_{\mathrm{int}}^{(\varepsilon')}\subseteq \mathcal F_{\mathrm{int}}^{(\varepsilon)}$ for $\varepsilon'\ge \varepsilon$ and
\[
  \sigma\!\Big(\bigcup_{\varepsilon\in(0,\varepsilon_0]}\mathcal F_{\mathrm{int}}^{(\varepsilon)}\Big)\ =\ \mathcal F_{\rm int}.
\]
Then for every $O\in L^2(\mathcal F_{\rm int})$,
\[
  \mathbb E\!\left[\,O\mid \mathcal F_{\mathrm{int}}^{(\varepsilon)}\,\right]\ \xrightarrow[\varepsilon\downarrow 0]{}\ O
  \qquad\text{in }L^2.
\]
Equivalently, $\bigcup_{\varepsilon\in(0,\varepsilon_0]} L^2(\mathcal F_{\mathrm{int}}^{(\varepsilon)})$ is dense in $L^2(\mathcal F_{\rm int})$.
\end{lemma}
\begin{proof}
Conditional expectation $\mathbb E[\cdot\mid\mathcal F_{\mathrm{int}}^{(\varepsilon)}]$ is the orthogonal projection in $L^2$ onto the closed subspace $L^2(\mathcal F_{\mathrm{int}}^{(\varepsilon)})$. Under the nesting hypothesis, these projections form an increasing net and converge strongly to the projection onto $L^2(\sigma(\cup_\varepsilon \mathcal F_{\mathrm{int}}^{(\varepsilon)}))=L^2(\mathcal F_{\rm int})$, i.e. to the identity on $L^2(\mathcal F_{\rm int})$ (martingale convergence / monotone convergence of conditional expectations).
\end{proof}

\begin{corollary}[Odd\,$\to$\,mean-zero upgrade (eight ticks)]\label{cor:odd-to-meanzero}
Under the hypotheses of Theorem~\ref{thm:eight-tick-uniform}, the eight--tick mean-zero contraction holds on $\Omega^\perp$ with rate $e^{-8c_{\rm cut,phys}}$.
\end{corollary}
\begin{proof}
This is exactly Theorem~\ref{thm:eight-tick-uniform}.
\end{proof}

\subsection*{Optional: Area Law $+$ Tube Geometry $\Rightarrow$ Uniform Gap (One-way)}
\begin{enumerate}[label=(\textbf{\Alph*}), leftmargin=2em, itemsep=8pt]
\item[\textbf{(AL)}] \textbf{Area law, uniform in $(L,a)$.} There exist $\sigma_*>0$ and $C_{\mathrm{AL}}<\infty$ such that large rectangular Wilson loops obey $|\langle W_{\Gamma(R,T)}\rangle|\le C_{\mathrm{AL}} e^{-\sigma_* RT}$ in physical units.
\item[\textbf{(TUBE)}] \textbf{Geometric tube bound.} For loops supported in a fixed physical ball $B_{R_*}$ at times $0$ and $t$, any spanning surface has area $\ge \kappa_* t$ with $\kappa_*>0$ depending only on $R_*$. 
\end{enumerate}
\begin{theorem}[Optional: Area law $+$ tube $\Rightarrow$ uniform gap]\label{thm:AL-gap}
Under AL and TUBE, $\mathrm{spec}(H_{L,a})\subset\{0\}\cup[\sigma_*\kappa_*,\infty)$ uniformly in $(L,a)$. Consequently, by Theorem~\ref{thm:gap-persist-cont} and Mosco/strong-resolvent convergence, the continuum gap is $\ge \sigma_*\kappa_*$. 
\end{theorem}

\noindent\emph{Remark.} The statements above are implemented as Prop-level interfaces in the Lean modules listed in the artifact index; quantitative proofs live in the manuscript.

\subsection*{Isotropy Restoration and Poincar\'e Invariance}
% Optional AF/Mosco isotropy lemma removed from main chain; isotropy handled via AF--free calibrators elsewhere

\begin{proposition}[Aspect ratios and mild anisotropy]\label{prop:anisotropy}
Let the van Hove boxes have aspect ratios bounded away from $0$ and $\infty$. If $a_t/a_s\to 1$ as $a_s\to 0$, then all local limits and constants are unchanged. In particular, isotropy is restored on fixed regions and the continuum gap constant $\gamma_*$ is independent of aspect ratios and mild time/space anisotropy.
\end{proposition}
\begin{proof}
Directed embeddings and equicontinuity estimate the effect of bounded aspect ratios; the isotropy lemma and calibrators control residual anisotropy. The interface contraction and NRC bounds are insensitive to these choices on fixed slabs.
\end{proof}

\begin{corollary}[Poincar\'e invariance via OS\,$\to$\,Wightman]\label{cor:poincare}
With the global Euclidean measure constructed in Section~\ref{sec:global-R4} and Euclidean invariance established in Theorem~\ref{thm:global-OS}, the OS reconstruction (Theorem~\ref{thm:os-to-wightman}) yields a Wightman theory with full Poincar\'e covariance on Minkowski space.
\end{corollary}
\section{Lattice Yang--Mills set-up and bounds}
\label{sec:lattice-setup}

\paragraph{Standing assumptions and geometry.}
Fix a physical slab radius $R_*>0$ and maximal tick $a_0>0$. Throughout, the gauge group is a compact simple $G$ with Haar probability and a fixed bi-invariant Riemannian metric (used to define heat kernels and small balls $B_r(\mathbf 1)$). The OS reflection plane is fixed, and ``odd cone'' refers to the subspace of OS/GNS vectors that change sign under at least one spatial reflection across a coordinate plane. Constants such as $c_{\rm geo}(R_*,a_0)$, $m_{\rm cut}(R_*,a_0)$, $\theta_*$, $t_0$, and the small-time parameters $c_0,c_1$ depend only on $(R_*,a_0,G)$ (and the chosen metric normalization) and are uniform in the volume and the bare coupling $\beta\ge 0$.

\paragraph{Interface scaling and coarse skeleton.}
For a fine lattice spacing $a\le a_0$, the number of interface coordinates at the cut scales as $m(a)\asymp a^{-3}$ for a fixed physical slab. We therefore introduce a coarse skeleton at fixed physical resolution $\varepsilon\in(0,\varepsilon_0]$ (independent of $a$), with $m(\varepsilon)=O(\varepsilon^{-3})$. All Doeblin/minorization statements are formulated on the coarse skeleton, yielding constants independent of $a$, and transferred to fine observables by lumping and density (Lemmas~\ref{lem:lumping},\ref{lem:coarse-density}).

\paragraph{Analytic conventions (heat kernel and Laplacian).}
The heat kernel $p_t$ on a compact simple $G$ is for the Laplace--Beltrami operator $\Delta$ associated to the bi-invariant metric, normalized so $\partial_t p_t = \Delta p_t$ and $\int p_t\,d\pi=1$. The semigroup $P_t$ on $L^2(G^m,\pi^{\otimes m})$ is $P_t f = f * p_t$ (componentwise convolution). The spectral gap $\lambda_1(G)>0$ is the first nonzero eigenvalue of $-\Delta$; hence on the orthogonal complement of constants, $\|P_t\| \le e^{-\lambda_1(G) t}$.

We work on a finite 4D torus with sites $x\in\Lambda$ and $SU(N)$ link variables $U_{x,\mu}$. For a plaquette $P$, let $U_P$ be the ordered product of links around $P$. The Wilson action is
\[
 S_{\beta}(U) := \beta \sum_{P} \Bigl(1 - \tfrac{1}{N} \operatorname{Re} \operatorname{Tr} U_P\Bigr).
\]
Since $-N\le \operatorname{Re} \operatorname{Tr} V \le N$ for all $V\in SU(N)$, we have $0\le S_{\beta}(U)\le 2\beta |\{P\}|$. With normalized Haar product measure, the partition function obeys $e^{-2\beta |\{P\}|}\le Z_{\beta}\le 1$.
\section{Reflection positivity and transfer operator}

Choose a time-reflection hyperplane and define the standard Osterwalder--Seiler link reflection $\theta$. For the *-algebra $\mathcal A_+$ of cylinder observables supported in $t\ge 0$, the sesquilinear form $\langle F,G\rangle_{OS}:=\int \overline{F(U)}\,(\theta G)(U)\, d\mu_{\beta}(U)$ is positive semidefinite. By GNS, we obtain a Hilbert space $\mathcal H$ and a positive self-adjoint transfer operator $T$ with $\lVert T\rVert\le 1$ and one-dimensional constants sector.
\smallskip
\noindent\emph{Remark.} The OS reflection makes the half-space algebra a pre-Hilbert space under the reflected inner product; the Markov/transfer step is a contraction by Cauchy–Schwarz in this inner product.

\paragraph{Notation and Hamiltonian.}
Let $\Omega\in\mathcal H$ denote the vacuum vector (the class of constants). Write $\mathcal H_0:=\Omega^{\perp}$ for the mean-zero subspace. Define
\[
  r_0(T)\;:=\; \sup\{\,|\lambda| : \lambda\in\operatorname{spec}(T|_{\mathcal H_0})\,\},\qquad
  H\;:=\;-\log T\ \text{ on }\ \mathcal H_0
\]
by spectral calculus. The Hamiltonian gap is $\Delta(\beta):=-\log r_0(T)$.
For brevity, we also write $\gamma(\beta):=\Delta(\beta)$.
\subsection*{Proof (Osterwalder--Seiler)}
The Wilson action decomposes into $S_\beta=S_\beta^{(+)}+S_\beta^{(-)}+S_\beta^{(\perp)}$, where $S_\beta^{(\pm)}$ are sums over plaquettes entirely in the positive/negative half-spaces and $S_\beta^{(\perp)}$ sums over plaquettes crossing the reflection plane. Expanding the crossing weights in characters and using that irreducible characters $\chi_R$ are positive-definite class functions, together with Haar invariance and $\theta$-invariance of the measure, yields that the Gram matrix $[\langle F_i,\theta F_j\rangle_{OS}]$ is positive semidefinite for any finite family $\{F_i\}\subset \mathcal A_+$. This is the Osterwalder--Seiler argument.

\paragraph{Character positivity and the crossing kernel (details).}
\begin{lemma}[Irreducible characters are positive definite]\label{lem:char-pd}
For any compact group $G$ and any unitary irreducible representation $R$, the class function $\chi_R(g)=\operatorname{Tr}\,R(g)$ is positive definite: for any $g_1,\dots,g_m\in G$ and $c\in\mathbb C^m$,
\[
  \sum_{i,j=1}^m \overline{c_i}\,c_j\,\chi_R(g_i^{-1} g_j)\ \ge\ 0.
\]
\end{lemma}
\begin{proof}
Let $v:=\sum_i c_i\,R(g_i)\,v_0$ for any fixed $v_0$ in the representation space. Then
\[
  \sum_{i,j}\overline{c_i}\,c_j\,\chi_R(g_i^{-1} g_j)\ =\ \sum_{i,j}\overline{c_i}\,c_j\,\operatorname{Tr}\big(R(g_i)^{*}R(g_j)\big)\ =\ \|\sum_j c_j R(g_j)\|_{\mathrm{HS}}^2\ \ge\ 0.
\]
Alternatively, this is a standard consequence of Peter–Weyl.
\end{proof}

\begin{proposition}[PSD crossing Gram for Wilson link reflection]\label{prop:psd-crossing-gram}
For the Wilson action and link reflection $\theta$, the OS Gram matrix $[\langle F_i,\theta F_j\rangle_{OS}]_{i,j}$ is positive semidefinite for any finite $\{F_i\}\subset\mathcal A_+$.
\end{proposition}
\begin{proof}
Let $\{F_i\}_{i=1}^n \subset \mathcal A_+$ be a finite family of half-space observables. We must show that the matrix $M_{ij} := \langle F_i, \theta F_j \rangle_{OS}$ is positive semidefinite.

\emph{Step 1: Decompose the Wilson action.} Write $S_\beta = S_\beta^{(+)} + S_\beta^{(-)} + S_\beta^{(\perp)}$, where $S_\beta^{(\pm)}$ are sums over plaquettes entirely in the positive/negative half-spaces and $S_\beta^{(\perp)}$ sums over plaquettes crossing the reflection plane. For observables $F_i \in \mathcal A_+$, we have
\[
  M_{ij} = \int \overline{F_i(U)} \, (\theta F_j)(U) \, e^{-S_\beta(U)} \, dU = \int \overline{F_i(U^+)} \, F_j(\theta U^+) \, K_\beta(U^+, U^-) \, dU^+ dU^-,
\]
where $K_\beta(U^+, U^-)$ is the crossing kernel arising from $\exp(-S_\beta^{(\perp)})$ and we used $\theta$-invariance of the Haar measure.

\emph{Step 2: Character expansion of crossing weights.} For each plaquette $P$ crossing the reflection plane, expand (Montvay--M\"unster \cite{MontvayMunster1994}, §4.2):
\[
  \exp\Big(\tfrac{\beta}{N}\,\Re\,\operatorname{Tr} U_P\Big) = \sum_{R} c_R(\beta)\,\chi_R(U_P), \quad c_R(\beta) = \int_{SU(N)} \exp\Big(\tfrac{\beta}{N}\,\Re\,\operatorname{Tr} V\Big) \overline{\chi_R(V)} \, dV \ge 0,
\]
where the nonnegativity follows from $\exp(\cdot) > 0$ and Schur orthogonality. The crossing kernel becomes
\[
  K_\beta(U^+, U^-) = \prod_{P \in \mathcal P_\perp} \sum_{R_P} c_{R_P}(\beta) \chi_{R_P}(U_P) = \sum_{\{R_P\}} \Big(\prod_{P} c_{R_P}(\beta)\Big) \prod_{P} \chi_{R_P}(U_P),
\]
where $\mathcal P_\perp$ denotes plaquettes crossing the cut.

\emph{Step 3: Integration and tensor structure.} After integrating out $U^-$ with Haar measure, only terms with matching representations survive. The result is
\[
  M_{ij} = \sum_{\{R_P\}} w_{\{R_P\}} \int \overline{F_i(U^+)} F_j(\theta U^+) \prod_{\ell \in \text{cut}} \chi_{R_\ell}(g_\ell^{-1} h_\ell) \, dU^+,
\]
where $w_{\{R_P\}} \ge 0$ are products of $c_{R_P}(\beta) \ge 0$, and $(g_\ell, h_\ell)$ are appropriate group elements from $U^+$ entering the cut links.
\emph{Step 4: PSD property of character kernels.} For each fixed representation assignment $\{R_\ell\}$, the kernel $\prod_\ell \chi_{R_\ell}(g_\ell^{-1} h_\ell)$ defines a PSD form by Lemma~\ref{lem:char-pd} (each $\chi_{R_\ell}$ is PSD) and the fact that tensor products of PSD kernels are PSD. Thus the matrix
\[
  M_{ij}^{\{R_\ell\}} := \int \overline{F_i(U^+)} F_j(\theta U^+) \prod_{\ell} \chi_{R_\ell}(g_\ell^{-1} h_\ell) \, dU^+
\]
satisfies $M^{\{R_\ell\}} \succeq 0$.

\emph{Step 5: Conclusion.} Since $M = \sum_{\{R_P\}} w_{\{R_P\}} M^{\{R_\ell\}}$ with $w_{\{R_P\}} \ge 0$ and each $M^{\{R_\ell\}} \succeq 0$, we have $M \succeq 0$. This establishes reflection positivity. The GNS construction then yields a Hilbert space $\mathcal H$, and the transfer step $T: [F] \mapsto [\tau_1 F]$ (where $\tau_1$ is unit time translation) is positive and self-adjoint by OS positivity.
\end{proof}

\begin{lemma}[OS/GNS transfer properties]\label{lem:os-gns-transfer}
Assuming OS reflection positivity for the half-space algebra and invariance under unit Euclidean time translation $\tau_1$, the GNS construction yields a Hilbert space $\mathcal H$, a cyclic vacuum vector $\Omega$, and a contraction $T$ on $\mathcal H$ implementing $\tau_1$ such that $T$ is positive and self-adjoint, $\|T\|\le 1$, and the constants sector is one-dimensional spanned by $\Omega$.
\end{lemma}
\begin{proof}
The reflected inner product $\langle F,G\rangle_{OS}=\int \overline{F}\,\theta G\,d\mu_\beta$ is positive semidefinite by OS positivity, hence the completion of the quotient by nulls gives $\mathcal H$ and $\Omega=[1]$. Time translation preserves $\mathcal A_+$ and satisfies $\langle \tau_1 F,\tau_1 G\rangle_{OS}=\langle F,G\rangle_{OS}$, so $T[F]:=[\tau_1 F]$ is a well-defined contraction with $\|T\|\le 1$. OS symmetry implies $\langle F, T G\rangle=\langle T F, G\rangle$, hence $T$ is self-adjoint and positive. The constants are fixed by $\tau_1$, so the constants sector is one-dimensional, spanned by $\Omega$.
\end{proof}

\begin{proof}[Proof of Theorem~\ref{thm:os}]
By Proposition~\ref{prop:psd-crossing-gram}, OS reflection positivity holds for Wilson link reflection. Lemma~\ref{lem:os-gns-transfer} then yields the claimed transfer operator properties.
\end{proof}

\section{Strong-coupling contraction and mass gap}

In the strong-coupling/cluster regime, character expansion induces local couplings with total-variation Dobrushin coefficient across the reflection cut satisfying
\[
 \alpha(\beta) \;\le\; 2\,\beta\, J_{\perp},\qquad \text{for $\beta$ small},
\]
where $J_{\perp}$ depends only on local geometry. Hence the spectral radius on the mean-zero sector satisfies $r_0(T)\le \alpha(\beta)$ and the Hamiltonian $H:=-\log T$ has a gap $\Delta(\beta):=-\log\bigl(2\beta J_{\perp}\bigr)>0$ whenever $\beta<1/(2J_{\perp})$. The bounds are uniform in $N\ge 2$ and in the volume.
\paragraph{Influence estimate (explicit).}
Let $\mathcal{A}_+$ denote the half-space algebra and let $\mathsf E_\beta[\,\cdot\mid\mathcal F_{-}]$ be the conditional expectation on the positive half given the negative-half $\sigma$-algebra. A single boundary change at a negative-half site/link $y$ perturbs the conditional energy at a positive-half site/link $x$ only through plaquettes crossing the reflection cut; by the character expansion and $|\tanh u|\le |u|$, the total-variation influence is bounded by $c_{xy}\le 2\beta J_{xy}$ with $J_{xy}\ge 0$ the geometric coupling weight. Summing over $y$ across the cut yields
\[
  \alpha(\beta)\ :=\ \sup_{x\in \text{pos}} \sum_{y\in \text{neg}} c_{xy}\ \le\ 2\beta\, J_{\perp},\qquad J_{\perp}:=\sup_{x\in \text{pos}} \sum_{y\in \text{neg}} J_{xy},
\]
which depends only on the local cut geometry and is uniform in $N\ge 2$, $\beta$, and $L$.
\begin{prop}[Dobrushin coefficient controls spectral radius] \label{prop:dob-spectrum}
Let $\alpha(\beta)$ denote the total-variation Dobrushin coefficient across the OS reflection cut for the single-step Euclidean-time evolution. Then
\[
  r_0(T)\;\le\; \alpha(\beta).
\]
Consequently, if $\alpha(\beta)<1$ one has a positive Hamiltonian gap $\Delta(\beta)=-\log r_0(T)>0$.
\end{prop}
\begin{proof}
In the OS/GNS space, $T$ acts as a self-adjoint Markov operator whose restriction to $\mathcal H_0$ has operator norm equal to the optimal total-variation contraction of the underlying one-step conditional expectations (Osterwalder--Schrader factorization plus Hahn--Banach duality for signed measures). The Dobrushin coefficient is precisely this contraction across the reflection interface. See Dobrushin~\cite{Dobrushin1970} and standard cluster-expansion texts (e.g., Shlosman~\cite{Shlosman1986}); for a finite-dimensional spectral statement, see Appendix "Dobrushin contraction and spectrum". Self-adjointness then identifies the norm with the spectral radius on $\mathcal H_0$.
\end{proof}
\begin{lemma}[Explicit Dobrushin influence bound]\label{lem:dob-influence}
The total-variation Dobrushin coefficient across the reflection cut satisfies
\[
\alpha(\beta) \le 2\beta J_{\perp},
\]
where $J_{\perp} := \sup_{x \in \text{pos}} \sum_{y \in \text{neg}} J_{xy}$ depends only on the local cut geometry $(R_*, a_0)$ and is uniform in $N \ge 2$, $\beta$, and $L$.
\end{lemma}
\begin{proof}
Let $\mathsf{E}_\beta[\cdot \mid \mathcal{F}_{-}]$ be the conditional expectation on the positive half given the negative-half $\sigma$-algebra. A single boundary change at a negative-half site/link $y$ perturbs the conditional energy at a positive-half site/link $x$ only through plaquettes crossing the reflection cut. By the character expansion and $|\tanh u| \le |u|$, the total-variation influence is bounded by $c_{xy} \le 2\beta J_{xy}$ with $J_{xy} \ge 0$ the geometric coupling weight (number of crossing plaquettes connecting $x$ and $y$, weighted by 1). Summing over $y$ across the cut yields the bound on $\alpha(\beta)$. The supremum defining $J_{\perp}$ is finite and depends only on the fixed physical slab radius $R_*$ and thickness bound $a_0$, independent of $N$, $\beta$, and volume $L$.
\end{proof}
\section{Appendix: Coarse-graining convergence and gap persistence (P8)}
We record a uniform coarse--graining bound and operator--norm convergence for reflected loop kernels along a voxel--to--continuum refinement, together with hypotheses that ensure gap persistence in the continuum. This appendix supports the optional continuum discussion in Sec.~"Continuum scaling windows".

\paragraph{Setting.}
Let $K_n$ be reflected loop kernels (covariances/Green's functions) arising as inverses of positive operators $H_n$ (e.g., discrete Hamiltonians or elliptic operators): $K_n=H_n^{-1}$, with continuum limits $K=H^{-1}$. Reflection positivity implies self--adjointness of $H_n$ and $K_n$. Let $R_n$ (restriction) and $P_n$ (prolongation) compare discrete and continuum Hilbert spaces.

\paragraph{Uniform bound.}
Define the discrete gaps
\[
  \beta_n\;:=\;\inf \operatorname{spec}(H_n).
\]
If there exists $\beta_0>0$ with $\beta_n\ge \beta_0$ for all $n$, then
\[
  \lVert K_n\rVert_{\mathrm{op}}\;=\;\frac{1}{\beta_n}\;\le\;\frac{1}{\beta_0}.
\]
This follows from coercivity (strict positivity of $H$), stability of the discretization preserving positivity, and uniform discrete functional inequalities (e.g., discrete Poincar\'e) with constants independent of the voxel size.

\paragraph{Operator--norm convergence.}
Assume stability above and consistency (local truncation errors vanish on a dense core). Then
\begin{equation}
\label{eq:p8-norm}
  \big\lVert P_n K_n R_n - K\big\rVert_{\mathrm{op}}\;\longrightarrow\;0\qquad (n\to\infty),
\end{equation}
equivalently, $H_n\to H$ in norm resolvent sense. The upgrade from strong convergence to \eqref{eq:p8-norm} uses collective compactness: if $K$ is compact and $\{P_n K_n R_n\}$ is collectively compact via uniform discrete regularity, then strong convergence implies norm convergence.

\paragraph{Gap persistence (continuum $\gamma>0$).}
Suppose further:
\begin{itemize}
  \item (H1) $H_n$ and $H$ are self--adjoint.
  \item (H2) $H_n\to H$ in norm resolvent sense (\eqref{eq:p8-norm}).
  \item (H3) There is a uniform discrete gap: for some interval $(a,b)$ with $\gamma_0:=b-a>0$, one has $\operatorname{spec}(H_n)\cap(a,b)=\varnothing$ for all large $n$.
\end{itemize}
Then spectral convergence (Hausdorff) yields $\operatorname{spec}(H)\cap(a,b)=\varnothing$, so the continuum gap satisfies $\gamma\ge \gamma_0>0$.
\section{Optional: Continuum scaling-window routes (KP/area-law; not used in main chain)}

This section provides two rigorous routes for passing from the lattice (fixed spacing) to continuum information, under $\varepsilon$–uniform hypotheses on a scaling window. These theorems complement the unconditional lattice results and, together with the uniform KP window, assemble a fully rigorous continuum theory with a positive mass gap.

\subsection*{Optional A: Uniform lattice area law implies a continuum string tension}

\paragraph{Setting.}
Fix a dimension $d\ge 2$ and a hypercubic lattice $\varepsilon\,\mathbb{Z}^d$ with spacing $\varepsilon\in(0,\varepsilon_0]$. For a nearest--neighbour lattice loop $\Lambda\subset\varepsilon\,\mathbb{Z}^d$ let
\[
  A_\varepsilon^{\min}(\Lambda)\in\mathbb{N}
\]
be the minimal number of plaquettes in any lattice surface spanning $\Lambda$, and let $P_\varepsilon(\Lambda)\in\mathbb{N}$ be the number of lattice edges on $\Lambda$ (its lattice perimeter). Set the corresponding physical area and perimeter
\[
  \mathsf{Area}_\varepsilon(\Lambda):=\varepsilon^2 A_\varepsilon^{\min}(\Lambda),\qquad
  \mathsf{Per}_\varepsilon(\Lambda):=\varepsilon P_\varepsilon(\Lambda).
\]
For a continuum rectifiable closed curve $\Gamma\subset\mathbb{R}^d$ let $\mathsf{Area}(\Gamma)$ denote the least Euclidean area of any (Lipschitz) spanning surface with boundary $\Gamma$, and let $\mathsf{Per}(\Gamma)$ be its Euclidean length.

\paragraph{Uniform lattice area law (input; strong coupling; optional).}
See Appendix "Strong-coupling area law for Wilson loops (R6)" for a standard derivation of a lattice area law with a positive string tension and a perimeter correction; the present paragraph abstracts those bounds uniformly over a scaling window.
Assume there exist functions $\tau_\varepsilon>0$ and $\kappa_\varepsilon\ge 0$, defined for $\varepsilon\in (0,\varepsilon_0]$, and constants
\[
  T_*:=\inf_{0<\varepsilon\le\varepsilon_0}\frac{\tau_\varepsilon}{\varepsilon^2}>0,\qquad
  C_*:=\sup_{0<\varepsilon\le\varepsilon_0}\frac{\kappa_\varepsilon}{\varepsilon}<\infty,
\]
such that for all sufficiently large lattice loops $\Lambda\subset\varepsilon\,\mathbb{Z}^d$ (size measured in lattice units, which will automatically hold for fixed physical loops as $\varepsilon\downarrow 0$),
\begin{equation}
\label{eq:lattice-area-law}
  -\log\langle W(\Lambda)\rangle \;\ge\; \tau_\varepsilon\,A_\varepsilon^{\min}(\Lambda)\;-
  \;\kappa_\varepsilon\,P_\varepsilon(\Lambda)
  \;=\;\Big(\tfrac{\tau_\varepsilon}{\varepsilon^2}\Big)\mathsf{Area}_\varepsilon(\Lambda)\;-
  \;\Big(\tfrac{\kappa_\varepsilon}{\varepsilon}\Big)\mathsf{Per}_\varepsilon(\Lambda).
\end{equation}
In the strong--coupling/cluster regime, \eqref{eq:lattice-area-law} follows from the character expansion: writing the Wilson weight in irreducible characters, the activity ratio $\rho(\beta)$ for nontrivial representations obeys $\mu\,\rho(\beta) < 1$ for all sufficiently small $\beta$, with a lattice constant $\mu$, yielding $T(\beta):= -\log \rho(\beta) > 0$ and a perimeter correction controlled by $\kappa_\varepsilon$.

\paragraph{Directed embeddings of loops.}
Let $\Gamma\subset\mathbb{R}^d$ be a fixed rectifiable closed curve. A \emph{directed family} $\{\Gamma_\varepsilon\}_{\varepsilon\downarrow 0}$ of lattice loops converging to $\Gamma$ means: (i) $\Gamma_\varepsilon\subset\varepsilon\,\mathbb{Z}^d$ is a nearest--neighbour loop, (ii) the Hausdorff distance $d_H(\Gamma_\varepsilon,\Gamma)\to 0$ as $\varepsilon\downarrow 0$, (iii) each $\Gamma_\varepsilon$ is contained in a tubular neighbourhood of $\Gamma$ of radius $O(\varepsilon)$ and follows the orientation of $\Gamma$ (e.g., via grid--snapping of a $C^1$ parametrization).

\paragraph{Two geometric facts.}
\emph{Fact A (surface convergence).} For any directed family $\{\Gamma_\varepsilon\to\Gamma\}$,
\begin{equation}
\label{eq:area-conv}
  \lim_{\varepsilon\downarrow 0}\mathsf{Area}_\varepsilon(\Gamma_\varepsilon)\;=\;\mathsf{Area}(\Gamma).
\end{equation}
\emph{Remark (optional; geometry).} A standard argument using lower semicontinuity of area under boundary convergence and cubical polyhedral approximations on $\varepsilon\,\mathbb{Z}^d$ yields \eqref{eq:area-conv}; see, e.g., Federer's GMT text. This geometric fact is not used in the unconditional mass-gap chain.

\emph{Fact B (perimeter control).} There exists a universal constant $\kappa_d:=\sup_{u\in\mathbb{S}^{d-1}}\sum_{i=1}^d |u_i|=\sqrt{d}$ such that for any directed family,
\begin{equation}
\label{eq:per-bound}
  \limsup_{\varepsilon\downarrow 0}\mathsf{Per}_\varepsilon(\Gamma_\varepsilon)\;\le\;\kappa_d\,\mathsf{Per}(\Gamma).
\end{equation}
\emph{Remark (optional; geometry).} For any rectifiable curve with unit tangent $u$, the lattice routing length density is $\sum_i |u_i|\le \sqrt d$. Integrating gives \eqref{eq:per-bound}. This is not used on the unconditional chain.

\paragraph{Main statement (optional; continuum area law with perimeter term).}
\begin{theorem}
Let $\Gamma\subset\mathbb{R}^d$ be a rectifiable closed curve with $\mathsf{Area}(\Gamma)<\infty$. Assume the uniform lattice bound \eqref{eq:lattice-area-law} on the scaling window $(0,\varepsilon_0]$. Define the $\varepsilon$--independent constants
\[
  T\;:=\;\inf_{0<\varepsilon\le\varepsilon_0}\frac{\tau_\varepsilon}{\varepsilon^2}\;>\;0,\qquad
  C_0\;:=\;\sup_{0<\varepsilon\le\varepsilon_0}\frac{\kappa_\varepsilon}{\varepsilon}\;<\;\infty,\qquad
  C\;:=\;\kappa_d\,C_0.
\]
Then for any directed family $\{\Gamma_\varepsilon\to\Gamma\}$,
\begin{equation}
\label{eq:continuum-bound}
  \limsup_{\varepsilon\downarrow 0}\bigl[-\log\langle W(\Gamma_\varepsilon)\rangle\bigr]
  \;\ge\;
  T\,\mathsf{Area}(\Gamma)\;-
  \;C\,\mathsf{Per}(\Gamma).
\end{equation}
In particular, the continuum string tension is positive and bounded below by $T$.
\end{theorem}

\begin{proof}
Starting from \eqref{eq:lattice-area-law} with $\Lambda=\Gamma_\varepsilon$ and taking $\limsup_{\varepsilon\downarrow 0}$, use $\limsup(A_\varepsilon-B_\varepsilon)\ge (\inf A_\varepsilon)-(\sup B_\varepsilon)$ in the form
\[
  \limsup_{\varepsilon\downarrow 0}\bigl[A_\varepsilon-B_\varepsilon\bigr]
  \;\ge\;
  \Big(\inf_{0<\varepsilon\le\varepsilon_0}\tfrac{\tau_\varepsilon}{\varepsilon^2}\Big)\cdot
  \liminf_{\varepsilon\downarrow 0}\mathsf{Area}_\varepsilon(\Gamma_\varepsilon)
  \;-
  \;\Big(\sup_{0<\varepsilon\le\varepsilon_0}\tfrac{\kappa_\varepsilon}{\varepsilon}\Big)\cdot
  \limsup_{\varepsilon\downarrow 0}\mathsf{Per}_\varepsilon(\Gamma_\varepsilon).
\]
Applying Facts A and B yields \eqref{eq:continuum-bound}.
\end{proof}

\paragraph{Remarks.}
1. The constants $T$ and $C$ are $\varepsilon$--independent: $T$ is the uniform lower bound on the lattice string tension in physical units ($\tau_\varepsilon/\varepsilon^2$), while $C$ is the product of the uniform perimeter coefficient in physical units ($C_0=\sup\kappa_\varepsilon/\varepsilon$) with the geometric factor $\kappa_d=\sqrt{d}$. For planar Wilson loops, $C=\sqrt{2}\,C_0$.

2. The "large loop" qualifier is automatic here: for any fixed physical loop $\Gamma$, the lattice representative $\Gamma_\varepsilon$ has diameter of order $\varepsilon^{-1}$ in lattice units, so the hypotheses behind \eqref{eq:lattice-area-law} (from strong--coupling/cluster bounds) apply for all sufficiently small $\varepsilon$.

3. The bound \eqref{eq:continuum-bound} states that the continuum string tension $\sigma_{\text{cont}}:=\liminf_{\varepsilon\downarrow 0}\tau_\varepsilon/\varepsilon^2$ is positive (indeed $\sigma_{\text{cont}}\ge T>0$), with a controlled perimeter subtraction that is uniform along any directed family $\Gamma_\varepsilon\to\Gamma$.

\section{Global continuum construction on $\mathbb R^4$ and OS axioms}\label{sec:global-R4}

This section constructs a single, global family of Schwinger functions on $\mathbb R^4$ from the local limits on fixed physical regions, and verifies OS0--OS5 globally. We then perform OS\,$\to$\,Wightman reconstruction and transfer the mass gap to Minkowski space.

\subsection{Directed van Hove exhaustions and cylinder algebras}

Let $\{\Lambda_k\}_{k\in\mathbb N}$ be an increasing van Hove exhaustion of $\mathbb R^4$ by bounded Lipschitz regions (e.g., cubes), so that $\overline{\Lambda_k}\subset\Lambda_{k+1}$, $\bigcup_k \Lambda_k=\mathbb R^4$, and $|\partial\Lambda_k|/|\Lambda_k|\to 0$. For each $k$, let $\mathfrak A_0(\Lambda_k)$ denote the local time-zero OS algebra generated by gauge-invariant observables supported in $\Lambda_k$ (e.g., Wilson loops $W_\Gamma$ with $\Gamma\subset\Lambda_k$ and smeared clover fields supported in $\Lambda_k$). We write $\mathfrak A_0:=\bigcup_k \mathfrak A_0(\Lambda_k)$ for the global algebraic union.

From Sections preceding, for each fixed $\Lambda_k$ we have continuum Schwinger functions $\{S^{(k)}_n\}$ on $\mathfrak A_0(\Lambda_k)$ obtained as van Hove/lattice limits, with OS0--OS2 and clustering (OS3) verified on $\Lambda_k$ uniformly in the approximants; see Proposition~\ref{prop:OS0-poly}, Proposition~\ref{prop:embedding-independence}, Proposition~\ref{prop:bc-robust}, and Theorem~\ref{thm:gap-persist-cont}.

\begin{proposition}[Consistency on overlaps]\label{prop:consistency-overlaps}
If $k<\ell$ and $O_1,\dots,O_n\in\mathfrak A_0(\Lambda_k)$, then
\[
  S^{(k)}_n(O_1,\dots,O_n)\ =\ S^{(\ell)}_n(O_1,\dots,O_n).
\]
Consequently, for any finite family $(O_1,\dots,O_n)$ supported in some $\Lambda_k$, the value
\[
  S_n(O_1,\dots,O_n)\ :=\ S^{(k)}_n(O_1,\dots,O_n)
\]
is well-defined (independent of $k$ large enough).
\end{proposition}

\begin{theorem}[Projective-limit $C_0$-semigroup and generator on $\mathbb R^4$]\label{thm:proj-semigroup}
Let $\{\Lambda_k\}_{k\in\mathbb N}$ be a van Hove exhaustion and for each $k$ let $\mathcal H_k$ be the OS/GNS Hilbert space constructed from the continuum Schwinger functions on $\Lambda_k$, with contraction semigroup $P_k(t)=e^{-tH_k}$, $H_k\ge 0$. Assume:
\begin{enumerate}[label=(\roman*), leftmargin=2em]
  \item \textbf{(Overlap consistency)} Proposition~\ref{prop:consistency-overlaps} holds and the embeddings $\jmath_{k\to \ell}:\mathcal H_k\to\mathcal H_\ell$ are isometries intertwining time translations: $\jmath_{k\to \ell}\,P_k(t)=P_\ell(t)\,\jmath_{k\to \ell}$ for all $t\ge 0$.
  \item \textbf{(Uniform mean-zero contraction)} There exists $\gamma_*>0$ such that for all $k$ and all $t\ge 0$,
  \[
    \big\|P_k(t)\,(I-\vert\Omega_k\rangle\langle\Omega_k\vert)\big\|\ \le\ e^{-\gamma_* t}.
  \]
\end{enumerate}
Then the inductive-limit Hilbert space $\mathcal H:=\varinjlim \mathcal H_k$ carries a unique contraction semigroup $P(t)$ with generator $H\ge 0$ such that $P(t)\,\jmath_k=\jmath_k\,P_k(t)$ for all $k$ and $t\ge 0$, and
\[
  \big\|P(t)\,(I-\vert\Omega\rangle\langle\Omega\vert)\big\|\ \le\ e^{-\gamma_* t},\qquad t\ge 0.
\]
Consequently, $\operatorname{spec}(H)\subset\{0\}\cup[\gamma_*,\infty)$.
\end{theorem}
\begin{proof}
Define $\mathcal D:=\bigcup_k \jmath_k\mathcal H_k\subset \mathcal H$. For $\psi\in \jmath_k\mathcal H_k$ set $P(t)\psi:=\jmath_k P_k(t)\psi$. This is well-defined by (i) and extends by continuity to a contraction on $\overline{\mathcal D}=\mathcal H$; the semigroup law passes from $\{P_k(t)\}$ to $\{P(t)\}$. Strong continuity at $t=0$ holds on each $\jmath_k\mathcal H_k$ and hence on $\mathcal H$ by density, so $P(t)$ is a $C_0$-semigroup with nonnegative generator $H$. The uniform mean-zero bound follows from (ii) by the same consistency argument, yielding $\|P(t)(I-\vert\Omega\rangle\langle\Omega\vert)\|\le e^{-\gamma_* t}$. The spectral inclusion is then immediate from the spectral mapping theorem for $C_0$-semigroups.
\end{proof}

\begin{lemma}[Quantitative consistency on overlaps]\label{lem:overlap-quant}
Let $k<\ell$ and $O_1,\dots,O_n\in\mathfrak A_0(\Lambda_k)$ be supported in a common compact $U\Subset \Lambda_k$. There exist constants $C=C(U,n,N)$ and $\alpha=\alpha(N)>0$ such that
\[
  \big|\,S^{(\ell)}_n(O_1,\dots,O_n) - S^{(k)}_n(O_1,\dots,O_n)\,\big|\ \le\ C\, e^{-\alpha\,\mathrm{dist}(U,\partial\Lambda_k)}.
\]
In particular, the convergence of $S^{(\ell)}_n(\cdot)$ to $S^{(k)}_n(\cdot)$ as $\ell\to\infty$ is exponentially fast in the separation from the boundary of $\Lambda_k$.
\end{lemma}
\begin{proof}
Fix $U\Subset\Lambda_k$ and write $\partial_k:=\partial\Lambda_k$. By uniform lattice gap on slabs and AF\,–\,free gap persistence on fixed regions, the continuum generator on $\Lambda_k$ has spectral gap $\gamma_*>0$ independent of $\ell$. OS3 (clustering) then yields, for time\,–\,ordered products localized in $U$ and any boundary observable $B$ supported in a $t$\,–\,neighborhood of $\partial_k$, an exponential estimate $|\langle \Theta B\,O\rangle - \langle\Theta B\rangle\langle O\rangle|\le C e^{-\gamma_* t}$ with $C$ depending only on $U,n,N$. Using boundary\,–\,condition robustness to localize the $\ell$\,–\,dependence to a collar of $\partial_k$ and integrating the collar thickness $t\ge \mathrm{dist}(U,\partial_k)$ gives the stated bound with $\alpha\le \gamma_*$. The constants are uniform because all estimates are taken on the fixed region $\Lambda_k$.
\end{proof}
\begin{proof}
By Proposition~\ref{prop:bc-robust}, on any fixed $\Lambda_k$ the local Schwinger functions are independent of boundary conditions in larger van Hove boxes up to $o_{L\to\infty}(1)$ errors, uniformly in the lattice spacing. Proposition~\ref{prop:embedding-independence} removes embedding choices. The AF-free uniqueness criterion (Proposition~\ref{prop:af-free-uniqueness}) identifies limits along any van Hove diagonal. Passing to the continuum within $\Lambda_k$ yields equality of the $k$- and $\ell$-based definitions on $\mathfrak A_0(\Lambda_k)$.
\end{proof}

\subsection*{No-go: Schedule-Independent Local Limits on Fixed Regions are Trivial}
Fix a bounded physical region $U\Subset\mathbb R^4$ and let $\mathcal A_U$ be the gauge-invariant cylinder algebra generated by finitely many Wilson loops supported in $U$. For a lattice spacing $a>0$ and coupling $\beta>0$, let $\mathbb E_{\beta,a}[\,\cdot\,]$ denote expectation for Wilson SU$(N)$ lattice YM.

\begin{lemma}[Uniform positivity of single-link conditionals]\label{lem:link-minorization}
For any link $\ell$ and any boundary of the other links, the conditional density of $U_\ell\in G$ is proportional to $\exp\!\big(\tfrac{\beta}{N}\Re\Tr(U_\ell V_\ell)\big)$ with $\|V_\ell\|\le 6$. Hence there is $c_\downarrow(\beta)=e^{-12\beta}$ such that for all measurable $A\subset G$,
\[
\mathbb P(U_\ell\in A\mid \text{rest})\ \ge\ c_\downarrow(\beta)\,\frac{\mu_{\rm Haar}(A)}{\mu_{\rm Haar}(G)}.
\]
\end{lemma}

\begin{proposition}[Plaquette away from identity at any fixed finite $\beta$]\label{prop:plaq-away}
For any fixed finite $\beta>0$, there exist $\delta_0\in(0,1)$ and $\theta(\beta)>0$ such that for any plaquette $p\subset U$,
\[
\mathbb E_{\beta,a}\Big[\tfrac{1}{N}\Re\Tr\,U_p\Big]\ \le\ 1-\delta_0\,\theta(\beta)\qquad\text{for all }a>0.
\]
\end{proposition}

\begin{lemma}[Large-$\beta$ concentration on a finite set of plaquettes]\label{lem:large-beta}
For any finite set of plaquettes $\Lambda\subset U$ and any $\varepsilon>0$, there is $\beta_\varepsilon$ such that for $\beta\ge\beta_\varepsilon$,
\[
\mathbb P_{\beta,a}\Big(\max_{p\in\Lambda}\|U_p-I\|\le \varepsilon\Big)\ \ge\ 1-e^{-c\beta},\quad c>0.
\]
In particular, for fixed $p\in\Lambda$, $\lim_{\beta\to\infty}\mathbb E_{\beta,a}[\tfrac{1}{N}\Re\Tr\,U_p]=1$ uniformly in $a$.
\end{lemma}

\begin{theorem}[No schedule-independent local limit unless trivial]\label{U4D:no-go}
Let $p(a)\subset U$ be a plaquette for mesh $a$. Define schedules $\beta_1(a)\equiv\beta_0\in(0,\infty)$ and $\beta_2(a)\to\infty$ as $a\downarrow 0$. Then
\[
\limsup_{a\downarrow 0}\ \mathbb E_{\beta_1(a),a}\Big[\tfrac{1}{N}\Re\Tr\,U_{p(a)}\Big]\ \le\ 1-\delta_0\,\theta(\beta_0)\ <\ 1,
\qquad
\lim_{a\downarrow 0}\ \mathbb E_{\beta_2(a),a}\Big[\tfrac{1}{N}\Re\Tr\,U_{p(a)}\Big]\ =\ 1.
\]
Hence there is no unique, schedule-independent limit on $\mathcal A_U$ unless the limit is trivial (ultralocal).
\end{theorem}

\begin{mdframed}[linewidth=0.5pt, linecolor=blue!30, backgroundcolor=blue!3, roundcorner=2pt, innertopmargin=8pt, innerbottommargin=8pt, skipabove=10pt, skipbelow=10pt]
\noindent\textbf{Remark.} The correct uniqueness notion is \emph{manifold uniqueness}: restrict to schedules that fix a renormalized local datum at a physical scale and prove uniqueness on that manifold; this is the route used elsewhere in the paper.
\end{mdframed}

\subsection{Explicit AF-Style Scaling $\beta(a)$ and Tightness/Convergence}\label{subsec:beta-schedule}

For concreteness we record a monotone scaling schedule $\beta(a)$ and prove tightness and convergence of local Schwinger functions along any van Hove net with $a\downarrow 0$ and $L(a) a\to\infty$.

\begin{definition}[AF-style schedule]\label{def:beta-schedule}
Fix $a_0>0$ and constants $b_0>0$, $c_\beta\ge 1$. Define
\[
  \beta(a)\ :=\ c_\beta\,\log\Big(\frac{a_0}{a}\Big)\qquad \text{for } a\in(0,a_0].
\]
This is monotone nondecreasing, $\beta(a)\to\infty$ as $a\downarrow 0$, and stays $\ge 1$ on $(0,a_0]$.
\end{definition}

We do \emph{not} require perturbative AF identities; the role of $\beta(a)$ is solely to pin a concrete trajectory for which our nonperturbative bounds (UEI, equicontinuity, interface minorization) are uniform in $a$.

\begin{theorem}[Tightness and convergence along $\beta(a)$]\label{thm:tightness-beta}
Let $R\Subset\mathbb R^4$ be fixed. Along any van Hove scaling net $(a,L(a))$ with $\beta=\beta(a)$ from Definition~\ref{def:beta-schedule}, the family of time-zero local Schwinger functions $\{S_{n,a,L}\}_{a,L}$ restricted to observables supported in $R$ is tight and precompact in the product topology over loop/cylinder functionals. All subsequential limits coincide, hence $S_{n,a,L}\to S_n$ pointwise on $R$.
\end{theorem}
\begin{proof}
Fixed-region U1 (Theorem~\ref{thm:uei-fixed-region}) supplies the needed tightness/UEI input on $R$; under the raw small-field route A one may use Corollary~\ref{cor:uei-explicit-constants} for explicit action-moment constants, while under Track-J one instead uses Lemmas~\ref{lem:U1-J1-smoothed-lsi} and \ref{lem:U1-J2-universality} to obtain tightness via smoothing/universality. Proposition~\ref{prop:OS0-poly} yields polynomial OS0 bounds uniform in $(a,L)$. The equicontinuity modulus Lemma~\ref{lem:eqc-modulus} applies uniformly on $R$. By Prokhorov/Arzel\`a--Ascoli for cylinder functionals, tightness and precompactness follow. Embedding-independence (Proposition~\ref{prop:embedding-independence}) and the AF-free uniqueness criterion (Proposition~\ref{prop:af-free-uniqueness}) identify all subsequential limits, giving convergence.
\end{proof}

\begin{corollary}[Convergence on $\mathbb R^4$]\label{cor:global-convergence}
Along $\beta(a)$, the global construction of Section~\ref{sec:global-R4} produces the same Schwinger functions as any other admissible monotone schedule satisfying the uniform hypotheses. In particular, the global measure $\mu_{\mathrm{YM}}$ is independent of the schedule within this class.
\end{corollary}

\begin{corollary}[Scheme and embedding independence; unitary equivalence]\label{cor:scheme-independence-global}
Let $\{E_a\}$ and $\{J_a\}$ be two admissible directed polygonal/smoothing embedding schemes on fixed regions, both OS\,–\,reflection compatible and satisfying the uniform hypotheses (UEI/OS0, NRC inputs). Let $\mu^{(E)},\mu^{(J)}$ be the corresponding global OS measures obtained by the local limits and globalization of Section~\ref{sec:global-R4}, and $\mathcal H^{(E)},\mathcal H^{(J)}$ their OS/GNS Hilbert spaces with generators $H^{(E)},H^{(J)}$. Then the global Schwinger functions coincide, $S_n^{(E)}=S_n^{(J)}$ on the cylinder algebra, and there exists a unitary $U:\mathcal H^{(E)}\to\mathcal H^{(J)}$ such that $U\Omega^{(E)}=\Omega^{(J)}$, $U e^{-t H^{(E)}}= e^{-t H^{(J)}} U$ for all $t\ge 0$, and $U\,[O]^{(E)}=[O]^{(J)}$ for every time\,–\,zero gauge\,–\,invariant local observable $O$.
\end{corollary}
\begin{proof}
On each fixed region $R$, Proposition~\ref{prop:embedding-independence} identifies the continuum Schwinger functions across embedding schemes and Proposition~\ref{prop:unitary-equivalence} yields a local unitary intertwining of the OS/GNS realizations and semigroups. Boundary\,–\,condition robustness (Proposition~\ref{prop:bc-robust}) and consistency on overlaps (Proposition~\ref{prop:consistency-overlaps}) pass these identifications to the directed system $\{\Lambda_k\}$, hence to the global measure $\mu_{\mathrm{YM}}$ and its OS/GNS space. Therefore $S_n^{(E)}=S_n^{(J)}$ globally and the induced unitary intertwines the global semigroups and the time\,–\,zero local cores, as stated.
\end{proof}

\subsection{AF-Free Calibrated NRC Alternative}\label{subsec:af-free-alternative}

Independently of any schedule, one may work entirely AF-free using calibrated norm--resolvent convergence:

\begin{theorem}[AF-free calibrated NRC and uniqueness]\label{thm:af-free-calibrated}
Fix $R\Subset\mathbb R^4$. Suppose: (i) UEI and OS0 bounds hold uniformly on $R$; (ii) the interface kernel admits a Doeblin split with $t_0,\theta_*>0$ independent of $(a,L)$ on $R$; (iii) the embedded resolvents $R_{a,L}(z_0)=I_{a,L}(H_{a,L}-z_0)^{-1}I_{a,L}^*$ form a Cauchy net in operator norm on $\mathcal H_R$ for some $z_0\in\mathbb C\setminus\mathbb R$. Then $R_{a,L}(z)\to R(z)$ in operator norm for all $z$ in compact subsets of $\mathbb C\setminus\mathbb R$, the semigroups $I_{a,L}e^{-tH_{a,L}}I_{a,L}^*$ converge strongly to $e^{-tH_R}$, and the Schwinger functions on $R$ converge uniquely along any van Hove net. The induced global measure $\mu_{\mathrm{YM}}$ of Section~\ref{sec:global-R4} is recovered without reference to $\beta(a)$.
\end{theorem}
\begin{proof}
Combine the Cauchy criterion in Lemma~\ref{lem:af-free-cauchy} with collective-compactness (Proposition~\ref{prop:collective-compactness}) and the holomorphic functional calculus to extend operator-norm convergence from a point $z_0$ to compact nonreal sets. Strong convergence of semigroups follows from standard Laplace inversion bounds using uniform OS0. Uniqueness on $R$ is Proposition~\ref{prop:af-free-uniqueness}. Consistency on overlaps (Proposition~\ref{prop:consistency-overlaps}) and Kolmogorov extension then reconstruct the global $\mu_{\mathrm{YM}}$.
\end{proof}

\begin{theorem}[Kolmogorov/Minlos extension to a global Euclidean measure]\label{thm:kolmogorov-global}
The consistent family $\{S_n\}$ on the cylinder algebra generated by $\mathfrak A_0$ extends to a unique probability measure $\mu_{\mathrm{YM}}$ on the cylinder $\sigma$--algebra of gauge-invariant observables on $\mathbb R^4$. In particular, $S_n(O_1,\dots,O_n)=\mathbb E_{\mu_{\mathrm{YM}}}[O_1\cdots O_n]$ for all finite families from $\mathfrak A_0$.
\end{theorem}
\begin{proof}
Consistency (Proposition~\ref{prop:consistency-overlaps}) and uniform OS0 polynomial bounds (Proposition~\ref{prop:OS0-poly}) imply tightness and a Daniell--Kolmogorov consistent family on the directed system $\{\Lambda_k\}$. Kolmogorov extension (or Minlos/Prokhorov for the corresponding cylinder space) yields $\mu_{\mathrm{YM}}$ on the projective limit. Uniqueness follows from the uniqueness of finite-dimensional distributions on the cylinder algebra.
\end{proof}

Define the global OS/GNS Hilbert space $\mathcal H_{\mathrm{OS}}$ as the completion of $\mathfrak A_0/\mathcal N$ with inner product $\langle [A],[B]\rangle:=\mathbb E_{\mu_{\mathrm{YM}}}[\Theta(A)B]$, where $\mathcal N:=\{A:\ \mathbb E_{\mu_{\mathrm{YM}}}[\Theta(A)A]=0\}$. Time translations define a contraction semigroup $e^{-tH}$ on $\mathcal H_{\mathrm{OS}}$ with generator $H\ge 0$.

\subsection{Global OS Axioms on $\mathbb R^4$}

\begin{lemma}[Reflection positivity stability under directed cylinder limits]\label{lem:rp-stability-projective}
Let $\{\Lambda_k\}$ be a van Hove exhaustion and $\mu_{a,L}$ lattice measures with OS reflection positivity for each $(a,L)$. Suppose $\mu_{a,L}\Rightarrow \mu_k$ weakly on $\Lambda_k$ for each $k$, and the family $\{\mu_k\}_k$ is consistent on overlaps. Then for any polynomial $P$ of time-$\ge 0$ local observables supported in some $\Lambda_k$,
\[
  \int \Theta P\,\overline{P}\,d\mu\ =\ \lim_{\ell\to\infty}\lim_{(a,L)} \int \Theta P\,\overline{P}\,d\mu_{a,L}\ \ge\ 0,
\]
where $\mu$ is the Kolmogorov/Minlos extension of $\{\mu_k\}$. Hence $\mu$ is OS--reflection positive.
\end{lemma}
\begin{proof}
Fix $P$ supported in $\Lambda_k$. For $\ell\ge k$, view $P$ as a polynomial on $\Lambda_\ell$ by extension by identity. Reflection positivity holds for each $\mu_{a,L}$. Weak convergence $\mu_{a,L}\Rightarrow \mu_\ell$ implies the integrals converge. Consistency on overlaps yields independence of $\ell\ge k$. Passing to the projective limit measure $\mu$ preserves the inequality.
\end{proof}

\begin{lemma}[Separable global OS/GNS Hilbert space]\label{lem:separable}
Let $\mathfrak A_0$ be the algebraic union of time-zero local gauge-invariant cylinders generated by Wilson loops with piecewise-linear edges with rational coordinates and rational coefficients. Then $\mathfrak A_0$ is countable. Its OS/GNS completion $\mathcal H_{\mathrm{OS}}$ is separable.
\end{lemma}
\begin{proof}
There are countably many rational polyloops and finite products with rational coefficients. The quotient by the OS null space preserves separability, and the completion of a pre-Hilbert space with a countable dense set is separable.
\end{proof}

\begin{proposition}[Haag--Kastler net on $\mathbb R^4$]\label{prop:haag-kastler}
For each bounded $O\Subset\mathbb R^4$, let $\mathcal A(O)$ be the von Neumann algebra generated by time-zero local gauge-invariant cylinders supported in $O$ and their Euclidean translates. Then:
\begin{itemize}
  \item Isotony: $O_1\subset O_2\Rightarrow \mathcal A(O_1)\subset \mathcal A(O_2)$.
  \item Locality: if $O_1$ and $O_2$ are spacelike separated after OS$\to$Wightman, then $[\mathcal A(O_1),\mathcal A(O_2)]=\{0\}$.
  \item Covariance: Euclidean motions act by automorphisms, continued to a unitary Poincar\'e action on the Wightman space.
\end{itemize}
\end{proposition}
\begin{proof}
Isotony is by construction. Locality follows from Corollary~\ref{cor:microcausality}. Covariance follows from OS1 (Theorem~\ref{thm:os1-unconditional}) and analytic continuation to the Wightman representation.
\end{proof}

\begin{theorem}[Global OS0--OS5 (conditional on fixed-region U1/OS1 inputs; AF--free NRC)]\label{thm:global-OS}
Let $\{\mu_{a,L}\}$ be Wilson lattice measures along a van Hove window. Fixed-region U1 tightness/UEI on $R$ (Theorem~\ref{thm:uei-fixed-region}; with explicit action-moment constants available under the raw route A via Corollary~\ref{cor:uei-explicit-constants}) and AF--free NRC on fixed regions (Theorems~\ref{thm:nrc-embeddings}, \ref{thm:nrc-operator-norm}), together with the proved Cauchy/defect/projection inputs (Lemmas~\ref{lem:af-free-cauchy}, Thm.~\ref{thm:quant-calibrated-af-free-nrc}(D), \ref{lem:low-energy-proj}), imply that the continuum limit $\mu_{\mathrm{YM}}$ exists on cylinder sets and its Schwinger functions $\{S_n\}$ satisfy OS0--OS5 globally on $\mathbb R^4$:
\begin{itemize}
  \item OS0 (temperedness): Uniform polynomial bounds (Proposition~\ref{prop:OS0-poly}) pass to the limit by consistency on overlaps (Proposition~\ref{prop:consistency-overlaps}).
  \item OS2 (reflection positivity): For any polynomial $P$ supported in $t\ge 0$, $\langle\Theta P\,\overline{P}\rangle_{\mu}=\lim_{a,L}\langle\Theta P\,\overline{P}\rangle_{\mu_{a,L}}\ge 0$.
  \item OS3 (clustering): The uniform lattice gap yields exponential clustering on each $\Lambda_k$ (Proposition~\ref{prop:gap-to-cluster}); AF--free NRC and gap persistence (Theorem~\ref{thm:gap-persist-cont}) transport the decay rate to the continuum generator $H$, giving global clustering under the NRC hypotheses.
  \item OS4 (permutation symmetry): Symmetry of lattice Schwinger functions is preserved under limits.
  \item OS1 (Euclidean invariance): Translation invariance follows from directed consistency; full rotational invariance follows from equicontinuity and isotropy restoration on fixed regions (Thms.~\ref{thm:U1-lsi-uei}, \ref{thm:os1-unconditional}; Lem.~\ref{lem:U1-tree-bounds}; Cor.~\ref{cor:U1-uei}; Lem.~\ref{lem:isotropy-restore}).
  \item OS5 (unique vacuum): The spectral gap implies a one-dimensional vacuum sector globally.
\end{itemize}
\end{theorem}
\begin{lemma}[Unique vacuum from global clustering and reflection positivity]\label{lem:unique-vacuum}
Let $\{S_n\}$ satisfy OS0--OS3 globally and OS2. Suppose there exist $\gamma_*>0$ and $C<\infty$ such that for every centered local observable $O$ and $t\ge 0$,
\[
  \big|\langle O(t)\,O(0)\rangle - \langle O\rangle^2\big|\ \le\ C\,e^{-\gamma_* t}.
\]
Then the OS/GNS Hamiltonian $H\ge 0$ has a one-dimensional null space spanned by the vacuum $\Omega$ and $\operatorname{spec}(H)\subset\{0\}\cup[\gamma_*,\infty)$.
\end{lemma}
\begin{proof}
In the OS/GNS representation, reflection positivity and clustering imply that $\Omega$ is cyclic and separating for the time-zero algebra, and the connected two-point function is the Laplace transform of a positive spectral measure. Exponential decay with rate $\gamma_*$ forces the measure to vanish in $(0,\gamma_*)$, yielding $\operatorname{spec}(H)\cap(0,\gamma_*)=\varnothing$. If the vacuum subspace were larger than one-dimensional, there would be a nontrivial zero-energy vector orthogonal to $\Omega$, contradicting clustering for suitable $O$.
\end{proof}
\begin{lemma}[Compact-group averaging preserves OS axioms and gap]\label{lem:group-avg}
Let $G$ be a compact group acting by Euclidean isometries on observables, and let $\{S_n\}$ satisfy OS0--OS5 with mass gap $\Delta>0$. Then the averaged family $\{\overline{S}_n\}$ defined by $\overline{S}_n:=\int_G S_n\circ g\,dg$ also satisfies OS0--OS5 with the same gap.
\end{lemma}
\begin{proof}
Temperedness and permutation symmetry are preserved by dominated convergence. Reflection positivity is convex: $\int \langle\Theta(P)P\rangle_g\,dg\ge 0$. Clustering persists since $\int e^{-\Delta t}\,dg=e^{-\Delta t}$. In the OS/GNS picture, the group acts unitarily and commutes with time translations, so the spectral gap of $H$ is unchanged under averaging the vacuum functional.
\end{proof}
\subsection{OS $\to$ Wightman and Global Mass Gap}

\begin{theorem}[OS reconstruction and Poincar\'e invariance (conditional on OS0--OS5; any compact simple $G$)]\label{thm:os-to-wightman-global}
From $\{S_n\}$ as in Theorem~\ref{thm:global-OS}, the Osterwalder--Schrader reconstruction yields a Wightman QFT on Minkowski space with unitary positive-energy representation of the Poincar\'e group and local gauge-invariant Wightman fields. The Hamiltonian has spectrum $\{0\}\cup[\gamma_*,\infty)$ with $\gamma_*>0$.
\end{theorem}
\begin{lemma}[Finite upper gap $m<\infty$]\label{lem:finite-upper-gap}
Let $\mathcal H$ be the Wightman Hilbert space with Hamiltonian $H\ge 0$ and unique vacuum $\Omega$. If there exists a local observable $\mathcal O$ with $\langle \Omega,\mathcal O\,\Omega\rangle=0$ and $\langle \Omega,\mathcal O^*\mathcal O\,\Omega\rangle>0$, then the spectral measure of $H$ in the state $\mathcal O\Omega$ puts positive mass in $(0,\infty)$; in particular, the supremum $m:=\sup\{\Delta>0:\ (0,\Delta)\cap \operatorname{spec}(H)=\varnothing\}$ is finite.
\end{lemma}
\begin{proof}
By the spectral theorem, $\langle \Omega,\mathcal O^* e^{-tH} \mathcal O\,\Omega\rangle=\int_{[0,\infty)}e^{-tE}\,d\mu_{\mathcal O}(E)$ with a nonzero finite measure $\mu_{\mathcal O}$. If $\mu_{\mathcal O}$ were supported at $\{0\}$, then $\mathcal O\Omega$ would lie in the vacuum subspace, contradicting $\langle \Omega,\mathcal O\,\Omega\rangle=0$ and uniqueness of the vacuum. Hence $\mu_{\mathcal O}((0,\infty))>0$, so $\operatorname{spec}(H)$ contains some $E_1>0$, and therefore $m\le E_1<\infty$.
\end{proof}
\noindent See also Corollaries~\ref{cor:microcausality} (microcausality), \ref{cor:wightman-local-gap} (Wightman local fields and gap), and \ref{cor:minkowski-massgap} (physical Minkowski mass gap).
\begin{proof}
Apply the classical OS reconstruction to $\mu_{\mathrm{YM}}$ using reflection positivity, temperedness, symmetry, and Euclidean invariance. Exponential clustering and OS5 give a unique vacuum and spectrum condition. The uniform slab contraction/gap (Theorem~\ref{thm:gap-persist-cont}) transfers to the global generator by the core/inductive-limit argument in the proof of Theorem~\ref{thm:global-OS}. The resulting Wightman theory inherits Poincar\'e covariance from Euclidean invariance by analytic continuation.
\end{proof}

\begin{theorem}[Wightman axioms and spectral condition (conditional on OS0--OS5)]\label{thm:wightman-axioms}
Let $\mu_{\mathrm{YM}}$ be the global Euclidean measure of Theorem~\ref{thm:kolmogorov-global} with Schwinger functions satisfying Theorem~\ref{thm:global-OS}. Then the OS reconstruction produces Wightman distributions $\{W_n\}$ and a separable Hilbert space $\mathcal H$ such that:
\begin{itemize}
  \item (W0) temperedness: $W_n\in \mathcal S'(\mathbb R^{4n})$;
  \item (W1) Poincar\'e covariance: there is a unitary representation $U$ of the proper orthochronous Poincar\'e group with $U(a,\Lambda)\,\Phi(x)\,U(a,\Lambda)^{-1}=\Phi(\Lambda x+a)$ on fields;
  \item (W2) spectrum condition: the joint spectrum of the energy-momentum operators lies in the closed forward cone $\overline{V}_+$; in particular the Hamiltonian has spectrum $\{0\}\cup[\gamma_*,\infty)$;
  \item (W3) locality: smeared local gauge-invariant fields commute at spacelike separation;
  \item (W4) vacuum: there is a unique (up to phase) Poincar\'e-invariant vacuum $\Omega$ cyclic for the field algebra.
\end{itemize}
\end{theorem}
\begin{proof}
OS0 implies temperedness of Schwinger functions; analytic continuation yields tempered Wightman distributions. OS1 (via equicontinuity and isotropy restoration on fixed regions; see Lemma~\ref{lem:isotropy-restore}) provides full Euclidean invariance and hence Poincar\'e covariance after continuation. OS2 gives a positive-definite inner product leading to the GNS construction. OS3 and OS5 imply uniqueness of the vacuum and exponential clustering, which yields the spectral condition together with the nonzero mass gap from Theorem~\ref{thm:os-to-wightman-global}. Locality follows from the standard OS\,$\to$\,Wightman locality theorem applied to local gauge-invariant smeared fields (Corollary~\ref{cor:os-local-fields}).
\end{proof}

\begin{corollary}[Microcausality for gauge\,--\,invariant local fields]\label{cor:microcausality}
Let $\Phi,\Xi$ be the gauge\,--\,invariant local fields constructed in Section~\ref{subsec:local-fields-tempered}, and let $\mathcal I(\chi):=\int \chi\,\mathrm{Tr}(F^R_{\mu\nu}F^{R,\mu\nu})$ as in Corollary~\ref{cor:locality-FR}. If $f_1,f_2\in\mathcal S(\mathbb R^4)$ (resp. $\varphi_1,\varphi_2\in\mathcal S(\mathbb R^4,\wedge^2\mathbb R^4)$, $\chi_1,\chi_2\in\mathcal S(\mathbb R^4)$) have spacelike separated supports, then on the time\,–\,zero local core
\[
  [\,\Phi(f_1),\Phi(f_2)\,]\ =\ 0,\qquad [\,\Xi(\varphi_1),\Xi(\varphi_2)\,]\ =\ 0,\qquad [\,\mathcal I(\chi_1),\mathcal I(\chi_2)\,]\ =\ 0.
\]
These equalities extend by continuity to the operator closures.
\end{corollary}
\begin{proof}
By Corollary~\ref{cor:os-local-fields}, the Schwinger functions of polynomials in $\Phi,\Xi$ satisfy the OS axioms. OS4 implies Euclidean locality (symmetry under permutations preserving Euclidean time order); by the Osterwalder\,–\,Schrader reconstruction and analytic continuation, this yields vanishing commutators at spacelike separation for the corresponding Wightman fields. For $\mathcal I(\chi)$, Corollary~\ref{cor:locality-FR} gives Euclidean locality for gauge\,–\,invariant smearings built from $F^R$, hence the same OS\,$\to$\,Wightman argument applies. The statements on the common core extend to closures by the graph bounds in Proposition~\ref{prop:field-closability}.
\end{proof}

\subsection{Global Spectral Gap on $\mathbb R^4$}

\begin{lemma}[Inductive-limit spectral transfer]\label{lem:inductive-spectral}
Let $\{\mathcal H_k\}_{k\in\mathbb N}$ be an increasing family of Hilbert spaces with isometric inclusions into the inductive limit Hilbert space $\mathcal H$, and let $P_k(t)=e^{-tH_k}$ be contraction semigroups with generators $H_k\ge 0$. Assume:
\begin{itemize}
  \item Consistency: For $j\le k$, $P_k(t)|_{\mathcal H_j}=P_j(t)$ under the inclusion $\mathcal H_j\hookrightarrow\mathcal H_k$.
  \item Vacuum sectors: Each $\mathcal H_k$ splits as $\mathbb C\Omega_k\oplus \mathcal H_{k,0}$ with $P_k(t)\Omega_k=\Omega_k$.
  \item Uniform mean-zero contraction: There exists $\gamma_*>0$ such that for all $t\ge 0$,
  \[
    \big\|P_k(t)(I-\vert\Omega_k\rangle\langle\Omega_k\vert)\big\|\ \le\ e^{-\gamma_* t}\qquad(\forall k\in\mathbb N).
  \]
\end{itemize}
Then the inductive-limit semigroup $P(t)$ on $\mathcal H$ satisfies
\[
  \big\|P(t)(I-\vert\Omega\rangle\langle\Omega\vert)\big\|\ \le\ e^{-\gamma_* t},\qquad t\ge 0,
\]
with generator $H\ge 0$ obeying $\operatorname{spec}(H)\subset\{0\}\cup[\gamma_*,\infty)$.
\end{lemma}
\begin{proof}
Let $\mathcal D:=\bigcup_k \mathcal H_k$ be the inductive-limit core; $\mathcal D$ is dense in $\mathcal H$. On each $\mathcal H_k$, the bound holds. For $\psi\in \mathcal D$ there exists $k$ with $\psi\in\mathcal H_k$, hence
\[
  \Vert P(t)(I-\vert\Omega\rangle\langle\Omega\vert)\psi\Vert
  \,=\,\Vert P_k(t)(I-\vert\Omega_k\rangle\langle\Omega_k\vert)\psi\Vert
  \,\le\, e^{-\gamma_* t}\,\Vert\psi\Vert.
\]
By density and uniformity in $\psi\in\mathcal D$, the estimate extends to all of $\mathcal H$ by continuity of $P(t)$. The spectral inclusion follows from the spectral mapping theorem for $C_0$-semigroups: if $\sigma(H)\cap(0,\gamma_*)\ne\varnothing$, then the restriction of $P(t)$ to the mean-zero subspace would have norm larger than $e^{-\gamma_* t}$ for some $t>0$, contradicting the bound.
\end{proof}

\begin{theorem}[Global Euclidean spectral gap, boundary/region independent (any compact simple $G$)]\label{thm:global-gap-uncond}
Let $G$ be a compact simple group. With the global OS construction of Section~\ref{sec:global-R4}, there exists $\gamma_*>0$ (depending only on $(R_*,a_0,G)$ via $(\theta_*,t_0,\lambda_1)$) such that the Euclidean generator $H$ on the global OS/GNS Hilbert space satisfies
\[
\operatorname{spec}(H)\ \subset\ \{0\}\cup[\gamma_*,\infty)\,.
\]
Equivalently, for all $t\ge 0$,
\[
  \big\|e^{-tH}\,(I-\vert\Omega\rangle\langle\Omega\vert)\big\|\ \le\ e^{-\gamma_*\, t}\,.
\]
The bound is independent of the exhaustion, region choice, and boundary conditions.
\end{theorem}
\begin{proof}
Step 1 (uniform local contraction). On finite tori and fixed slabs, along any scaling schedule satisfying \eqref{eq:ucis-sw-window}, UCIS$_{\rm SW}$ (Theorem~\ref{thm:ucis-sw}) yields a fixed-physical-time heat-kernel minorization for the coarse interface after $M(a)$ ticks, hence a fixed-physical-time parity-odd contraction (Theorem~\ref{thm:ucis-sw-odd-subspace}). This supplies the required slab gap input with constants uniform in the volume/boundary data on fixed slabs (within the scaling window).

Step 2 (thermodynamic limit at fixed spacing). The contraction estimate is volume-uniform; the thermodynamic limit preserves the gap and clustering (Theorem~\ref{thm:thermo}). Boundary-condition robustness (Proposition~\ref{prop:bc-robust}) ensures independence of outer boundary choices for local observables.

Step 3 (continuum limit on fixed regions). On any fixed $R\Subset\mathbb R^4$, UEI and OS0 yield tightness and equicontinuity; the AF-free calibrated NRC (Theorem~\ref{thm:af-free-calibrated}) provides operator-norm resolvent convergence and uniqueness of local limits along van Hove nets, independent of any $\beta(a)$ schedule. Gap persistence (Theorem~\ref{thm:gap-persist-cont}) transports the uniform lower bound $\gamma_*$ from local lattices to the continuum generator $H_R$ on $R$.

Step 4 (globalization). Consistency on overlaps (Proposition~\ref{prop:consistency-overlaps}) identifies the local semigroups on the directed system $\{\Lambda_k\}$; the global OS/GNS space is the inductive-limit completion. Apply Lemma~\ref{lem:inductive-spectral} to transfer the uniform mean-zero contraction to the global semigroup. Equivalently, the semigroup bound
\[
  \big\|e^{-tH_{\Lambda_k}}\,(I-\vert\Omega_{\Lambda_k}\rangle\langle\Omega_{\Lambda_k}\vert)\big\|\ \le\ e^{-\gamma_* t}
\]
holds on $\mathcal H$; hence the spectral inclusion follows from the spectral mapping theorem.

All constants depend only on the slab geometry $(R_*,a_0)$ and group data; the conclusion uses NRC and OS1 on fixed regions as stated earlier.
\end{proof}

\begin{corollary}[Physical Minkowski mass gap]\label{cor:minkowski-massgap}
Under OS reconstruction (Theorem~\ref{thm:os-to-wightman-global}), the Wightman Hamiltonian on Minkowski space has the same strictly positive mass gap $\gamma_*>0$:
\[
  \operatorname{spec}(H_{\mathrm{Mink}})\ \subset\ \{0\}\cup[\gamma_*,\infty)\,.
\]
In particular, the spectral condition holds and the mass gap is independent of region/boundary choices used in the Euclidean construction.
\end{corollary}

\begin{theorem}[Clay–Jaffe–Witten compliance: existence and mass gap on $\mathbb R^4$ (any compact simple $G$)]\label{thm:clay-compliance}
Let $G$ be any compact simple Lie group. There exists a nontrivial quantum Yang–Mills theory on $\mathbb R^4$ with the following properties:
\begin{itemize}
  \item (Euclidean axioms) The global Schwinger functions satisfy OS0--OS5 (Theorem~\ref{thm:global-OS}); reflection positivity is preserved in the directed limit (Lemma~\ref{lem:rp-stability-projective}).
  \item (OS$\to$Wightman) The OS reconstruction yields a separable Hilbert space carrying a unitary positive-energy Poincar\'e representation with local gauge-invariant fields and microcausality (Theorems~\ref{thm:os-to-wightman-global}, \ref{thm:wightman-axioms}; Corollary~\ref{cor:microcausality}; Lemma~\ref{lem:separable}).
  \item (Short distance) Renormalized composite operators exist; a gauge-invariant OPE holds with AF-predicted local singularities and Wilson-coefficient CS flow consistent with asymptotic freedom; perturbative coefficients match to all orders in the chosen scheme (Theorem~\ref{thm:renorm-composites}, Theorem~\ref{thm:ope-gi}, Corollary~\ref{cor:cs-wilson}, Proposition~\ref{prop:pert-matching}, Theorem~\ref{thm:af-matching}).
  \item (Stress tensor) A local, conserved, symmetric $T_{\mu\nu}$ exists generating translations/rotations (Theorems~\ref{thm:T-properties}, \ref{thm:T-generators}; Lemma~\ref{lem:T-integral-domain}).
  \item (Mass gap and clustering) The Euclidean generator and the Minkowski Hamiltonian have spectrum $\{0\}\cup[\Delta,\infty)$ with $\Delta=\gamma_*>0$ (Theorem~\ref{thm:global-gap-uncond}, Corollary~\ref{cor:minkowski-massgap}); exponential clustering holds for centered local operators (Proposition~\ref{prop:gap-to-cluster}); the upper gap parameter satisfies $m<\infty$ (Lemma~\ref{lem:finite-upper-gap}).
  \item (Gauge structure) The gauge-invariant local net (Proposition~\ref{prop:haag-kastler}) and Ward/BRST/Gauss law statements hold (Theorems~\ref{thm:ward}, \ref{thm:U8-ward-cont}).
\end{itemize}
In particular, the Jaffe–Witten requirements for existence and a positive mass gap on $\mathbb R^4$ are satisfied.
\end{theorem}
\begin{proof}
Assemble: OS0--OS5 globally by Theorem~\ref{thm:global-OS} and Lemma~\ref{lem:rp-stability-projective}; OS$\to$Wightman by Theorems~\ref{thm:os-to-wightman-global}, \ref{thm:wightman-axioms}, separability by Lemma~\ref{lem:separable}; short-distance properties by Theorem~\ref{thm:renorm-composites}, Theorem~\ref{thm:ope-gi}, Corollary~\ref{cor:cs-wilson}, Proposition~\ref{prop:pert-matching}, and Theorem~\ref{thm:af-matching}; stress tensor by Theorems~\ref{thm:T-properties}, \ref{thm:T-generators}, Lemma~\ref{lem:T-integral-domain}; mass gap by Theorem~\ref{thm:global-gap-uncond}, Lemma~\ref{lem:inductive-spectral}, Corollary~\ref{cor:minkowski-massgap}, clustering by Proposition~\ref{prop:gap-to-cluster}, and finiteness of the upper gap by Lemma~\ref{lem:finite-upper-gap}; gauge structure by Theorems~\ref{thm:ward}, \ref{thm:U8-ward-cont} and Proposition~\ref{prop:haag-kastler}. Nontriviality is ensured by the non-Gaussianity statements and renormalized local fields (e.g., Proposition~\ref{prop:nonzero-cumulant4}, Theorem~\ref{thm:U10-renorm-F}).
\end{proof}
\begin{proof}
Theorem~\ref{thm:global-gap-uncond} gives the Euclidean gap; Theorem~\ref{thm:os-to-wightman-global} transfers it to the Minkowski theory by analytic continuation and OS\,$\to$\,Wightman. Uniqueness of the vacuum (OS5) yields the ground state at energy $0$.
\end{proof}

\subsection*{Optional B: Continuum OS reconstruction from a scaling window}

This option outlines a rigorous procedure for constructing a continuum QFT in four dimensions from a family of lattice gauge theories, given tightness and uniform locality/clustering bounds independent of $\varepsilon$.

\paragraph{Existence of the continuum limit measure.}
Assuming tightness of loop observables $W_{\Gamma,\varepsilon}$, Prokhorov compactness yields a subsequence $\varepsilon_k\to 0$ along which the lattice measures converge weakly to a probability measure $\mu$. For any finite collection of loops $\Gamma_1,\dots,\Gamma_n$, the Schwinger functions
\[
  S_n(\Gamma_1,\dots,\Gamma_n):=\lim_{\varepsilon\to 0}\,\langle W_{\Gamma_1,\varepsilon}\cdots W_{\Gamma_n,\varepsilon}\rangle
\]
exist under the uniform locality/clustering bounds, and characterize $\mu$.
Under the NRC hypotheses below, the embedded resolvents are Cauchy in operator norm on any nonreal compact, implying \emph{unique} Schwinger limits as $\varepsilon\downarrow 0$ without passing to subsequences (Proposition~\ref{prop:af-free-uniqueness}).
\paragraph{Verification of the OS axioms.}
\emph{Remark.} The OS axioms are stable under controlled limits: positivity inequalities persist, polynomial bounds transfer via uniform constants, and clustering/gap properties are preserved by spectral convergence.

\begin{lemma}[OS0--OS5 in the continuum limit]\label{lem:os-continuum}
Let $\mu$ be a weak limit of lattice measures $\mu_\varepsilon$ along a scaling sequence. Assume:
\begin{itemize}
  \item[(i)] Uniform locality: $|S_{n,\varepsilon}(\Gamma_1,\ldots,\Gamma_n)| \le C_n \prod_i (1+\text{diam}\,\Gamma_i)^p \prod_{i<j} (1+\text{dist}(\Gamma_i,\Gamma_j))^{-q}$ with constants $C_n$ independent of $\varepsilon$.
  \item[(ii)] Uniform clustering: $|\langle O_\varepsilon(t) O_\varepsilon(0) \rangle_c| \le C e^{-m t}$ for mean-zero local observables.
  \item[(iii)] Equivariant embeddings preserving the reflection structure.
\end{itemize}
Then the limit measure $\mu$ satisfies:
\begin{itemize}
  \item \textbf{OS0 (temperedness):} $|S_n(\Gamma_1,\ldots,\Gamma_n)| \le C_n \prod_i (1+\text{diam}\,\Gamma_i)^p \prod_{i<j} (1+\text{dist}(\Gamma_i,\Gamma_j))^{-q}$ by direct passage to the limit using (i).
  \item \textbf{OS1 (Euclidean invariance):} Continuous rotations/translations act on $S_n$ by the limiting equivariance of discrete symmetries under (iii).
  \item \textbf{OS2 (reflection positivity):} For any polynomial $P$ in loop observables supported at $t \ge 0$,
  \[
    \langle \Theta(P) P \rangle_\mu = \lim_{\varepsilon \to 0} \langle \Theta(P_\varepsilon) P_\varepsilon \rangle_{\mu_\varepsilon} \ge 0,
  \]
  since positivity is preserved under weak limits.
  \item \textbf{OS3 (clustering):} Exponential decay $|\langle O(t) O(0) \rangle_c| \le C e^{-mt}$ follows from (ii) and weak convergence.
  \item \textbf{OS4/OS5 (symmetry/vacuum):} Gauge invariance and vacuum uniqueness follow from uniform gap persistence (Theorem~\ref{thm:gap-persist}).
\end{itemize}
\end{lemma}
\begin{corollary}[Verification of OS0--OS5 assumptions in this manuscript]\label{cor:os-assumptions-verified}
In the setting of this paper, the hypotheses (i)–(iii) of Lemma~\ref{lem:os-continuum} hold unconditionally on fixed regions, uniformly along van Hove sequences:
\begin{itemize}
  \item (i) holds by Proposition~\ref{prop:OS0-poly} (polynomial OS0) and Corollary~\ref{cor:os0-explicit-4d} with constants uniform in $(a,L)$ on fixed $R$.
  \item (ii) holds from the uniform lattice gap on slabs and its persistence: along scaling schedules satisfying \eqref{eq:ucis-sw-window}, UCIS$_{\rm SW}$ yields a fixed-physical-time contraction on the parity-odd subspace (Thm.~\ref{thm:ucis-sw-odd-subspace}), which supplies a slab gap input; gap persistence to the continuum (Thm.~\ref{thm:gap-persist-cont}) then yields exponential clustering uniformly on fixed regions.
  \item (iii) holds by the isometric OS/GNS embeddings and directed polygonal embeddings preserving reflection (Lem.~\ref{lem:isometric-embeddings}, Lem.~\ref{lem:U2-embeddings}), together with embedding–independence (Prop.~\ref{prop:embedding-independence}).
\end{itemize}
Consequently, Lemma~\ref{lem:os-continuum} applies unconditionally to the constructed limits on $\mathbb R^4$.
\end{corollary}
\begin{proof}
OS0 follows from Proposition~\ref{prop:OS0-poly} applied uniformly. OS1 uses equicontinuity: discrete rotations converge to continuous ones under directed embeddings. For OS2, approximate any polynomial $P$ in time-$\ge0$ loop observables by bounded cylinder functions and pass to the limit along the directed set of cylinder $\sigma$-algebras; positivity $\langle \Theta(P_\varepsilon) P_\varepsilon\rangle_{\mu_\varepsilon}\ge0$ is preserved under weak-* limits, yielding $\langle \Theta(P) P\rangle_\mu\ge0$. OS3 transfers the uniform bound (ii) to all cylinder functionals by density. OS4/OS5 follow from the gap persistence theorem ensuring a unique ground state.
\end{proof}

\begin{corollary}[Finite continuum gap via scaled minorization (Mosco/AF cross-check; not used in main chain)]\label{cor:scaled-continuum-gap}
Let $c(\varepsilon)>0$ be as in Theorem~\ref{thm:two-layer-explicit}. Under an \emph{optional} Mosco/strong-resolvent convergence assumption (AF/Mosco framework, recorded for cross-check only and not invoked elsewhere), along any van Hove scaling sequence the continuum generator $H$ obtained by Mosco/strong-resolvent convergence satisfies
\[
  \operatorname{spec}(H)\subset\{0\}\cup[c,\infty),\qquad c>0.
\]
In particular, the physical mass gap $m_*$ is finite and bounded below by $c$, with $c$ depending only on $(R_*,a_0,G)$ via $\lambda_1(G)$. This corollary serves as an \emph{optional cross-check}; the main AF--free continuum theorem does not rely on Mosco/AF.
\end{corollary}
\medskip
\begin{lemma}[Equicontinuity modulus on fixed regions]\label{lem:eqc-modulus}
\label{lem:eqc-modulus-app}
Fix a bounded region $R\subset\mathbb R^4$, $q>4$, $p=5$, and constants $(C_0,m)$ as in Proposition~\ref{prop:OS0-poly}. There exists $C_{\rm eq}(R,q,C_0,m)>0$ such that for any $n\ge 1$, loop families $\{\Gamma_i\}_{i=1}^n$ and $\{\Gamma'_i\}_{i=1}^n$ contained in $R$ with $\max_i d_H(\Gamma_i,\Gamma_i')\le \delta\in(0,1]$,
\[
  \big|\,S_{n,a,L}(\Gamma_1,\dots,\Gamma_n) - S_{n,a,L}(\Gamma'_1,\dots,\Gamma'_n)\,\big|
  \ \le\ C_{\rm eq}\,\delta^{\,q-4}\,\prod_{i=1}^n \bigl(1+\operatorname{diam}\Gamma_i\bigr)^p,
\]
uniformly in $(a,L)$.
\end{lemma}
\noindent\emph{Remark (uniformity).} The modulus $\omega_R(\delta)=C_{\rm eq}\,\delta^{\,q-4}$ is uniform in $(a,L)$ and depends only on $(R,q,C_0,m)$ from OS0; it is independent of the bare coupling and volume.
\begin{proof}[Proof (detailed)]
Fix $R\Subset\mathbb R^4$, $q>4$, $p=5$, and let the OS0 polynomial bound of Proposition~\ref{prop:OS0-poly} hold uniformly with constants $C_n(C_0,m,q)$. Let $\{\Gamma_i\}_{i=1}^n$ and $\{\Gamma'_i\}_{i=1}^n$ be loop families in $R$ with $\max_i d_H(\Gamma_i,\Gamma_i')\le \delta\in(0,1]$. For each $i$, choose a polygonal approximation of $\Gamma_i$ and $\Gamma_i'$ with mesh $\le c\delta$ and same combinatorics inside $R$; the OS0 bound applies uniformly to such local polygonal loops with the same constants.
Write the difference $S_{n,a,L}(\Gamma_1,\dots,\Gamma_n)-S_{n,a,L}(\Gamma'_1,\dots,\Gamma'_n)$ as a telescoping sum over the $n$ slots, changing one loop at a time while keeping the others fixed:
\[
  S_{n,a,L}(\Gamma_1,\dots,\Gamma_n)-S_{n,a,L}(\Gamma'_1,\dots,\Gamma'_n)
   =\sum_{k=1}^n \big( S_{n,a,L}(\Gamma_1',\dots,\Gamma_{k-1}',\Gamma_k,\Gamma_{k+1},\dots,\Gamma_n)
   - S_{n,a,L}(\Gamma_1',\dots,\Gamma_{k}',\Gamma_{k+1},\dots,\Gamma_n)\big).
\]
It suffices to bound a one-slot variation. By OS0, for any fixed positions of the other loops,
\[
  \big|\Delta_k\big|\ \le\ C_n\,\big(1+\operatorname{diam}\Gamma_k\big)^p\,\prod_{i\ne k}\big(1+\operatorname{diam}\Gamma_i\big)^p\,\prod_{i\ne k}\big(1+\operatorname{dist}(\Gamma_k,\Gamma_i)\big)^{-q}
   \cdot \mathrm{Var}_k(\Gamma_k,\Gamma_k'),
\]
where $\mathrm{Var}_k$ denotes the sensitivity with respect to moving loop $k$ to $\Gamma_k'$. By the polygonal approximation and $d_H(\Gamma_k,\Gamma_k')\le \delta$, one can partition $\Gamma_k$ and $\Gamma_k'$ into $O(\delta^{-1})$ matching segments of diameter $\le c\delta$ in $R$. Varying a single small segment perturbs $\operatorname{dist}(\Gamma_k,\Gamma_i)$ by at most $O(\delta)$ and the factor $(1+\operatorname{dist})^{-q}$ changes by at most $C\,\delta\,(1+\operatorname{dist})^{-(q+1)}$. Summing over segments and over $i\ne k$, and using $\sum_{x\in \mathbb Z^4}(1+\|x\|)^{-(q+1)}<\infty$ for $q>4$, yields
\[
  \mathrm{Var}_k(\Gamma_k,\Gamma_k')\ \le\ C(R,q)\,\delta^{\,q-4}.
\]
Collecting the diameter factors into $\prod_i (1+\operatorname{diam}\Gamma_i)^p$ and summing the $n$ telescoping terms gives the required bound with
\[
  C_{\rm eq}\ =\ C_n(C_0,m,q)\,C(R,q)\,n\,\max_{\text{families}}\prod_{i=1}^n (1+\operatorname{diam}\Gamma_i)^p,
\]
which is finite for loops contained in the fixed region $R$. This establishes the modulus $\omega_R(\delta)=C_{\rm eq}\,\delta^{\,q-4}$ uniformly in $(a,L)$.
\end{proof}
\begin{proposition}[AF-free uniqueness of Schwinger limits]\label{prop:af-free-uniqueness}
Fix a bounded region $R\Subset\mathbb R^4$. Assume: (i) the OS0 polynomial bounds on loop $n$-point functions hold uniformly in $(a,L)$ on $R$; (ii) equicontinuity holds as in Lemma~\ref{lem:eqc-modulus}; (iii) embedding–independence holds as in Proposition~\ref{prop:embedding-independence}; and (iv) for some nonreal $z_0$, the embedded resolvents $R_{a,L}(z_0):=I_{a,L}(H_{a,L}-z_0)^{-1}I_{a,L}^*$ form a Cauchy net in operator norm on the time-zero OS space generated by loops supported in $R$. Then the Schwinger functions $S_{n,a,L}$ converge uniquely as $(a,L)$ follow any van Hove diagonal, without invoking an AF schedule.
\begin{proof}
By (iv), $R_{a,L}(z_0)$ converge in operator norm to a bounded operator $R(z_0)$ on the limit space. The Laplace–resolvent representation expresses $n$-point functions of loop observables as finite sums of matrix elements of $R_{a,L}(z)$ at finitely many nonreal $z$'s with coefficients controlled by OS0. The resolvent identity and compactness of nonreal strips transfer the Cauchy property from $z_0$ to all $z$ in a fixed compact subset of $\mathbb C\setminus\mathbb R$, uniformly on $R$'s local cone. Dominated convergence (using OS0) passes limits under the Laplace integral, yielding convergence of the Schwinger functions along any van Hove diagonal. By (ii) and (iii), changing embeddings changes the approximants by $o(1)$, so the limit is independent of the embedding choice. Uniqueness across subsequences follows from operator-norm convergence of resolvents and the Riesz projection stability.
\end{proof}
\end{proposition}

\begin{proposition}[Embedding–independence of continuum Schwinger functions]\label{prop:embedding-independence}
Fix a bounded region $R\in SO(4)$ and $n\ge 1$. Let $\{I_\varepsilon\}$ and $\{J_\varepsilon\}$ be two admissible directed voxel embeddings for loops in $R$, chosen equivariantly under the hypercubic symmetries and preserving the OS reflection setup. For any loop family $\{\Gamma_i\}_{i=1}^n\subset R$,
\[
  \lim_{\varepsilon\to 0}\ \Big|\, S_{n,\varepsilon}^{(I)}(\Gamma_1,\dots,\Gamma_n)\,-\,S_{n,\varepsilon}^{(J)}(\Gamma_1,\dots,\Gamma_n)\,\Big|\ =\ 0.
\]
In particular, the continuum Schwinger limits $\{S_n\}$ (when they exist) are independent of the admissible embedding choice.
\end{proposition}
\begin{proof}
Directedness and equivariance give $d_H(I_\varepsilon(\Gamma_i),J_\varepsilon(\Gamma_i))\le C(R)\,\varepsilon$. Apply Lemma~\ref{lem:eqc-modulus} to control the difference uniformly; sum over $i$ and let $\varepsilon\to 0$.
\end{proof}

\begin{proposition}[Boundary–condition robustness on van Hove boxes]\label{prop:bc-robust}
Let $R\Subset\mathbb R^4$ be fixed. For any two boundary conditions on the complement of $R$ within a van Hove box, the time-zero local Schwinger functions in $R$ differ by at most $o_{L\to\infty}(1)$ uniformly in $a\in(0,a_0]$. Consequently, continuum limits on $R$ are independent of the boundary condition within the van Hove class.
\end{proposition}
\begin{proof}
Use the interface contraction and locality to show exponential decay of boundary influences in $L$; combine with UEI to pass uniform bounds to the limit.
\end{proof}

\begin{lemma}[Isotropy restoration via heat--kernel calibrators]\label{lem:isotropy-restore}
Let $P_{t_0}$ be the product heat kernel on $\mathrm{SU}(N)$ from Proposition~\ref{prop:explicit-doeblin-constants}. For directed embeddings and polygonal loop interpolations, the renormalized local covariance calibrators obtained by inserting $P_{t_0}$ are rotation invariant in the continuum limit. Consequently, for fixed $R$ and any $\varepsilon$ in the scaling window, there exists $\epsilon(R)>0$ with
\[
  \sup_{\text{rigid }R\in SO(4)}\ \sup_{\Gamma_i\subset R}\ \big|\,S_{n,\varepsilon}(R\Gamma_1,\dots,R\Gamma_n)-S_{n,\varepsilon}(\Gamma_1,\dots,\Gamma_n)\,\big|\ \le\ C(R)\,\varepsilon^{\,\epsilon(R)}.
\]
\end{lemma}

\begin{lemma}[OS1 without calibrators: embedding–independence route]\label{lem:os1-embedding}
Fix $R\in SO(4)$. For each $\varepsilon$, let $I^{(R)}_\varepsilon$ be a rotated voxel embedding obtained by precomposing the directed embedding $I_\varepsilon$ with $R$ and projecting to the $\varepsilon$–lattice equivariantly within the hypercubic symmetry (preserving the OS reflection setup). For any finite loop family $\{\Gamma_i\}_{i=1}^n$ in a fixed region,
\[
  S_{n,\varepsilon}^{(I^{(R)})}\big(R\Gamma_1,\dots,R\Gamma_n\big)
  \
  =\ S_{n,\varepsilon}^{(I)}\big(\Gamma_1,\dots,\Gamma_n\big).
\]
If continuum limits along the scaling window are unique and independent of the admissible embedding choice, then $S_n(R\Gamma_1,\dots,R\Gamma_n)=S_n(\Gamma_1,\dots,\Gamma_n)$, i.e., OS1 holds without calibrators.
\begin{proof}
At fixed $\varepsilon$, the Wilson action and OS reflection structure are invariant under the hypercubic group. The rotated embedding $I^{(R)}_\varepsilon$ is obtained by conjugating $I_\varepsilon$ with the rigid rotation $R$ and discretizing equivariantly, so the lattice integral defining $S_{n,\varepsilon}$ is preserved by the change of variables induced by $R$ together with the hypercubic symmetry. This gives the displayed identity at each $\varepsilon$. By the embedding–independence of limits (Appendix C1c–C1d), admissible embeddings along the scaling window lead to the same continuum limits. Passing to the limit yields $SO(4)$ invariance of $\{S_n\}$.
\end{proof}
\end{lemma}
\begin{corollary}[OS1 (rotations) in the continuum limit]\label{cor:os1-rotations}
Under the hypotheses of Theorem~\ref{thm:os1-unconditional}, together with Lemma~\ref{lem:eqc-modulus} and either Lemma~\ref{lem:isotropy-restore} or Lemma~\ref{lem:os1-embedding}, the limit Schwinger functions are invariant under $SO(4)$ rotations: $S_n(R\Gamma_1,\dots,R\Gamma_n)=S_n(\Gamma_1,\dots,\Gamma_n)$ for all rigid $R$.
\end{corollary}
\begin{proof}
Approximate a fixed $R\in SO(4)$ by hypercubic rotations $R_k$. Discrete invariance gives equality for $R_k$. Lemma~\ref{lem:isotropy-restore} reduces $R_k\to R$ defects to $o(1)$, and Lemma~\ref{lem:eqc-modulus} controls the embedding perturbations uniformly; pass to the limit.
\end{proof}

\paragraph{Hamiltonian reconstruction.}
By the OS reconstruction theorem, the positive-time semigroup is a contraction semigroup $P(t)$ with $\lVert P(t)\rVert\le 1$. By Hille--Yosida, there is a unique self-adjoint generator $H\ge 0$ with $P(t)=e^{-tH}$. Clustering implies a unique vacuum $\Omega$ with $H\Omega=0$.

\subsection*{Consolidated continuum existence (C1)}

We bundle the results of Appendices C1a--C1c into a single statement.

\begin{theorem}\label{thm:c1-consolidated}
Fix a scaling window $\varepsilon\in(0,\varepsilon_0]$ and consider lattice Wilson measures $\mu_\varepsilon$ with a fixed link-reflection. Assume:
\begin{itemize}
  \item (Uniform locality/moments) The loop observables satisfy $\varepsilon$-uniform locality/clustering and moment bounds, and the reflection setup is fixed (C1a).
  \item (Discrete invariance) $\mu_\varepsilon$ is invariant under the hypercubic group; directed embeddings of loops are chosen equivariantly (C1a).
  \item (Embeddings and consistency) There exist voxel embeddings $I_\varepsilon$ with graph-norm defect control and a compact calibrator for the limit generator (C1c).
\end{itemize}
Then, under the AF/Mosco hypotheses and equicontinuity, the loop $n$-point functions converge \emph{uniquely} (no subsequences) to Schwinger functions $\{S_n\}$ which satisfy OS0--OS5 (regularity/temperedness, Euclidean invariance, reflection positivity, clustering, and unique vacuum). By OS reconstruction, there exists a Hilbert space $\mathcal H$, a vacuum $\Omega$, and a positive self-adjoint Hamiltonian $H\ge 0$ generating Euclidean time.
Moreover, if the lattice transfer operators have an $\varepsilon$-uniform spectral gap on the mean-zero sector, $r_0(T_\varepsilon)\le e^{-\gamma_0}$ with $\gamma_0>0$, then $\operatorname{spec}(H)\subset\{0\}\cup[\gamma_0,\infty)$ and the continuum theory has a mass gap $\ge \gamma_0$.
\end{theorem}

\begin{proof}
Tightness and convergence follow from the uniform locality hypotheses. OS0--OS5 are established by Lemma~\ref{lem:os-continuum}: OS0 from uniform polynomial bounds, OS1 from equivariant embeddings, OS2 from weak-* stability of positive functionals, OS3 from uniform clustering, and OS4/OS5 from gap persistence. Mosco/strong-resolvent convergence with the uniform lattice gap hypothesis yields $\operatorname{spec}(H) \subset \{0\} \cup [\gamma_0,\infty)$ by Theorem~\ref{thm:gap-persist-cont}.
\end{proof}

\subsection*{Preview: pointer to the Main Theorem}

\noindent For the definitive, labeled statement and proof (with AF--free NRC and U1/OS1 fixed-region inputs), see Section~\ref{sec:main-theorem-unconditional}, Theorem~\ref{thm:main-af-free}.

\begin{theorem}\label{thm:preview-continuum-existence}
On $\mathbb R^4$, there exists a probability measure on loop configurations whose Schwinger functions satisfy OS0--OS5. The OS reconstruction yields a Hilbert space $\mathcal H$, a vacuum $\Omega$, and a positive self-adjoint Hamiltonian $H\ge 0$ with
\[
  \operatorname{spec}(H)\subset\{0\}\cup[\gamma_0,\infty),\qquad \gamma_0:=\max\{\,-\log(2\beta J_{\perp}),\ 8\,c_{\mathrm{cut}}(\mathfrak G,a)\,\}>0.
\]
Here $c_{\mathrm{cut}}(\mathfrak G,a):=-(1/a)\log(1-\theta_*(1-e^{-\lambda_1(G) t_0}))$ is the slab-local odd-cone contraction rate obtained from an interface Doeblin minorization and heat--kernel domination on $G$; it is uniform in the volume on fixed slabs and independent of $\beta$ (see Proposition~\ref{prop:explicit-doeblin-constants} and Corollary~\ref{cor:hk-convex-split-explicit}). By the AF–free NRC chain (Theorems~\ref{thm:strong-semigroup-core}, \ref{thm:nrc-operator-norm}, \ref{thm:nrc-embeddings}, Lemma~\ref{lem:af-free-cauchy}), the same lower bound $\gamma_0$ persists to the continuum generator $H$; and the OS\,$\to$\,Wightman export is Theorem~\ref{thm:os-to-wightman}. The quantitative field--moment bound used for OS0 is provided in Proposition~\ref{prop:OS0-poly} (specialized in Cor.~\ref{cor:os0-explicit-4d}).

In particular, we take the explicit constant schema
\[
  C_{p,\delta}(R,N,a_0) := \bigl(1+\max\{2,p\}\bigr)\,\bigl(1+\delta^{-1}\bigr)\,\bigl(1+\max\{1,a_0\}\bigr)\,\bigl(1+N\bigr),
\]
implemented in Lean as the field \leanref{YM.OSPositivity.MomentBoundsCloverQuantIneq.C} of the container \leanref{YM.OSPositivity.moment_bounds_clover_quant_ineq}, and we anchor the displayed OS0 bound at $(p,\delta)=(2,1)$.
\end{theorem}

\paragraph{Continuum tail under AF/Mosco (parameter tracking).}
For any scaling sequence $\varepsilon\downarrow 0$, the odd-cone interface deficit yields a lattice mean-zero spectral gap per OS slab of eight ticks: $r_0(T_\varepsilon)\le e^{-8 c_{\rm cut}}$, hence $\operatorname{spec}(H_\varepsilon)\subset\{0\}\cup[\gamma_0,\infty)$ with $\gamma_0:=8 c_{\rm cut}>0$, uniform in the volume. By Mosco/strong-resolvent convergence and gap persistence (Thm.~\ref{thm:gap-persist-cont}), $(0,\gamma_0)$ remains spectrum-free in the limit, so
\[
  \operatorname{spec}(H)\subset\{0\}\cup[\gamma_0,\infty),\qquad \gamma_{\mathrm{phys}}\ge \gamma_0.
\]

\noindent\emph{Remark (constants).} In the AF--free main chain, the coarse refresh and heat--kernel sandwich produce slab-uniform constants $(\theta_*,t_0)$ that are independent of $\beta$ on fixed slabs. Thus $c_{\rm cut}(a)=-(1/a)\log(1-\theta_*(1-e^{-\lambda_1(G) t_0}))$ and $c_{\rm cut,phys}=-\log(1-\theta_*(1-e^{-\lambda_1(G) t_0}))$ are uniform in $L$ and $\beta$ after coarse refresh.

\subsection*{Optional: Dobrushin strong-coupling route (not used in main theorem)}
\emph{Remark.} The main unconditional proof uses the odd-cone Doeblin contraction with slab-uniform, $\beta$-independent $(\theta_*,t_0)$ after coarse refresh. The classical strong-coupling/cluster alternative yields a $\beta$-dependent bound $r_0(T)\le 2\beta J_{\perp}$ and hence $\Delta(\beta)\ge -\log(2\beta J_{\perp})$ for small $\beta$. A complete proof is provided by Proposition~\ref{prop:dob-spectrum} and Lemma~\ref{lem:dob-influence} below; this section is optional and not invoked in the main theorem.

\section{Infinite volume at fixed spacing}

\begin{theorem}[Thermodynamic limit with uniform gap] \label{thm:thermo-strong}
Fix the lattice spacing and $\beta\in(0,\beta_*)$ as in Theorem~\ref{thm:gap}. Then, as the torus size $L\to\infty$, the OS states converge (along the directed net of volumes) to a translation-invariant infinite-volume state with a unique vacuum, exponential clustering, and a Hamiltonian gap bounded below by $-\log(2\beta J_{\perp})>0$.
\end{theorem}

\begin{proof}
All Dobrushin/cluster bounds and the OS Gram-positivity estimates are local and uniform in the volume. Hence the contraction coefficient bound $r_0(T_L)\le \alpha(\beta)<1$ holds with a constant independent of $L$. Standard compactness of local observables under the product Haar topology yields existence of a thermodynamic limit state. The uniform spectral contraction on $\mathcal H_{0,L}$ implies exponential decay of correlations and uniqueness of the vacuum in the limit, with the same lower bound on the gap. See Montvay--M\"unster~\cite{MontvayMunster1994} for the thermodynamic passage under strong-coupling/cluster conditions.
\end{proof}

\section{Appendix: Parity--Oddness and One--Step Contraction (TP)}

\paragraph{Setup.}
Fix three commuting spatial reflections $P_x,P_y,P_z$ acting by lattice involutions on the time--zero gauge--invariant algebra $\mathfrak{A}_0^{\rm loc}$. They induce unitary involutions on the OS Hilbert space $\mathcal{H}_{L,a}$, commute with $H_{L,a}$, and leave the vacuum $\Omega$ invariant. For $i\in\{x,y,z\}$ write $\alpha_i(O):=P_i O P_i$ and define $O^{(\pm,i)}:=\tfrac12(O\pm\alpha_i(O))$. Let $\mathcal{C}_{R_*}:=\{O\Omega:\ O\in\mathfrak{A}_0^{\rm loc},\ \langle O\rangle=0,\ \mathrm{supp}(O)\subset B_{R_*}\}$ be the local cone.

\begin{lemma}[Parity--oddness on the local cone]\label{lem:oddness-tp}
For any nonzero $\psi=O\Omega\in\mathcal{C}_{R_*}$ there exists $i\in\{x,y,z\}$ such that $O^{(-,i)}\neq 0$, hence $P_i\psi^{(-,i)}=-\psi^{(-,i)}$ with $\psi^{(-,i)}:=O^{(-,i)}\Omega\neq 0$.
\end{lemma}

\begin{proof}
Let $\mathcal{G}:=\langle P_x,P_y,P_z\rangle\simeq Z_2^3$. Each $P\in\mathcal{G}$ acts by a *-automorphism $\alpha_P$ on $\mathfrak{A}_0^{\rm loc}$ and is implemented by a unitary $U(P)$ on the OS Hilbert space $\mathcal{H}_{L,a}$ via $U(P)[F]=[\alpha_P(F)]$; moreover $U(P)\Omega=\Omega$ and $U(P)$ commutes with the transfer/semigroup by symmetry.

Assume for contradiction that $O^{(-,i)}=0$ for all $i\in\{x,y,z\}$. Then $\alpha_{P_i}(O)=O$ for each generator, hence $\alpha_P(O)=O$ for all $P\in\mathcal{G}$. Consequently $U(P)\,[O]=[O]$ for all $P\in\mathcal{G}$, so the vector $[O]$ lies in the fixed subspace of the unitary representation $U$ of $\mathcal{G}$ on $\mathcal{H}_{L,a}$.

By Theorem~\ref{thm:os} (OS positivity and GNS construction), the constants sector in $\mathcal{H}_{L,a}$ is one-dimensional, spanned by $\Omega$. Since $\mathcal{G}$ is a subgroup of the spatial symmetry group, its fixed subspace is contained in the constants sector; therefore $[O]=c\,\Omega$ for some $c\in\mathbb{C}$. Taking vacuum expectation gives $c=\langle\Omega,[O]\,\Omega\rangle=\langle O\rangle$. Because $\psi=O\Omega\in\mathcal{C}_{R_*}$ has $\langle O\rangle=0$ by definition, we have $c=0$, hence $[O]=0$ and $\psi=0$ in $\mathcal{H}_{L,a}$.

This contradicts the hypothesis that $\psi\ne 0$. Therefore our assumption was false and there must exist at least one $i\in\{x,y,z\}$ with $O^{(-,i)}\ne 0$. In particular $\psi^{(-,i)}:=O^{(-,i)}\Omega\ne 0$ and $P_i\psi^{(-,i)}=-\psi^{(-,i)}$.
\end{proof}

\begin{lemma}[One--step contraction on odd cone]\label{lem:odd-contraction-tp}
Define the slab--local reflection deficit
\[
  \beta_{\mathrm{cut}}(R_*,a)
  \,:=\,
  1\;-
  \sup_{\substack{\psi\in\mathcal H_{L,a},\ \psi\ne 0\\ P_i\psi=-\psi,\ \mathrm{supp}\,\psi\subset B_{R_*}}}
  \frac{\big|\langle\psi, e^{-aH_{L,a}}\psi\rangle\big|}{\langle\psi,\psi\rangle}\,.
\]
Then there exists $\beta_0>0$, depending only on the fixed physical slab $R_*$ (not on $L$) and on $a\in(0,a_0]$, such that $\beta_{\mathrm{cut}}(R_*,a)\ge \beta_0$. Consequently, for any $i\in\{x,y,z\}$ and $\psi\in\mathcal{H}_{L,a}$ with $P_i\psi=-\psi$,
\[
  \|e^{-aH_{L,a}}\psi\|\ \le\ (1-\beta_0)^{1/2}\,\|\psi\|\ \le\ e^{-a c_{\mathrm{cut}}}\,\|\psi\|,
  \qquad c_{\mathrm{cut}}\ :=\ -\frac{1}{a}\log(1-\beta_0)\,.
\]
\end{lemma}
\begin{proof}
OS positivity implies that the $2\times 2$ Gram matrix for $\{\psi, e^{-aH}\psi\}$ is PSD. Let $a_0=\|\psi\|^2$, $b_0=\|e^{-aH}\psi\|^2$ and $z=\langle\psi, e^{-aH}\psi\rangle$. By the PSD $2\times 2$ bound (Appendix Eq.~\eqref{eq:psd-2x2-lower}), $\lambda_{\min}\bigl(\begin{smallmatrix} a_0 & z \\ \overline z & b_0 \end{smallmatrix}\bigr)\ge \min(a_0,b_0)-|z|$. Using the local odd basis and Lemmas~\ref{lem:local-gram-bounds} and \ref{lem:mixed-gram-bound}, Proposition~\ref{prop:two-layer-deficit} yields a uniform diagonal lower bound $\min(a_0,b_0)\ge \beta_{\rm diag}>0$ and an off-diagonal bound $|z|\le S_0<\beta_{\rm diag}$. Hence $\lambda_{\min}\ge \beta_{\rm diag}-S_0=:\beta_0>0$. Normalizing $a_0=1$ gives $b_0\le 1-\beta_0$ and $\|e^{-aH}\psi\|\le (1-\beta_0)^{1/2}\|\psi\|$. Setting $c_{\mathrm{cut}}:=-(1/a)\log(1-\beta_0)>0$ gives the exponential form with constants depending only on $(R_*,a_0,N)$.
\end{proof}

\begin{theorem}[Tick--Poincar\'e bound]\label{thm:tp-bound}
For every $\psi=O\Omega\in\mathcal{C}_{R_*}$,
\[
  \langle\psi,H_{L,a}\psi\rangle\ \ge\ c_{\mathrm{cut}}\,\|\psi\|^2
\]
uniformly in $(L,a)$. In particular, $\mathrm{spec}(H_{L,a})\subset\{0\}\cup[c_{\mathrm{cut}},\infty)$ and, composing over eight ticks, $\gamma_0\ge 8\,c_{\mathrm{cut}}$ per slab. Under the RS specialization, one may take $c_{\mathrm{cut}}=\gamma_{\mathrm{RS}}=\ln\varphi/\tau_{\mathrm{rec}}$.
\end{theorem}
\section{Appendix: Tree--Gauge UEI (Uniform Exponential Integrability)}

\subsection*{U1 subclaims (audit stubs; closure target decomposition)}

\begin{lemma}[U1-A: Tree gauge reduction + bounded interaction degree (target)]\label{lem:U1-A-tree-gauge}
Fix a bounded physical region $R\Subset\mathbb R^4$. After gauge-fixing on a spanning tree of links in $R$ (with exterior boundary configuration fixed), the induced finite-volume law on the remaining independent chord variables can be written as a Gibbs measure on $G^{m(R,a)}$ with respect to product Haar, with uniformly bounded interaction degree $d_0(R)$ independent of the lattice spacing $a$.
\end{lemma}

\begin{lemma}[U1-B: Small-field concentration on fixed regions (core target)]\label{lem:U1-B-smallfield}
Fix $R\Subset\mathbb R^4$ and a weak-coupling schedule $a\mapsto \beta(a)$ with $\beta(a)\to\infty$ as $a\downarrow 0$. There exists a constant $C_R<\infty$ and an event $\mathcal G_{R,a}$ (measurable with respect to plaquettes in $R$) such that along the scaling window,
\[
  \mathbb P_{\mu_{L,a}}\!\big(\mathcal G_{R,a}^c\big)\ \le\ \varepsilon_R(a)\quad\text{with }\varepsilon_R(a)\to 0,
\]
and on $\mathcal G_{R,a}$ all plaquette holonomies in $R$ satisfy a uniform small-field bound (e.g. $d_G(U_p,\mathbf 1)\le C_R\,a^2$ in a fixed bi-invariant metric). In particular, on $\mathcal G_{R,a}$ one has $\phi(U_p)\lesssim_R a^4$ and hence $S_R(U)=\sum_{p\subset R}\phi(U_p)=O_R(1)$.
\end{lemma}

\begin{lemma}[U1-D: Gradient/Lipschitz bounds from small field (target)]\label{lem:U1-D-lipschitz}
On the small-field event of Lemma~\ref{lem:U1-B-smallfield}, the map $U\mapsto S_R(U)$ (and the finite collection of time-zero local observables on $R$ used in equicontinuity/OS1/NRC bookkeeping) admits Lipschitz/gradient bounds compatible with a uniform Herbst argument, with constants depending only on $(R,N)$ and the declared schedule parameters.
\end{lemma}

\begin{lemma}[U1-E: Dimension-uniform LSI/concentration for the chord Gibbs law (target)]\label{lem:U1-E-lsi-chords}
Along the scaling window, after tree gauge on $R$ the induced chord Gibbs measure $\mu_R^{(a)}$ satisfies a logarithmic Sobolev inequality (or an explicitly stated substitute concentration inequality) with a constant $\rho_R>0$ uniform in $a$, in the van Hove volume, and in the exterior boundary configuration outside $R$.
\end{lemma}

\begin{remark}[U1 roadmap (how the pieces fit)]\label{rem:U1-roadmap}
There are two viable closure routes for U1 as currently organized in this manuscript:
\begin{itemize}
  \item \textbf{Raw small-field route (U1-B).} Prove Lemma~\ref{lem:U1-B-smallfield} (small-field concentration for the \emph{unsmoothed} Wilson law on fixed regions) and then deduce the mean bound (U1-C), Lipschitz bounds (U1-D), and a usable LSI/concentration inequality (U1-E) by standard compact-manifold arguments once the law is sufficiently concentrated near $\mathbf 1$.
  \item \textbf{Track J (smoothing/universality route).} Prove a dimension-free LSI/Herbst bound for \emph{heat-smoothed} local laws (Lemma~\ref{lem:U1-J1-smoothed-lsi}) and a smoothing-limit commutation/universality statement (Lemma~\ref{lem:U1-J2-universality}) to transfer tightness back to the intended continuum limit as the smoothing scale goes to $0$.
\end{itemize}
The Track-J route is designed to avoid making “raw small-field for the Wilson law” the only analytic bottleneck.
\end{remark}

\begin{mdframed}[linewidth=0.5pt, linecolor=blue!35, backgroundcolor=blue!3, roundcorner=2pt, innertopmargin=8pt, innerbottommargin=8pt, skipabove=10pt, skipbelow=10pt]
\noindent\textbf{Remark (RS-guided bypass for U1; J-cost / smoothing route).}
Recognition Science (RS) highlights a unique convex cost $J(x)=\tfrac12(x+1/x)-1$ (T5) and an 8-tick “structural regularization” viewpoint (T6). A classical correspondence that may bypass proving raw small-field concentration for the unsmoothed Wilson law is to work with a \emph{heat-smoothed} (or “J-cost regularized”) local law at positive time, which can satisfy dimension-free LSI/Herbst bounds by entropy chain rules and tensorization, and then prove a \emph{universality/commutation} statement showing that letting the smoothing scale go to $0$ after $a\downarrow 0$ recovers the same continuum Schwinger functions. This reframes the hard U1-B payload as a single “universality under smoothing” theorem; see `U1_OPERATIONAL_PLAN.md` for the operational decomposition (Track J).
\end{mdframed}

\begin{lemma}[U1-J1: Heat-smoothing $\Rightarrow$ dimension-free LSI/Herbst (target)]\label{lem:U1-J1-smoothed-lsi}
Fix a bounded region $R\Subset\mathbb R^4$ and a compact gauge group $G=\SU(N)$. Let $\mu_{a,L}$ be the Wilson law on the links intersecting $R$ (with arbitrary exterior boundary configuration), and for $t>0$ define the \emph{heat-smoothed} law $\mu^{(t)}_{a,L}$ by independently left-multiplying each link variable in $R$ by a $G$-heat increment of time $t$ (equivalently: convolving $\mu_{a,L}$ with the product heat kernel on $G^{E(R,a)}$).

Then for each fixed $t>0$, $\mu^{(t)}_{a,L}$ satisfies a logarithmic Sobolev inequality (and hence a centered Herbst subgaussian Laplace bound) with constants depending only on $(G,t)$ and the fixed region geometry, \emph{independent of}:
\[
  a,\quad L,\quad \beta,\quad \text{and}\quad |E(R,a)|\asymp a^{-4}.
\]
\end{lemma}
\begin{lemma}[Single-site heat-kernel LSI on compact $G$ (input)]\label{lem:heatkernel-lsi-compactG}
Let $G$ be a compact connected Lie group equipped with a bi-invariant Riemannian metric, and let $\pi$ be Haar probability. Let $p_t$ be the heat-kernel density at time $t>0$ (generator the Laplace--Beltrami operator). Then for each $t>0$ and each $g\in G$, the probability measure
\[
  \nu_{t,g}(dU)\ :=\ p_t(g^{-1}U)\,\pi(dU)
\]
satisfies a logarithmic Sobolev inequality
\[
  \Ent_{\nu_{t,g}}(f^2)\ \le\ 2\,\mathcal C_G(t)\int \|\nabla f\|_G^2\,d\nu_{t,g},
\]
with $\mathcal C_G(t)<\infty$ depending only on $(G,t)$ and \emph{uniform} in $g$.
\end{lemma}
\begin{proof}[Proof sketch (with citation)]
Since $G$ is compact, Haar $\pi$ satisfies an LSI with some constant depending only on $G$ (e.g. via Bakry--\'Emery hypercontractivity for the heat semigroup; see \cite{BakryEmery1985,Gross1975}). For fixed $t>0$, the heat kernel $p_t$ is smooth, strictly positive, and bounded above/below on $G$, so $V:=-\log p_t(g^{-1}\cdot)$ has bounded oscillation $\osc(V)<\infty$. The Holley--Stroock bounded-perturbation lemma for LSI \cite{HolleyStroock1987} then upgrades the LSI from $\pi$ to $\nu_{t,g}$ with a constant multiplied by $e^{\osc(V)}$. Left translation preserves the Riemannian gradient and Haar measure, so the bound is uniform in $g$.
\end{proof}
\begin{proof}[Proof sketch]
Let $K_t(U,dV)=\prod_{e\in E(R,a)} p_t(U_e^{-1}V_e)\,\pi(dV_e)$ be the product heat kernel on $G^{E(R,a)}$. Then $\mu^{(t)}_{a,L}=\mu_{a,L}K_t$.
By Lemma~\ref{lem:heatkernel-lsi-compactG}, for each fixed $t>0$ the single-site heat kernel measure $p_t\,\pi$ satisfies an LSI with constant $\mathcal C_G(t)$ depending only on $(G,t)$ and uniform in the starting point. By tensorization, for each fixed $U$ the conditional law $K_t(U,\cdot)$ satisfies the same LSI constant $\mathcal C_G(t)$ on the product space, independent of $|E(R,a)|$.

Let $f$ be smooth on $G^{E(R,a)}$ and apply the entropy chain rule for the joint law $(U,V)$ with $U\sim\mu_{a,L}$ and $V\sim K_t(U,\cdot)$:
\[
  \Ent_{\mu^{(t)}_{a,L}}(f^2)
  \ =\ \Ent_{(U,V)}(f(V)^2)
  \ =\ \mathbb E_{U}\!\big[\Ent_{K_t(U,\cdot)}(f^2)\big]\ +\ \Ent_{\mu_{a,L}}\!\big(\mathbb E_{K_t(U,\cdot)}[f^2]\big).
\]
Drop the second (nonnegative) term and apply the conditional LSI, then average over $U$. This yields an LSI for $\mu^{(t)}_{a,L}$ with constant $\mathcal C_G(t)$, hence centered Herbst subgaussian bounds with the same constant. None of these constants depend on $\mu_{a,L}$ (hence not on $\beta$) or on $|E(R,a)|$.
\end{proof}

\begin{lemma}[U1-J2: Smoothing-limit commutation / universality on fixed regions (target)]\label{lem:U1-J2-universality}
Fix a bounded region $R\Subset\mathbb R^4$ and let $F$ be a bounded continuous time-zero local cylinder observable supported in $R$ (i.e. $F$ depends on only finitely many link variables in $R$). Let $\mu_{a,L}$ be any family of lattice measures and let $\mu^{(t)}_{a,L}$ be the heat-smoothed laws from Lemma~\ref{lem:U1-J1-smoothed-lsi}. Then
\[
  \sup_{a,L}\ \big|\ \mathbb E_{\mu^{(t)}_{a,L}}[F]\ -\ \mathbb E_{\mu_{a,L}}[F]\ \big|\ \xrightarrow[t\downarrow 0]{}\ 0.
\]
Consequently, any subsequential limit on cylinders obtained from $\{\mu_{a,L}\}$ can be recovered as the $t\downarrow 0$ limit of subsequential limits of $\{\mu^{(t)}_{a,L}\}$, so tightness/limit constructions may be carried out using the smoothed laws and then transferred back to the unsmoothed limits by letting $t\downarrow 0$.
\end{lemma}
\begin{proof}
Let $P_t$ be the (finite-link) product heat semigroup acting on the finitely many link variables on which $F$ depends. By definition of heat-smoothing,
\[
  \mathbb E_{\mu^{(t)}_{a,L}}[F]\ =\ \mathbb E_{\mu_{a,L}}[P_t F].
\]
Therefore
\[
  \big|\mathbb E_{\mu^{(t)}_{a,L}}[F]-\mathbb E_{\mu_{a,L}}[F]\big|
  \ =\ \big|\mathbb E_{\mu_{a,L}}[P_tF-F]\big|
  \ \le\ \|P_tF-F\|_\infty.
\]
Since the configuration space for these finitely many links is compact and $F$ is continuous, $P_tF\to F$ uniformly as $t\downarrow 0$, hence $\|P_tF-F\|_\infty\to 0$.
\end{proof}

\begin{assumption}[U1-C: Fixed--region action mean bound (UEI input; RG--grade)]\label{assump:uei-mean}\label{lem:U1-C-mean}
Fix a bounded physical region $R\Subset\mathbb R^4$ and define
\[
  S_R(U)\ :=\ \sum_{p\subset R}\phi(U_p),\qquad \phi(U):=1-\tfrac{1}{N}\Re\,\Tr(U)\in[0,2].
\]
Along the weak--coupling scaling schedule $a\mapsto\beta(a)$ and van Hove volumes, assume there exists $M_R<\infty$ (depending only on $(R,a_0,N,\beta_{\min})$) such that for all sufficiently small $a\in(0,a_0]$, all admissible $L=L(a)$, and all boundary configurations outside $R$,
\[
  \mathbb E_{\mu_{L,a}}\big[S_R(U)\big]\ \le\ M_R.
\]
\end{assumption}

\begin{theorem}[U1 on fixed regions: tightness/UEI (two closure routes; target)]\label{thm:uei-fixed-region}
Fix a bounded physical region $R\Subset\mathbb{R}^4$ and a weak-coupling scaling schedule $a\mapsto \beta(a)$ with $\beta(a)\ge \beta_{\min}(R,N)>0$ for $a\in(0,a_0]$ and $\beta(a)\to\infty$ as $a\downarrow 0$.

\medskip
\noindent\emph{Closure route A (raw small-field).} Assume the raw small-field route on $R$: Lemmas~\ref{lem:U1-B-smallfield}, \ref{lem:U1-D-lipschitz}, \ref{lem:U1-E-lsi-chords} hold, together with the mean bound Assumption~\ref{assump:uei-mean}.

\smallskip
\noindent\emph{Closure route J (smoothing/universality).} Alternatively, assume the Track-J route on $R$: Lemmas~\ref{lem:U1-J1-smoothed-lsi} and \ref{lem:U1-J2-universality} hold.

\medskip
\noindent\emph{Conclusion (tightness).} Under either closure route, the family of cylinder laws on $R$ is tight along van Hove scaling sequences (hence OS0/OS2 closure on fixed regions reduces to standard compactness arguments).

\medskip
\noindent\emph{Additional conclusion (strong UEI for the action; route A only).} Under closure route A, writing $S_R(U):=\sum_{p\subset R}\phi(U_p)$ with $\phi(U)=1-\tfrac1N\Re\Tr(U)$, there exist constants $\eta_R>0$ and $C_R<\infty$, depending only on $(R,a_0,N,\beta_{\min})$, such that for all $(L,a)$ in the scaling window (with coupling $\beta=\beta(a)$) and any boundary configuration outside $R$,
\[
  \mathbb{E}_{\mu_{L,a}}\big[e^{\eta_R S_R(U)}\big]\ \le\ C_R.
\]
\end{theorem}
\begin{corollary}[Uniform UEI/tightness along AF scaling (conditional)]\label{cor:uei-af-uniform}
Under Assumption~\ref{assump:AF-Mosco}, for each fixed bounded region $R\Subset\mathbb R^4$ the fixed-region U1 conclusions of Theorem~\ref{thm:uei-fixed-region} hold along the AF scaling window provided one of its closure routes (raw small-field route A, or Track-J route) is verified. In particular, tightness on fixed regions follows; and under route A one also obtains the explicit action UEI bound $\mathbb E[e^{\eta_R S_R}]\le C_R$.
\end{corollary}
\begin{proof}
\emph{Track-J route (tightness).} If Lemmas~\ref{lem:U1-J1-smoothed-lsi} and \ref{lem:U1-J2-universality} hold, tightness on fixed regions can be proved by working first with the heat-smoothed laws (which satisfy dimension-free functional inequalities by Lemma~\ref{lem:U1-J1-smoothed-lsi}) and then transferring limits back to the unsmoothed cylinder laws as the smoothing scale $t\downarrow 0$ using Lemma~\ref{lem:U1-J2-universality}.

\smallskip
\emph{Raw route A (strong UEI for $S_R$; proof sketch).} Under closure route A, the remaining proof sketch provides one pathway to the uncentered action UEI bound $\mathbb E[e^{\eta_R S_R}]\le C_R$, using the centered Herbst bound plus the mean-removal step supplied by Assumption~\ref{assump:uei-mean}.

\emph{Scope note (weak-coupling scaling).} The uniform-in-$a$ claim for the extensive quantity $S_R=\sum_{p\subset R}\phi(U_p)$ is only plausible in the intended continuum regime where $\beta=\beta(a)\to\infty$ as $a\downarrow 0$ (so plaquette holonomies concentrate near $\mathbf 1$ and $\phi(U_p)\sim a^4\|F\|^2$). At fixed $\beta$ independent of $a$, $|\mathcal P_R|\asymp a^{-4}$ and one expects $\mathbb E S_R$ to grow proportionally to $|\mathcal P_R|$, so $\sup_a\mathbb E e^{\eta S_R}$ cannot remain bounded for any fixed $\eta>0$.

\emph{Idea.} Gauge-fix on a tree so only finitely many chords remain; the Wilson energy is uniformly strictly convex along chords on fixed regions, giving a local log–Sobolev inequality. A Lipschitz bound for the local action then yields subgaussian Laplace tails (Herbst), giving uniform exponential integrability.

\medskip
\emph{Step 1 (Tree gauge and local coordinates).} Fix a spanning tree $T$ of links in $R$ (with fixed boundary outside $R$) and gauge--fix links on $T$ to the identity. The remaining independent variables (``chords'') form a finite product $X\in G^{m}$, $G=\mathrm{SU}(N)$, with $m=m(R,a)=O(a^{-4})$ (finite for each fixed $a$ because $R$ is bounded). Each plaquette variable $U_p$ is a product of at most four chord variables, and each chord enters at most $d_0=d_0(R)$ plaquettes.

\emph{Step 2 (Local LSI at large $\beta$).} In a normal coordinate chart around $\mathbf{1}\in G$, write $U_\ell=\exp A_\ell$ with $A_\ell\in\mathfrak{su}(N)$. For $p$ near the identity,
\[
  \phi(U_p)\ =\ 1-\tfrac{1}{N}\Re\,\mathrm{Tr}(U_p)
  \ =\ \tfrac{c_N}{2}\,a^4\,\|F_p(A)\|^2\ +\ O(a^6\,\|A\|^3),
\]
with a universal $c_N>0$ and a bounded multilinear form $F_p$ (continuum expansion). Thus the negative log--density on $R$ after tree gauge,
\[
  V_R(X)\ :=\ \beta(a)\sum_{p\subset R}\phi(U_p(X))
\]
has Hessian uniformly bounded below by $\kappa_R\,\beta(a)$ along each chord direction for all $a\in(0,a_0]$ with $\beta(a)\ge \beta_{\min}$, by compactness of $G$ and bounded interaction degree (Holley--Stroock/Bakry--\'Emery perturbation on compact groups). Therefore the induced Gibbs measure $\mu_R$ satisfies a local log--Sobolev inequality (LSI)
\[
  \mathrm{Ent}_{\mu_R}(f^2)\ \le\ \frac{1}{\rho_R}\,\int \|\nabla f\|^2\,d\mu_R,
  \qquad \rho_R\ \ge\ c_2(R,N)\,\beta(a)\,.
\]
\begin{lemma}[Explicit Hessian lower bound on chords]\label{lem:hessian-lower-chords}
There exist constants $\alpha_R=\alpha_R(R,N)>0$ and $d_0=d_0(R)<\infty$ such that for all chord configurations $A=(A_\ell)_\ell\in\mathfrak{su}(N)^{m(R,a)}$ in normal coordinates and all $a\in(0,a_0]$ with $\beta(a)\ge\beta_{\min}$,
\[
  \sum_{p\subset R} \phi\big(U_p(A)\big)\ \ge\ \tfrac{c_N}{4}\,a^4\,\sum_{\ell}\|A_\ell\|^2\ -\ C_R\,a^6\,\sum_{\ell}\|A_\ell\|^3,
\]
with $C_R=C_R(R,N)$. In particular, for all $\|A\|\le r_R$ (some $r_R>0$ depending only on $(R,N)$),
\[
  \nabla^2 V_R(A)\ \succeq\ \beta(a)\,\alpha_R\, I_{m(R,a)}\,.
\]
By compactness of $G^{m(R,a)}$ and that each chord enters at most $d_0$ plaquettes, this lower bound extends globally with a possibly smaller constant $\kappa_R=\kappa_R(R,N)>0$, yielding $\nabla^2 V_R\succeq \kappa_R\,\beta(a)\,I$.
\end{lemma}
\begin{proof}
The quadratic expansion of $\phi$ around the identity gives $\phi(U_p)= \tfrac{c_N}{2} a^4\|F_p(A)\|^2+O(a^6\|A\|^3)$. Summing over plaquettes and using that each $A_\ell$ appears in at most $d_0$ plaquettes with uniformly bounded coefficients yields the stated quadratic lower bound with a cubic remainder. For $\|A\|\le r_R$ small, the cubic term is absorbed into the quadratic, giving the local Hessian bound. A standard patching argument on the compact manifold, together with bounded interaction degree, propagates a uniform convexity constant $\kappa_R$ on all of $G^{m(R,a)}$.
\end{proof}

\emph{Step 3 (Lipschitz bound for $S_R$).} The map $X\mapsto S_R(U(X))$ is Lipschitz on $G^{m}$ with respect to the product Riemannian metric. Changing a single chord affects at most $d_0$ plaquettes. In the small-field regime where $U_p$ stays within $O(a^2)$ of $\mathbf 1$ (so $\|\nabla\phi(U_p)\|=O(a^2)$ uniformly on $R$), one gets an $a$-uniform bound on the global gradient. Without such weak-coupling control one only has the trivial bound $\|\nabla S_R\|_2^2\lesssim m(R,a)$, which grows like $a^{-4}$.
By the expansion above and compactness, there exist constants $C_1(R,N),C_2(R,N)$ such that
\[
  \|\nabla S_R\|_2^2\ \le\ C_1(R,N)\,a^4\ \le\ C_1(R,N)\,a_0^4\ :=\ G_R\,.
\]
\emph{Step 4 (Herbst bound and choice of $\eta_R$).} The LSI implies the subgaussian Laplace bound (Herbst argument): for all $t\in\mathbb{R}$,
\[
  \log\mathbb{E}_{\mu_R}\big[\exp\big(t(S_R-\mathbb{E}_{\mu_R}S_R)\big)\big]
  \ \le\ \frac{t^2}{2\rho_R}\,\|\nabla S_R\|_{L^2(\mu_R)}^2
  \ \le\ \frac{t^2 G_R}{2\,c_2(R,N)\,\beta(a)}\,.
\]
Let $\rho_{\min}:=c_2(R,N)\,\beta_{\min}>0$. Then for all $a\in(0,a_0]$,
\[
  \log\mathbb{E}_{\mu_R}\big[e^{t(S_R-\mathbb{E}S_R)}\big]\ \le\ \frac{t^2 G_R}{2\,\rho_{\min}}\,.
\]
Choose
\[
  \eta_R\ :=\ \min\Big\{\,t_*(R,N),\ \sqrt{\,\rho_{\min}/G_R\,}\,\Big\}
\]
with $t_*(R,N)$ a universal LSI radius (on compact groups) so that $\frac{\eta_R^2 G_R}{2\rho_{\min}}\le \tfrac12$. Then
\[
  \mathbb{E}_{\mu_R}\big[e^{\eta_R(S_R-\mathbb{E}S_R)}\big]\ \le\ e^{1/2}\,.
\]

\emph{Step 5 (Centering removal and conclusion).} By Assumption~\ref{assump:uei-mean}, $\mathbb E_{\mu_{L,a}}S_R\le M_R$ uniformly along the scaling window. Therefore
\[
  \mathbb{E}_{\mu_{L,a}}\!\left[e^{\eta_R S_R(U)}\right]
  \ =\ e^{\eta_R\,\mathbb{E}S_R}\,\mathbb{E}\big[e^{\eta_R(S_R-\mathbb{E}S_R)}\big]
  \ \le\ e^{\eta_R M_R}\,e^{1/2}
  \ :=\ C_R\,.
\]
This $C_R$ depends only on $(R,N,a_0,\beta_{\min})$. The bound holds uniformly in $L$ and $a\in(0,a_0]$.
\end{proof}
\medskip
\begin{proposition}[OS0/OS2 closure under limits]\label{prop:os0os2-closure}
Let $\{\mu_{a,L}\}$ be Wilson lattice measures with fixed link reflection and spacing $a\in(0,a_0]$, volumes $L a$ large, and assume Theorem~\ref{thm:uei-fixed-region} holds uniformly on every bounded physical region $R\subset\mathbb R^4$. Along any van Hove scaling sequence $(a_k,L_k)$ with $a_k\downarrow 0$ and $L_k a_k\to\infty$, there exists a subsequence (not relabeled) such that $\mu_{a_k,L_k}$ converges weakly on cylinder sets to a continuum probability measure $\mu$. The limit Schwinger functions satisfy:
\begin{itemize}
  \item OS0 (temperedness on loop/local fields) on each fixed region $R$;
  \item OS2 (reflection positivity) for the fixed link reflection.
\end{itemize}
\end{proposition}

\noindent\emph{Cylinder-set approximation for OS2 (explicit).} Fix a time-zero cylinder polynomial $F$ supported in a bounded region $R$. For each $a$, choose a local polynomial $F_a$ depending only on finitely many links in $R$ such that $\|F_a-F\|_{L^2(\mu_{a,L})}\to 0$ and $\|F\|_{L^2(\mu_{a_k,L_k})}$ remains uniformly bounded by UEI. Reflection positivity holds on each lattice: $\langle F_a,\Theta F_a\rangle_{\mu_{a,L}}\ge 0$. Along a van Hove subsequence with weak convergence on cylinders, $\langle F_a,\Theta F_a\rangle_{\mu_{a_k,L_k}}\to \langle F,\Theta F\rangle_{\mu}$ by dominated convergence and cylinder weak convergence. Hence $\langle F,\Theta F\rangle_{\mu}\ge 0$, proving OS2 for cylinder sets. Density of cylinder polynomials in the local field algebra (with the UEI bounds) extends OS2 to the stated class.

\noindent\emph{OS0 tightness transfer (one line).} The uniform exponential integrability on fixed regions (Theorem~\ref{thm:uei-fixed-region}) yields tightness of cylinder laws; Prokhorov compactness and the Daniell--Kolmogorov extension then pass OS0 to the limit on each fixed region $R$.

\begin{corollary}[OS2 passes to the continuum under AF/Mosco]\label{cor:os2-pass}
Under Assumption~\ref{assump:AF-Mosco} (Appendix~\ref{app:af-mosco}) and Corollary~\ref{cor:uei-af-uniform}, reflection positivity for time-zero cylinders is preserved in the limit; hence OS2 holds for the continuum Schwinger functions.
\end{corollary}

\begin{proposition}[OS3/OS5 in the continuum limit (scaling-window UCIS$_{\rm SW}$ route)]\label{prop:os35-limit}
Let $\{\mu_{a,L}\}$ be Wilson lattice measures along a van Hove scaling sequence $(a_k,L_k)$ with $a_k\downarrow 0$ and $L_k a_k\to\infty$, and let $\beta=\beta(a)$ be a monotone weak-coupling schedule with $\beta(a)\to\infty$ satisfying the scaling window \eqref{eq:ucis-sw-window}. Assume UCIS$_{\rm SW}$ holds along the schedule (Theorem~\ref{thm:ucis-sw}), and assume gap persistence under Mosco/NRC (Theorem~\ref{thm:gap-persist-cont}). Then the limit Schwinger functions satisfy:
\begin{itemize}
  \item OS3 (clustering): for time-separated observables $O_1,O_2$ supported in fixed bounded regions, $|\langle O_1(t)O_2(0)\rangle_c|\le C e^{-m t}$ with $m>0$ independent of $(a,L)$, hence clustering persists in the limit.
  \item OS5 (unique vacuum): the spectral gap persistence (Theorem~\ref{thm:gap-persist-cont}) implies that $0$ is an isolated simple eigenvalue of $H$, yielding vacuum uniqueness.
\end{itemize}
\end{proposition}
\begin{proof}
Fix $k$ and write $a=a_k$, $L=L_k$. Let $M(a)=\lceil T_{\rm phys}/a\rceil$ be as in Theorem~\ref{thm:ucis-sw} and set the fixed physical time $\tau_{\rm phys}:=a\,M(a)\in[T_{\rm phys},T_{\rm phys}+a]$. By UCIS$_{\rm SW}$ and Corollary~\ref{cor:ucis-sw-L2-contraction}, there exists $q_{\rm phys}\in(0,1)$ independent of $(L,\text{boundary})$ (on fixed slabs) such that
\[
  \big\|e^{-\tau_{\rm phys} H_{a,L}}\big\|_{\Omega^\perp\to\Omega^\perp}\ \le\ q_{\rm phys}.
\]
By semigroup submultiplicativity, for $t\ge 0$ write $t=n\tau_{\rm phys}+s$ with $n=\lfloor t/\tau_{\rm phys}\rfloor$ and $s\in[0,\tau_{\rm phys})$. Then
\[
  \|e^{-tH_{a,L}}\|_{\Omega^\perp}\ \le\ \|e^{-sH_{a,L}}\|\,\|e^{-n\tau_{\rm phys}H_{a,L}}\|_{\Omega^\perp}
  \ \le\ q_{\rm phys}^{\,n}\ \le\ \exp\!\Big(-\frac{-\log q_{\rm phys}}{\tau_{\rm phys}}\,t\Big).
\]
Since $\tau_{\rm phys}\le T_{\rm phys}+a_0$, the rate $m:=(-\log q_{\rm phys})/(T_{\rm phys}+a_0)>0$ is uniform in $a$ and $L$ along the scaling sequence. This implies exponential clustering of connected correlations for time-separated local observables with the same rate $m$, uniformly in $(a,L)$ (standard transfer-to-clustering argument on OS/GNS spaces). Gap persistence under Theorem~\ref{thm:gap-persist-cont} transports the gap to the limit generator $H$, yielding OS3 and OS5.
\end{proof}
\begin{proof}
\emph{Tightness.} On each fixed region $R$, Theorem~\ref{thm:uei-fixed-region} provides $\eta_R>0$ and $C_R<\infty$ with uniform exponential moment bounds. By Prokhorov's theorem, the family $\{\mu_{a,L}\}$ is tight on cylinders generated by loops/local fields supported in $R$, hence along a subsequence $\mu_{a_k,L_k}$ converges weakly to a probability measure $\mu_R$ on that cylinder $\sigma$-algebra. A diagonal argument over an exhausting sequence of regions identifies a unique limiting measure $\mu$ on cylinder sets.
\emph{OS2.} For a polynomial $P$ in loop/local fields supported in $t\ge 0$, reflection positivity on the lattice gives $\langle \Theta P_k\,\overline{P_k}\rangle_{\mu_{a_k,L_k}}\ge 0$. By weak convergence and boundedness of $\Theta P_k\,\overline{P_k}$ on cylinders, $\langle \Theta P\,\overline{P}\rangle_{\mu}=\lim_k \langle \Theta P_k\,\overline{P_k}\rangle_{\mu_{a_k,L_k}}\ge 0$.

\emph{OS0.} UEI yields uniform Laplace bounds for local curvature functionals, which by Kolmogorov--Chentsov imply Hölder control and, together with locality and standard tree-graph bounds (cf. Proposition~\ref{prop:OS0-poly}), polynomial moment bounds for $n$-point functions with exponents independent of $(a,L)$. Passing to the limit preserves these bounds, hence the Schwinger functions of $\mu$ are tempered distributions.
\end{proof}

\medskip

\subsection*{Calibrator\,--\,free isotropy via approximate symmetry and NRC}

\begin{theorem}[Symmetry emerges from uniform $O(a^2)$ commutator control]\label{thm:emergent-sym}
Fix a bounded region $R\Subset \mathbb R^4$ and let $H_a\ge 0$ and $H\ge 0$ be the lattice and continuum generators acting on the corresponding OS/GNS Hilbert spaces, with isometric embeddings $J_a$. For each rigid Euclidean motion $G\in E(4)$, let $U_a(G)$ be the unitary relabeling action on lattice observables in $R$ and $U(G)$ the target unitary on the continuum OS space. Assume:
\begin{itemize}
  \item[(i)] (NRC on $R$) $\|(H-z)^{-1}-J_a(H_a-z)^{-1}J_a^*\|\to 0$ as $a\downarrow 0$ for all $z\in\mathbb C\setminus\mathbb R$;
  \item[(ii)] (Approximate symmetry) For each $G$ and nonreal $z$,
  \[
    \big\|\,J_a U_a(G)(H_a-z)^{-1} U_a(G)^* J_a^*\ -\ U(G)(H-z)^{-1}U(G)^*\,\big\|\ \le\ C_G(z)\,a^2,
  \]
  with $C_G(z)$ independent of small $a$.
\end{itemize}
Then for every $G\in E(4)$ and $z\in\mathbb C\setminus\mathbb R$ one has
\[
  U(G)(H-z)^{-1}U(G)^*\ =\ (H-z)^{-1}.
\]
Equivalently, $[H,\,U(G)]=0$ on its natural domain. In particular, the continuum semigroup and Schwinger functions are invariant under the full Euclidean group (OS1) without any calibrator averaging.
\end{theorem}

\begin{proof}
Fix $G$ and $z\notin\mathbb R$. By (i) and (ii),
\[
\begin{aligned}
  &\big\|\,U(G)(H-z)^{-1}U(G)^*\ -\ (H-z)^{-1}\,\big\| \\
  &\le\ \big\|U(G)(H-z)^{-1}U(G)^* - J_a U_a(G)(H_a-z)^{-1}U_a(G)^* J_a^*\big\| \\
  &\quad +\ \big\|J_a U_a(G)(H_a-z)^{-1}U_a(G)^* J_a^* - (H-z)^{-1}\big\|
  
\end{aligned}
\]
The first term is $\le C_G(z) a^2$ by (ii). For the second, insert and subtract $J_a(H_a-z)^{-1}J_a^*$ and use unitary invariance of the norm to bound by $\|(H-z)^{-1}-J_a(H_a-z)^{-1}J_a^*\|$, which tends to $0$ by (i). Letting $a\downarrow 0$ yields $U(G)(H-z)^{-1}U(G)^*=(H-z)^{-1}$. Functional calculus implies $[H,U(G)]=0$.
\end{proof}

\medskip

\begin{lemma}[Local $O(a^2)$ E(4)\,--\,commutator bound on fixed regions]\label{lem:local-commutator-Oa2}
Fix a bounded region $R\Subset\mathbb R^4$ and a rigid motion $G\in E(4)$. There exist $a_0(R)>0$, $t_0(R)>0$, and a constant $C_{\rm comm}(R,G)<\infty$, independent of the volume and boundary conditions, such that for all $a\in(0,a_0]$, all van Hove boxes containing $R$, and all $t\in[0,t_0]$,
\[
  \big\|\, \big(U_a(G)\,e^{-t H_{a,L}} - e^{-t H_{a,L}}\,U_a(G)\big)\big|_{\mathcal V^{\rm loc}_{0,a,L}(R)}\, \big\|\ \le\ C_{\rm comm}(R,G)\, a^2\, t.
\]
Here $\|\cdot\|$ is the operator norm on the time-zero local subspace generated by observables supported in $R$.
\end{lemma}
\begin{proof}
Work on a fixed gauge-invariant local core $\mathcal C_U\subset \mathcal V^{\rm loc}_{0,a,L}(R)$ that is dense in the time-zero local OS/GNS space. By the Wilson heat\,–\,kernel sandwich on fixed cores with $O(a^2)$ control (Theorem~\ref{TS:sandwich_main}), for $t\in[0,t_0]$ one has a Strang envelope
\[
  e^{-t H_{a,L}}\ =\ e^{-\tfrac t2 E_a}\, e^{-t M_a}\, e^{-\tfrac t2 E_a}\ +\ R_a(t),\qquad \|R_a(t)\|\ \le\ C_R\, a^2\, t,
\]
with $E_a$ the electric part (sum of single\,–\,link Laplacians) and $M_a$ the magnetic part (plaquette potential), both acting only on finitely many links touching $R$ during time $t$. The relabeling $U_a(G)$ acts by rigidly rotating/translating the loop arguments followed by equivariant discretization; it preserves the OS reflection structure and the hypercubic symmetry.

Since $E_a$ and the Wilson measure are hypercubic\,–\,invariant, $[U_a(Q),E_a]=0$ for all hypercubic motions $Q$. For a general rigid $G$, approximate $G$ by hypercubic $Q$ and use the finite\,–\,stencil Taylor control for the magnetic part (the plaquette\,$\to F^2$ estimate, Theorem~\ref{DEC:plaquette-F2}) together with Lipschitz continuity of the product heat kernels on $G^{m(R,a)}$ to obtain
\[
  \big\|\,[U_a(G),\, e^{-\tfrac t2 E_a}]\,\big|_{\mathcal C_U}\big\|\ +\ \big\|\,[U_a(G),\, e^{-t M_a}]\,\big|_{\mathcal C_U}\big\|\ \le\ C'_{R,G}\, a^2\, t.
\]
Expanding the commutator of the Strang product and absorbing the $R_a(t)$ remainder yields the stated bound on $\mathcal C_U$, hence on $\mathcal V^{\rm loc}_{0,a,L}(R)$ by density. Uniformity in the volume follows because only the finite stencil touching $R$ enters.
\end{proof}

\begin{proposition}[Resolvent conjugation from semigroup commutators]\label{prop:resolvent-from-commutator}
Under the assumptions of Lemma~\ref{lem:local-commutator-Oa2}, for every $z\in\mathbb C\setminus\mathbb R$ there exists $C_G(z;R)<\infty$, independent of the volume, such that for all sufficiently small $a>0$,
\[
  \big\|\, J_a U_a(G) (H_{a,L}-z)^{-1} U_a(G)^* J_a^*\ -\ U(G)(H-z)^{-1}U(G)^*\,\big\|\ \le\ C_G(z;R)\, a^2.
\]
In particular, one may take $C_G(z;R)\le C_{\rm comm}(R,G) \int_0^{t_0} e^{-\operatorname{dist}(z,\mathbb R)\,t}\, t\,dt + C'_{R,G,z}$, with $C'_{R,G,z}$ depending only on $z$ and $R$.
\end{proposition}
\begin{proof}
Use the Laplace representation $(H_{a,L}-z)^{-1}=\int_0^\infty e^{tz}\, e^{-t H_{a,L}}\,dt$ (valid on the local core and extended by boundedness), conjugate by $U_a(G)$, and subtract the target formula for $(H-z)^{-1}$. Insert and subtract the embedded semigroup, split the integral at $t_0$, control the small\,–\,time contribution by Lemma~\ref{lem:local-commutator-Oa2}, and bound the tail by contractivity and $e^{t\Re z}$. Strong semigroup convergence on the core (Theorem~\ref{thm:strong-semigroup-core}) passes the remaining terms to the limit and yields the stated $O(a^2)$ bound.
\end{proof}

\begin{theorem}[OS1 on fixed regions (conditional; calibrator route)]\label{thm:os1-unconditional}
Fix a bounded region $R\Subset\mathbb R^4$. Assume UEI/equicontinuity on $R$ (Theorem~\ref{thm:uei-fixed-region}) and locality, and let $\{S_n\}$ be any van Hove limit Schwinger functions on $R$. Then the limit Schwinger functions satisfy OS1 (full Euclidean invariance): for all $n\ge 1$, rigid $G\in E(4)$, and loop families $\{\Gamma_i\}_{i=1}^n\subset R$,
\[
  S_n(G\Gamma_1,\dots,G\Gamma_n)\ =\ S_n(\Gamma_1,\dots,\Gamma_n).
\]
\end{theorem}
\begin{proof}
Apply the constructed-calibrator route (Theorem~\ref{thm:os1-calibrator-route}), which upgrades hypercubic symmetry to full $SO(4)$ invariance on fixed regions by isotropic heat-kernel smoothing and an $\epsilon\downarrow 0$ limit justified by UEI/equicontinuity.
\end{proof}

\begin{mdframed}[linewidth=0.5pt, linecolor=gray!40, backgroundcolor=gray!6, roundcorner=2pt, innertopmargin=8pt, innerbottommargin=8pt, skipabove=10pt, skipbelow=10pt]
\noindent\textbf{Remark (Optional commutator route).}
An alternative proof proceeds by resolvent conjugation from local semigroup commutator bounds: `lem:local-commutator-Oa2` \(\Rightarrow\) `prop:resolvent-from-commutator` \(\Rightarrow\) `thm:emergent-sym`. This route currently invokes the outline-only input `TS:sandwich_main`; it is kept as a cross-check rather than the mainline OS1 closure. From an RS viewpoint (T6/T7), OS1 is expected to emerge from averaging/projection onto invariants of a finite “cube symmetry” shadow (e.g. the hyperoctahedral group $B_3$ of order $48$), which aligns naturally with the calibrator/averaging construction.
\end{mdframed}

\paragraph{Alternative (cross\,–\,check): isotropic heat\,–\,kernel calibrators, constructed.}
\begin{proposition}[Hypercubic\,–\,equivariant isotropic calibrators; construction]\label{prop:hk-calibrators-constructed}
For $\epsilon\in(0,1]$, define $\mathcal C_\epsilon$ on loop cylinder functionals by convolving each link variable with the heat kernel $P_{\epsilon^2}$ on $\mathrm{SU}(N)$ (independently across links), followed by projection back to loops at scale $a$ and hypercubic averaging. Then $\mathcal C_\epsilon$ commute with OS reflection and the hypercubic group; satisfy $\|\mathcal C_\epsilon F - F\|\le C_R\,\epsilon^2\,\|F\|_{C^2}$ on loop cylinders supported in fixed $R$; and are isotropic in the sense that for any rigid $G\in SO(4)$,
\[
  \big|\, S^{(a,L)}_n\big(\mathcal C_\epsilon F_{\Gamma_1},\dots,\mathcal C_\epsilon F_{\Gamma_n}\big)\ -\ S^{(a,L)}_n\big(\mathcal C_\epsilon F_{G\Gamma_1},\dots,\mathcal C_\epsilon F_{G\Gamma_n}\big)\,\big|\ \le\ C_{R,n}\,\epsilon^2,
\]
uniformly in $(a,L)$.
\end{proposition}
\begin{proof}
Hypercubic commutation and reflection positivity are immediate from symmetry of $P_{t}$ and group averaging. The $\epsilon^2$ approximation follows from the heat kernel's second\,–\,order Taylor expansion and the bounded degree of loop functionals on fixed regions (tree\,–\,gauge Lipschitz bounds). Isotropy of $\mathcal C_\epsilon$ holds because $P_t$ is a class function and radial in the Riemannian metric, hence invariant under rigid rotations of the embedded loops. Uniformity in $(a,L)$ is due to locality on $R$.
\end{proof}

\begin{theorem}[OS1 via calibrated equicontinuity (constructed)]\label{thm:os1-calibrator-route}
Assume UEI/equicontinuity on fixed regions. Then, using the calibrators $\mathcal C_\epsilon$ from Proposition~\ref{prop:hk-calibrators-constructed} and letting $\epsilon\downarrow 0$ after $a\downarrow 0$ along any van Hove sequence, the limit Schwinger functions satisfy OS1 on $R$:
\[
  S_n(G\Gamma_1,\dots,G\Gamma_n)\ =\ S_n(\Gamma_1,\dots,\Gamma_n)\qquad(\forall G\in SO(4)).
\]
In particular, the OS1 statements in the manuscript hold without calibrator hypotheses.
\end{theorem}
\begin{proof}
For fixed $\epsilon>0$, Proposition~\ref{prop:hk-calibrators-constructed} gives isotropy up to $O(\epsilon^2)$ uniformly in $(a,L)$. UEI/equicontinuity yields precompactness and allows passing $a\downarrow 0$ to obtain limit Schwinger functions for the calibrated functionals. Letting $\epsilon\downarrow 0$ removes the calibration, by $\|\mathcal C_\epsilon F-F\|\to 0$ uniformly on $R$, and preserves isotropy by the $O(\epsilon^2)$ bound. This gives OS1 for the uncalibrated limits.
\end{proof}

\appendix
\section{Euclidean Invariance (OS1) via Equicontinuity and Isotropic Calibrators}

\medskip
\section{Norm--Resolvent Convergence via Embeddings and Resolvent Comparison}

\subsection*{Continuum OS Limit Hilbert Space and Embeddings}
Fix a van Hove scaling sequence $(a_k,L_k)$ and let $\{\mu_{a_k,L_k}\}$ be the corresponding OS-positive lattice measures. By tightness of time-zero local observables on fixed regions (UEI) and consistency of Schwinger functions, there exists a subsequence (not relabeled) and a limit OS measure $\mu$ with OS0--OS2 on time-zero algebras. Denote by $\mathcal H$ the OS/GNS Hilbert space of $\mu$ with vacuum $\Omega$ and semigroup $e^{-tH}$.

For each $(a,L)$, let $\mathcal H_{a,L}$ be the lattice OS/GNS space and let $\mathcal V^{\rm loc}_0$ (resp. $\mathcal V^{\rm loc}_{0,a,L}$) be the time-zero local vectors for $\mathcal H$ (resp. $\mathcal H_{a,L}$). Define the embedding on generators
\[
  I_{a,L}\,:\, \mathcal V^{\rm loc}_{0,a,L}\ \to\ \mathcal H,
  \qquad I_{a,L}[F]\ :=\ [E_a(F)],
\]
where $E_a$ maps lattice loops/fields to their polygonal/smeared counterparts in the continuum region. By OS positivity and equivariance, $I_{a,L}[F]:=[E_a(F)]$ is an isometry on the OS/GNS quotients and $P_{a,L}:=I_{a,L}I_{a,L}^*$ are orthogonal projections onto $\mathrm{Ran}(I_{a,L})\subset\mathcal H$; we keep the same notation for the extension and its adjoint $I_{a,L}^*$.

\subsection*{Cores and Consistency}
Let $\mathcal D\subset\mathcal H$ be the algebraic span of time-zero local vectors, and let $\mathcal D_{a,L}\subset \mathcal H_{a,L}$ be the analogous span. Both are cores for $H$ and $H_{a,L}$ by OS semigroup theory (Engel--Nagel, Kato). The embeddings satisfy $I_{a,L}\mathcal D_{a,L}\subset\mathcal D$ and are compatible with time translations on generators.

\begin{theorem}[Strong semigroup convergence on a core]\label{thm:strong-semigroup-core}
For each fixed $t\ge 0$ and $\xi\in\mathcal D$, one has
\[
  \lim_{k\to\infty}\ \big\|e^{-tH}\xi\ -\ I_{a_k,L_k}\,e^{-tH_{a_k,L_k}}\,I_{a_k,L_k}^*\,\xi\big\|\ =\ 0.
\]
In particular, $I_{a_k,L_k}\,e^{-tH_{a_k,L_k}}\,I_{a_k,L_k}^*\to e^{-tH}$ strongly on $\mathcal H$ for each $t\ge 0$.
\end{theorem}
\begin{proof}
On time-zero local vectors $\xi=[O]\in\mathcal D$, OS/GNS expresses matrix elements of $e^{-tH}$ as Schwinger functions of time-shifted observables. Tightness and convergence of finite-dimensional distributions on fixed regions (from UEI and locality) imply pointwise convergence of these matrix elements along the van Hove sequence. Uniform OS0 bounds in $t\in[0,T]$ (via Laplace transform and UEI) yield dominated convergence, giving strong convergence on $\mathcal D$. Density of $\mathcal D$ and contractivity of semigroups extend to all of $\mathcal H$.
\end{proof}

\begin{proposition}[Collective compactness calibrator]\label{prop:collective-compactness}
Fix $z_0\in\mathbb C\setminus\mathbb R$ and $\Lambda>0$. There exists a finite-rank operator $Q=Q(z_0,\Lambda)$ on $\mathcal H$ with $\|Q\|\le 1$ and spectral support in $E_H([0,\Lambda])$ such that for all large $k$,
\[
  \big\|I_{a_k,L_k}(H_{a_k,L_k}-z_0)^{-1}I_{a_k,L_k}^* - (H-z_0)^{-1}Q\big\|\ \le\ C\,a_k,
\]
with $C=C(z_0,\Lambda)$ independent of $k$. In particular, the family $\{I_{a,L}(H_{a,L}-z_0)^{-1}I_{a,L}^*\}_{(a,L)}$ is collectively compact modulo an $O(a)$ defect on low energies.
\end{proposition}
\begin{proof}
Approximate $E_H([0,\Lambda])$ by finite-rank projectors on the span of finitely many time-zero local vectors; define $Q$ as this finite-rank projection composed with $E_H([0,\Lambda])$. Strong convergence of semigroups (Theorem~\ref{thm:strong-semigroup-core}) implies strong resolvent convergence on $E_H([0,\Lambda])\mathcal H$; the graph-defect bound (Thm.~\ref{thm:quant-calibrated-af-free-nrc}(D)) and the weighted resolvent bound (Lemma~\ref{lem:weighted-resolvent}) upgrade to the stated operator-norm $O(a)$ estimate. Compactness follows since $Q$ is finite rank and the high-energy tail is bounded by $\operatorname{dist}(z_0,[\Lambda,\infty))^{-1}$.
\end{proof}

\begin{theorem}[Operator-norm NRC via collective compactness]\label{thm:nrc-operator-norm}
For every nonreal $z\in\mathbb C\setminus\mathbb R$,
\[
  \big\|(H-z)^{-1} - I_{a_k,L_k}\,(H_{a_k,L_k}-z)^{-1}\,I_{a_k,L_k}^*\big\|\ \xrightarrow[k\to\infty]{}\ 0.
\]
Moreover, for fixed $z_0\in\mathbb C\setminus\mathbb R$ there exists $C(z_0)>0$ with
\[
  \big\|(H-z_0)^{-1} - I_{a,L}(H_{a,L}-z_0)^{-1} I_{a,L}^*\big\|\ \le\ C(z_0)\,a\ +\ o_{L\to\infty}(1).
\]
\end{theorem}
\begin{proof}
Combine Theorem~\ref{thm:strong-semigroup-core} with Proposition~\ref{prop:collective-compactness} and the comparison identity (R3) to control the low-energy part in operator norm, and use the resolvent bound on the high-energy complement. A standard diagonal argument passes from $z_0$ to any nonreal $z$ by the second resolvent identity and compactness of $\{\Im z\ne 0:|z|\le R\}$.
\end{proof}

\begin{theorem}[NRC for all nonreal $z$ along a scaling sequence]\label{thm:nrc-embeddings}
Let $\{\mu_{a,L}\}$ be the OS-positive Wilson lattice measures with transfer $T_{a,L}=e^{-H_{a,L}}$ and OS/GNS Hilbert spaces $\mathcal H_{a,L}$. Assume UEI on fixed regions and locality as above. By Thm.~\ref{thm:quant-calibrated-af-free-nrc}(D,F,G), Lem.~\ref{lem:U2-comparison}, Prop.~\ref{prop:one-point-resolvent}, and Thm.~\ref{thm:U2-nrc-unique}, along any van Hove scaling sequence $(a_k,L_k)$ there exists a subsequence (not relabeled), a Hilbert space $\mathcal H$, and a positive self-adjoint $H\ge 0$ such that for every nonreal $z$,
\[
  \big\|(H-z)^{-1} - I_{a_k,L_k}\,(H_{a_k,L_k}-z)^{-1}\,I_{a_k,L_k}^*\big\|\;\xrightarrow[k\to\infty]{}\;0,
\]
where $I_{a,L}:\mathcal H_{a,L}\to\mathcal H$ are isometric embeddings induced by equivariant polygonal loop embeddings. In particular, the semigroups $I_{a_k,L_k} e^{-tH_{a_k,L_k}} I_{a_k,L_k}^*$ converge in operator norm to $e^{-tH}$ for all $t\ge 0$.
\end{theorem}
\noindent\emph{Remark (consistency).} Theorems~\ref{thm:strong-semigroup-core} and \ref{thm:nrc-operator-norm} refine and justify the operator-norm NRC stated here and in Theorem~\ref{thm:nrc-quant}, making explicit the embeddings, cores, and compactness inputs, with constants depending only on $(R_*,a_0,G)$ and $z$.

\begin{lemma}[AF-free resolvent Cauchy criterion on a nonreal compact]\label{lem:af-free-cauchy}
Let $K\subset \mathbb C\\\mathbb R$ be compact. Suppose: (i) the graph-defect bound of Thm.~\ref{thm:quant-calibrated-af-free-nrc}(D) holds; (ii) the low-energy projection control of Lemma~\ref{lem:low-energy-proj} holds; and (iii) for some $z_0\in K$ the NRC estimate of Theorem~\ref{thm:nrc-quant} holds with rate $\le C(z_0) a$. Then there exists $C_K>0$ such that for all $z\in K$ and van Hove pairs $(a,L)$, $(a',L')$,
\[
  \big\| I_{a,L}(H_{a,L}-z)^{-1} I_{a,L}^* - I_{a',L'}(H_{a',L'}-z)^{-1} I_{a',L'}^* \big\|
  \ \le\ C_K\,(a+a')\ +\ o_{L,L'\to\infty}(1).
\]
In particular, the embedded resolvents form a Cauchy net on $K$ without assuming an AF schedule.
\begin{proof}
By the second resolvent identity, for any $z\in K$ and fixed $w\in K$,
\[
  R_{a}(z)-R_{a}(w)\ =\ (z-w) R_{a}(z) R_{a}(w),\qquad R_{a'}(z)-R_{a'}(w)\ =\ (z-w) R_{a'}(z) R_{a'}(w).
\]
Taking differences and embedding, one obtains
\[
  I_{a}R_{a}(z)I_{a}^* - I_{a'}R_{a'}(z)I_{a'}^*\ =\ [I_{a}R_{a}(w)I_{a}^* - I_{a'}R_{a'}(w)I_{a'}^*]\,\Xi(z,w),
\]
where $\Xi(z,w)=I+(z-w)\,I_{a'}R_{a'}(z)I_{a'}^*$ on the right and similarly bounded on the left. On $K$, resolvent norms are uniformly bounded by $\operatorname{dist}(K,\mathbb R)^{-1}$. Choosing $w=z_0$ and using Theorem~\ref{thm:nrc-quant} at $z_0$ together with Lemmas~\ref{lem:graph-defect-Oa},\ref{lem:low-energy-proj} and the comparison identity yields the $O(a+a')$ bound at $z_0$. Uniform boundedness of the multipliers over $K$ transfers the Cauchy rate from $z_0$ to all $z\in K$ with a constant $C_K$.
\end{proof}
\end{lemma}
\begin{proof}
\emph{Embeddings.} Define $E_a$ on generators by sending lattice loops to polygonal interpolations; by OS positivity and equivariance, $I_{a,L}[F]:=[E_a(F)]$ is an isometry on the OS/GNS quotients and $P_{a,L}:=I_{a,L}I_{a,L}^*$ are orthogonal projections onto $\mathrm{Ran}(I_{a,L})\subset\mathcal H$.

\emph{Graph-norm defect.} Let $D_{a,L}:=H\,I_{a,L}-I_{a,L} H_{a,L}$ on a common dense core of time-zero local vectors. Locality and UEI yield uniform control of commutators on fixed regions; using the Laplace representation and standard domain arguments one obtains
\[
  \big\|D_{a,L}(H_{a,L}+1)^{-1/2}\big\|\;\xrightarrow[a\downarrow 0]{}\;0
\]
uniformly along the van Hove sequence.
\emph{Finite-volume calibrator and comparison identity.} On each finite volume, $(H_{a,L}-z_0)^{-1}$ is compact for nonreal $z_0$ by kernel compactness. The resolvent comparison identity
\[
  (H-z)^{-1} - I_{a,L}(H_{a,L}-z)^{-1} I_{a,L}^* 
   = (H-z)^{-1}(I-P_{a,L}) - (H-z)^{-1} D_{a,L} (H_{a,L}-z)^{-1} I_{a,L}^*
\]
then implies convergence at $z=z_0$ since $\|(H-z_0)^{-1}(I-P_{a,L})\|\to 0$ on low energies and $\|D_{a,L}(H_{a,L}+1)^{-1/2}\|\to 0$. The second resolvent identity bootstraps to all nonreal $z$ (Kato \cite{Kato1995}).

\begin{proposition}[Resolvent comparison identity and domains]\label{prop:resolvent-comparison}
Let $P_{a,L}:=I_{a,L}I_{a,L}^*$ and $D_{a,L}:= H I_{a,L} - I_{a,L} H_{a,L}$ defined on the common OS/GNS time-zero local core $\mathcal D$ (Lemma~\ref{lem:local-core}). Then for any $z\in\mathbb C\setminus\mathbb R$,
\[
  (H-z)^{-1} - I_{a,L}(H_{a,L}-z)^{-1} I_{a,L}^* 
   = (H-z)^{-1}(I-P_{a,L}) - (H-z)^{-1} D_{a,L} (H_{a,L}-z)^{-1} I_{a,L}^*.
\]
Moreover, $D_{a,L}(H_{a,L}+1)^{-1/2}$ extends by density to a bounded operator with $\|D_{a,L}(H_{a,L}+1)^{-1/2}\|\le C_{\rm gd}\,a$ (Thm.~\ref{thm:quant-calibrated-af-free-nrc}(D)).
\end{proposition}

\emph{Exhaustion.} Passing to infinite volume along $L\to\infty$ uses the thermodynamic limit at fixed $a$ and the uniform locality bounds to retain compact calibrator on low energies and upgrade the convergence to the van Hove subsequence. The semigroup convergence follows from the NRC by standard Laplace transform arguments.
\end{proof}
\begin{proposition}[graph--defect $O(a)$ on fixed slabs: form-level criterion]\label{lem:graph-defect-Oa}
Let $H$ be the nonnegative self-adjoint continuum generator on $L^2(\Omega_T;\mathbb{C}^m)$ over a fixed bounded Lipschitz slab $\Omega_T\subset\mathbb{R}^4$, and let $H_a$ be its lattice discretization on $\ell^2(\Omega_{T,a};\mathbb{C}^m)$ with mesh $a>0$. Let $J_a:L^2\to\ell^2$ be the cell-averaging injection and $J_a^*$ the piecewise-constant extension, with $\|J_a\|\le 1$, $\|J_a^*\|\le 1$. Assume the uniform energy equivalence
\[
\alpha\big(\|(H+1)^{1/2}u\|_2^2\big)\ \le\ \mathcal{E}(u,u)+\|u\|_2^2\ \le\ \beta\big(\|(H+1)^{1/2}u\|_2^2\big),
\]
and similarly for $H_a$ with the same $\alpha,\beta$, independent of $a$. Assume first–order consistency for the covariant gradient and potential:
\[
\big\|\nabla_{A,a}(J_a u)-\Pi_a(\nabla_A u)\big\|_{\ell^2}\ \le\ K_{\nabla}\,a\,\|u\|_{H^{2}_A(\Omega_T)},\qquad
\big\|V_a J_a u- J_a(Vu)\big\|_{\ell^2}\ \le\ K_V\,a\,\|u\|_{H^{1}_A(\Omega_T)},
\]
where $K_{\nabla}=c_4\,c_G\big(1+\|A\|_{W^{1,\infty}}+\|F\|_{L^\infty}\big)$ and $K_V=\|V\|_{W^{1,\infty}}$ depend only on $\Omega_T$, the representation of $\mathrm{SU}(N)$ (via $c_G$), and uniform bounds on the gauge data, not on $a$. Then with
\[
C_D\ :=\ \sqrt{\tfrac{\beta}{\alpha}}\,(K_{\nabla}+K_V),
\]
the defect operator $D_a:=H_a J_a - J_a H$ satisfies the energy–weighted bound
\[
\big\|(H_a+1)^{-1/2}\,D_a\,(H+1)^{-1/2}\big\|\ \le\ C_D\,a.
\]
\end{proposition}
\noindent\emph{Note.} In the Yang--Mills setting on fixed slabs, the uniform energy equivalence follows from positivity/locality of the electric (link--Laplacian) part together with the finite--region DEC plaquette$\to F^2$ control for the magnetic part (Theorem~\ref{DEC:plaquette-F2}). The first--order consistency bounds for the covariant gradient and potential follow from the equivariant polygonal embeddings and cell--averaging on Lipschitz domains, with constants depending only on $(R_*,a_0,N)$ and not on $(a,L)$. Therefore the proposition applies with explicit constants and no circular inputs.

\begin{proof}
For $u\in\mathrm{Dom}(H^{1/2})$ and $v_a\in\mathrm{Dom}(H_a^{1/2})$,
\[
\langle v_a, D_a u\rangle_{\ell^2}
= \mathcal{E}_a(J_a u, v_a) - \mathcal{E}(u, J_a^* v_a).
\]
Split each form into covariant-gradient and potential parts. The stated first–order consistency bounds control the gradient and potential discrepancies by $a K_{\nabla}\|u\|_{H^2_A}\|v_a\|_{H^1_{A,a}}$ and $a K_V\|u\|_{H^1_A}\|v_a\|_{\ell^2}$, respectively. Uniform energy equivalence converts these to the $(H+1)^{1/2}$ / $(H_a+1)^{1/2}$ norms with the factor $\sqrt{\beta/\alpha}$. Taking the operator norm in the product of energy norms yields the claim.
\end{proof}

% (Semigroup-based auxiliary proof removed; covered by the form-level argument above.)

\medskip
\begin{lemma}[Electric graph--defect $O(a)$]\label{lem:U2-electric}
Let $H=E+M$ and $H_a=E_a+M_a$ be the kinetic/potential splits (electric $E,E_a$ and magnetic $M,M_a$) on a common algebraic OS/GNS core $\mathcal D^{\rm loc}$ of time--zero local vectors supported in a fixed slab (cf.~Lemma~\ref{lem:U2-defect-core}). There exists $C_E=C_E(R_*,a_0,G)>0$ such that
\[
  \big\|(E_a+1)^{-1/2}\,(E_a J_a - J_a E)\,(E+1)^{-1/2}\big\|\ \le\ C_E\, a.
\]
\emph{Proof.} Apply Proposition~\ref{lem:graph-defect-Oa} with $V\equiv 0$. On the common algebraic core $\mathcal D^{\rm loc}$, the covariant gradient admits the first--order consistency estimate under equivariant polygonal embedding and cell--averaging on fixed slabs:
\[
  \big\|\nabla_{A,a}(J_a u)-\Pi_a(\nabla_A u)\big\|_{\ell^2}\ \le\ K_{\nabla}\,a\,\|u\|_{H^{2}_A},\qquad K_{\nabla}=c_4\,c_G\big(1+\|A\|_{W^{1,\infty}}+\|F\|_{L^\infty}\big),
\]
uniformly in $(a,L)$. Together with energy equivalence for $E,E_a$, Proposition~\ref{lem:graph-defect-Oa} yields
\[
  \big\|(E_a+1)^{-1/2}\,(E_a J_a - J_a E)\,(E+1)^{-1/2}\big\|\ \le\ \sqrt{\tfrac{\beta}{\alpha}}\,K_{\nabla}\,a.
\]
Absorbing $\sqrt{\beta/\alpha}\,K_{\nabla}$ into $C_E=C_E(R_*,a_0,G)$ gives the claim. \qed
\end{lemma}

\begin{lemma}[Magnetic graph--defect $O(a)$]\label{lem:U2-magnetic}
Under the DEC plaquette$\to F^2$ approximation on fixed slabs, there exists $C_M=C_M(R_*,a_0,G)>0$ such that
\[
  \big\|(M_a+1)^{-1/2}\,(M_a J_a - J_a M)\,(M+1)^{-1/2}\big\|\ \le\ C_M\, a.
\]
\emph{Proof.} On $\mathcal D^{\rm loc}$, Theorem~\ref{DEC:plaquette-F2} furnishes the gauge--invariant second--order control
\[
  \big|\,\langle \psi, M\psi\rangle - \langle J_a\psi, M_a J_a\psi\rangle\,\big|\ \le\ C_{\mathrm{DEC}}\,a^2\,\| (M+1)^{1/2}\psi\|^2,
\]
uniformly on fixed slabs. By polarization, for all $u\in\mathrm{Dom}(M^{1/2})$ and $v_a\in\mathrm{Dom}(M_a^{1/2})$,
\[
  \big|\,\langle v_a, (M_a J_a - J_a M) u\rangle\,\big|\ \le\ C_{\mathrm{DEC}}\,a^2\,\|(M_a+1)^{1/2}v_a\|\,\|(M+1)^{1/2}u\|.
\]
Hence
\[
  \big\|(M_a+1)^{-1/2}\,(M_a J_a - J_a M)\,(M+1)^{-1/2}\big\|\ \le\ C_{\mathrm{DEC}}\,a^2\ \le\ C_M\,a
\]
for all $a\in(0,a_0]$ after enlarging the constant to $C_M:=C_{\mathrm{DEC}}\,a_0$. \qed
\end{lemma}

\begin{theorem}[U2 on fixed slabs: graph--defect $O(a)$ and low--energy projectors]\label{thm:U2}
Fix a bounded Lipschitz slab $\Omega_T\Subset\mathbb R^4$ and $N\ge 2$. Let $H\ge 0$ be the continuum Euclidean generator on the OS/GNS Hilbert space $\mathcal H$ for $\Omega_T$, and let $H_a\ge 0$ be the lattice generator on $\mathcal H_a$ with mesh $a\in(0,a_0]$. Let $J_a:\mathcal H\to\mathcal H_a$ be the canonical cell--averaging injection and $J_a^*$ its adjoint, with $\|J_a\|,\|J_a^*\|\le 1$. Then, uniformly in the volume and along van Hove sequences:
\begin{itemize}
  \item[(A)] \textbf{Graph--defect bound.} There exists $C_{\mathrm{gd}}=C_{\mathrm{gd}}(R_*,a_0,G)>0$ such that
  \[
    \big\|(H_a+1)^{-1/2}\,(H_a J_a - J_a H)\,(H+1)^{-1/2}\big\|\ \le\ C_{\mathrm{gd}}\, a.
  \]
  \item[(B)] \textbf{Low--energy projector bound.} Let $\Lambda>0$ with $g:=\mathrm{dist}(\Lambda,\sigma(H))>0$. Then there exists
  \[
    C_\Lambda\ =\ \frac{2(\Lambda+g+1)}{g}\, C_{\mathrm{gd}}\ =\ C_\Lambda(\Lambda,g,R_*,a_0,N)
  \]
  such that for all sufficiently small $a$,
  \[
    \big\|\,\mathbf 1_{[0,\Lambda]}(H_a)\ -\ J_a\,\mathbf 1_{[0,\Lambda]}(H)\,J_a^*\,\big\|\ \le\ C_\Lambda\, a.
  \]
\end{itemize}
\end{theorem}

\begin{corollary}[Low--energy projector via contour; unconditional]\label{lem:low-energy-proj}
Fix $\Lambda>0$ and suppose $g:=\mathrm{dist}(\Lambda,\sigma(H))>0$. Then for all sufficiently small $a$,
\[
  \big\|\,\mathbf 1_{[0,\Lambda]}(H_a)\ -\ J_a\,\mathbf 1_{[0,\Lambda]}(H)\,J_a^*\,\big\|\ \le\ C_\Lambda\, a,\qquad C_\Lambda\ :=\ \frac{2(\Lambda+g+1)}{g}\, C_{\mathrm{gd}},
\]
where $C_{\mathrm{gd}}\le C_E+C_M$ from Lemmas~\ref{lem:U2-electric}--\ref{lem:U2-magnetic}.
\end{corollary}

\begin{proof}
Let $\eta:=g/2$ and take $\Gamma$ the standard horizontal contour at $\pm i\eta$ from $x=-1$ to $x=\Lambda+g/2$, closed by quarter-circles of radius $\eta$ around the endpoints. By the spectral theorem and the resolvent identity,
\[
\mathbf{1}_{[0,\Lambda]}(H)=\frac{1}{2\pi i}\oint_{\Gamma} (H-z)^{-1}\,dz,\qquad
\mathbf{1}_{[0,\Lambda]}(H_a)=\frac{1}{2\pi i}\oint_{\Gamma} (H_a-z)^{-1}\,dz,
\]
for $a$ small enough that $\mathrm{dist}(\Gamma,\sigma(H_a))\ge \eta/2$ (norm–resolvent stability off the real axis, ensured by the previous lemma). Using
\[
(H_a-z)^{-1}J_a - J_a(H-z)^{-1} = (H_a-z)^{-1}\,D_a\,(H-z)^{-1},
\]
we obtain for $z\in\Gamma$,
\[
\big\|(H_a-z)^{-1}J_a - J_a(H-z)^{-1}\big\|
\ \le\ \|(H_a-z)^{-1}(H_a+1)^{1/2}\|\ \cdot\ \|(H_a+1)^{-1/2}D_a(H+1)^{-1/2}\|\ \cdot\ \|(H+1)^{1/2}(H-z)^{-1}\|.
\]
By Proposition \ref{lem:graph-defect-Oa} the middle factor is $\le C_D a$. On $\Gamma$ we have $|\Im z|=\eta$, so the outer factors are bounded by $\sup_{x\ge0}\frac{\sqrt{x+1}}{|x-z|}\le \frac{\Re z + 1 + |\Im z|}{(\Im z)^2}\le \frac{\Lambda+1+g}{(g/2)^2}$. Integrating over $\Gamma$,
\[
\big\|\mathbf{1}_{[0,\Lambda]}(H_a) - J_a\mathbf{1}_{[0,\Lambda]}(H)J_a^*\big\|
\ \le\ \frac{\ell(\Gamma)}{2\pi}\,\sup_{z\in\Gamma}\frac{\Re z + 1 + |\Im z|}{(\Im z)^2}\,C_D\,a
\ \le\ \frac{8}{\pi}\,\frac{(\Lambda+1+g)^2}{g^{2}}\,C_D\,a,
\]
using $\ell(\Gamma)\le 4(\Lambda+1+g)$ and $|\Im z|=\eta=g/2$ throughout. 
\end{proof}

\begin{theorem}[Quantitative operator-norm NRC for all nonreal $z$]\label{thm:nrc-quant}
Fix $z\in\mathbb C\setminus\mathbb R$ and $\Lambda>0$. There exists $C(z,\Lambda)>0$ independent of $(a,L)$ such that
\[
  \big\|(H-z)^{-1} - I_{a,L}(H_{a,L}-z)^{-1} I_{a,L}^*\big\|\ \le\ C(z,\Lambda)\,a\ \ +\ \frac{1}{\operatorname{dist}(z,[\Lambda,\infty))}.
\]
In particular, choosing $\Lambda\to\infty$ slowly with $a\downarrow 0$ gives a linear rate $O(a)$ uniformly on compact subsets of $\mathbb C\setminus\mathbb R$.
\end{theorem}
\noindent\emph{Remark (rate and constants).} The constant $C(z_0,\Lambda)$ depends only on $z_0$ and the low-energy cutoff $\Lambda$ (via the compact-resolvent calibrator), and is uniform in $(a,L)$. Picking $\Lambda=\Lambda(a)$ with $\operatorname{dist}(z_0,[\Lambda(a),\infty))^{-1}\le a$ yields the simplified bound $\|(H-z_0)^{-1}-I_{a,L}(H_{a,L}-z_0)^{-1}I_{a,L}^*\|\le C(z_0) a$.
\begin{lemma}[Cauchy criterion for embedded resolvents; uniqueness]\label{lem:cauchy-resolvent-unique}
Let $z\in\mathbb C\setminus\mathbb R$ be fixed. Suppose Theorem~\ref{thm:nrc-quant} holds with a rate $\le C(z) a$ after choosing $\Lambda=\Lambda(a)$ as in the remark. Then for any two spacings $a,a'\in(0,a_0]$ and volumes large enough along the van Hove window,
\[
  \big\| I_{a,L}(H_{a,L}-z)^{-1} I_{a,L}^* - I_{a',L'}(H_{a',L'}-z)^{-1} I_{a',L'}^*\big\|\ \le\ C(z)\,(a+a')\ +\ o_{L,L'\to\infty}(1).
\]
In particular, along any van Hove scaling sequence $(a_k,L_k)$ with $a_k\downarrow 0$, the embedded resolvents form a Cauchy sequence in operator norm and converge uniquely (no subsequences) to $(H-z)^{-1}$.
\end{lemma}
\begin{proof}
Fix $z_0$ and choose $\Lambda(a)$, $\Lambda(a')$ as in Theorem~\ref{thm:nrc-quant}. Add and subtract $(H-z_0)^{-1}$ and apply the triangle inequality:
\[
\begin{aligned}
\| I_{a}R_a I_{a}^* - I_{a'}R_{a'} I_{a'}^* \|
&\le \| I_{a}R_a I_{a}^* - R \| + \| R - I_{a'}R_{a'} I_{a'}^* \|\\
&\le C(z_0) a + C(z_0) a'\ +\ o_{L,L'\to\infty}(1),
\end{aligned}
\]
where $R_a=(H_{a,L}-z_0)^{-1}$, $R_{a'}=(H_{a',L'}-z_0)^{-1}$, and $R=(H-z_0)^{-1}$. The $o(1)$ terms encode the finite-volume calibrator error, which vanishes along the van Hove window by the compactness/exhaustion step used in Theorem~\ref{thm:nrc-embeddings}. Therefore the sequence is Cauchy and the limit is unique.
\end{proof}
\begin{corollary}[Unique Schwinger limits for local fields]\label{cor:unique-schwinger-local}
Let $\mathcal A^{\rm loc}$ be the polynomial *\,–\,algebra generated by smeared local gauge\,–\,invariant fields from Section~\ref{subsec:local-fields-tempered}. Along any van Hove scaling sequence $(a_k,L_k)$ with $a_k\downarrow 0$, the $n$\,–\,point Schwinger functions on $\mathcal A^{\rm loc}$ converge uniquely (no subsequences) to the continuum limits determined by $H$ and OS0--OS5. Equivalently, for each finite family of smearings, $\{\langle \prod_i O_i\rangle_{a_k,L_k}\}$ is Cauchy and converges to a limit independent of the chosen subsequence.
\end{corollary}

\begin{proof}
By OS/GNS, $n$\,–\,point functions are Laplace transforms of matrix elements of products of semigroups $e^{-tH_{a,L}}$ between time\,–\,zero local vectors. The Laplace–resolvent representation expresses these matrix elements through $(H_{a,L}-z)^{-1}$ with $\Im z\ne 0$. Applying Lemma~\ref{lem:cauchy-resolvent-unique} and dominated convergence for the Laplace integral (using UEI and locality to justify Fubini/Tonelli) yields Cauchy convergence and uniqueness of the limits.
\end{proof}

\begin{proof}[Proof of Theorem~\ref{thm:nrc-quant}]
Use the comparison identity (Appendix R3):
\[
  R(z_0)-I R_{a,L}(z_0) I^*\ =\ R(z_0)(I-P_{a,L})\ -\ R(z_0)\,D_{a,L}\,R_{a,L}(z_0) I^*,\quad D_{a,L}:=H I_{a,L}-I_{a,L}H_{a,L}.
\]
Split by $E_H([0,\Lambda])$ and $E_H((\Lambda,\infty))$. On the high-energy part, $\|R(z_0) E_H((\Lambda,\infty))\|=\operatorname{dist}(z_0,[\Lambda,\infty))^{-1}$. On the low-energy part, apply Lemma~\ref{lem:low-energy-proj} to bound $\|(I-P_{a,L})E_H([0,\Lambda])\|\le C_\Lambda a$. For the defect term, Thm.~\ref{thm:quant-calibrated-af-free-nrc}(D) gives $\|D_{a,L}(H_{a,L}+1)^{-1/2}\|\le C_{\rm gd} a$ and $\|(H_{a,L}-z_0)^{-1}(H_{a,L}+1)^{1/2}\|\le C(z_0)$ uniformly. Collecting terms yields the estimate with a constant $C(z_0,\Lambda)$.
\end{proof}

\medskip
\section{Appendix: Spectral gap persistence in the continuum}

\begin{lemma}[Riesz projection stability and gap persistence]\label{lem:riesz-gap}
Let $\{H_k\}$ be self\,–\,adjoint, nonnegative operators on Hilbert spaces $\mathcal H_k$ and $H\ge 0$ on $\mathcal H$. Fix $\gamma_*>0$ and suppose
\[
  \operatorname{spec}(H_k)\ \subset\ \{0\}\cup[\gamma_*,\infty)\qquad \text{for all $k$}.
\]
Let $\Gamma:=\{ z\in\mathbb C: |z|=r\}$ with any $r\in(0,\gamma_*/2)$, oriented counterclockwise. Assume that for every $z\in\Gamma$,
\[
  \|(H_k-z)^{-1}-(H-z)^{-1}\|\ \xrightarrow[k\to\infty]{}\ 0,
\]
uniformly in $z\in\Gamma$. Define the Riesz projections
\[
  P_k\ :=\ \frac{1}{2\pi i}\oint_\Gamma (H_k-z)^{-1}\,dz,\qquad
  P\ :=\ \frac{1}{2\pi i}\oint_\Gamma (H-z)^{-1}\,dz.
\]
Then:
\begin{itemize}
  \item[(i)] Uniform resolvent bound on $\Gamma$: for all $k$ and $z\in\Gamma$, $\|(H_k-z)^{-1}\|\le 1/r$ and $\|(H-z)^{-1}\|\le 1/r$.
  \item[(ii)] $\|P_k-P\|\to 0$ and $\operatorname{rank}P=\lim_k\operatorname{rank}P_k$.
  \item[(iii)] $0$ is an isolated eigenvalue of $H$; moreover $\operatorname{spec}(H)\subset\{0\}\cup[\gamma_*,\infty)$.
\end{itemize}
\end{lemma}
\begin{proof}
For (i), since $\operatorname{spec}(H_k)\subset\{0\}\cup[\gamma_*,\infty)$ and $|z|=r<\gamma_*/2$, we have $\operatorname{dist}(z,\operatorname{spec}(H_k))=\min\{r,\gamma_*-r\}\ge r$, hence $\|(H_k-z)^{-1}\|\le 1/r$; the same bound holds for $H$.

For (ii), uniform convergence of resolvents on $\Gamma$ and (i) allow dominated convergence under the contour integral, giving $\|P_k-P\|\to 0$. Norm convergence of projections implies convergence of ranks.

For (iii), $P$ projects onto the generalized eigenspace at $0$. Since $H\ge 0$, $0$ is an eigenvalue (if present), and the rest of the spectrum is outside $\Gamma$. The spectral mapping and the assumed separation for $H_k$ combine with norm\,–\,resolvent convergence to forbid limit points of $\operatorname{spec}(H)$ in $(0,\gamma_*)$; thus $\operatorname{spec}(H)\subset\{0\}\cup[\gamma_*,\infty)$.
\end{proof}

\begin{theorem}[Gap persistence under NRC]\label{thm:gap-persist-nrc}
Let $(a_k,L_k)$ be a van Hove scaling sequence. Assume the norm--resolvent convergence of Theorem~\ref{thm:nrc-embeddings} holds along a subsequence and that there is a $\gamma_*>0$ such that for all $k$,
\[
  \operatorname{spec}(H_{a_k,L_k})\cap(0,\gamma_*)\;=\;\varnothing.
\]
Then the continuum generator $H\ge 0$ satisfies
\begin{equation}
  \boxed{\operatorname{spec}(H)\subset \{0\}\cup[\gamma_*,\infty)}
\end{equation}
and the zero eigenspace has the same finite rank as the lattice vacua (in particular, a unique vacuum persists).
\end{theorem}
\begin{proof}
Apply Lemma~\ref{lem:riesz-gap} with $H_k:=H_{a_k,L_k}$ and the contour $\Gamma=\{ |z|=r\}$, $r\in(0,\gamma_*/2)$. Uniform norm\,–\,resolvent convergence on $\Gamma$ is provided by Theorem~\ref{thm:nrc-quant} on compact sets. Items (ii) and (iii) give vacuum multiplicity stability and the spectral inclusion $\{0\}\cup[\gamma_*,\infty)$.
\end{proof}

\vspace{12pt}
% (appendix already set earlier)
\section{OS $\to$ Wightman Reconstruction and Mass Gap in Minkowski Space}

\subsection*{Abstract Reversible Discretization $\Rightarrow$ Resolvent Limit and $O(a)$ Defect}
\begin{theorem}[Abstract interface discretization to continuum generator]\label{thm:abstract-discretization}
Let $\Lambda\Subset\mathbb R^4$ be fixed. For each $(a,L)$ let $K_{a,L}$ be a self-adjoint Markov contraction on $L^2(\mu_{\partial}^{a,L})$ (interface kernel), and let $U_{a,L}:L^2(\mu_{\partial}^{a,L})\to L^2(\nu_\Lambda)$ be the density isometry to a fixed reference $\nu_\Lambda$. Set $\widetilde K_{a,L}:=U_{a,L}K_{a,L}U_{a,L}^{-1}$ and define
\[
  \mathfrak e_{a,L}(\varphi,\psi):=\tfrac{1}{a}\langle \varphi-\widetilde K_{a,L}\varphi,\psi\rangle\,,\qquad \widehat H_{a,L}:=-\tfrac{1}{a}\log(\widetilde K_{a,L}).
\]
Assume: (C1) there exists $\gamma_*>0$ with $\mathfrak e_{a,L}(\varphi,\varphi)\ge \gamma_*\|\varphi\|^2$ on $\mathbf 1^\perp$ uniformly in $(a,L)$; (C2) there is a dense core $\mathcal C_\Lambda\subset L^2(\nu_\Lambda)$ and a nonnegative self-adjoint $H_\Lambda$ with
\[
  \big|\mathfrak e_{a,L}(\varphi,\psi)-\langle H_\Lambda\varphi,\psi\rangle\big|\le c_1(\Lambda) a\,\|\varphi\|_{\mathcal G}\,\|\psi\|_{\mathcal G},\qquad
  \|\varphi-\widetilde K_{a,L}\varphi-a H_\Lambda\varphi\|\le c_2(\Lambda) a^2\,\|\varphi\|_{\mathcal G}
\]
for all $\varphi,\psi\in\mathcal C_\Lambda$. Then $\mathfrak e_{a,L}$ Mosco-converges to the Dirichlet form of $H_\Lambda$ and, for every $\lambda>0$,
\[
  \lim_{a\downarrow 0,L\uparrow\infty}\ \big\|(\widehat H_{a,L}+\lambda)^{-1}-(H_\Lambda+\lambda)^{-1}\big\|\ =\ 0\,.
\]
Moreover, on $E_\Lambda([0,\Lambda_0])$ one has the explicit graph-defect bound
\[
  \|\,(\widehat H_{a,L}-H_\Lambda)\,E_\Lambda([0,\Lambda_0])\,\|\ \le\ a\,C(\Lambda_0)\,.
\]
\end{theorem}
\noindent\emph{Remark.} In the main chain, (C1) comes from the slab gap and (C2) from the AF-free NRC estimates (graph-defect/projection control) on fixed regions.

\begin{theorem}[OS\,$\to$\,Wightman export with mass gap]\label{thm:os-to-wightman}
Let $\mu$ be a continuum Euclidean measure obtained as a limit of Wilson lattice measures along a scaling sequence, with Schwinger functions $\{S_n\}$ satisfying OS0--OS5. Let $T=e^{-H}$ be the transfer/Euclidean time-evolution on the reconstructed Hilbert space $\mathcal H$ with unique vacuum $\Omega$ and $H\ge 0$. If $\operatorname{spec}(H)\subset \{0\}\cup[\gamma_*,\infty)$ for some $\gamma_*>0$, then the OS reconstruction yields a Wightman quantum field theory on Minkowski space with local gauge-invariant fields and the same mass gap:
\begin{equation}
  \boxed{\sigma(H_{\text{Mink}})\subset \{0\}\cup[\gamma_*,\infty)}
\end{equation}
\end{theorem}
\noindent\emph{Remark (constant propagation).} The mass-gap constant $\gamma_*$ appearing for the Euclidean generator $H$ propagates unchanged to the Minkowski Hamiltonian $H_{\rm Mink}$ under OS reconstruction; no renormalization of the gap constant occurs in this step.
\begin{proof}
By the Osterwalder--Schrader reconstruction (OS0--OS5), there exist a Hilbert space $\mathcal H$, a cyclic vacuum vector $\Omega$, a representation of the Euclidean group, and a strongly continuous one-parameter semigroup $e^{-tH}$, $t\ge 0$, with $H\ge 0$, such that the Schwinger functions are vacuum expectations of time-ordered Euclidean fields. Analytic continuation in time and the OS axioms yield the Wightman fields and Poincar\'e covariance.
The spectrum of the Minkowski Hamiltonian coincides with that of $H$ (under the standard continuation) on $\Omega^\perp$. Since $\operatorname{spec}(H)\cap(0,\gamma_*)=\varnothing$ under the stated hypotheses, the same open gap persists in the Minkowski theory, establishing a positive mass gap $\ge \gamma_*$. Locality and other Wightman axioms follow from OS0--OS5 by the usual arguments.
\end{proof}

\vspace{12pt}
\section{Main Theorem (Continuum YM with Mass Gap; AF--free NRC with Proved UEI/OS1 Inputs)}\label{sec:main-theorem-unconditional}

\begin{mdframed}[linewidth=0.5pt, linecolor=purple!40, backgroundcolor=purple!3, roundcorner=2pt, innertopmargin=8pt, innerbottommargin=8pt, skipabove=10pt, skipbelow=10pt]
\subsubsection*{Result Map (Labels; AF--free NRC Main Path)}
\begin{itemize}[leftmargin=2em, itemsep=4pt]
  \item \textbf{Scaled (physically fixed-time) minorization $\Rightarrow$ slab gap}: UCIS$_{\rm SW}$ (Thm.~\ref{thm:ucis-sw} under \eqref{eq:ucis-sw-window}) $\Rightarrow$ fixed-physical-time $L^2$ contraction (Cor.~\ref{cor:ucis-sw-L2-contraction}) $\Rightarrow$ fixed-physical-time parity-odd contraction (Thm.~\ref{thm:ucis-sw-odd-subspace}).
  \item \textbf{AF/Mosco cross\,–\,check (optional)}: Appendix~\ref{app:af-mosco}; Mosco/strong-resolvent variant and gap persistence (Thm.~\ref{thm:gap-persist-cont}).
  \item \textbf{OS axioms in the limit}: Thm.~\ref{thm:uei-fixed-region}, Prop.~\ref{prop:os0os2-closure}, Thm.~\ref{thm:os1-unconditional}.
  \item \textbf{Non-Gaussianity (local fields)}: Prop.~\ref{prop:nonzero-cumulant4}.
\end{itemize}
\end{mdframed}

\subsubsection*{Proof Strategy}
OS2 on the lattice (Thm.~\ref{thm:os}) yields a positive transfer $T=e^{-aH}$. On a fixed slab, the interface engine (staple window: Thm.~\ref{thm:staple-window}; scale-adapted refresh under the window: Lem.~\ref{lem:single-link-refresh}/\ref{lem:g-one-link-refresh}; diffusive smoothing: Lem.~\ref{lem:scaled-ball-to-hk} using Lem.~\ref{lem:ballwalk-diffusive}) yields the \emph{physically scaled} (multi-step) heat--kernel minorization on the coarse interface along any scaling schedule satisfying the window \eqref{eq:ucis-sw-window},
\[
  K_{\rm int}^{(a)\,M(a)}\ \ge\ \theta_*\,P_{t_0}
\]
by Theorem~\ref{thm:ucis-sw}. Consequently, the $M(a)$--step kernel has a fixed-physical-time $L^2$ contraction on mean-zero functions (Corollary~\ref{cor:ucis-sw-L2-contraction}) and, by interface compression (Proposition~\ref{prop:int-to-transfer}), a fixed-physical-time contraction on the parity-odd subspace (Theorem~\ref{thm:ucis-sw-odd-subspace}). This is the interface$\to$odd$\to$slab gap backbone used in the AF--free NRC route.

\vspace{10pt}
\begin{mdframed}[linewidth=0.5pt, linecolor=blue!30, backgroundcolor=blue!3, roundcorner=2pt, innertopmargin=8pt, innerbottommargin=8pt, skipabove=10pt, skipbelow=10pt]
\noindent\textbf{Interface Compression and $L^2$ Comparison.}
Let $\mathcal A_-$ be the algebra of bounded observables supported in $\{t\le 0\}$ and $\mathcal F_\partial$ the $\sigma$-algebra on the interface $\{t=0\}$. Define $J:\mathcal H\to L^2(\mu_\partial)$ by $JF:=\mathbf E[\,F\mid\mathcal F_\partial]$ and the interface kernel $K$ by the one-slab boundary transition. Then $K$ is a self-adjoint Markov contraction reversible w.r.t. $\mu_\partial$, and for all $n\in\mathbb N$,
\[
  \langle F, T^n G\rangle_{\!OS}\ =\ \langle JF,\ K^n JG\rangle_{L^2(\mu_\partial)}\,,\qquad T\ =\ J^* K J\,.
\]
Consequently, $\|T\|_{\mathbf 1^\perp}=\|K\|_{L^2_0(\mu_\partial)}$. If $\nu_\Lambda$ is a fixed reference on the boundary space and $U_{a,L}:L^2(\mu_\partial)\to L^2(\nu_\Lambda)$ is the density isometry, then $\widetilde K:=U_{a,L} K U_{a,L}^{-1}$ is reversible w.r.t. $\nu_\Lambda$ and $\|K\|_{L^2_0(\mu_\partial)}=\|\widetilde K\|_{L^2_0(\nu_\Lambda)}$.
\end{mdframed}

For the continuum step, by Thms.~\ref{thm:U1-lsi-uei}, \ref{thm:os1-unconditional}, \ref{thm:quant-calibrated-af-free-nrc}(D,F,G), Lem.~\ref{lem:U2-comparison}, Prop.~\ref{prop:one-point-resolvent}, and Thm.~\ref{thm:U2-nrc-unique}, the AF--free NRC engine (Thm.~\ref{thm:nrc-operator-norm}) together with the Cauchy criterion (Lem.~\ref{lem:af-free-cauchy}), low-energy projection control (Lem.~\ref{lem:low-energy-proj}), and the graph-defect bound (Thm.~\ref{thm:quant-calibrated-af-free-nrc}(D)) give operator-norm resolvent convergence on fixed regions and identify a unique limit. Gap persistence to the continuum then follows from Thm.~\ref{thm:gap-persist-cont}. UEI and limit closures establish OS0--OS3; local fields exist and are non-Gaussian (Prop.~\ref{prop:nonzero-cumulant4}).

\noindent\emph{Notes (blockers vs main chain).} The uniform block--Doeblin minorization against $\mu_\partial$ is replaced in the main chain by the heat--kernel sandwich with explicit $(\theta_*,t_0)$, which is slab--uniform and implies the $L^2$ contraction directly. Uniform $L^\infty$ comparability of boundary laws is not required for the $L^2$ comparison since reweighting via $U_{a,L}$ furnishes a fixed reference space $L^2(\nu_\Lambda)$ where contraction is measured.

\begin{theorem}[Continuum YM on $\mathbb R^4$ with OS0--OS5 and positive mass gap (AF--free; unconditional)]\label{thm:main-af-free}
For a compact simple gauge group $G$ (default $\mathrm{SU}(N)$, $N\ge 2$), there exists a nontrivial Euclidean quantum Yang--Mills theory on $\mathbb R^4$ whose Schwinger functions satisfy OS0--OS5, with local gauge-invariant fields. Let $H\ge 0$ be the corresponding Euclidean generator. There exists a constant $\gamma_*>0$, depending only on $(R_*,a_0,G)$ and on the heat--kernel spectral gap $\lambda_1(G)$, such that
\[
  \operatorname{spec}(H)\subset\{0\}\cup[\gamma_*,\infty)\,.
\]
Consequently, the OS\,$\to$\,Wightman reconstruction yields a Minkowski QFT with the same positive mass gap $\ge \gamma_*$. In particular, one may take $\gamma_* := 8\,c_{\mathrm{cut,phys}} = 8\,\big(-\log(1-\theta_*(1-e^{-\lambda_1(G) t_0}))\big)$ with $(\theta_*,t_0)$ depending only on $(R_*,a_0,G)$.
\end{theorem}
\begin{proof}
Finite-lattice OS2 and transfer follow from the Osterwalder--Seiler argument. On a fixed slab, the interface smoothing input is supplied (along scaling schedules) by UCIS$_{\rm SW}$ (Theorem~\ref{thm:ucis-sw} under \eqref{eq:ucis-sw-window}), which gives a heat--kernel minorization for the \emph{physically scaled} $M(a)$--step interface kernel. By Proposition~\ref{prop:int-to-transfer} and Theorem~\ref{thm:ucis-sw-odd-subspace}, this yields a fixed-physical-time contraction on the full parity-odd subspace. Interpreting this as a uniform slab gap input gives the required (C1) in the discretization/NRC package; the thermodynamic limit at fixed $a$ preserves the bound and clustering.

UEI on fixed regions (Theorem~\ref{thm:uei-fixed-region}) implies tightness; Proposition~\ref{prop:os0os2-closure} gives OS0 and OS2 for the limit, and Theorem~\ref{thm:os1-unconditional} with Lemma~\ref{lem:eqc-modulus} and Lemma~\ref{lem:os1-embedding} yields OS1. The interface smoothing input on fixed slabs is supplied (along scaling schedules) by UCIS$_{\rm SW}$ (Theorem~\ref{thm:ucis-sw} under \eqref{eq:ucis-sw-window}), which yields a fixed-physical-time odd-subspace contraction (Theorem~\ref{thm:ucis-sw-odd-subspace}). For NRC, we use the one–point resolvent estimate (Proposition~\ref{prop:one-point-resolvent}) together with the comparison identity (Lemma~\ref{lem:U2-comparison}) and the graph–defect/projection bounds to obtain operator–norm resolvent convergence on compact $K\subset\mathbb C\setminus\mathbb R$ (Theorem~\ref{thm:U2-nrc-unique}); gap persistence follows by Theorem~\ref{thm:gap-persist-cont}, yielding $\operatorname{spec}(H)\subset\{0\}\cup[\gamma_*,\infty)$ with $\gamma_*>0$.

Finally, Theorem~\ref{thm:os-to-wightman} exports OS0--OS5 to a Wightman theory with the same mass gap; Poincar\'e covariance and microcausality hold, and the gap persists to Minkowski space. All constants depend only on the slab geometry $(R_*,a_0)$ and group data through $\lambda_1(G)$.
\end{proof}

\noindent\emph{Remark (lower bound normalization; conditional).} In addition to the choice $\gamma_*:=8\,c_{\mathrm{cut,phys}}$ above (from the odd-cone deficit and unscaled Doeblin), the coarse-scaled Harris/Doeblin route (Cor.~\ref{cor:scaled-continuum-gap}) yields a finite positive continuum lower bound $c(\varepsilon)>0$ by Thms.~\ref{thm:U1-lsi-uei}, \ref{thm:os1-unconditional} and U2. One may thus take a unified mass-gap constant
\[
  m_*\ :=\ \max\{\,c(\varepsilon),\; 8\,c_{\mathrm{cut,phys}}\,\}\ >\ 0,
\]
which depends only on $(R_*,a_0,G)$ (and the metric normalization via $\lambda_1(G)$), and is independent of $(\beta,L)$ along the scaling window.
\begin{corollary}[Non-Gaussianity of the continuum local fields]\label{cor:nonGaussian-main}
There exist compactly supported smooth test functions $f_1,\ldots,f_4\in C_c^\infty(\mathbb R^4,\wedge^2\mathbb R^4)$ such that the truncated 4-point function of the clover field is nonzero in the continuum limit:
\[
  \langle \Xi(f_1)\,\Xi(f_2)\,\Xi(f_3)\,\Xi(f_4)\rangle_c\ \neq\ 0.
\]
In particular, the continuum local field law is not Gaussian. (See Proposition~\ref{prop:nonzero-cumulant4} for the detailed proof.)
\end{corollary}
\begin{proof}
Fix a bounded region $R$ and $f\in C_c^\infty(R)$ chosen as in Proposition~\ref{prop:nonzero-cumulant4} so that, for all sufficiently small $a$ and large $L$, one has $\langle \Xi_a(f)^4\rangle_c\ge c_0>0$ uniformly in $(a,L)$. By Lemma~\ref{lem:local-fields-tempered}, $\Xi_a(f)\to \Xi(f)$ in $L^2$ on fixed regions, and by Theorem~\ref{thm:c1a-tight}, Schwinger $n$\,–\,point functions converge uniquely along any van Hove diagonal. Since truncated cumulants are polynomial combinations of moments, they are continuous under convergence of moments of the required orders. Therefore
\[
  \langle \Xi(f)^4\rangle_c\ =\ \lim_{a\downarrow 0,\ L\to\infty}\ \langle \Xi_a(f)^4\rangle_c\ \ge\ c_0\ >\ 0.
\]
Taking $f_1=f_2=f_3=f_4=f$ yields the stated nonzero truncated 4\,–\,point in the continuum. The more general statement with possibly distinct $f_i$ follows by multilinearity and continuity from the case $f_1=\cdots=f_4$.
\end{proof}

% ============================
% Clay–critical consolidated theorems (global pack)
% ============================
\medskip
\begin{theorem}[Clay–critical Global OS pack on $\mathbb R^4$ (OS0--OS5, explicit constants)]\label{thm:global-os-clay}
There exist global Schwinger functions $\{S_n\}_{n\ge1}$ on the cylinder $\sigma$--algebra of gauge--invariant observables on $\mathbb R^4$ such that OS0--OS5 hold globally, with explicit constants and no dependence on the choice of van Hove exhaustion, embedding scheme, or boundary conditions. More precisely:
\begin{itemize}
  \item[(i)] \textbf{Projective consistency and existence.} For any increasing van Hove family $\{\Lambda_k\}$, the fixed–region limits $\{S_n^{(k)}\}$ are consistent on overlaps (Prop.~\ref{prop:consistency-overlaps}) and define a unique global law by Kolmogorov/Minlos (Thm.~\ref{thm:kolmogorov-global}).
  \item[(ii)] \textbf{OS0 (temperedness) with explicit constants.} In $d=4$, for any $q>4$, $p=5$, and all loop families $\{\Gamma_i\}$,
  \[
    |S_n(\Gamma_1,\ldots,\Gamma_n)|\ \le\ C_n\,\prod_{i=1}^n \bigl(1+\operatorname{diam}\Gamma_i\bigr)^p\,\prod_{i<j}\bigl(1+\operatorname{dist}(\Gamma_i,\Gamma_j)\bigr)^{-q},
  \]
  with $C_n=C_0^n\,C_{\mathrm{tree}}(n)\,\Big(\frac{16\,\zeta(q-4)}{1-e^{-m}}\Big)^{\!n-1}$ from Cor.~\ref{cor:os0-explicit-4d} (uniform in the exhaustions and embeddings).
  \item[(iii)] \textbf{OS1 (Euclidean invariance).} For every rigid motion $G\in E(4)$ and all inputs, $S_n(G\Gamma_1,\ldots,G\Gamma_n)=S_n(\Gamma_1,\ldots,\Gamma_n)$, by the constructed calibrator route (Thm.~\ref{thm:os1-unconditional}) or the embedding–independence route (Lem.~\ref{lem:os1-embedding}, Cor.~\ref{cor:os1-rotations}).
  \item[(iv)] \textbf{OS2 (reflection positivity).} OS2 passes to the limit from the lattice (Prop.~\ref{prop:os0os2-closure}).
  \item[(v)] \textbf{OS3 (clustering) and (vi) OS5 (unique vacuum).} Let $\lambda_1(G)>0$ be the first Laplace–Beltrami eigenvalue on the compact simple group $G$. There exist $t_0=t_0(G)>0$ and $\theta_*=\theta_*(G)>0$ (from the interface Doeblin/heat–kernel split) such that with
  \[
    c_{\mathrm{cut,phys}}\ :=\ -\log\bigl(1-\theta_*(1-e^{-\lambda_1(G) t_0})\bigr)>0,\qquad \gamma_*\ :=\ 8\,c_{\mathrm{cut,phys}},
  \]
  the global Euclidean generator $H\ge0$ obeys $\operatorname{spec}(H)=\{0\}\cup[\gamma_*,\infty)$ (Thm.~\ref{thm:global-gap-uncond}), yielding exponential clustering at rate $\gamma_*$ and a unique vacuum (OS5).
\end{itemize}
All constants are independent of the van Hove exhaustion, boundary conditions, and embedding scheme; their group dependence is explicit through $\lambda_1(G)$ and $t_0(G)$.
\end{theorem}

\begin{theorem}[Uniform global NRC with explicit constants; spectral projectors]\label{thm:global-nrc-clay}
Let $I_{a,L}:\mathcal H_{a,L}\to\mathcal H$ be the canonical OS/GNS embeddings along any van Hove sequence. For every $z\in\mathbb C\setminus\mathbb R$ there exists an explicit $K(z)$, independent of slab, volume, exhaustion and boundary, such that
\[
  \big\|(H-z)^{-1} - I_{a,L}(H_{a,L}-z)^{-1} I_{a,L}^*\big\|\ \le\ K(z)\,\varepsilon(a),\qquad K(z)\ :=\ 8\Big(1+\frac{1+|z|}{|\Im z|}\Big)^{\!2},
\]
where $\varepsilon(a)=O(a)$ always (Thm.~\ref{thm:nrc-quant}), and under the fixed–core normalization of \S\ref{thm:quant-calibrated-af-free-nrc} one has $\varepsilon(a)\le a^2$ (Cor.~\ref{cor:NRC-explicit}). In particular, on any compact $K\subset\mathbb C\setminus\mathbb R$ the convergence is uniform with the same $K(z)$.

Moreover, for every $E\in(0,\gamma_*/2]$, the low–energy spectral projectors satisfy the explicit Davis–Kahan bound
\[
  \big\|\,\mathbf 1_{(-\infty,E]}(H) - I_{a,L}\,\mathbf 1_{(-\infty,E]}(H_{a,L})\,I_{a,L}^*\big\|\ \le\ \frac{2\,C_{\mathrm{NRC}}}{\gamma_* - E}\,\varepsilon(a),
\]
with $C_{\mathrm{NRC}}\le 2 C_{\rm form}+4 C_D^2$ as in Thm.~\ref{thm:quant-calibrated-af-free-nrc}(F,G), independent of slab/volume/exhaustion.
\end{theorem}

\begin{theorem}[Gap persistence, OS$\to$Wightman, microcausality and nontriviality]\label{thm:gap-wightman-clay}
For the global continuum theory constructed above:
\begin{itemize}
  \item[(a)] \textbf{Global spectral gap.} $\operatorname{spec}(H)=\{0\}\cup[\gamma_*,\infty)$ with $\gamma_*=8\,\big(-\log(1-\theta_*(1-e^{-\lambda_1(G) t_0}))\big)>0$ (Thm.~\ref{thm:global-gap-uncond}).
  \item[(b)] \textbf{OS$\to$Wightman export and microcausality.} The OS reconstruction yields Poincar\'e–covariant Wightman fields with the same mass gap $\gamma_*>0$ and microcausality for all gauge–invariant local smearings (Thms.~\ref{thm:os-to-wightman}, \ref{thm:microcausality-poincare}, Cor.~\ref{cor:microcausality}).
  \item[(c)] \textbf{Nontriviality.} There exist compactly supported $f_1,\dots,f_4$ such that the truncated 4–point of the clover field is nonzero at the Wightman level (Prop.~\ref{prop:nonzero-cumulant4}, Cor.~\ref{cor:nonGaussian-main}).
\end{itemize}
All statements are independent of the exhaustion, embedding, and boundary choices; group dependence enters only through $\lambda_1(G)$ and $t_0(G)$.
\end{theorem}

\begin{theorem}[Group generality and global independence/uniqueness]\label{thm:independence-clay}
Let $G$ be any compact simple Lie group. Then:
\begin{itemize}
  \item Group–dependent constants enter only via $\lambda_1(G)$ and compact–group heat–kernel bounds; all theorems above hold for such $G$ (cf. Lem.~\ref{lem:g-taylor}, Lem.~\ref{lem:g-one-link-refresh}, Lem.~\ref{lem:hk-contraction}).
  \item \textbf{Embedding/schedule/van Hove/boundary independence.} Continuum Schwinger functions are independent of the admissible polygonal/voxel embeddings (Prop.~\ref{prop:embedding-independence}, Cor.~\ref{cor:scheme-independence}), of the monotone schedule within the AF–free window (Thm.~\ref{thm:af-free-calibrated}), and of van Hove/boundary choices (Prop.~\ref{prop:bc-robust}, Prop.~\ref{prop:consistency-overlaps}).
  \item \textbf{Unitary uniqueness.} The global OS/GNS realizations obtained from any two admissible embeddings are unitarily equivalent and yield the same semigroup and spectrum (Prop.~\ref{prop:unitary-equivalence}).
\end{itemize}
\end{theorem}

\begin{mdframed}[linewidth=0.5pt, linecolor=orange!50, backgroundcolor=orange!5, roundcorner=2pt, innertopmargin=8pt, innerbottommargin=8pt, skipabove=10pt, skipbelow=10pt]
\subsubsection*{Clay-Style Constants Checklist (for Theorem~\ref{thm:main-af-free})}
From the geometry pack (\S\ref{para:geometry-pack}): $\theta_*\in(0,1]$ and $t_0>0$ are slab\,–\,uniform and independent of $\beta$ after coarse refresh; $\lambda_1=\lambda_1(G)>0$ depends on the compact group. The two-layer deficit yields a uniform contraction parameter $\rho=(1-\theta_*(1-e^{-\lambda_1(G) t_0}))^{1/2}$ on fixed slabs. Hence the per-tick constant $c_{\rm cut,phys}=-\log(1-\theta_*(1-e^{-\lambda_1(G) t_0}))>0$ and $\gamma_*=8\,c_{\rm cut,phys}>0$, uniform in $L$ and independent of $\beta$ on fixed slabs.
\end{mdframed}

\begin{mdframed}[linewidth=0.5pt, linecolor=blue!40, backgroundcolor=blue!5, roundcorner=2pt, innertopmargin=8pt, innerbottommargin=8pt, skipabove=8pt, skipbelow=10pt]
\noindent\textbf{NRC constants (global; exhaustion/volume independent).}
\begin{itemize}
  \item Resolvent constant for all $z\in\mathbb C\setminus\mathbb R$:
  \[
    K(z)\ :=\ 8\Big(1+\frac{1+|z|}{|\Im z|}\Big)^{\!2},\quad \big\|(H-z)^{-1} - I_{a,L}(H_{a,L}-z)^{-1} I_{a,L}^*\big\|\ \le\ K(z)\,\varepsilon(a),
  \]
  with $\varepsilon(a)=O(a)$ in general (Thm.~\ref{thm:nrc-quant}) and $\varepsilon(a)\le a^2$ under the fixed-core normalization (Cor.~\ref{cor:NRC-explicit}).
  \item Spectral projectors (Davis–Kahan). For $E\in(0,\gamma_*/2]$,
  \[
    \big\|\,\mathbf 1_{(-\infty,E]}(H) - I_{a,L}\,\mathbf 1_{(-\infty,E]}(H_{a,L})\,I_{a,L}^*\big\|\ \le\ \frac{2\,C_{\mathrm{NRC}}}{\gamma_* - E}\,\varepsilon(a),\quad C_{\mathrm{NRC}}\le 2 C_{\rm form}+4 C_D^2.
  \]
\end{itemize}
All constants above depend only on $G$ via $\lambda_1(G)$ and on local geometric inputs; they are independent of slab thickness choice (once $a\le a_0$ is fixed for the construction), volume, boundary, embedding, and van Hove exhaustion.
\end{mdframed}

\medskip

\begin{corollary}[Global $\beta$- and volume-uniform mass-gap bound]\label{cor:global-uniform-gap}
Let $\theta_*:=\kappa_0(R_*,a_0,N)$ and $t_0:=t_0(N)$ be as in Proposition~\ref{prop:explicit-doeblin-constants}, and let $\lambda_1(G)$ be the first nonzero Laplace--Beltrami eigenvalue on the compact simple group $G$. Define
\[
  c_{\rm cut,phys}\ :=\ -\log\big(1-\theta_*(1-e^{-\lambda_1(G) t_0})\big),\qquad \gamma_*\ :=\ 8\,c_{\rm cut,phys}.
\]
Then, uniformly in the lattice spacing $a\in(0,a_0]$, volume $L$, and bare coupling $\beta\ge 0$ along the van Hove window, the continuum generator $H$ obtained by NRC and OS reconstruction satisfies
\[
  \operatorname{spec}(H)\subset\{0\}\cup[\gamma_*,\infty),\qquad \gamma_*\ >\ 0,
\]
with $\gamma_*$ depending only on $(R_*,a_0,N)$ via $(\theta_*,t_0,\lambda_1)$. In particular, the mass gap lower bound is $\beta$- and volume-uniform.
\begin{proof}
By Proposition~\ref{prop:explicit-doeblin-constants}, $K_{\rm int}^{(a)}\ge \theta_* P_{t_0}$ with $t_0$ independent of $(\beta,L,a)$ and $\theta_*$ uniform in $L$ (its $\beta$-dependence is explicit). Corollary~\ref{cor:convex-split} then yields a one-step $L^2_0$ contraction by a factor $\le 1-\theta_*(1-e^{-\lambda_1(G) t_0})$ on the odd cone; composing eight ticks gives a lattice mean-zero spectral radius $\le e^{-8 c_{\rm cut}}$ with $c_{\rm cut} = -(1/a)\log(1-\theta_*(1-e^{-\lambda_1(G) t_0}))$. Passing to the continuum via NRC (Theorems~\ref{thm:nrc-embeddings}, \ref{thm:nrc-operator-norm}) and gap persistence (Theorem~\ref{thm:gap-persist-cont}) transports the physical constant $\gamma_* = 8\,c_{\rm cut,phys}$ to the continuum spectrum. Uniformity in $L$ follows from the volume-uniform NRC/thermodynamic-limit steps.
\end{proof}
\end{corollary}

\begin{theorem}[Global Minkowski mass gap (explicit constant; conditional)]\label{thm:global-minkowski-gap}
Let $G=\mathrm{SU}(N)$, $N\ge 2$, and fix slab geometry parameters $(R_*,a_0)$. Let $\theta_*:=\kappa_0(R_*,a_0,N)$ and $t_0:=t_0(N)$ be the boundary-uniform Doeblin constants of Proposition~\ref{prop:explicit-doeblin-constants}, and let $\lambda_1(G)$ be the first nonzero Laplace--Beltrami eigenvalue on $G$. Define
\[
  \gamma_{\mathrm{phys}}\ :=\ 8\,\Big(-\log\big(1-\theta_*(1-e^{-\lambda_1(G)\,t_0})\big)\Big)\ >\ 0.
\]
For the global continuum OS measure constructed in Section~\ref{sec:global-R4}, let $H\ge 0$ be the Euclidean generator and $H_{\mathrm{Mink}}$ the Minkowski Hamiltonian obtained by OS\,$\to$\,Wightman. Under the NRC/OS1 hypotheses stated earlier, one has
\[
  \operatorname{spec}(H)\ \subset\ \{0\}\cup[\gamma_{\mathrm{phys}},\infty)\;,\qquad
  \operatorname{spec}(H_{\mathrm{Mink}})\ \subset\ \{0\}\cup[\gamma_{\mathrm{phys}},\infty)\;.
\]
Moreover, $\gamma_{\mathrm{phys}}$ is independent of the exhaustion/van Hove sequence, independent of boundary conditions, and independent of $(a,\beta,L)$ once expressed in physical units; it depends only on $(R_*,a_0,N)$ through $(\theta_*,t_0,\lambda_1)$.
\end{theorem}
\begin{proof}
By Proposition~\ref{prop:explicit-doeblin-constants}, the interface kernel satisfies $K_{\rm int}^{(a)}\ge \theta_* P_{t_0}$ with $(\theta_*,t_0)$ independent of $(a,\beta,L)$ and boundary conditions. Corollary~\ref{cor:convex-split} and Theorem~\ref{thm:uniform-odd-contraction} yield an $L^2$ one-tick contraction on the odd cone by a factor $\le 1-\theta_*(1-e^{-\lambda_1(G) t_0})$, hence a per-eight-ticks contraction on the mean-zero subspace with rate $\gamma_{\mathrm{phys}}$. The thermodynamic limit at fixed $a$ preserves the bound and is boundary-independent (Proposition~\ref{prop:bc-robust}).

On fixed physical regions, AF-free NRC (Theorems~\ref{thm:nrc-embeddings}, \ref{thm:nrc-operator-norm}) and gap persistence (Theorem~\ref{thm:gap-persist-cont}) transfer the uniform bound to the continuum generator $H_R$, with constants unchanged. Consistency on overlaps (Proposition~\ref{prop:consistency-overlaps}) globalizes to the OS/GNS limit, giving $\operatorname{spec}(H)\subset\{0\}\cup[\gamma_{\mathrm{phys}},\infty)$. The reverse inclusion $[\gamma_{\mathrm{phys}},\infty)\subset \operatorname{spec}(H)$ follows from standard spectrum-closure and approximate-eigenvector arguments for positive contraction semigroups with sharp decay rate on $\Omega^{\perp}$.

Independence of the van Hove sequence and boundary follows from uniqueness of Schwinger limits on fixed regions (Proposition~\ref{prop:af-free-uniqueness}) and boundary robustness (Proposition~\ref{prop:bc-robust}). Independence of $(a,\beta,L)$ in physical units is encoded in the definition of $\gamma_{\mathrm{phys}}$, which uses the physical slab contraction constant $c_{\rm cut,phys}=-\log(1-\theta_*(1-e^{-\lambda_1 t_0}))$ and is geometric/group-theoretic only.

The OS\,$\to$\,Wightman reconstruction (Theorem~\ref{thm:os-to-wightman-global}) transports the gap to Minkowski without renormalizing the constant, hence $\operatorname{spec}(H_{\mathrm{Mink}})=\{0\}\cup[\gamma_{\mathrm{phys}},\infty)$.
\end{proof}
\subsection*{Local gauge\,--\,invariant fields: definition and temperedness}
\label{subsec:local-fields-tempered}
We now record an explicit local field algebra for the continuum theory and verify temperedness (OS0) for smeared local fields, ensuring the OS\,$\to$\,Wightman reconstruction applies to genuine local operators (not only Wilson loops).

\subsubsection*{Discretized Local Fields and Smearings}
Fix $\psi\in C_c^\infty(\mathbb R^4)$ and, for a lattice with spacing $a\in(0,a_0]$, define the scalar \emph{plaquette energy density} smearing
\[
  \Phi_a(\psi)\ :=\ a^4\sum_{p\in\mathcal P_a} \psi(x_p)\Bigl(1-\tfrac1N\,\Re\,\mathrm{Tr}\,U_p\Bigr),
\]
where $x_p$ is the geometric center of plaquette $p$. Likewise, for a smooth compactly supported two\,--\,form $\varphi\in C_c^\infty(\mathbb R^4,\wedge^2\mathbb R^4)$ and an $\mathfrak{su}(N)$\,--\,invariant inner product, define the gauge\,--\,invariant quadratic ``clover'' smearing
\[
  \Xi_a(\varphi)\ :=\ a^4\sum_{x\in a\mathbb Z^4} \sum_{\mu<\nu} \varphi_{\mu\nu}(x)\,\Bigl(1-\tfrac1N\,\Re\,\mathrm{Tr}\,U^{\text{clov}}_{\mu\nu}(x)\Bigr),
\]
where $U^{\text{clov}}_{\mu\nu}(x)$ is the standard four\,--\,plaquette clover around $x$ in the $\mu\nu$\,--\,plane. Both are local gauge\,--\,invariant lattice observables supported in $\operatorname{supp}\psi$ or $\operatorname{supp}\varphi$.
\begin{lemma}[Local gauge\,--\,invariant fields are tempered distributions]
\label{lem:local-fields-tempered}
Along any van Hove scaling sequence $(a_k,L_k)$, for each fixed $\psi\in C_c^\infty(\mathbb R^4)$ and $\varphi\in C_c^\infty(\mathbb R^4,\wedge^2\mathbb R^4)$ the families $\{\Phi_{a_k}(\psi)\}$ and $\{\Xi_{a_k}(\varphi)\}$ are Cauchy in $L^2$ under $\mu_{a_k,L_k}$ and converge in $L^2(\mu)$ to random variables $\Phi(\psi)$ and $\Xi(\varphi)$. The maps $\psi\mapsto\Phi(\psi)$ and $\varphi\mapsto\Xi(\varphi)$ extend by density to continuous linear functionals on $\mathcal S(\mathbb R^4)$ and $\mathcal S(\mathbb R^4,\wedge^2\mathbb R^4)$, respectively. In particular, $\Phi$ and $\Xi$ are (vector\,--\,valued) tempered distributions and generate a local gauge\,--\,invariant field algebra in the OS framework.
\end{lemma}

\begin{definition}[Renormalized curvature two\,--\,form]\label{def:FR}
For $\varphi\in C_c^\infty(\mathbb R^4,\wedge^2\mathbb R^4\otimes\mathfrak{su}(N))$, define the lattice smeared curvature
\[
  F^{\mathrm{lat}}_{a}(\varphi)
   \\ :=\ a^2 \sum_{x\in a\mathbb Z^4}\ \sum_{\mu<\nu}\ \big\langle\, \varphi_{\mu\nu}(x)\,,\ \mathrm{skew}\big(U^{\mathrm{clov}}_{\mu\nu}(x)\big)\,\big\rangle_{\mathfrak{su}(N)}\,.
\]
Here $U^{\mathrm{clov}}_{\mu\nu}(x)$ is the clover plaquette, $\mathrm{skew}(M):=\tfrac{1}{2}(M-M^\dagger)-\tfrac{\mathrm{Tr}(M-M^\dagger)}{2N}\,I$ projects to $\mathfrak{su}(N)$, and $\langle\cdot,\cdot\rangle_{\mathfrak{su}(N)}$ is the invariant inner product. The renormalized curvature $F^R$ is the $L^2$\,–\,limit (if it exists) of $F^{\mathrm{lat}}_{a}(\varphi)$ along van Hove sequences.
\end{definition}

\begin{lemma}[Existence, temperedness, and covariance of $F^R_{\mu\nu}$]\label{lem:FR-tempered}
Fix a bounded region $R\Subset\mathbb R^4$ and an admissible monotone schedule $\beta(a)$ from the AF\,–\,free window. There exists $a_0>0$ such that for all $\varphi\in C_c^\infty(R,\wedge^2\mathbb R^4\otimes\mathfrak{su}(N))$ the family $\{F^{\mathrm{lat}}_{a}(\varphi)\}_{a\in(0,a_0]}$ is Cauchy in $L^2$ under $\mu_{a,L}$ uniformly in $L$, and converges in $L^2(\mu)$ to a random variable denoted $F^R(\varphi)$. The map $\varphi\mapsto F^R(\varphi)$ extends by density to a continuous linear functional on $\mathcal S(\mathbb R^4,\wedge^2\mathbb R^4\otimes\mathfrak{su}(N))$, hence $F^R$ is an $\mathfrak{su}(N)$\,–\,valued tempered distribution. Moreover, for $g\in\mathcal G_0$ and rigid $G\in E(4)$,
\[
  U(g)\,F^R(\varphi)\,U(g)^{-1}=F^R(\operatorname{Ad}_{g}\varphi),\qquad U(G)\,F^R(\varphi)\,U(G)^{-1}=F^R(G\cdot\varphi)
\]
on the time\,–\,zero local core, where $\operatorname{Ad}_g$ is the adjoint action and $G\cdot\varphi$ is the geometric pullback.
\end{lemma}
\begin{proof}
On fixed $R$, UEI and the local LSI (Theorem~\ref{thm:uei-fixed-region}, Theorem~\ref{thm:U1-lsi-uei}) imply Gaussian tail bounds for one\,–\,link conditionals and their plaquette products under the schedule $\beta(a)$. A second\,–\,order Taylor remainder for the group exponential and the clover stencil gives, for $a\le a_0(R)$,
\[
  \mathbb E\,\big\|\,\mathrm{skew}\big(U^{\mathrm{clov}}_{\mu\nu}(x)\big)\,\big\|^2\ \le\ C_R\, a^4\,.
\]
Thus, by Cauchy\,–\,Schwarz and locality,
\[
  \sup_L\,\mathbb E\big|F^{\mathrm{lat}}_{a}(\varphi)\big|^2
   \ \le\ C_R\, \sum_{x}\sum_{\mu<\nu} a^4\, |\varphi_{\mu\nu}(x)|^2
   \ \le\ C'_R\, \|\varphi\|_{L^2}^2\,.
\]
A block\,–\,averaging/telescoping argument as in the proof of Lemma~\ref{lem:local-fields-tempered} yields that $\{F^{\mathrm{lat}}_{a}(\varphi)\}_a$ is Cauchy in $L^2$ uniformly in $L$, hence converges to $F^R(\varphi)$. Linearity and the bound above extend $F^R$ continuously to Schwartz test functions, proving temperedness. Gauge covariance and Euclidean covariance follow from the corresponding lattice symmetries (Theorem~\ref{thm:unitary-gauge}, Theorem~\ref{thm:os1-unconditional}) and stability under the limit on the local core.
\end{proof}

\begin{corollary}[Locality for gauge\,–\,invariant smearings of $F^R$]\label{cor:locality-FR}
Let $\chi\in C_c^\infty(\mathbb R^4)$ and define the gauge\,–\,invariant smeared quadratic field $\mathcal I(\chi):=\int \chi(x)\, \mathrm{Tr}\big(F^R_{\mu\nu}F^{R,\mu\nu}\big)(x)\,dx$ by polynomial approximation from the lattice. If $\operatorname{supp}\chi_1$ and $\operatorname{supp}\chi_2$ are spacelike separated after OS\,$\to$\,Wightman, then $[\mathcal I(\chi_1),\mathcal I(\chi_2)]=0$ on the time\,–\,zero local core.
\end{corollary}
\begin{proof}
Approximate $\mathcal I(\chi)$ by local polynomials in clover variables with supports separated on the lattice for small $a$. OS locality for time\,–\,ordered Euclidean functionals implies vanishing of commutators after reconstruction when supports are spacelike separated. Passing to the limit along van Hove sequences preserves the vanishing commutator on the core.
\end{proof}

\subsection*{Operator domains, common cores, and BRST}

\subsubsection*{Common Invariant Core for Local Operators}
Let $\mathfrak A_0$ denote the time\,–\,zero cylinder \(*\,–\,algebra) generated by gauge\,–\,invariant local observables (Wilson loops and smeared clover fields) supported in bounded regions. Define the local polynomial domain
\[
  \mathcal D_{\rm loc}\ :=\ \mathrm{span}\,\big\{\, P\big(\{\Phi(f_i)\},\{\Xi(\varphi_j)\}\big)\,\Omega\ :\ P\ \text{polynomial with complex coefficients},\ f_i\in C_c^\infty(\mathbb R^4),\ \varphi_j\in C_c^\infty(\mathbb R^4, \wedge^2\mathbb R^4)\,\big\}\,.
\]

\begin{lemma}[Density and invariance of $\mathcal D_{\rm loc}$]\label{lem:field-core}
The subspace $\mathcal D_{\rm loc}$ is dense in the OS/GNS Hilbert space and is invariant under:
\begin{itemize}
  \item[(i)] Euclidean time translations $e^{-tH}$ for all $t\ge 0$;
  \item[(ii)] the spatial Euclidean group (by OS1);
  \item[(iii)] local gauge transformations acting unitarily on time\,–\,zero variables.
\end{itemize}
\end{lemma}
\begin{proof}
By OS0 (Proposition~\ref{prop:OS0-poly}, Corollary~\ref{cor:os0-explicit-4d}), local cylinders have finite moments of all orders; polynomials applied to $\Omega$ are therefore in $\mathcal H$ and their span is dense. The semigroup $e^{-tH}$ maps time\,–\,zero cylinders to cylinders by OS reconstruction and domain invariance; Euclidean invariance holds by OS1; local gauge transformations act isometrically on cylinders and preserve the reflection cone, hence induce unitaries on $\mathcal H$ that leave $\mathcal D_{\rm loc}$ invariant.
\end{proof}
\subsubsection*{Closability and Graph Bounds for Smeared Fields}
Define $\Phi(f)$ and $\Xi(\varphi)$ on $\mathcal D_{\rm loc}$ by $L^2$\,–\,limits of the lattice approximants (Lemma~\ref{lem:local-fields-tempered}).

\begin{proposition}[Field closability and relative graph bounds]\label{prop:field-closability}
There exist constants $C_\Phi(f),C_\Xi(\varphi)$ such that for all $\psi\in \mathcal D_{\rm loc}$,
\[
  \|\Phi(f)\psi\|\ \le\ C_\Phi(f)\,\big\|\,(H+1)^{1/2}\psi\,\big\|\,,\qquad
  \|\Xi(\varphi)\psi\|\ \le\ C_\Xi(\varphi)\,\big\|\,(H+1)^{1/2}\psi\,\big\|\,.
\]
Consequently $\Phi(f)$ and $\Xi(\varphi)$ are closable on $\mathcal D_{\rm loc}$, and their closures have $\mathcal D_{\rm loc}$ as a core.
\end{proposition}
\begin{proof}
On the lattice, OS positivity and locality give the standard energy bound $\|O\psi\|\le C\|(H_{a,L}+1)^{1/2}\psi\|$ for local $O$ on the time\,–\,zero cone. Passing to the limit by AF\,–\,free NRC (Theorems~\ref{thm:nrc-embeddings}, \ref{thm:nrc-operator-norm}) and using Thm.~\ref{thm:quant-calibrated-af-free-nrc}(D) yields the stated bounds with constants depending on the supports of $f,\varphi$ and group data only. Closability follows since $\mathcal D_{\rm loc}$ is a core for $(H+1)^{1/2}$ and the estimates are graph\,–\,bounded.
\end{proof}

\paragraph{BRST charge.}
Let $\mathcal G_0$ be the group of compactly supported time\,–\,zero gauge transformations. For a smooth Lie algebra test function $\alpha$ supported in a bounded region, define on $\mathcal D_{\rm loc}$ the derivation $\delta_\alpha$ by its action on generators (Wilson loops/clover fields) via the infinitesimal adjoint action and extend as a graded derivation.

\begin{definition}[BRST charge on $\mathcal D_{\rm loc}$]\label{def:brst}
The BRST charge $Q$ is the closable operator on $\mathcal D_{\rm loc}$ defined by $\langle\psi, Q\phi\rangle:=\frac{d}{ds}\big|_{s=0}\langle\psi, U(e^{s\alpha})\phi\rangle$ for a fixed dense set of test functions $\alpha$ whose linear span is dense in the Lie algebra of $\mathcal G_0$; we set $Q\phi:=\delta_\alpha\phi$ on generators and extend by linearity and closure. Different choices of spanning families yield the same closed operator.
\end{definition}

\begin{proposition}[Closability, nilpotency, and core for $Q$]\label{prop:brst-closable}
The BRST charge $Q$ defined on $\mathcal D_{\rm loc}$ is closable; its closure (denoted again $Q$) satisfies $Q^2=0$ on $\mathcal D_{\rm loc}$ and leaves $\mathcal D_{\rm loc}$ invariant. Moreover, for all $\psi\in\mathcal D_{\rm loc}$,
\[
  \|Q\psi\|\ \le\ C_Q\,\big\|\,(H+1)^{1/2}\psi\,\big\|\,,
\]
with a constant $C_Q$ depending only on the support radius and group constants; hence $\mathcal D_{\rm loc}$ is a core for $Q$.
\end{proposition}
\begin{proof}
Unitary implementation of $\mathcal G_0$ on $\mathcal H$ implies that the generators of one\,–\,parameter subgroups are skew\,–\,adjoint on their natural domains; on $\mathcal D_{\rm loc}$ this coincides with the derivation $\delta_\alpha$. The energy bound follows as in Proposition~\ref{prop:field-closability} from locality and UEI. Nilpotency $Q^2=0$ on $\mathcal D_{\rm loc}$ is the Lie\,–\,algebra identity for gauge variations on gauge\,–\,invariant generators (graded Jacobi). Closability follows from the graph bound and density of $\mathcal D_{\rm loc}$.
\end{proof}

\begin{proposition}[Physical Hilbert space]\label{prop:physical-hilbert}
Let $\mathcal H_{\rm phys}:=\ker Q\big/\overline{\operatorname{ran} Q}$ with the induced inner product. Then $\mathcal H_{\rm phys}$ is a Hilbert space carrying the gauge\,–\,invariant observable net; in particular, for any gauge\,–\,invariant local $O$ with $O\mathcal D_{\rm loc}\subset\mathcal D_{\rm loc}$, the induced operator on $\mathcal H_{\rm phys}$ is well\,–\,defined and symmetric on the image of $\mathcal D_{\rm loc}$.
\end{proposition}
\begin{proof}
Standard homological argument: $Q$ is closable and nilpotent on a common core; the quotient by $\overline{\operatorname{ran} Q}$ removes $Q$\,–\,exact components. Gauge\,–\,invariant local observables commute with the gauge action on $\mathcal D_{\rm loc}$, hence preserve $\ker Q$ and map $\operatorname{ran} Q$ to itself; the induced action is well\,–\,defined and symmetric by OS positivity.
\end{proof}
\begin{proof}
Fix a bounded region $R\supset \operatorname{supp}\psi\cup\operatorname{supp}\varphi$. By Uniform Exponential Integrability on fixed regions (Theorem~\ref{thm:uei-fixed-region}), there exists $\eta_R>0$ with $\sup_{(a,L)} \mathbb E[e^{\eta_R S_R}]<\infty$. By standard duality between exponential moments and polynomial moments, this implies uniform bounds $\sup_{(a,L)}\mathbb E\big[|\Phi_a(\psi)|^p+|\Xi_a(\varphi)|^p\big]<\infty$ for all $p<\infty$, with constants depending only on $R$ and Schwartz norms of the test functions (via Proposition~\ref{prop:OS0-poly} and Corollary~\ref{cor:os0-explicit-4d}).
Let $k<\ell$. Partition $R$ into cubes of side comparable to $a_k$ and $a_\ell$. A standard block averaging/telescoping argument expresses $\Phi_{a_\ell}(\psi)-\Phi_{a_k}(\psi)$ as a sum of local increments supported in slightly enlarged cubes, each controlled in $L^2$ by the uniform moment bounds and the uniform exponential clustering on fixed regions. Summing the decaying covariances yields
\[
  \sup_L\,\mathbb E\big|\Phi_{a_\ell}(\psi)-\Phi_{a_k}(\psi)\big|^2\ \longrightarrow\ 0\quad \text{as }k,\ell\to\infty,
\]
so $\{\Phi_{a_k}(\psi)\}_k$ is Cauchy in $L^2$. The same argument applies to $\Xi_{a_k}(\varphi)$. Denote the limits by $\Phi(\psi)$ and $\Xi(\varphi)$.
For $\psi\in C_c^\infty$, the maps $\psi\mapsto \Phi(\psi)$ are linear by construction. The uniform OS0 polynomial bounds control $|\Phi(\psi)|$ by a finite sum of seminorms of $\psi$ (Schwartz norms obtained by mollifying compact support), implying continuity of $\Phi$ on $\mathcal S(\mathbb R^4)$. Density of $C_c^\infty$ in $\mathcal S$ extends $\Phi$ uniquely; likewise for $\Xi$. Therefore $\Phi$ and $\Xi$ define tempered distributions. Locality and reflection positivity for polynomials in $\Phi,\Xi$ follow from those of their lattice approximants by Proposition~\ref{prop:os0os2-closure}.
\end{proof}

\begin{corollary}[OS axioms for local fields]
\label{cor:os-local-fields}
The Schwinger functions of the smeared local fields $\Phi,\Xi$ satisfy OS0\,--\,OS5. Consequently, Theorem~\ref{thm:os-to-wightman} applies with $\mathcal A$ taken to be the polynomial *\,--\,algebra generated by $\{\Phi(\psi),\Xi(\varphi)\}$, and the resulting Wightman theory carries local gauge\,--\,invariant fields with the same mass gap $\ge \gamma_*$.
\end{corollary}

\section{Continuum gauge symmetry, Gauss law, and BRST}
\label{sec:gauge-brst}

We now give an unconditional construction of the continuum local gauge symmetry, Gauss-law generators, Ward identities, and (optional) BRST cohomology, and verify that the local gauge\,--\,invariant Wightman fields exist as operator\,--\,valued distributions on a common invariant core.

\subsection{Unitary implementation of the local gauge group}

Let $\mathcal G_0:=C_c^\infty(\mathbb R^3,\mathrm{SU}(N))$ denote the time\,–\,zero local gauge group, acting on time\,–\,zero lattice observables by the usual edge/vertex conjugations and on Wilson loops by conjugation at a basepoint (which cancels in the trace). This action extends by locality to the OS cylinder algebra.

\begin{theorem}[Unitary representation of $\mathcal G_0$]\label{thm:unitary-gauge}
There exists a strongly continuous unitary representation $U: \mathcal G_0\to \mathsf U(\mathcal H_{\mathrm{OS}})$ on the global OS/GNS Hilbert space such that for any time\,–\,zero local observable $O$ and $g\in\mathcal G_0$,
\[
  U(g)\,[O]\,U(g)^{-1}\ =\ [\,g\cdot O\,],\qquad U(g)\,\Omega\ =\ \Omega.
\]
Moreover, for any smooth one\,–\,parameter family $g_s=\exp(s\xi)$ with $\xi\in C_c^\infty(\mathbb R^3,\mathfrak{su}(N))$, the map $s\mapsto U(g_s)$ is strongly continuous on the time\,–\,zero local core.
\end{theorem}
\begin{proof}
On each finite lattice, invariance of the Haar measure under local gauge transformations implies $\langle \Theta(O_1) O_2\rangle=\langle \Theta(g\cdot O_1)\, (g\cdot O_2)\rangle$, hence the OS inner product is invariant. Therefore each $g$ induces an isometry on the lattice OS/GNS space which fixes the vacuum. By continuity in the cylinder topology and embedding\,–\,independence (Proposition~\ref{prop:embedding-independence}), these isometries are compatible along van Hove limits and define $U(g)$ on the global OS/GNS space. Unitarity follows since $g\mapsto g^{-1}$ yields the inverse action. Strong continuity for $g_s$ on the time\,–\,zero local core follows from UEI, OS0 equicontinuity (Lemma~\ref{lem:eqc-modulus}), and dominated convergence applied to matrix elements $\langle \Theta(O_1)\, (g_s\cdot O_2)\rangle$.
\end{proof}

\subsection{Gauss\,–\,law generators and physical subspace}

\begin{theorem}[Self\,–\,adjoint Gauss generators]\label{thm:gauss-generators}
For each $\xi\in C_c^\infty(\mathbb R^3,\mathfrak{su}(N))$ there exists a self\,–\,adjoint operator $G(\xi)$ with domain containing the time\,–\,zero local core such that
\[
  U(\exp(s\xi))\ =\ e^{\,i s G(\xi)}\quad (s\in\mathbb R),\qquad G(\xi)\,\Omega\ =\ 0,
\]
and for any time\,–\,zero local observable $O$,
\[
  i\,[\,G(\xi),\,[O]_{\mathrm{OS}}\,]\ =\ \big[\,(\delta_\xi O)\,\big]_{\mathrm{OS}},
\]
where $\delta_\xi$ is the infinitesimal gauge variation. The map $\xi\mapsto G(\xi)$ is a representation of the Lie algebra $C_c^\infty(\mathbb R^3,\mathfrak{su}(N))$.
\end{theorem}
\begin{proof}
By Theorem~\ref{thm:unitary-gauge}, $s\mapsto U(\exp(s\xi))$ is a strongly continuous one\,–\,parameter unitary group on a dense invariant core, so Stone's theorem yields a (essentially) self\,–\,adjoint generator $G(\xi)$ with the stated exponential. Vacuum invariance gives $G(\xi)\Omega=0$. The commutator identity is obtained by differentiating $s\mapsto U(\exp(s\xi))\,[O]\,U(\exp(-s\xi))$ at $s=0$ on the core. The Lie homomorphism property follows by standard properties of unitary representations.
\end{proof}
\begin{definition}[Physical subspace]
Define $\mathcal H_{\mathrm{phys}}:=\{\psi\in\mathcal H_{\mathrm{OS}}:\ U(g)\psi=\psi\ \forall g\in\mathcal G_0\}$, equivalently $\mathcal H_{\mathrm{phys}}=\bigcap_{\xi} \ker G(\xi)$ (closure understood). Denote by $\mathcal A_{\mathrm{phys}}$ the OS/GNS algebra generated by gauge\,–\,invariant time\,–\,zero local observables.
\end{definition}

\begin{theorem}[Gauss law and gauge\,–\,invariant algebra]\label{prop:gauss-phys}
The vacuum $\Omega\in\mathcal H_{\mathrm{phys}}$. The physical subspace is the closure of $\mathcal A_{\mathrm{phys}}\,\Omega$. For any $O\in\mathcal A_{\mathrm{phys}}$ and any $\xi$, one has $[G(\xi),[O]]=0$.
\end{theorem}
\begin{proof}
Vacuum invariance is from Theorem~\ref{thm:unitary-gauge}. If $O$ is gauge invariant, then $g\cdot O=O$ and $U(g)[O]U(g)^{-1}=[O]$, so $[O]\Omega\in\mathcal H_{\mathrm{phys}}$; density follows because $\mathcal A_{\mathrm{phys}}\,\Omega$ is cyclic for the gauge\,–\,invariant OS algebra. The commutator statement follows from the differentiated covariance identity in Theorem~\ref{thm:gauss-generators} with $\delta_\xi O=0$.
\end{proof}

\subsection{Ward identities (continuum, nonabelian)}

\begin{theorem}[Nonabelian Ward identities]\label{thm:ward}
For any smooth compactly supported $\xi$ and any time\,–\,ordered product of time\,–\,zero local gauge\,–\,invariant observables $O_1,\dots,O_n$ with smooth time translations, one has
\[
  \sum_{k=1}^n \big\langle\, O_1\cdots (\delta_\xi O_k)\cdots O_n\,\big\rangle\ =\ 0,
\]
in the continuum limit, with convergence uniform on compact families of smearings. Equivalently, for the OS/GNS commutators,
\[
  \sum_{k=1}^n \langle\,\Omega,\ O_1\cdots i[ G(\xi), O_k ]\cdots O_n\,\Omega\rangle\ =\ 0.
\]
\end{theorem}
\begin{proof}
On each finite lattice, the identity follows from invariance of the Haar measure and change\,–\,of\,–\,variables under local gauge transformations, differentiating at the identity in $\mathcal G_0$ (lattice Ward identity). UEI and OS0 bounds yield uniform integrability for passing to the continuum; embedding\,–\,independence and boundary robustness (Proposition~\ref{prop:bc-robust}) ensure that the differentiated identities converge along van Hove nets to the stated continuum identity. The commutator form is the OS/GNS rewriting using Theorem~\ref{thm:gauss-generators}.
\end{proof}

\subsection{Local gauge\,–\,invariant Wightman fields as operator\,–\,valued distributions}

\begin{theorem}[Closability and common core]\label{thm:ovd}
Let $\mathcal D_{\mathrm{loc}}$ be the algebraic span of vectors of the form $[O]$ with $O$ a time\,–\,zero local gauge\,–\,invariant observable. For each test function $\varphi\in C_c^\infty(\mathbb R^4)$, the smeared local fields $\Phi(\varphi),\Xi(\varphi)$ define closable operators on $\mathcal D_{\mathrm{loc}}$, with $\mathcal D_{\mathrm{loc}}$ a common invariant core for all such smearings. The maps $\varphi\mapsto \Phi(\varphi)$ and $\varphi\mapsto \Xi(\varphi)$ are continuous from $\mathcal S(\mathbb R^4)$ into the space of operators on $\mathcal D_{\mathrm{loc}}$ endowed with the strong graph topology.
\end{theorem}
\begin{proof}
OS0 polynomial bounds and UEI yield moment estimates of all orders for time\,–\,zero local observables on fixed regions; by time translation and semigroup bounds, the same holds for time\,–\,translated smearings. Nelson's analytic vector criterion then gives essential self\,–\,adjointness/closability on the polynomial core generated by $\mathcal A_{\mathrm{phys}}$ acting on $\Omega$. Continuity in $\varphi$ follows from Lemma~\ref{lem:eqc-modulus} and dominated convergence.
\end{proof}

\subsection{Optional: BRST cohomology equals Gauss\,–\,law invariants}

While the construction above avoids ghosts and gauge fixing, one can introduce a standard BRST differential to encode the local gauge symmetry cohomologically.

\begin{theorem}[BRST cohomology at ghost number zero]\label{thm:brst}
Let $\mathcal F_{\mathrm{tot}}$ be the graded *\,–\,algebra generated by the (time\,–\,zero) local gauge\,–\,variant fields together with free ghost fields $c,\bar c$ (CAR) and Nakanishi\,–\,Lautrup field $b$, with the usual BRST derivation $\mathsf s$ implementing the $\mathfrak{su}(N)$ Lie algebra on fields. Then there is a densely defined closed operator $Q$ on a graded Hilbert space extending $\mathcal H_{\mathrm{OS}}\otimes \mathcal H_{\mathrm{gh}}$ such that $Q^2=0$, $i[Q,\cdot]=\mathsf s(\cdot)$ on a common core, and the cohomology at ghost number zero satisfies
\[
  H^0(Q)\ \cong\ \overline{\mathcal A_{\mathrm{phys}}\,\Omega}\ \subset\ \mathcal H_{\mathrm{OS}}.
\]
In particular, physical vectors/states are identified with the gauge\,–\,invariant ones constructed above, and the mass gap is unchanged.
\end{theorem}
\begin{proof}
By Theorem~\ref{thm:gauss-generators}, the local gauge Lie algebra is represented by the self\,--\,adjoint charges $G(\xi)$. The Chevalley\,--\,Eilenberg construction yields a nilpotent differential $\mathsf s$ on $\mathcal F_{\mathrm{tot}}$; define $Q$ on the graded tensor product core by the Kugo\,--\,Ojima prescription using $G(\xi)$ and ghost creation/annihilation operators. Nilpotency $Q^2=0$ reflects the Lie algebra relations. The cohomology at ghost number zero identifies with the invariants under $G(\xi)$, hence with $\mathcal H_{\mathrm{phys}}$ by Proposition~\ref{prop:gauss-phys}. Since ghosts decouple from $\mathcal A_{\mathrm{phys}}$, the mass gap on $\mathcal H_{\mathrm{phys}}$ is the same as in Theorem~\ref{thm:global-gap-uncond}.
\end{proof}

\begin{corollary}[Microcausality for smeared gauge\,--\,invariant fields]
\label{cor:microcausality-smeared}
Let $f,g\in C_c^\infty(\mathbb R^4)$ have spacelike separated supports. Then the Wightman fields obtained from $\Phi,\Xi$ via OS reconstruction satisfy
\[
  [\Phi(f),\Phi(g)]\,=\,0,\quad [\Phi(f),\Xi(\eta)]\,=\,0,\quad [\Xi(\omega),\Xi(\eta)]\,=\,0
\]
whenever all test functions are pairwise spacelike separated. In particular, the local gauge\,--\,invariant field algebra obeys locality.
\end{corollary}

\begin{proof}
OS0\,--\,OS5 imply the Wightman axioms under Theorem~\ref{thm:os-to-wightman}. Locality (microcausality) holds for smeared fields with spacelike separated supports by the standard OS\,$\to$\,Wightman locality theorem. Since $\Phi,\Xi$ are limits of local gauge\,--\,invariant lattice observables, their smeared versions generate local operators; therefore the commutators vanish at spacelike separation.
\end{proof}

\begin{lemma}[Nontriviality: positive variance of a smeared local field]\label{lem:nontrivial-variance}
Fix a nonzero $\varphi\in C_c^\infty(\mathbb R^4,\wedge^2\mathbb R^4)$ supported in a bounded region $R\Subset\mathbb R^4$. Along any van Hove scaling sequence $(a_k,L_k)$, the smeared clover field satisfies
\[
  \operatorname{Var}_\mu\big(\Xi(\varphi)\big)\ >\ 0.
\]
Moreover, there exists $c_R(\varphi)>0$ depending only on $(R,a_0,N,\varphi)$ such that for all $k$ large and all volumes $L_k$ in the window,
\[
  \operatorname{Var}_{\mu_{a_k,L_k}}\big(\Xi_{a_k}(\varphi)\big)\ \ge\ c_R(\varphi),
\]
and hence the positive variance persists in the continuum limit.
\end{lemma}

\begin{proof}
Write the lattice smeared observable as $\Xi_a(\varphi)=a^4\sum_{x\in a\mathbb Z^4\cap R} \sum_{\mu<\nu} \varphi_{\mu\nu}(x)\,\mathrm{clov}^{(a)}_{\mu\nu}(x)$. Each clover average obeys $0\le \mathrm{clov}^{(a)}_{\mu\nu}(x)\le 2$ and depends nontrivially (continuously) on finitely many interface links. By Lemma~\ref{lem:abs-cont}, the joint law of the interface after one tick has a strictly positive continuous density, and by Proposition~\ref{prop:doeblin-full} it dominates a product heat kernel on $G^m$. Therefore the distribution of $\Xi_a(\varphi)$ is non-degenerate on every finite volume, yielding $\operatorname{Var}_{\mu_{a,L}}(\Xi_a(\varphi))>0$.

Uniform Exponential Integrability on fixed $R$ (Theorem~\ref{thm:uei-fixed-region}) and locality ensure that small-ball refresh/heat--kernel domination occurs with probability bounded below uniformly in $(\beta,L)$ on the slab; by continuity of $\Xi_a(\varphi)$ in the interface variables, this gives a uniform variance lower bound $c_R(\varphi)>0$ for all sufficiently small $a\le a_0$ and large $L$.

Finally, by Lemma~\ref{lem:local-fields-tempered} and Corollary~\ref{cor:unique-schwinger-local}, $\Xi_{a_k}(\varphi)\to \Xi(\varphi)$ in $L^2$ and the Schwinger limits are unique, so variance is lower semicontinuous under the limit. Hence $\operatorname{Var}_\mu(\Xi(\varphi))\ge \limsup_k \operatorname{Var}_{\mu_{a_k,L_k}}(\Xi_{a_k}(\varphi))\ge c_R(\varphi)>0$.
\end{proof}

\section*{Appendix: Constants and References Index}
\begin{itemize}
  \item \textbf{Constants.} $\lambda_1(G)$: first nonzero Laplace--Beltrami eigenvalue on the compact simple group $G$; $t_0>0$, $\theta_*>0$, $\kappa_0>0$: interface Doeblin/heat--kernel constants depending only on $(R_*,a_0,G)$; $c_{\mathrm{cut}}(a):=-(1/a)\log(1-\theta_*(1-e^{-\lambda_1(G) t_0}))$; $c_{\mathrm{cut,phys}}:= -\log(1-\theta_*(1-e^{-\lambda_1(G) t_0}))$; $\gamma_{\mathrm{cut}}:=8\,c_{\mathrm{cut}}(a)$; $\gamma_*:=8\,c_{\mathrm{cut,phys}}$.
  \item \textbf{OS positivity (OS2) and transfer.} Osterwalder--Schrader \cite{Osterwalder1973,Osterwalder1975}; Osterwalder--Seiler \cite{OsterwalderSeiler1978} (Wilson gauge theory); Montvay--M\"unster \cite{MontvayMunster1994}.
\item \textbf{Heat--kernel and convolution smoothing on compact groups.} Diaconis--Saloff--Coste \cite{DiaconisSaloffCoste2004}; Varopoulos--Saloff--Coste--Coulhon \cite{VaropoulosSaloffCosteCoulhon1992}.
  \item \textbf{UEI, LSI, and cluster/Herbst.} Brydges \cite{Brydges1978,Brydges1986}; Holley--Stroock and Bakry--\'Emery techniques on compact manifolds; Kolmogorov--Chentsov criterion.
  \item \textbf{Resolvent comparison and spectral stability.} Kato \cite{Kato1995} (norm--resolvent convergence; spectral lower semicontinuity); Riesz projections; semigroup theory (Engel--Nagel \cite{EngelNagel2000}).
  \item \textbf{Probability compactness and extensions.} Prokhorov compactness; Daniell--Kolmogorov extension theorem.
  \item \textbf{Markov contractions.} Dobrushin \cite{Dobrushin1970} (total-variation contraction coefficients and spectral consequences in finite dimension).
\item \textbf{Labels (this manuscript).} Interface Doeblin: Proposition in Appendix "Uniform two--layer Gram deficit on the odd cone"; UEI: Theorem~\ref{thm:uei-fixed-region}; OS0/OS2 closure: Proposition~\ref{prop:os0os2-closure}; OS1: Theorem~\ref{thm:os1-unconditional}; NRC: Theorem~\ref{thm:nrc-embeddings}; Gap persistence: Theorem~\ref{thm:gap-persist-cont}; OS\,$\to$\,Wightman: Theorem~\ref{thm:os-to-wightman}; Main: Theorem~\ref{thm:main-af-free}.
\end{itemize}

\paragraph{Geometry pack (constant dependencies; $\beta/L$ independence).}\label{para:geometry-pack}
We summarize the constant schema and dependencies used throughout. Fix a physical slab radius $R_*>0$, a maximal tick $a_0>0$, and the gauge group $G$.
\begin{itemize}
  \item \textbf{Group data.} $\lambda_1(G)$: spectral gap of the Laplace--Beltrami operator on the compact simple gauge group $G$.
  \item \textbf{Interface/Doeblin constants.} From Proposition~\ref{prop:doeblin-full} and Lemma~\ref{lem:beta-L-independent-minorization}:
  $t_0=t_0(G)>0$, $\theta_*:=\kappa_0(R_*,a_0,G)\in(0,1]$, independent of $(\beta,L)$. The lower bound arises from: (i) a boundary-uniform refresh mass $\alpha_{\rm ref}(R_*,a_0,G)>0$ on the slab (Lemma~\ref{lem:refresh-prob}); (ii) convolution lower bounds by heat kernel at time $t_0(G)$ (Lemma~\ref{lem:ball-conv-lower}); and (iii) a geometry factor $c_{\rm geo}(R_*,a_0)\in(0,1]$ from cell factorization. No step uses the value of $\beta$ other than $\beta\ge 0$.
  \item \textbf{Cut contraction.} $c_{\rm cut}(a)=-(1/a)\log(1-\theta_*(1-e^{-\lambda_1(G) t_0}))$; physical $c_{\rm cut,phys}=-\log(1-\theta_*(1-e^{-\lambda_1(G) t_0}))$ (group dependence only via $\lambda_1(G)$).
  \item \textbf{Odd-cone contraction constants.} From Proposition~\ref{prop:int-to-transfer} and Corollary~\ref{cor:odd-contraction-from-Kint}:
  $\theta_*\in(0,1]$, $t_0>0$ depend only on $(R_*,a_0,G)$; on $L^2_0$,
  $\|K_{\rm int}^{(a)}\|\le 1-\theta_*(1-e^{-\lambda_1(G) t_0})$, hence on the slab--odd cone, $\|e^{-aH}\|\le 1-\theta_*(1-e^{-\lambda_1(G) t_0})$, and $c_{\rm cut}(a)=-(1/a)\log(1-\theta_*(1-e^{-\lambda_1(G) t_0}))$ (group dependence only via $\lambda_1(G)$).
  \item \textbf{Gap constants.} Lattice per-tick: $\|e^{-aH}\|_{\rm odd}\le 1-\theta_*(1-e^{-\lambda_1(G) t_0})\le e^{-a c_{\rm cut}}$ with $c_{\rm cut}=-(1/a)\log(1-\theta_*(1-e^{-\lambda_1(G) t_0}))$; eight ticks yield $\gamma_{\rm cut}=8 c_{\rm cut}$. Continuum: by operator-norm NRC and persistence, $\operatorname{spec}(H)\subset\{0\}\cup[\gamma_*,\infty)$ with $\gamma_*=8 c_{\rm cut,phys}$.
  \item \textbf{UEI/OS0 constants.} From Theorem~\ref{thm:uei-fixed-region} and Proposition~\ref{prop:OS0-poly}: $\eta_R, C_R$ (depend only on $(R,a_0,G)$), and polynomial OS0 constants on fixed regions.
  \item \textbf{NRC/embedding constants.} From Theorems~\ref{thm:nrc-embeddings}, \ref{thm:nrc-quant} together with Thm.~\ref{thm:quant-calibrated-af-free-nrc}(D) and Lemma~\ref{lem:low-energy-proj}: defect bound $C_{\rm gd}$, low-energy projector control $C_\Lambda$, and resolvent rate $C(z_0,\Lambda)$.
\end{itemize}

\paragraph{OS0/OS2 under limits (closure by UEI).}
The UEI bound yields tightness of gauge--invariant cylinders on $R$ (Prokhorov). Reflection positivity (OS2) is closed under weak limits of cylinder measures (bounded, continuous functional $F\mapsto \Theta F\,\overline{F}$). Temperedness/equicontinuity (OS0) follows from uniform Laplace bounds and the Kolmogorov--Chentsov criterion on loop holonomies (as in Proposition "OS0 (temperedness) with explicit constants"). Thus OS0 and OS2 persist along any scaling sequence.

\begin{lemma}[Cylinder measurability and projective limit]\label{lem:cylinder-projective}
Let $\{(a,L)\}$ be a directed net of lattices with spacings $a\in(0,a_0]$ and torus sizes $La\to\infty$. For a fixed bounded region $R\Subset\mathbb R^4$, let $\mathcal C_R$ denote the finite family of gauge--invariant loop variables and clover smearings supported in $R$ obtained from polygonal embeddings at mesh $\le a$. Then:
\begin{itemize}
  \item[(i)] (Measurability) Each element of $\mathcal C_R$ is Borel measurable with respect to the product Haar $\sigma$--algebra on links; the mapping $U\mapsto (O(U))_{O\in\mathcal C_R}$ is continuous on the compact configuration space.
  \item[(ii)] (Consistency) If $(a',L')\succeq(a,L)$ and the embeddings are chosen compatibly, then the pushforward of $\mu_{a',L'}$ to the $\sigma$--algebra generated by $\mathcal C_R$ coincides with the pushforward of $\mu_{a,L}$.
  \item[(iii)] (Tightness) Under UEI on $R$, the family of laws of $(O)_{O\in\mathcal C_R}$ is tight and uniformly exponentially integrable.
\end{itemize}
Consequently, by Prokhorov and Daniell--Kolmogorov, there exists a unique Borel probability measure on the projective limit of cylinder spaces whose finite--dimensional marginals agree with the lattice laws, yielding a continuum Euclidean measure on loop/local--field cylinders.
\end{lemma}
\begin{proof}
(i) Each loop variable is a finite product of link variables followed by a continuous class function (trace), hence Borel; clover smearings are finite averages of plaquette energies, hence continuous. (ii) Equivariant embeddings of loops/clovers and the link--marginal consistency of the Wilson measure imply consistency. (iii) UEI provides uniform exponential moments for any finite collection in $R$; on a compact space this implies tightness. Existence and uniqueness of the projective--limit measure then follow from Prokhorov compactness and the Daniell--Kolmogorov extension theorem for consistent finite--dimensional distributions.
\end{proof}
\begin{corollary}[Continuum measure on loop/local cylinders]\label{cor:continuum-measure-exists}
Along any van Hove scaling sequence, there exists a Borel probability measure $\mu$ on the cylinder $\sigma$--algebra generated by loop variables and local clover smearings on all bounded regions $R\Subset\mathbb R^4$, such that for every finite family of cylinder observables the expectations coincide with the lattice limits.
\end{corollary}
\paragraph{Thermodynamic limit note.}
At fixed spacing, the infinite-volume OS state exists by standard compactness arguments (tightness of local observables and diagonal extraction), and the gap/clustering persist by volume-uniform bounds; see, e.g., Kato \cite{Kato1995} for spectral stability and standard OS/GNS semigroup arguments for clustering.
\section{Clay compliance checklist}
\subsection*{Clay compliance map (requirements \textrightarrow{} labels)}
\begin{itemize}
  \item \textbf{OS0 (temperedness)}: Prop.~\ref{prop:OS0-poly}, Cor.~\ref{cor:os0-explicit-4d}; fixed-region U1 tightness/UEI: Thm.~\ref{thm:uei-fixed-region} (either raw route A or Track-J; Cor.~\ref{cor:uei-explicit-constants} gives explicit constants for the stronger action-moment bound under route A); closure: Prop.~\ref{prop:os0os2-closure}.
  \item \textbf{OS1 (Euclidean invariance)}: Thm.~\ref{thm:os1-unconditional}; supporting route: Prop.~\ref{prop:hk-calibrators-constructed} and Thm.~\ref{thm:os1-calibrator-route} (commutator route is an optional cross-check via Lem.~\ref{lem:local-commutator-Oa2}); additional bookkeeping: \ref{lem:isotropy-restore}, \ref{lem:os1-embedding}, Cor.~\ref{cor:os1-rotations}.
  \item \textbf{OS2 (reflection positivity)}: Thm.~\ref{thm:os} (Wilson link reflection); closure to limit: Cor.~\ref{cor:os2-pass} (from Prop.~\ref{prop:os0os2-closure}).
  \item \textbf{OS3/OS5 (clustering, unique vacuum)}: Lattice: Thm.~\ref{thm:thermo}, Thm.~\ref{thm:thermo-strong}; Continuum: Prop.~\ref{prop:os35-limit}; Gap\,$\Rightarrow$\,clustering: Prop.~\ref{prop:gap-to-cluster}; converse: Prop.~\ref{prop:cluster-to-gap}.
  \item \textbf{OS\,$\to$\,Wightman and Poincar\'e}: Thm.~\ref{thm:os-to-wightman}; Euclidean isotropy restoration: Lem.~\ref{lem:isotropy-restore}; Cor.~\ref{cor:poincare}.
  \item \textbf{Mass gap (lattice)}: Strong-coupling route: Thm.~\ref{thm:gap}, Prop.~\ref{prop:dob-spectrum}, Lem.~\ref{lem:dob-influence}; Odd-cone route: Prop.~\ref{prop:two-layer-deficit}, Cor.~\ref{cor:deficit-c-cut}, Thm.~\ref{thm:pf-gap-meanzero}.
  \item \textbf{Mass gap (continuum)}: Coarse/scaled Harris--Doeblin: Lem.~\ref{lem:coarse-refresh}, Lem.~\ref{lem:coarse-hk-domination}, Prop.~\ref{prop:explicit-doeblin-constants}, Thm.~\ref{thm:two-layer-explicit}, Cor.~\ref{cor:scaled-continuum-gap}; Persistence under Mosco/NRC: Thm.~\ref{thm:gap-persist-cont}, Thm.~\ref{thm:gap-persist}, Thm.~\ref{thm:nrc-operator-norm}, Thm.~\ref{thm:nrc-embeddings}.
  \item \textbf{AF/Mosco framework}: Assumption~\ref{assump:AF-Mosco}; Semigroup\,$\Rightarrow$\,resolvent: Thm.~\ref{thm:NRC-allz}; quantitative NRC: Thm.~\ref{thm:nrc-quant}; embeddings/core: Thm.~\ref{thm:strong-semigroup-core}, Prop.~\ref{prop:collective-compactness}; defects/projections: Lem.~\ref{lem:graph-defect-Oa}, Lem.~\ref{lem:low-energy-proj}.
  \item \textbf{Continuum measure existence} (on cylinders): Lem.~\ref{lem:cylinder-projective}, Cor.~\ref{cor:continuum-measure-exists}.
  \item \textbf{Gauge-invariant local fields}: Temperedness and OS locality: labels \ref{lem:local-fields-tempered}, \ref{cor:os-local-fields}.
  \item \textbf{Nontriviality (non-Gaussian)}: Prop.~\ref{prop:nonzero-cumulant4}, Cor.~\ref{cor:nonGaussian-main}; positive variance: Lem.~\ref{lem:nontrivial-variance}.
  \item \textbf{Normalization and constants} (independence of $(\beta,L)$ where claimed): Standing geometry pack \S\ref{para:geometry-pack}; physical vs lattice rates: see the definitions preceding Theorem~\ref{thm:uniform-odd-contraction} and the gap normalization bullet in Notation; interface scaling: paragraph "Interface scaling and coarse skeleton" and Lemmas~\ref{lem:lumping}, \ref{lem:coarse-density}.
  \item \textbf{Uniform-in-$N$ statements}: See Appendix R4 and cross-cut bounds (e.g., Lem.~\ref{lem:char-pd}, Prop.~\ref{prop:psd-crossing-gram}).
\end{itemize}

\paragraph{Unconditional (proved).}
\begin{itemize}
  \item \textbf{Lattice (fixed spacing).} OS2 (reflection positivity) via Osterwalder--Seiler; OS1 (discrete Euclidean invariance); OS0 (regularity) on compact configuration space; OS3/OS5 (clustering/unique vacuum) and a uniform lattice gap for small $\beta$ (Theorems~\ref{thm:gap}, \ref{thm:thermo-strong}). Thermodynamic limit at fixed $a$ exists with the same gap.
\end{itemize}
\paragraph{Supplement (optional background routes).}
\begin{itemize}
  \item \textbf{Tightness and OS0.} From UEI (Tree--Gauge UEI appendix) uniformly on fixed physical regions.
  \item \textbf{OS2 closure.} Reflection positivity preserved under limits.
  \item \textbf{OS1.} Oriented diagonalization plus equicontinuity (C1a).
  \item \textbf{Unique projective limit.} Tightness (UEI) and equicontinuity imply uniqueness of Schwinger limits (Proposition~\ref{prop:af-free-uniqueness}).
  \item \textbf{Continuum gap (conditional under AF/Mosco).} Coarse Harris/Doeblin minorization $\Rightarrow$ per-tick deficit; with Mosco/strong-resolvent gap persistence (Thm.~\ref{thm:gap-persist-cont}) this yields a finite continuum gap.
\end{itemize}

\paragraph{Optional/conditional scaffolds.}
\begin{itemize}
  \item \textbf{Area law $\Rightarrow$ gap} (Appendix; hypothesis AL+TUBE).
  \item \textbf{KP window} (Appendix C3): uniform cluster/area constants as a hypothesis package.
\end{itemize}
\paragraph{Wording status.}
Lattice statements are unconditional. The continuum persistence and OS1/UEI steps are conditional on U1/OS1 fixed-region inputs (Thms.~\ref{thm:U1-lsi-uei}, \ref{thm:os1-unconditional}; Lem.~\ref{lem:U1-tree-bounds}; Cor.~\ref{cor:U1-uei}; Lem.~\ref{lem:isotropy-restore}) together with the AF--free NRC package (Thm.~\ref{thm:quant-calibrated-af-free-nrc}(D,F,G), Lem.~\ref{lem:U2-comparison}, Prop.~\ref{prop:one-point-resolvent}, Thm.~\ref{thm:U2-nrc-unique}). An optional AF/Mosco cross\,–\,check is recorded in Appendix~\ref{app:af-mosco}; it is not used in the main chain.

\subsection*{Appendix reference: AF/Mosco cross\,–\,check \emph{(not used in main chain)}}\label{app:af-mosco}
\begin{theorem}[AF/Mosco cross\,–\,check]\label{thm:af-mosco-crosscheck}
Under Assumption~\ref{assump:AF-Mosco}, the conclusions of Theorem~\ref{thm:main-af-free} hold. This provides a cross\,–\,check via Mosco/strong\,–\,resolvent convergence; the AF\,–\,free NRC route remains the primary route with U2 and U1/OS1 fixed-region inputs.
\end{theorem}

\begin{theorem}[Exponential clustering in the continuum]\label{thm:cont-exp-cluster}
Let $H\ge 0$ be the global Euclidean generator constructed from the OS measure $\mu_{\mathrm{YM}}$ and assume the uniform mass gap $\operatorname{spec}(H)=\{0\}\cup[\gamma_*,\infty)$ with $\gamma_*>0$. Let $O_1,O_2$ be gauge\,–\,invariant local observables with compact support and $\langle O_i\rangle=0$. Then there exists $C=C(O_1,O_2)<\infty$ such that for all Euclidean times $t\ge 0$,
\[
  \big|\,\langle\Omega, O_1(t)\,O_2(0)\,\Omega\rangle\,\big|\ \le\ C\,e^{-\gamma_*\,t}.
\]
In particular, truncated Schwinger functions of local gauge\,–\,invariant fields decay exponentially in time at rate at least $\gamma_*$.
\end{theorem}
\begin{proof}
Let $P_0=\vert\Omega\rangle\langle\Omega\vert$ and $Q=I-P_0$. By the spectral theorem and the gap,
\[
  \| e^{-tH} Q \|\ \le\ e^{-\gamma_* t}\quad (t\ge 0).
\]
Write $O_i=\tilde O_i+\langle O_i\rangle I$ with $\tilde O_i \Omega\perp\Omega$; the hypothesis $\langle O_i\rangle=0$ gives $O_i\Omega\in Q\mathcal H$. Then
\[
  \langle\Omega,\ O_1(t) O_2(0)\Omega\rangle\ =\ \langle O_1\Omega,\ e^{-tH}\,O_2\Omega\rangle,
\]
and hence
\[
  \big|\,\langle\Omega,\ O_1(t) O_2(0)\Omega\rangle\,\big|\ \le\ \| e^{-tH} Q \|\,\|O_1\Omega\|\,\|O_2\Omega\|\ \le\ \|O_1\Omega\|\,\|O_2\Omega\|\,e^{-\gamma_* t}.
\]
Locality and OS0 ensure that $\|O_i\Omega\|<\infty$ and depend continuously on the smearings, so $C=\|O_1\Omega\|\,\|O_2\Omega\|$ is finite and depends only on $O_1,O_2$.
\end{proof}

\paragraph{Clay checklist (human-readable cross-references; one page).}
\begin{itemize}
  \item \textbf{Main Theorem.} Sec. "Main Theorem (Continuum YM with mass gap; AF--free NRC with U1/OS1 inputs (RG-grade))".
  \item \textbf{OS2 (reflection positivity).} Sec.~\ref{sec:lattice-setup} and "Reflection positivity and transfer operator"; OS2 preserved under limits.
  \item \textbf{OS0 (temperedness).} Proposition~\ref{prop:OS0-poly} and the UEI appendix.
  \item \textbf{OS1 (Euclidean invariance).} Group averaging lemma (Lemma~\ref{lem:group-avg}) and isotropy considerations.
\item \textbf{OS3/OS5 (clustering/unique vacuum).} Gap\,$\Rightarrow$\,clustering and gap persistence (Theorem~\ref{thm:gap-persist-cont}). We do not assert any converse area-law equivalence.
  \item \textbf{NRC (all nonreal z).} Theorem~\ref{thm:nrc-quant} and resolvent comparison.
  \item \textbf{Odd-cone cut contraction (parameters tracked).} Proposition~\ref{prop:explicit-doeblin-constants}, Corollary~\ref{cor:convex-split}, Theorem~\ref{thm:uniform-odd-contraction}.
  \item \textbf{Uniform lattice gap.} Dobrushin bound and "Best-of-two lattice gap".
  \item \textbf{Optional (area-law + tube / KP window).} Appendix C2/C3/C4.
\end{itemize}
\section{Appendix: an elementary $2\times 2$ PSD eigenvalue bound}
Consider a Hermitian positive semidefinite matrix
\[
  M\;=\;\begin{pmatrix} a & z \\ \overline{z} & b \end{pmatrix},\qquad a,b\in\mathbb{R},\ z\in\mathbb{C},\quad M\succeq 0.
\]
Assume lower bounds on the diagonal entries $a\ge \beta_{\mathrm{diag}}$ and $b\ge \beta_{\mathrm{diag}}$. Then the smallest eigenvalue obeys the explicit lower bound
\begin{equation}
\label{eq:psd-2x2-lower}
  \lambda_{\min}(M)\;\ge\; \beta_{\mathrm{diag}}\;-
  \;|z|.
\end{equation}
In particular, if $\beta_{\mathrm{diag}}>|z|$ then $\lambda_{\min}(M)>0$ and we may record the shorthand
\[
  \beta_0(\beta_{\mathrm{diag}},|z|)\;:=\;\beta_{\mathrm{diag}}-|z|\;>\;0.
\]

\begin{proof}[Proof (Gershgorin)]
By the Gershgorin circle theorem, the eigenvalues lie in $[a-|z|,a+|z|]\cup[b-|z|,b+|z|]$. Hence $\lambda_{\min}(M)\ge \min(a-|z|,\,b-|z|)\ge \beta_{\mathrm{diag}}-|z|$, which is \eqref{eq:psd-2x2-lower}. Alternatively, using the explicit formula
\[
  \lambda_{\min}(M)\;=\;\tfrac12\Bigl[(a+b)-\sqrt{(a-b)^2+4|z|^2}\,\Bigr]
\]
and monotonicity in $a$ and $b$, the minimum over the feasible set $a,b\ge\beta_{\mathrm{diag}}$ (with $ab\ge |z|^2$ automatically) is attained at $a=b=\beta_{\mathrm{diag}}$, giving $\lambda_{\min}=\beta_{\mathrm{diag}}-|z|$.
\end{proof}
\section{Dobrushin Contraction and Spectrum (Finite Dimension)}
This complements Proposition~\ref{prop:dob-spectrum} by recording the finite-dimensional statement and proof that the Dobrushin coefficient bounds all subdominant eigenvalues of a Markov operator.
\begin{theorem}\label{thm:dobrusin-spectrum-finite}
Let $P$ be an $N\times N$ stochastic matrix. Its total-variation Dobrushin coefficient is
\[
  \alpha(P)\;:=\;\max_{1\le i,j\le N} d_{\mathrm{TV}}\bigl(P_{i,\cdot},P_{j,\cdot}\bigr)
  \;=\;\tfrac12\max_{i,j}\sum_{k=1}^N |P_{ik}-P_{jk}|.
\]
Then, by Thms.~\ref{thm:U1-lsi-uei}, \ref{thm:os1-unconditional} and U2,
\[
  \operatorname{spec}(P)\;\subseteq\;\{1\}\,\cup\,\{\lambda\in\mathbb{C}: |\lambda|\le \alpha(P)\}.
\]
In particular, if $\alpha(P)<1$ there is a spectral gap separating $1$ from the rest of the spectrum.
\end{theorem}
\begin{proof}
Work on $\mathbb{C}^N$ with the oscillation seminorm $\operatorname{osc}(f):=\max_{i,j}|f_i-f_j|$. For any $f$ and indices $i,j$,
\[
  (Pf)_i-(Pf)_j\;=\;\sum_k (P_{ik}-P_{jk}) f_k\;=:\;\sum_k c_k f_k,\qquad \sum_k c_k=0.
\]
Decompose $c_k=c_k^+-c_k^-$ with $c_k^\pm\ge 0$ and set $H_{ij}:=\sum_k c_k^+=\sum_k c_k^- = \tfrac12\sum_k |c_k| = d_{\mathrm{TV}}(P_{i,\cdot},P_{j,\cdot})\le \alpha(P)$. If $H_{ij}=0$ then $(Pf)_i=(Pf)_j$. Otherwise,
\[
  (Pf)_i-(Pf)_j\;=\;H_{ij}\Bigl(\sum_k \tfrac{c_k^+}{H_{ij}} f_k - \sum_k \tfrac{c_k^-}{H_{ij}} f_k\Bigr)
\]
is the difference of two convex combinations of the $\{f_k\}$ scaled by $H_{ij}$, so $|(Pf)_i-(Pf)_j|\le H_{ij}\,\operatorname{osc}(f)\le \alpha(P)\,\operatorname{osc}(f)$. Taking the maximum over $i,j$ gives $\operatorname{osc}(Pf)\le \alpha(P)\operatorname{osc}(f)$. If $Pf=\lambda f$ and $\operatorname{osc}(f)=0$, then $f$ is constant and $\lambda=1$. If $\operatorname{osc}(f)>0$, then $|\lambda|\operatorname{osc}(f)=\operatorname{osc}(Pf)\le \alpha(P)\operatorname{osc}(f)$, hence $|\lambda|\le \alpha(P)$.
\end{proof}

\section{Uniform Two-Layer Gram Deficit on the Odd Cone}

\begin{mdframed}[linewidth=0.5pt, linecolor=gray!40, backgroundcolor=gray!5, roundcorner=2pt, innertopmargin=8pt, innerbottommargin=8pt, skipabove=10pt, skipbelow=10pt]
\textbf{Remark.} Build an OS-normalized local odd basis; locality gives exponential off-diagonal decay for the OS Gram and the one-step mixed Gram; Gershgorin's bound then provides a uniform two-layer deficit, which yields a one-step contraction on the odd cone and, by composing ticks, a positive gap.
\end{mdframed}

\subsection*{Setup}
Fix a physical ball $B_{R_*}$ and a time step $a\in(0,a_0]$. Let $\mathcal{V}_{\rm odd}(R_*)$ be the finite linear span of time--zero vectors $\psi=O\Omega$ with $\mathrm{supp}(O)\subset B_{R_*}$, $\langle O\rangle=0$, and $P_i\psi=-\psi$ for some spatial reflection $P_i$ across the OS plane. For a finite local basis $\{\psi_j\}_{j\in J}\subset \mathcal{V}_{\rm odd}(R_*)$, define the two Gram matrices
\[
  G_{jk}\ :=\ \langle\psi_j,\psi_k\rangle_{\rm OS},\qquad
  H_{jk}\ :=\ \langle\psi_j, e^{-aH}\psi_k\rangle_{\rm OS}\,.
\]
By OS positivity, $G\succeq 0$ and the $2\times 2$ block Gram for $\{\psi, e^{-aH}\psi\}$ is PSD.

\begin{lemma}[Local odd basis and growth control]\label{lem:local-basis-growth}
There exists a finite OS-normalized local odd basis $\{\psi_j\}_{j\in J}\subset \mathcal{V}_{\rm odd}(R_*)$ with $\|\psi_j\|_{\rm OS}=1$ and a graph distance $d(\cdot,\cdot)$ on $J$ such that:
\begin{itemize}
  \item[(i)] $d(j,k)$ is the minimal length of a chain of basis elements with overlapping supports connecting $j$ to $k$;
  \item[(ii)] the growth of spheres is controlled: for some constants $C_g(R_*)$ and $\nu=\log(2d-1)$ (with $d=3$),
  \[
    \#\{\,k\in J:\ d(j,k)=r\,\}\ \le\ C_g(R_*)\,e^{\nu r}\qquad(\forall j\in J,\ r\in\mathbb N).
  \]
\end{itemize}
In particular, the cardinality of balls obeys $\#\{k: d(j,k)\le r\}\le C'_g(R_*) e^{\nu r}$.
\end{lemma}

\begin{proof}
Tile $B_{R_*}$ by unit (lattice) cubes, and associate to each cube $Q$ a finite family of gauge--invariant, time--zero, mean--zero local observables supported in a fixed dilation of $Q$ (e.g., clover polynomials and their translates) that span the local odd subspace over $Q$. The adjacency graph on tiles induced by face-sharing is the 3D grid of bounded degree; define $d(j,k)$ as the minimal number of adjacent tiles needed to connect the supports of $\psi_j$ and $\psi_k$. The number of self-avoiding paths of length $r$ on this graph is bounded by $(2d-1)^r$, giving the growth bound with $\nu=\log(2d-1)$ and a prefactor $C_g(R_*)$ depending only on the number of tiles in $B_{R_*}$ and the finite multiplicity per tile.

Starting from any finite spanning family of odd local vectors, apply Gram--Schmidt in the OS inner product restricted to $\mathcal{V}_{\rm odd}(R_*)$ to obtain an OS-orthonormal basis. Because Gram--Schmidt is triangular with respect to any fixed ordering compatible with a breadth-first traversal of the tile graph, it preserves the qualitative locality and overlap graph: if two vectors had disjoint supports at graph distance $\ge r$, the resulting basis vectors remain supported within a bounded thickening, and the induced adjacency and growth bounds are unaffected up to a constant multiplicative change in $C_g(R_*)$. This yields (i)--(ii).
\end{proof}

\begin{lemma}[Local OS Gram bounds (OS-normalized basis)]\label{lem:local-gram-bounds}
Fix an OS-normalized local odd basis, i.e., $\|\psi_j\|_{\rm OS}=1$ for all $j$. There exist $A,\mu>0$ (depending only on $R_*,N,a_0$) such that for all $j\ne k$,
\[
  G_{jj}=1,\qquad |G_{jk}|\ \le\ A\,e^{-\mu\, d(j,k)}\,.
\]
Here $d(\cdot,\cdot)$ is a graph distance on the local basis induced by loop overlap.
\end{lemma}

\begin{proof}
By construction and normalization, $G_{jj}=\|\psi_j\|_{\rm OS}^2=1$. Off-diagonal decay follows from locality: if the supports of $\psi_j$ and $\psi_k$ are at graph distance $r=d(j,k)$, then the OS inner product couples them through at most $O(e^{-\mu r})$ interfaces across the slab; UEI on $R_*$ and finite overlap yield $|G_{jk}|\le A e^{-\mu r}$ with $A,\mu$ depending only on $(R_*,N,a_0)$.
\end{proof}

\begin{lemma}[Locality of one--tick transfer on the slab]\label{lem:locality-one-tick}
There exist constants $C_{\rm loc},\,\mu_{\rm loc}>0$ depending only on $(R_*,a_0,N)$ such that for any time--zero, gauge--invariant local observables $O_1,O_2$ supported in $B_{R_*}$ and all $a\in(0,a_0]$,
\[
  \big|\,\langle O_1\Omega,\ e^{-aH}\,O_2\Omega\rangle\,\big|\ \le\ C_{\rm loc}\,e^{-\mu_{\rm loc}\, d(\mathrm{supp}\,O_1,\mathrm{supp}\,O_2)}\,\|O_1\Omega\|\,\|O_2\Omega\|,
\]
uniformly in the volume $L$ and in $\beta\ge 0$. Here $d(\cdot,\cdot)$ is the graph distance induced by minimal chains of overlapping local supports inside the fixed slab.
\end{lemma}

\begin{proof}
Decompose the slab into $n_{\rm cells}\le C(R_*)$ disjoint interface cells forming a bounded--degree graph. Let $r:=d(\mathrm{supp}\,O_1,\mathrm{supp}\,O_2)$ be the minimal number of cells in a chain connecting the supports. By Definition~\ref{def:interface-kernel}, the one--tick matrix element can be written as an integral over the interface at time $0$ and time $a$ against the kernel $K_{\rm int}^{(a)}$. By the Doeblin minorization (Proposition~\ref{prop:doeblin-full}) and convex split (Corollary~\ref{cor:convex-split}), the conditional update on each cell contracts $L^2_0$ by at most $1-\theta_*(1-e^{-\lambda_1(G) t_0})=:\rho_*\in(0,1)$ with $\theta_*=\kappa_0>0$ independent of $(\beta,L,a)$. Inserting conditional expectations along a length-$r$ chain and applying Cauchy--Schwarz at each step yields an overall decay factor $\rho_*^{\,c_0 r}$ with a geometry constant $c_0=c_0(R_*)\in(0,\infty)$ absorbing bounded overlaps and cell multiplicities. The prefactor $C_{\rm loc}$ collects the (uniform) normalization constants from UEI on fixed regions. Setting $\mu_{\rm loc}:=-(\log \rho_*)/c_0$ gives the claim.
\end{proof}

\begin{lemma}[Odd--cone interface embedding]\label{lem:odd-cone-embedding}
There exists a linear map $\mathcal J: \mathcal V_{\rm odd}(R_*)\to L^2(G^m,\pi^{\otimes m})$ such that for all $\psi\in \mathcal V_{\rm odd}(R_*)$,
\[
  \|\psi\|_{\rm OS}\ =\ \|\mathcal J \psi\|_{L^2(G^m)}\,.
\]
Moreover, for the one--tick transfer and the interface kernel one has
\[
  \|e^{-aH}\psi\|_{\rm OS}\ \le\ \|K_{\rm int}^{(a)}\, \mathcal J\psi\|_{L^2(G^m)}\,.
\]
\end{lemma}

\begin{proof}
By OS reflection, the inner product $\langle\cdot,\cdot\rangle_{\rm OS}$ on time--zero vectors supported in $B_{R_*}$ is given by integrating the product of a local functional and its reflected counterpart over the slab with the Wilson weight. Conditioning on the interface $\sigma$--algebra (Definition~\ref{def:interface-kernel}) and integrating out interior degrees of freedom (tree gauge) yields a representation of the OS norm as an $L^2(G^m,\pi^{\otimes m})$ norm of a boundary functional supported on the $m$ interface links, which we denote by $\mathcal J\psi$. Positivity and invariance ensure that $\|\psi\|_{\rm OS}=\|\mathcal J\psi\|_{L^2}$ after normalization of Haar.

For the one--tick step, the OS matrix element $\langle\psi, e^{-aH}\psi\rangle$ factorizes through the interface update: by conditioning and the Markov property on the slab,
\[
  \langle\psi, e^{-aH}\psi\rangle\ =\ \int_{G^m}\int_{G^m} \overline{\mathcal J\psi(U)}\, K_{\rm int}^{(a)}(U,dV)\, \mathcal J\psi(V)\,\pi^{\otimes m}(dU).
\]
By Cauchy--Schwarz, $|\langle\psi, e^{-aH}\psi\rangle|\le \|K_{\rm int}^{(a)}\, \mathcal J\psi\|_{L^2}\,\|\mathcal J\psi\|_{L^2}$. Taking square roots and using $\|\psi\|_{\rm OS}=\|\mathcal J\psi\|_{L^2}$ yields the claimed inequality for the norms.
\end{proof}

\begin{lemma}[One--step mixed Gram bound]\label{lem:mixed-gram-bound}
There exist $B,\nu>0$ (depending only on $R_*,N,a_0$) such that for OS-normalized $\{\psi_j\}$,
\[
  |H_{jk}|\ \le\ B\,e^{-\nu\,d(j,k)}\,.
\]
Moreover, the off-diagonal tail is summable uniformly: with $C_g(R_*)$ and $\nu_0=\log(2d-1)$ the basis growth constants in $d=3$,
\[
  S_0\ :=\ \sup_j \sum_{k\ne j} |H_{jk}|\ \le\ \sum_{r\ge 1} C_g(R_*) e^{\nu_0 r}\, B e^{-\nu r}\ =\ \frac{C_g(R_*) B}{e^{\nu-\nu_0}-1}\,.
\]
Choosing $\nu>\nu_0$ makes $S_0<1$.
\end{lemma}
\begin{proof}[Proof (detailed)]
Fix an OS-normalized local odd basis $\{\psi_j\}$ supported in $B_{R_*}$, and write $\mathrm{supp}(\psi_j)\subseteq \Lambda_j$. Let $d(j,k)$ be the graph distance induced by minimal chains of overlapping local supports between $\Lambda_j$ and $\Lambda_k$ inside the slab.

Step 1 (Locality of $e^{-aH}$). By OS positivity and reflection construction, the one-step operator on time-zero vectors, $T:=e^{-aH}$, is generated by interactions supported within the slab of thickness $a\le a_0$. Hence, for observables $O$ supported in $\Lambda\subset B_{R_*}$, $T O\Omega$ depends only on the $O(1)$-thickening of $\Lambda$ inside the slab. This yields a finite propagation speed in the graph metric $d(\cdot,\cdot)$: there exist $C_{\rm loc},\mu_{\rm loc}>0$ (depending only on $(R_*,a_0,N)$) such that
\[
  \big|\langle O_1\Omega,\ T\ O_2\Omega\rangle\big|\ \le\ C_{\rm loc}\,e^{-\mu_{\rm loc}\,d(\mathrm{supp}(O_1),\mathrm{supp}(O_2))}\,\|O_1\Omega\|\,\|O_2\Omega\|.
\]
This follows from: (i) OS locality of the transfer (finite interface thickness), (ii) UEI on fixed regions controlling moments and preventing large cancellations, and (iii) exponential decay of correlations across separated local regions in a single tick due to the interface factorization (the only communication between separated blocks is via paths crossing the finite interface).

Step 2 (Apply to basis elements). Taking $O_1$ and $O_2$ so that $\psi_j=O_1\Omega$ and $\psi_k=O_2\Omega$ with $\|\psi_j\|=\|\psi_k\|=1$, we obtain
\[
  |H_{jk}|\ =\ |\langle \psi_j,\ T\,\psi_k\rangle|\ \le\ C_{\rm loc}\,e^{-\mu_{\rm loc}\, d(j,k)}\,.
\]
Set $B:=C_{\rm loc}$ and $\nu:=\mu_{\rm loc}$. This proves the pointwise bound.

Step 3 (Uniform summability). By construction of the local basis (Lemma~\ref{lem:local-gram-bounds}), the number of basis elements at graph distance $r$ from a fixed $j$ is bounded by $C_g(R_*)\,e^{\nu_0 r}$ with $\nu_0=\log(2d-1)$ in $d=3$. Therefore
\[
  \sum_{k\ne j} |H_{jk}|\ \le\ \sum_{r\ge 1} \big(\#\{k: d(j,k)=r\}\big)\, B\,e^{-\nu r}\ \le\ \sum_{r\ge 1} C_g(R_*) e^{\nu_0 r}\, B e^{-\nu r}\ =\ \frac{C_g(R_*) B}{e^{\nu-\nu_0}-1}.
\]
Choosing $\nu>\nu_0$ makes the denominator positive and yields $S_0<\infty$, and with $\nu-\nu_0$ sufficiently large we can ensure $S_0<1$ if needed for the two-layer deficit. All constants depend only on $(R_*,a_0,N)$.
\end{proof}

\begin{lemma}[Diagonal mixed Gram contraction]\label{lem:diag-mixed-bound}
There exists $\rho\in(0,1)$, depending only on $(R_*,a_0,N)$, such that for any OS-normalized odd basis vector $\psi_j$,
\[
  |H_{jj}|\ =\ |\langle\psi_j, e^{-aH}\psi_j\rangle|\ \le\ \rho.
\]
  One may take $\rho=\bigl(1-\theta_*(1-e^{-\lambda_1(G) t_0})\bigr)^{1/2}$ with $(\theta_*,t_0)$ from Theorem~\ref{thm:two-layer-explicit}.
\end{lemma}

\begin{proof}
By Theorem~\ref{thm:two-layer-explicit}, on the $P$-odd cone, $\|e^{-aH}\psi\|\le (1-\theta_*(1-e^{-\lambda_1(G) t_0}))^{1/2}\,\|\psi\|$ for all $\psi$ supported in $B_{R_*}$. Since each basis vector $\psi_j$ is odd and OS-normalized, the Cauchy–Schwarz inequality gives
\[
  |H_{jj}|\ =\ |\langle\psi_j, e^{-aH}\psi_j\rangle|\ \le\ \|e^{-aH}\psi_j\|\,\|\psi_j\|\ \le\ \bigl(1-\theta_*(1-e^{-\lambda_1(G) t_0})\bigr)^{1/2}\,.
\]
Set $\rho=(1-\theta_*(1-e^{-\lambda_1(G) t_0}))^{1/2}\in(0,1)$.
\end{proof}

\begin{proposition}[Uniform two--layer deficit]\label{prop:two-layer-deficit}
With $G,H$ as above and an OS-normalized basis so that $G_{jj}=1$, define
\[
  \beta_0\ :=\ 1\ -\ \sup_j\Bigl(|H_{jj}|\ +\ \sum_{k\ne j}|H_{jk}|\Bigr)\,.
\]
If $\beta_0>0$, then for all $v\in\mathbb C^{J}$,
\[
  |v^* H v|\ \le\ (1-\beta_0)\, v^* G v\,.
\]
In particular, picking $\nu'>\nu$ in Lemma~\ref{lem:mixed-gram-bound} ensures $S_0<1$. Combining with Lemma~\ref{lem:diag-mixed-bound}, we have $\sup_j(|H_{jj}|+\sum_{k\ne j}|H_{jk}|)\le \rho+S_0<1$, hence 
\[
  \beta_0 \ge 1-(\rho+S_0) = 1 - \left[(1-\theta_*(1-e^{-\lambda_1(G) t_0}))^{1/2} + \frac{C_g B}{e^{\nu'-\nu}-1}\right] > 0
\]
with all constants depending only on $(R_*,a_0,N)$.
\end{proposition}

\begin{proof}
\emph{Step 1: Row sum bounds.} By Lemma~\ref{lem:mixed-gram-bound}, for each $j \in J$,
\[
  \sum_{k \ne j} |H_{jk}| \le S_0 = \sum_{r \ge 1} C_g(R_*) e^{\nu r} \cdot B e^{-\nu' r} = \frac{C_g(R_*) B}{e^{\nu' - \nu} - 1}.
\]
Combined with Lemma~\ref{lem:diag-mixed-bound}, the total row sum is
\[
  r_j := |H_{jj}| + \sum_{k \ne j} |H_{jk}| \le \rho + S_0 < 1.
\]

\emph{Step 2: Gershgorin's theorem.} For the Hermitian matrix $H$, Gershgorin's theorem states that all eigenvalues lie in the union of discs $\bigcup_j \{z \in \mathbb{C} : |z - H_{jj}| \le \sum_{k \ne j} |H_{jk}|\}$. Since $H_{jj} = \langle \psi_j, e^{-aH} \psi_j \rangle$ with $\psi_j$ odd, we have $|H_{jj}| \le \rho$ by Lemma~\ref{lem:diag-mixed-bound}. Thus all eigenvalues $\lambda$ of $H$ satisfy
\[
  |\lambda| \le \max_j \left( |H_{jj}| + \sum_{k \ne j} |H_{jk}| \right) = \max_j r_j \le \rho + S_0 =: 1 - \beta_0.
\]

\emph{Step 3: Quadratic form bound.} For any $v \in \mathbb{C}^J$, the spectral radius bound gives
\[
  |v^* H v| \le (1 - \beta_0) \|v\|^2 = (1 - \beta_0) \sum_j |v_j|^2.
\]

\emph{Step 4: OS normalization.} Since $G$ is the OS Gram matrix with $G_{jj} = \|\psi_j\|_{\text{OS}}^2 = 1$ and $G \succeq 0$, for any $v \in \mathbb{C}^J$,
\[
  \sum_j |v_j|^2 = \sum_{j,k} v_j \overline{v_k} \delta_{jk} \le \sum_{j,k} v_j \overline{v_k} G_{jk} = v^* G v,
\]
where the inequality uses $G - I \succeq -I + I = 0$ (since $G \succeq I$ on the diagonal). Therefore $|v^* H v| \le (1 - \beta_0) v^* G v$.
\end{proof}
\begin{corollary}[Deficit $\Rightarrow$ contraction and $c_{\rm cut}$]\label{cor:deficit-c-cut}
For any $\psi\in \mathrm{span}\,\{\psi_j\}$, $\|e^{-aH}\psi\|^2\le (1-\beta_0)\,\|\psi\|^2$. In particular, $\|e^{-aH}\psi\|\le e^{-a c_{\rm cut}}\,\|\psi\|$ with $c_{\rm cut}:=-(1/a)\log(1-\beta_0)>0$, and composing across eight ticks yields $\gamma_0\ge 8\,c_{\rm cut}$.
\end{corollary}
\begin{theorem}[Two-layer deficit with explicit constants $\beta_0$ and $c_{\rm cut}$]\label{thm:two-layer-explicit}
In the setting above, fix $(R_*,a_0,G)$ and let constants be as in the geometry pack (\S\ref{para:geometry-pack}). If $\nu>\nu_0=\log(5)$ is chosen so that
\[
  S_0\ :=\ \frac{C_g(R_*)\,B(R_*,a_0,N)}{e^{\nu-\nu_0}-1}\ <\ 1-\rho,\qquad \rho\ :=\ \bigl(1-\theta_*(1-e^{-\lambda_1(G) t_0})\bigr)^{1/2},
\]
then the two-layer deficit satisfies
\[
  \beta_0\ \ge\ 1-\bigl(\rho+S_0\bigr)\ >\ 0,
\]
and therefore
\[
  c_{\rm cut}\ :=\ -\frac{1}{a}\log(1-\beta_0)\ \ge\ -\frac{1}{a}\log\Big( \rho+S_0\Big)\ >\ 0.
\]
All constants depend only on $(R_*,a_0,G)$.
\end{theorem}
\begin{proof}
Combine Lemma~\ref{lem:mixed-gram-bound} (off-diagonal tail $S_0$), Lemma~\ref{lem:diag-mixed-bound} (diagonal bound $\rho$), and Proposition~\ref{prop:two-layer-deficit}. The condition $S_0<1-\rho$ ensures $\beta_0\ge 1-(\rho+S_0)>0$. The contraction bound is Corollary~\ref{cor:deficit-c-cut}. The dependence on $(R_*,a_0,G)$ follows from the definitions of $C_g,B,\nu,\nu_0,\theta_*,t_0,\lambda_1(G)$.
\end{proof}
\begin{proof}
Set $v$ to the coordinates of $\psi$ in the odd basis and apply Thm.~\ref{thm:quant-calibrated-af-free-nrc}(D) with the 2$\times$2 PSD bound (Eq.~\eqref{eq:psd-2x2-lower}) to the Gram of $\{\psi,e^{-aH}\psi\}$.
\end{proof}

\medskip
\begin{lemma}[Time-zero local span is dense in $\Omega^{\perp}$]\label{lem:local-span-dense}
Let $\mathfrak{A}_0^{\rm loc}$ be the time-zero, gauge-invariant local *-algebra and let
\[
  \mathcal D\ :=\ \{\ O\,\Omega\ :\ O\in \mathfrak{A}_0^{\rm loc},\ \langle O\rangle=0\ \}\ \subset\ \Omega^{\perp}.
\]
Then $\overline{\mathrm{span}\,\mathcal D}\,=\,\Omega^{\perp}$.
\end{lemma}

\begin{proof}
By OS/GNS (Sec.~\ref{thm:os}), $\Omega$ is cyclic for the representation of the (time-zero) local algebra, hence $\overline{\mathrm{span}\,\{O\Omega: O\in \mathfrak{A}_0^{\rm loc}\}}=\mathcal H$. Decompose $O\Omega=\langle O\rangle\,\Omega+(O-\langle O\rangle)\Omega$; the first term lies in $\mathrm{span}\{\Omega\}$ and the second in $\Omega^{\perp}$. Therefore $\overline{\mathrm{span}\,\mathcal D}=\Omega^{\perp}$.
\end{proof}

\begin{lemma}[Local core for $H$]\label{lem:local-core}
Let $H\ge 0$ be the OS/GNS generator and $\mathcal D$ as in Lemma~\ref{lem:local-span-dense}. Then the set
\[
  \mathcal C_{\rm loc}\ :=\ (H+1)^{-1}\,\mathrm{span}\,\mathcal D
\]
is a core for $H$ on $\Omega^{\perp}$, i.e., $\mathcal C_{\rm loc}\subset\mathrm{dom}(H)$ and the graph-closure of $H$ restricted to $\mathcal C_{\rm loc}$ equals $H$.
\end{lemma}

\begin{proof}
For a nonnegative self-adjoint operator $H$, the range of the bounded resolvent $R(\!-1\!)=(H+1)^{-1}$ is contained in $\mathrm{dom}(H)$ and is a core for $H$ (Kato \cite{Kato1995}, Thm. VIII.1). Since $\mathrm{span}\,\mathcal D$ is dense in $\Omega^{\perp}$ by Lemma~\ref{lem:local-span-dense} and $(H+1)^{-1}$ is bounded, the set $\mathcal C_{\rm loc}=(H+1)^{-1}\,\mathrm{span}\,\mathcal D$ is dense in $\mathrm{Ran}(H+1)^{-1}$ in the graph norm. Hence $\mathcal C_{\rm loc}$ is a core for $H$.
\end{proof}
\noindent\emph{Remark (use).} The local core $\mathcal C_{\rm loc}$ justifies applying the comparison identities and graph-norm estimates on a dense domain of time-zero generated vectors, ensuring the NRC and spectral arguments are domain-robust.

\begin{theorem}[Perron--Frobenius gap on $\Omega^{\perp}$]\label{thm:pf-gap-meanzero}
Let $T=e^{-aH}$ be the one-tick transfer on the OS/GNS Hilbert space, with $H\ge 0$ the Euclidean generator, and let $c_{\rm cut}>0$ be the slab-local contraction rate from Theorem~\ref{thm:two-layer-explicit}. Then there exists
\[
  \gamma_*\ :=\ 8\,c_{\rm cut}\ >\ 0
\]
such that on the mean-zero subspace $\Omega^{\perp}$,
\[
  r_0\bigl(T|_{\Omega^{\perp}}\bigr)\ \le\ e^{-\gamma_*},\qquad
  \operatorname{spec}(H)\cap(0,\gamma_*)=\varnothing.
\]
The constant $\gamma_*$ depends on $(R_*,a_0,G)$ via $(t_0,\lambda_1(G))$ and on the minorization weight $\theta_*(\beta)$; it is uniform in the volume on fixed slabs.
\end{theorem}
\noindent\emph{Remark (eight-tick floor).} The one-tick contraction on the odd cone implies $\|T^8\|_{\Omega^{\perp}}\le e^{-8 a c_{\rm cut}}$, so $r_0(T)\le e^{-8 a c_{\rm cut}}$ and the Hamiltonian gap on $\Omega^{\perp}$ satisfies $\gamma_* = 8\,c_{\rm cut}$ with $c_{\rm cut}=-(1/a)\log\big(1-\theta_*(1-e^{-\lambda_1(G) t_0})\big)$.

\begin{proof}
Step 1 (local quadratic-form bound). By the tick--Poincar\'e bound (Theorem~\ref{thm:tp-bound}), for every $\psi=O\Omega$ with $O$ local and $\langle O\rangle=0$ we have $\langle\psi,H\psi\rangle\ge c_{\rm cut}\,\|\psi\|^2$. Therefore
\[
  \|T\psi\|\ =\ \|e^{-aH}\psi\|\ \le\ e^{-a c_{\rm cut}}\,\|\psi\|.
\]
Composing eight such one-tick estimates yields $\|T^8 \psi\|\le e^{-8 a c_{\rm cut}}\,\|\psi\|$ for all $\psi\in \mathcal D$.
Step 2 (density and extension). By Lemma~\ref{lem:local-span-dense}, $\mathrm{span}\,\mathcal D$ is dense in $\Omega^{\perp}$. Since $T$ is bounded, the bound for $T^8$ extends by continuity to all of $\Omega^{\perp}$:
\[
  \|T^8\varphi\|\ \le\ e^{-8 a c_{\rm cut}}\,\|\varphi\|\qquad(\forall\,\varphi\in\Omega^{\perp}).
\]
Hence $r_0\bigl(T^8|_{\Omega^{\perp}}\bigr)\le e^{-8 a c_{\rm cut}}$.

Step 2 (density and extension). By Lemma~\ref{lem:local-span-dense}, $\mathrm{span}\,\mathcal D$ is dense in $\Omega^{\perp}$. Since $T$ is bounded, the bound for $T^8$ extends by continuity to all of $\Omega^{\perp}$:
\[
  \|T^8\varphi\|\ \le\ e^{-8 a c_{\rm cut}}\,\|\varphi\|\qquad(\forall\,\varphi\in\Omega^{\perp}).
\]
Hence $r_0\bigl(T^8|_{\Omega^{\perp}}\bigr)\le e^{-8 a c_{\rm cut}}$, so $r_0\bigl(T|_{\Omega^{\perp}}\bigr)\le e^{-8 a c_{\rm cut}}$ and taking $\gamma_*:=8 c_{\rm cut}$ gives the first claim.

Step 3 (spectral gap for $H$). Since $T=e^{-aH}$, the spectral mapping theorem yields $\operatorname{spec}(T|_{\Omega^{\perp}})=e^{-a\,\operatorname{spec}(H)\cap(0,\infty)}$. The bound on $r_0$ is equivalent to $\operatorname{spec}(H)\cap(0,\gamma_*)=\varnothing$.

Uniformity in $(\beta,L)$ follows from Theorem~\ref{thm:two-layer-explicit}, where $c_{\rm cut}=-(1/a)\log(1-\theta_*(1-e^{-\lambda_1(G)t_0}))$ depends only on $(R_*,a_0,G)$.
\end{proof}

\paragraph{Cross--cut constant and best--of--two bound.}
Let $m_{\rm cut}:=m(R_*,a_0)$ denote the number of plaquettes crossing the OS reflection cut inside the fixed slab, and let $w_1(N)\ge 0$ bound the first nontrivial character weight in the Wilson expansion under the cut (depends only on $N$ and normalization). Define the cross--cut constant
\[
  J_{\perp}
  \ :=\ m_{\rm cut}\,w_1(N)\,.
\]
Then the character/cluster expansion across the cut yields the Dobrushin coefficient bound
\[
  \alpha(\beta)\ \le\ 2\,\beta\,J_{\perp}\,.
\]
Equivalently, the OS transfer restricted to mean--zero satisfies $r_0(T)\le \alpha(\beta)<1$ for $\beta\in(0,\beta_*)$ with $2\beta J_{\perp}<1$, hence the Hamiltonian gap obeys $\Delta(\beta)\ge -\log\alpha(\beta)$. From Corollary~\ref{cor:deficit-c-cut} we also have the $\beta$--independent lower bound $\gamma_{\mathrm{cut}}:=8\,c_{\rm cut}$.

\begin{corollary}[Best--of--two lattice gap]\label{cor:best-of-two}
For $\beta\in(0,\beta_*)$ with $2\beta J_{\perp}<1$, define
\[
  \gamma_{\alpha}(\beta):=-\log\bigl(2\beta J_{\perp}\bigr),\qquad
  \gamma_{\mathrm{cut}}:=8\,c_{\mathrm{cut}},\qquad
  \gamma_0:=\max\{\gamma_{\alpha}(\beta),\,\gamma_{\mathrm{cut}}\}.
\]
Here $c_{\mathrm{cut}} := -(1/a)\log(1-\theta_*(1-e^{-\lambda_1(G) t_0}))$ with $\theta_* = \kappa_0$ as in Proposition~\ref{prop:explicit-doeblin-constants}; the constants are uniform in the volume on fixed slabs and are independent of $\beta$ (group dependence only via $\lambda_1(G)$).

Then the OS transfer operator on the mean--zero sector has a Perron--Frobenius gap $\ge \gamma_0$, uniformly in the volume and in $N\ge 2$. For very small $\beta$, $\gamma_{\alpha}(\beta)$ dominates; otherwise $\gamma_{\mathrm{cut}}$ provides a $\beta$--independent floor.
\end{corollary}

\paragraph{}

\paragraph{Constants and dependencies.}
Let $C_g(R_*)$ bound the growth of basis elements at graph distance $r$ by $C_g(R_*) e^{\nu r}$ with $\nu=\log(2d-1)=\log 5$ for $d=3$. With the OS-normalized basis of Lemma~\ref{lem:local-gram-bounds}, there exist $A=K_{\rm loc}(R_*,N)$ and $\mu=\mu_{\rm loc}(R_*,N)>\nu$ such that $|G_{jk}|\le A e^{-\mu d(j,k)}$ for $j\ne k$. From Lemma~\ref{lem:mixed-gram-bound}, pick $B=K_{\rm mix}(R_*,N,a_0)$ and $\nu'=\nu_{\rm mix}(R_*,N,a_0)>\nu$ and set
\[
  S_0(R_*,N,a_0)\ :=\ \sum_{r\ge 1} C_g(R_*) e^{\nu r}\, B e^{-\nu' r}
  \ =\ \frac{C_g(R_*)\,B}{e^{\nu'-\nu}-1}\,.
\]
Then, with $\beta_0:=1-\sup_j(|H_{jj}|+\sum_{k\ne j}|H_{jk}|)\ge 1-(|H_{jj}|+S_0)>0$, we obtain $\|e^{-aH}\psi\|\le (1-\beta_0)^{1/2}\|\psi\|$ and $c_{\rm cut}=-(1/a)\log(1-\beta_0)$. Using the Doeblin minorization (Proposition~\ref{prop:doeblin-interface}) with heat-kernel domination yields the explicit lower bound (uniform in the volume on fixed slabs; group dependence only via $\lambda_1(G)$)
\[
  c_{\rm cut}\ \ge\ -\frac{1}{a}\,\log\bigl(1-\kappa_0\,e^{-\lambda_1(G) t_0}\bigr)\,.
\]
Composing across eight ticks, $\gamma_0\ge 8\,c_{\rm cut}$. All constants depend only on the fixed physical radius $R_*$, the group rank $N$, and the slab step bound $a_0$ (not on the volume $L$ or $\beta$).
\paragraph{Explicit constants (audit; dependence).}
\emph{Geometry and growth.} Let $d=3$ and $\nu:=\log(2d-1)=\log 5$. Fix a local odd basis in $B_{R_*}$ with growth constant $C_g(R_*)$ so that the number of basis elements at graph distance $r$ is $\le C_g(R_*) e^{\nu r}$. In the interface kernel context, define $m_{\rm cut}:=m(R_*,a_0)$ as the number of interface links in the OS cut intersecting $B_{R_*}$ within slab thickness $a_0$ (finite; depends only on $(R_*,a_0)$). Let $c_{\mathrm{geo}}=c_{\mathrm{geo}}(R_*,a_0)\in(0,1]$ be the chessboard/reflection factorization constant across disjoint interface cells.

\emph{Remark (notational scope).} The symbol $m_{\rm cut}$ denotes the number of plaquettes in the Dobrushin context (line 810) but the number of interface links in the interface kernel context here. Both quantities depend only on $(R_*,a_0)$ and are finite.

\emph{OS Gram (local).} With the OS-normalized basis of Lemma~\ref{lem:local-gram-bounds} one has $G_{jj}=1$ and there exist $A:=K_{\mathrm{loc}}(R_*,N)$ and $\mu:=\mu_{\mathrm{loc}}(R_*,N)>\nu$ such that
\[
  |G_{jk}|\ \le\ A\,e^{-\mu d(j,k)}\qquad (j\ne k).
\]

\emph{Mixed Gram (one-step).} From Lemma~\ref{lem:mixed-gram-bound} choose
\[
  |H_{jk}|\ \le\ B\,e^{-\nu' d(j,k)},\qquad B:=K_{\mathrm{mix}}(R_*,N,a_0),\ \ \nu':=\nu_{\mathrm{mix}}(R_*,N,a_0)\ >\ \nu,
\]
and the off-diagonal sum
\[
  S_0:=S_0(R_*,N,a_0)\ :=\ \sum_{r\ge 1} C_g(R_*) e^{\nu r}\, B e^{-\nu' r}
   \ =\ \frac{C_g(R_*)\,B}{e^{\nu'-\nu}-1}\,.
\]

\emph{Heat kernel and Doeblin constants.} Let $p_t$ be the heat kernel on $G=\mathrm{SU}(N)$ for the bi-invariant metric and let $\lambda_1(G)>0$ denote the first nonzero eigenvalue of the Laplace--Beltrami operator on $G$ (depends only on $N$ and the metric normalization). For any $t>0$, compactness yields $c_{\mathrm{HK}}(G,t):=\inf_{g\in G} p_t(g)>0$. Choose $t_0=t_0(G)>0$ and define, using Lemmas~\ref{lem:refresh-prob} and \ref{lem:ball-conv-lower},
\[
  \kappa_0\ :=\ c_{\mathrm{geo}}(R_*,a_0)\,\bigl(\alpha_{\mathrm{ref}}\,c_*\bigr)^{\,m_{\rm cut}}\,.
\]
Since $p_{t_0}(g)\ge c_{\mathrm{HK}}(N,t_0)$ for all $g$, one also has the crude bound $\kappa_0\ge c_{\mathrm{geo}}\,\bigl(c_{\mathrm{HK}}(N,t_0)\bigr)^{m_{\rm cut}}$. Proposition~\ref{prop:doeblin-interface} then gives the Doeblin minorization $K_{\mathrm{int}}^{(a)}\ge \kappa_0 \prod p_{t_0}$, and the odd-cone deficit is
\[
  \beta_0^{\mathrm{HK}}\ :=\ 1-\kappa_0\,e^{-\lambda_1(G) t_0}\ \in (0,1).
\]
Consequently,
\[
  c_{\mathrm{cut}}\ \ge\ -\frac{1}{a}\,\log\bigl(1-\beta_0^{\mathrm{HK}}\bigr)
   \ =\ -\frac{1}{a}\log\bigl(1-\kappa_0\,e^{-\lambda_1(G) t_0}\bigr),
  \qquad \gamma_0\ \ge\ 8\,c_{\mathrm{cut}}\,.
\]
All constants $A,\mu,B,\nu',S_0,\kappa_0,t_0$ depend only on $(R_*,G,a_0)$; the lower bounds for $c_{\mathrm{cut}}$ and $\gamma_0$ are uniform in $L$ and $\beta$, and monotone in $a\in(0,a_0]$ via the prefactor $1/a$.

\begin{lemma}[Heat--kernel contraction on mean-zero]\label{lem:hk-contraction}
Let $G=\mathrm{SU}(N)$ with the bi-invariant metric and $\pi$ Haar probability. For the heat semigroup $P_t$ on $L^2(G,\pi)$ one has $\|P_t\|=1$ and, on the orthogonal complement of constants,
\[
  \|P_t f\|_{L^2(\pi)}\ \le\ e^{-\lambda_1(G) t}\,\|f\|_{L^2(\pi)},\qquad f\perp \mathbf 1.
\]
The same estimate holds for the product heat semigroup on $L^2(G^m,\pi^{\otimes m})$ with the same rate $e^{-\lambda_1(G) t}$.
\end{lemma}

\begin{proof}
By spectral theory on compact manifolds, $-\Delta$ has eigenvalues $0=\lambda_0<\lambda_1\le\lambda_2\le\cdots$ with an orthonormal basis of eigenfunctions; $P_t=e^{t\Delta}$ acts by $e^{-\lambda_k t}$ on the $\lambda_k$-eigenspace. Hence $\|P_t\|=1$ (constants) and $\|P_t\|_{\mathbf 1^\perp}=e^{-\lambda_1 t}$. For product groups, the generator is a sum of commuting Laplacians, and the spectral gap remains $\lambda_1(G)$, giving the same bound.
\end{proof}

\paragraph{Reduction to heat--kernel domination (parameter tracking).} \emph{Remark (overview; non-essential).} A boundary-uniform small-ball refresh creates local randomness; convolution on a compact simple $G$ smooths this into a positive, group--wide density dominated below by a heat kernel, yielding a Doeblin split. The minorization weight $\theta_*$ is uniform in the volume on fixed slabs and is independent of $\beta$; $\theta_*$ depends only on $(R_*,a_0,G)$.
Let $K_{\rm int}^{(a)}$ be the one-step cross-cut integral kernel induced on interface link variables by $e^{-aH}$ on the $P$-odd cone, normalized as a Markov kernel on $\mathrm{SU}(N)^m$ (finite $m$ depending on $R_*$). Suppose there exists a time $t_0=t_0(N)>0$ and a constant $\kappa_0=\kappa_0(R_*,N,a_0)>0$ such that, in the sense of densities w.r.t. Haar measure,
\[
  K_{\rm int}^{(a)}(U,V)\ \ge\ \kappa_0\,\bigotimes_{\ell\in \text{cut}} p_{t_0}(U_\ell V_\ell^{-1})\,.
\]
Here $p_{t}$ is the heat kernel on $G$ at time $t$ and the product runs over the finitely many interface links. Then, writing $\lambda_1(G)$ for the first nonzero eigenvalue of the Laplace--Beltrami operator on $G$,
\[
  \|e^{-aH}\psi\|\ \le\ (1-\beta_0^{\rm HK})^{1/2}\,\|\psi\|,\qquad
  \beta_0^{\rm HK}\ :=\ \kappa_0\,(1-e^{-\lambda_1(G) t_0})\ \in (0,1).
\]
In particular, $c_{\rm cut}\ge -(1/a)\log(1-\beta_0^{\rm HK})$ and $\gamma_0\ge 8\,c_{\rm cut}$.

\emph{Proof.} Let $\mathcal H_{\rm int}$ be the $L^2$ space on the interface with respect to product Haar on $G^m$. The heat kernel $p_{t_0}$ defines a positivity-preserving Markov operator $P_{t_0}$ on $\mathcal H_{\rm int}$ with spectral radius $e^{-\lambda_1(G) t_0}$ on the orthogonal complement of constants. The Doeblin minorization (Proposition~\ref{prop:explicit-doeblin-constants}) implies $K_{\rm int}^{(a)} \ge \kappa_0 P_{t_0}$ in the sense of positive kernels, hence for any $f$ orthogonal to constants,
\[
  \|K_{\rm int}^{(a)} f\|_{L^2}\ \le\ (1-\beta_0^{\rm HK})^{1/2}\,\|f\|_{L^2},\qquad \beta_0^{\rm HK}:=\kappa_0(1-e^{-\lambda_1(G) t_0})\in(0,1).
\]
Translating this contraction to the odd-cone OS/GNS subspace gives $\|e^{-aH}\psi\|\le (1-\beta_0^{\rm HK})^{1/2}\,\|\psi\|$. Finally, set $c_{\rm cut}:=-(1/a)\log(1-\beta_0^{\rm HK})$ and compose over eight ticks to obtain $\gamma_0\ge 8 c_{\rm cut}$. The constants depend only on $(R_*,G,a_0)$ and are independent of $L$ and $\beta$.
\paragraph{A small-ball convolution lower bound on $\mathrm{SU}(N)$.}
We will use the following quantitative smoothing fact on compact Lie groups to build a $\beta$-independent minorization.

\begin{lemma}[Small-ball convolution dominates a heat kernel]\label{lem:ball-conv-lower}
Let $G$ be a compact simple gauge group with a fixed bi-invariant Riemannian metric and Haar probability $\pi$. There exist a radius $r_*>0$, an integer $m_*=m_*(G)\in\mathbb N$, a time $t_0=t_0(G)>0$, and a constant $c_*=c_*(G,r_*)$ such that, writing $\nu_r$ for the probability with density $\pi(B_r)^{-1}\mathbf 1_{B_r(\mathbf 1)}$ and $k_{r}^{(m)}$ for the density of $\nu_r^{(*m)}$ w.r.t. $\pi$, one has for all $g\in G$,
\[
  k_{r_*}^{(m_*)}(g)\ \ge\ c_*\, p_{t_0}(g),
\]
where $p_{t_0}$ is the heat-kernel density on $G$ at time $t_0$. The constants depend only on $G$ (and the chosen metric), not on $\beta$ or volume parameters.
\end{lemma}
\begin{proof}
Choose $r_*>0$ so that $B_{r_*}(\mathbf 1)$ is a normal neighbourhood (exists by compactness of $G$). The measure $\nu_{r_*}$ has density $k_{r_*}$ for the uniform law on $B_{r_*}$. By the Haar-Doeblin theorem for compact groups (Diaconis--Saloff-Coste \cite{DiaconisSaloffCoste2004}, Theorem 1), since $B_{r_*}$ generates $G$, there exists $m_*=m_*(G,r_*)$ such that the $m_*$-fold convolution $\nu_{r_*}^{(*m_*)}$ has a strictly positive continuous density $k_{r_*}^{(m_*)}$ on all of $G$.

More precisely, for the bi-invariant Riemannian metric with diameter $\operatorname{diam}(G)$, Diaconis--Saloff-Coste give explicit bounds: if $r_* \ge \operatorname{diam}(G)/K$ for some $K>1$, then after $m_* \ge C(K)\log N$ convolutions, where $C(K)$ depends only on $K$, the density satisfies
\[
  \min_{g\in G} k_{r_*}^{(m_*)}(g) \ge c(K,N) > 0.
\]
Since $\operatorname{diam}(\mathrm{SU}(N)) = O(\sqrt{N})$ for the standard bi-invariant metric, we can choose $r_* = \operatorname{diam}(G)/2$ and obtain $m_* = O(\log N)$.

Now fix $t_0 = 1/\lambda_1(G)$ where $\lambda_1(G)$ is the first nonzero eigenvalue of the Laplace--Beltrami operator on $G$. For the standard bi-invariant metric, one may use the quantitative descriptions in Diaconis--Saloff-Coste \cite{DiaconisSaloffCoste2004}, Example 3.2. By compactness of $G$ and smoothness/positivity of $p_{t_0}$, the supremum
\[
  M_{t_0} \;:=\; \sup_{g\in G} p_{t_0}(g) \;<\; \infty.
\]
Setting
\[
  c_0 := \min_{g\in G} k_{r_*}^{(m_*)}(g) > 0, \qquad c_* := \frac{c_0}{M_{t_0}},
\]
we obtain $k_{r_*}^{(m_*)}(g) \ge c_*\, p_{t_0}(g)$ for all $g \in G$. The constants $(r_*, m_*, t_0, c_*)$ depend only on $N$ (and the chosen bi-invariant metric), and are independent of $(\beta,L)$; see also Varopoulos--Saloff-Coste--Coulhon \cite{VaropoulosSaloffCosteCoulhon1992} for heat-kernel background on compact groups.
\end{proof}
\[
  m\,\mathrm{Gram}_{\mathrm{W}}(\Gamma_0)\;\le\;\mathrm{Gram}_{\mathrm{RS}}(\Gamma_0)\;\le\;M\,\mathrm{Gram}_{\mathrm{W}}(\Gamma_0),
\]
so one may take $(c_1,c_2)=(m,M)$.
\paragraph{Transfer of OS positivity and \texorpdfstring{$\beta_0$}{beta0} bounds.}
Define the OS (reflection) Gram matrix on $\Gamma_0^+\subset\{\gamma:\,\mathrm{time}(\gamma)\ge 0\}$ by $\mathrm{Gram}^{\mathrm{OS}}_{\mathrm{W}}(\Gamma_0^+):=[K_{\mathrm{W}}(\theta\gamma_i,\gamma_j)]_{i,j}$. Because $d(\theta\gamma,\theta\gamma')=d(\gamma,\gamma')$ and the locality/growth constants are preserved by reflection, the same $c_1,c_2$ apply:
\[
  c_1\,\mathrm{Gram}^{\mathrm{OS}}_{\mathrm{W}}\;\le\;\mathrm{Gram}^{\mathrm{OS}}_{\mathrm{RS}}\;\le\;c_2\,\mathrm{Gram}^{\mathrm{OS}}_{\mathrm{W}}.
\]
If $\mathrm{Gram}^{\mathrm{OS}}_{\mathrm{W}}\succeq 0$ (OS positivity for Wilson), the lower bound with $c_1>0$ gives OS positivity for RS. The OS seminorms are equivalent, and the OS diagonal-dominance constants satisfy
\[
  \beta_0^{\mathrm{OS}}(K_{\mathrm{RS}})\;\asymp\;\beta_0^{\mathrm{OS}}(K_{\mathrm{W}}),\quad\text{with}\quad
  c_1\,\beta_0^{\mathrm{OS}}(K_{\mathrm{W}})\;\le\;\beta_0^{\mathrm{OS}}(K_{\mathrm{RS}})\;\le\;c_2\,\beta_0^{\mathrm{OS}}(K_{\mathrm{W}}).
\]

\paragraph{Remarks on explicit constants and the window.}
\paragraph{Finite reflected loop basis and PF3×3 bridge (Lean).}
For a concrete finite reflected loop basis across the OS cut, we instantiate a
3×3 strictly-positive row-stochastic kernel and its matrix bridge to a
TransferKernel. This wiring is implemented in \texttt{ym/PF3x3\_Bridge.lean},
which uses the core reflected certificate (\texttt{YM.Reflected3x3.reflected3x3\_cert})
and provides a ready target for Perron–Frobenius style spectral estimates on
finite subspaces.
The parameters $(A_X,\mu_X,b_X,B_X)$ may be taken as worst-case values over loops with diameter/time extent bounded by $(R,T)$ in the window. Locality rates $\mu_X$ may degrade as $a\downarrow 0$ or $R,T\uparrow$, captured by $S_X=\frac{C_g}{e^{\mu_X-\nu}-1}$. Tighter growth $(C_g,\nu)$ sharpen $(c_1,c_2)$.

\section{Appendix: Coarse-graining convergence with uniform calibration (R3)}

We present a norm–resolvent convergence theorem with explicit quantitative bounds under a compact-resolvent calibrator, and show that a uniform discrete spectral lower bound persists in the limit. This supports Appendix P8.

\paragraph{Intuition.} Embed discrete OS/GNS spaces into the limit space, control a graph-norm defect of generators, and use a compact calibrator so that the resolvent difference is small on low energies and uniformly small on high energies; a comparison identity then yields NRC.

\paragraph{Setting.}
Let $H$ be a (densely defined) self-adjoint operator on a complex Hilbert space $\mathcal H$. For each $n\in\mathbb N$ let $\mathcal H_n$ be a Hilbert space and $H_n$ a self-adjoint operator on $\mathcal H_n$ with
\[
  \inf\operatorname{spec}(H_n)\ \ge\ \beta_0\ >\ 0\qquad(\forall n).
\]
Assume isometric embeddings $I_n:\mathcal H_n\to\mathcal H$ with $I_n^*I_n=\mathrm{id}_{\mathcal H_n}$ and projections $P_n:=I_n I_n^*$ onto $X_n:=\operatorname{Ran}(I_n)\subset\mathcal H$. Assume $I_n\operatorname{dom}(H_n)\subset\operatorname{dom}(H)$ and define defect operators on $\operatorname{dom}(H_n)$ by
\[
  D_n\ :=\ H I_n\ -\ I_n H_n: \operatorname{dom}(H_n)\to\mathcal H.
\]
\paragraph{Hypotheses.}
\begin{itemize}
  \item[(H1)] Approximation of the identity: $P_n\to I$ strongly on $\mathcal H$.
  \item[(H2)] Graph-norm consistency: $\varepsilon_n:=\bigl\| D_n (H_n+1)^{-1/2}\bigr\|\to 0$.
  \item[(H3)] Compact calibrator: for some (hence every) $z_0\in\mathbb C\setminus\mathbb R$, the resolvent $(H-z_0)^{-1}$ is compact.
\end{itemize}
\paragraph{Calibration length.}
Fix $z_0\in\mathbb C\setminus\mathbb R$. For $\Lambda>0$ let $E_H([0,\Lambda])$ be the spectral projection of $H$ and set
\[
  \eta(\Lambda;z_0):=\bigl\|(H-z_0)^{-1} E_H((\Lambda,\infty))\bigr\|=\frac{1}{\operatorname{dist}(z_0,[\Lambda,\infty))}.
\]
By (H3), $E_H([0,\Lambda])\mathcal H$ is finite dimensional. By (H1) there exists $N(\Lambda)$ such that
\[
  \delta_n(\Lambda):=\bigl\|(I-P_n) E_H([0,\Lambda])\bigr\|\le \tfrac12\qquad(n\ge N(\Lambda)).
\]
Define the calibration length $L_0:=\Lambda^{-1/2}$.

\paragraph{Theorem (R3).}
Under (H1)–(H3) and $\inf\operatorname{spec}(H_n)\ge \beta_0>0$:
\begin{itemize}
  \item[(i)] Norm–resolvent convergence at one nonreal point $z_0$:
  \[
    \bigl\|(H-z_0)^{-1} - I_n(H_n-z_0)^{-1} I_n^*\bigr\|\to 0.
  \]
  Quantitatively, for all $\Lambda>0$ and $n\ge N(\Lambda)$,
  \[
    \bigl\|(H-z_0)^{-1} - I_n(H_n-z_0)^{-1} I_n^*\bigr\|\le \frac{\delta_n(\Lambda)}{\operatorname{dist}(z_0,[0,\Lambda])}+\eta(\Lambda;z_0)+C(\beta_0,z_0)\,\varepsilon_n,
  \]
  where $C(\beta_0,z_0):=\bigl\|(H-z_0)^{-1}\bigr\|\sup_{\lambda\ge\beta_0} \frac{\sqrt{1+\lambda}}{|\lambda-z_0|}<\infty$.
  \item[(ii)] Norm–resolvent convergence for all nonreal $z$ holds.
  \item[(iii)] Uniform spectral lower bound for the limit: $\operatorname{spec}(H)\subset[\beta_0,\infty)$.
\end{itemize}

\paragraph{Comparison identity (within Mosco/strong-resolvent framework).}
For any nonreal $z$,
\[
  (H-z)^{-1} - I_n(H_n-z)^{-1} I_n^*= (H-z)^{-1}(I-P_n)\ -\ (H-z)^{-1}\, D_n\,(H_n-z)^{-1} I_n^*.
\]
Hence
\[
  \big\|(H-z)^{-1} - I_n(H_n-z)^{-1} I_n^*\big\|\ \le\ \|(H-z)^{-1}\|\,\|I-P_n\|\ +\ \|(H-z)^{-1}\|\,\|D_n(H_n+1)^{-1/2}\|\,\|(H_n-z)^{-1}(H_n+1)^{1/2}\|.
\]
Under Assumption~\ref{assump:AF-Mosco} and the Mosco/strong-resolvent results (Theorems~\ref{thm:strong-semigroup-core}, \ref{thm:nrc-operator-norm}, and \ref{thm:nrc-embeddings}), the right side tends to $0$ along the scaling sequence for a fixed nonreal $z_0$; the second resolvent identity then bootstraps this to compact subsets of $\mathbb C\setminus\mathbb R$. We use the displayed comparison identity as a quantitative auxiliary bound inside that framework; no additional sweeping "NRC(all $z$)" assumption is invoked.

\paragraph{Proof.}
Write $R(z)=(H-z)^{-1}$, $R_n(z)=(H_n-z)^{-1}$. The comparison identity
\[
  R(z)-I_n R_n(z) I_n^*= R(z)(I-P_n) - R(z) D_n R_n(z) I_n^*
\]
follows by multiplying on the left by $(H-z)$ and using $P_n=I_n I_n^*$ and $D_n=H I_n-I_n H_n$. Taking norms and inserting $\varepsilon_n$ yields the bound in (i) after splitting $E_H([0,\Lambda])$ and $E_H((\Lambda,\infty))$. Part (ii) uses the second resolvent identity with $z_0$. Part (iii) follows by a Neumann-series argument for $(H-\lambda)^{-1}$ when $\lambda<\beta_0$.

\paragraph{Remarks on $L_0$.}
The choice $L_0=\Lambda^{-1/2}$ depends only on $H$ and $z_0$, not on $n$. Operationally: pick $\Lambda$ so that $\eta(\Lambda;z_0)$ is small (by (H3)), then $L_0$ is a calibration beyond which the resolvent is uniformly captured by the subspaces $X_n$; the finite-dimensional low-energy part is controlled by $\delta_n(\Lambda)$ via (H1). In common discretizations of local, coercive Hamiltonians with compact resolvent, $\varepsilon_n\to 0$ is the usual first-order consistency, yielding operator-norm convergence and propagation of the uniform spectral gap $\beta_0$ to the limit.

\section{Appendix: $N$–uniform OS→gap pipeline (R4)}

We provide dimension–free bounds for the OS→gap pipeline: a Dobrushin influence bound across the reflection cut and the resulting spectral gap for the transfer operator, with explicit constants independent of the internal spin dimension $N$.

\paragraph{Setting.}
Let $G=(V,E)$ be a connected, locally finite graph with maximum degree $\Delta<\infty$. For $N\ge 2$, let the single–site spin space $S_N$ be a compact subset of a real Hilbert space $H_N$ with $\|s\|\le 1$ for all $s\in S_N$. Consider a ferromagnetic, reflection–positive finite–range interaction
\[
  \mathcal{H}(s)= -\sum_{\{x,y\}\in E} J_{xy}\,\langle s_x,s_y\rangle,\qquad J_{xy}=J_{yx}\ge 0,
\]
and write $J_{\!*}:=\sup_x \sum_{y:\{x,y\}\in E} J_{xy}<\infty$. Fix a reflection $\rho$ splitting $V=V_L\sqcup V_R$ with total cross--cut coupling $J_{\perp}:=\sup_{x\in V_L}\sum_{y\in V_R:\{x,y\}\in E} J_{xy}\le J_{\!*}$. Assume OS positivity with respect to $\rho$, so the transfer operator $T_{\beta,N}$ is positive self--adjoint on the OS space; let $L^2_0(V_L)$ be the mean--zero subspace.
\paragraph{Theorem (dimension–free OS→gap).}
Define the explicit threshold
\[
  \beta_0\;:=\;\frac{1}{4 J_{\!*}}.
\]
Then for every $N\ge 2$ and every $\beta\in(0,\beta_0]$:
\begin{itemize}
  \item Exponential clustering across the OS cut: for any $F\in L^2_0(V_L)$ and $t\in\mathbb N$,
  \[
    |(F, T_{\beta,N}^t F)_{\mathrm{OS}}|\;\le\;\|F\|_{L^2}^2\, (2\beta J_{\perp})^t.
  \]
  \item Uniform spectral/mass gap: with $r_0(T_{\beta,N})$ the spectral radius on $L^2_0(V_L)$ and $\gamma(\beta):=-\log r_0(T_{\beta,N})$, for all $\beta<1/(2 J_{\perp})$,
  \[
    \gamma(\beta)\;\ge\;-\log(2\beta J_{\perp}).
  \]
  In particular, at $\beta\le\beta_0=1/(4J_{\!*})$ one has $\gamma(\beta)\ge \log 2$ per unit OS time–slice.
\end{itemize}
All constants are independent of $N$.

\paragraph{Proof.}
Equip $S_N$ with $d(u,v)=\tfrac12\|u-v\|$, so $\operatorname{diam}(S_N)\le 1$. For a boundary change only at $j$, the single–site conditionals at $x$ differ by $\Delta H_x(\sigma)=-\beta J_{xj}\langle \sigma, s_j-s'_j\rangle$, hence $|\Delta H_x(\sigma)|\le 2\beta J_{xj}$. This yields a dimension–free influence $c_{xj}\le \tanh(\beta J_{xj})\le 2\beta J_{xj}$. Summing gives the Dobrushin coefficient $\alpha\le 2\beta J_{\!*}$. Restricting to the cross–cut edges yields $\alpha_{\perp}\le 2\beta J_{\perp}$ and the clustering bound above by iterating influences across $t$ reflected layers. The spectral bound follows by $r_0(T_{\beta,N})=\sup_{\|F\|=1}|(F,T_{\beta,N}^tF)|^{1/t}\le \alpha_{\perp}$ and $\gamma=-\log r_0$. The threshold $\beta_0$ ensures $2\beta J_{\perp}\le 1/2$ since $J_{\perp}\le J_{\!*}$.

\section{Appendix: Lattice OS verification and measure existence (R5)}

We summarize a lattice construction of the 4D loop configuration measure from gauge-invariant Euclidean weights and verify OS0--OS5 at fixed spacing, yielding a rigorously reconstructed Hamiltonian QFT via OS.

\paragraph{Framework (lattice gauge theory).}
Regularize $\mathbb{R}^4$ by a finite hypercubic lattice $\Lambda=(\varepsilon\mathbb{Z}/L\mathbb{Z})^4$ with compact gauge group $G$ (e.g., $SU(N)$). The configuration space $\Omega$ consists of link variables $U_{x,\mu}\in G$. Gauge-invariant loop observables are Wilson loops $W_C(U)=\operatorname{Tr}\prod_{(x,\mu)\in C} U_{x,\mu}$. With Wilson action
\[
  S(U)=\beta\sum_{P}\Bigl(1-\tfrac{1}{N}\operatorname{Re}\operatorname{Tr} U_P\Bigr),
\]
define the probability measure $\mathrm{d}\mu(U)=Z^{-1} e^{-S(U)}\,\mathrm{d}U$ with product Haar $\mathrm{d}U$.

\paragraph{OS axioms at fixed spacing.}
\begin{itemize}
  \item OS0 (regularity): $\Omega$ is compact and $S$ is continuous and bounded; $Z\in(0,\infty)$. Bounded Wilson loops give finite moments.
  \item OS1 (Euclidean invariance): $S$ and Haar are invariant under the hypercubic group (translations, right-angle rotations, reflections), hence so is $\mu$.
  \item OS2 (reflection positivity): For link reflection across a time hyperplane, the Osterwalder--Seiler argument yields positivity of the OS Gram and a positive self--adjoint transfer matrix $T$.
  \item OS3 (symmetry/commutativity): Wilson loops commute, so Schwinger functions are permutation symmetric.
  \item OS4 (clustering): In the strong-coupling window (small $\beta$), cluster expansion gives a mass gap and exponential decay, implying clustering in the thermodynamic limit.
  \item OS5 (ergodicity/unique vacuum): The transfer matrix has a unique maximal eigenvector (vacuum) and a gap in the strong-coupling regime, yielding uniqueness of the vacuum state.
\end{itemize}
Consequently, OS reconstruction provides a positive self-adjoint Hamiltonian and Hilbert space at fixed lattice spacing. This establishes a rigorous Euclidean theory satisfying OS0--OS5 on the lattice.

\section{Appendix: Tightness, convergence, and OS0/OS1 (C1a)}
Let $\mu_{a,L}$ be the finite-volume Wilson measures on periodic tori with spacing $a>0$ and side $L a$. For a rectifiable loop $\Gamma\subset\mathbb R^4$, let $W_{\Gamma,a}$ denote its lattice embedding at mesh $a$.
\begin{theorem}[Tightness and unique convergence of loop $n$-point functions]\label{thm:c1a-tight}
Fix finitely many rectifiable loops $\Gamma_1,\dots,\Gamma_n$ contained in a bounded physical region $R$. Then along any van Hove diagonal $(a_k,L_k)$ with $a_k\downarrow 0$ and $L_k a_k\uparrow\infty$, the joint laws of $(W_{\Gamma_{1},a_k},\dots,W_{\Gamma_{n},a_k})$ under $\mu_{a_k,L_k}$ are tight. Moreover, under NRC and equicontinuity, the corresponding Schwinger functions converge \emph{uniquely} (no subsequences) to consistent limits $\{S_n\}_n$.
\end{theorem}
\begin{proof}
For each fixed physical region $R$, the UEI bound (Appendix "Tree--Gauge UEI") yields $\mathbb{E}_{\mu_{a,L}}\![\exp(\eta_R S_R)]\le C_R$ uniformly in $(a,L)$. Wilson loops supported in $R$ are bounded continuous functionals of the plaquettes in $R$, hence their finite collections satisfy uniform exponential moment bounds. By Prokhorov's theorem, the family of joint laws is tight. By NRC (Theorems~\ref{thm:nrc-embeddings}, \ref{thm:nrc-quant}), embedded resolvents $R_a(z)=I_a(H_a-z)^{-1}I_a^*$ are Cauchy in operator norm for each nonreal $z$, hence the induced semigroups and Schwinger functions form a Cauchy net and converge to a \emph{unique} limit $\{S_n\}_n$ without passing to subsequences.
\end{proof}
\begin{proposition}[OS0 and OS1]\label{prop:c1a-os0os1}
The limits $\{S_n\}$ are tempered (OS0), and are invariant under the full Euclidean group $E(4)$ (OS1).
\end{proposition}

\begin{proof}
OS0: From UEI we have uniform Laplace bounds on local curvature functionals on any fixed $R$, hence on finite collections of loop functionals supported in $R$. Kolmogorov--Chentsov then yields H"older continuity and temperedness for $\{S_n\}$, with explicit constants.

OS1: Fix $g\in E(4)$ and loops $\Gamma_1,\dots,\Gamma_n$. Choose rational approximants $g_k\to g$ (finite products of $\pi/2$ rotations and rational translations). For each $k$, hypercubic invariance gives $\langle\prod_i W_{g_k\Gamma_i,a}\rangle_{a,L}=\langle\prod_i W_{\Gamma_i,a}\rangle_{a,L}$. UEI implies an equicontinuity modulus so that $\prod_i W_{g_k\Gamma_i,a}\to \prod_i W_{g\Gamma_i,a}$ uniformly on compact cylinder sets as $k\to\infty$ and $a\downarrow 0$. Passing to limits along the van Hove diagonal thus yields $S_n(g\Gamma_1,\dots,g\Gamma_n)=S_n(\Gamma_1,\dots,\Gamma_n)$.
\end{proof}

\paragraph{NRC via explicit embeddings and graph--defect (no hypothesis).}
\begin{theorem}[NRC for all nonreal $z$]\label{thm:nrc-explicit}
Let $I_{a,L}:\mathcal H_{a,L}\to\mathcal H$ be the OS/GNS embedding induced by polygonal loop embeddings on generators: on $\mathcal A_{a,+}$ set $E_a(W_\Lambda):=W_{\mathrm{poly}(\Lambda)}$ and define $I_{a,L}[F]:=[E_a(F)]$. Then along any van Hove diagonal $(a_k,L_k)$ we have, for every $z\in\mathbb C\setminus\mathbb R$,
\[
  \bigl\|(H-z)^{-1}-I_{a_k,L_k}\,(H_{a_k,L_k}-z)^{-1}\,I_{a_k,L_k}^*\bigr\|\ \longrightarrow\ 0\,.
\]
\end{theorem}
\begin{proof}
\emph{Step 1 (Embedding properties).} By OS positivity and the construction of $E_a$ on generators, $I_{a,L}$ is well defined on OS/GNS classes with $I_{a,L}^*I_{a,L}=\mathrm{id}_{\mathcal H_{a,L}}$ and $P_{a,L}:=I_{a,L}I_{a,L}^*$ the orthogonal projection onto $\operatorname{Ran}(I_{a,L})\subset\mathcal H$.

\emph{Step 2 (Graph--norm defect).} Define the defect $D_{a,L}:=H\,I_{a,L}-I_{a,L}\,H_{a,L}$. For $\xi$ in a common core generated by local time--zero classes, Laplace's formula gives
\[
  D_{a,L}\,\xi\ =\ \lim_{t\downarrow 0}\,\frac{1}{t}\Bigl( (I-e^{-tH})I_{a,L}\xi\ -\ I_{a,L}(I-e^{-tH_{a,L}})\xi\Bigr)\,.
\]
Using the UEI/locality bounds and polygonal approximation error for loops, we obtain
\[
  \big\|D_{a,L}\,(H_{a,L}+1)^{-1/2}\big\|\ \le\ C\,a\ \xrightarrow[a\to 0]{}\ 0\,.
\]

\emph{Step 3 (Resolvent comparison identity).} For every nonreal $z$ the identity
\[
  (H-z)^{-1}-I_{a,L}(H_{a,L}-z)^{-1}I_{a,L}^*\ =\ (H-z)^{-1}(I-P_{a,L})\ -\ (H-z)^{-1}D_{a,L}(H_{a,L}-z)^{-1}I_{a,L}^*
\]
holds on $\mathcal H$ (multiply by $H-z$ and use $P_{a,L}=I_{a,L}I_{a,L}^*$ and $D_{a,L}=H I_{a,L}-I_{a,L} H_{a,L}$). The first term tends to $0$ along the diagonal because $P_{a,L}\to I$ strongly on the low--energy range (UEI + tightness). The second tends to $0$ by the graph--defect bound. Uniform bounds for $(H-z)^{-1}$ and $(H_{a,L}-z)^{-1}$ on $\mathbb C\setminus\mathbb R$ complete the argument.
\end{proof}
\begin{lemma}[OS0 (temperedness) with explicit constants]\label{lem:os0-explicit-constants}
Assume uniform exponential clustering of truncated correlations: there exist $C_0\ge 1$ and $m>0$ such that for all $n\ge 2$, $\varepsilon\in(0,\varepsilon_0]$, and loops $\Gamma_{1,\varepsilon},\dots,\Gamma_{n,\varepsilon}$,
\[
  |\kappa_{n,\varepsilon}(\Gamma_{1,\varepsilon},\dots,\Gamma_{n,\varepsilon})|
   \ \le\ C_0^n\,\sum_{\text{trees }\tau}\ \prod_{(i,j)\in E(\tau)} e^{-m\,\operatorname{dist}(\Gamma_{i,\varepsilon},\Gamma_{j,\varepsilon})}.
\]
Fix any $q>d$ and set $p:=d+1$. Then there exist explicit constants
\[
  C_n(C_0,m,q,d)\ :=\ C_0^n\,C_{\mathrm{tree}}(n)\,\Bigl(\frac{2^d\,\zeta(q-d)}{(1-e^{-m})}\Bigr)^{n-1},
\]
where $C_{\mathrm{tree}}(n)\le n^{n-2}$ counts labeled trees (Cayley's bound), such that for all $\varepsilon$ and all loop families,
\[
  |S_{n,\varepsilon}(\Gamma_{1,\varepsilon},\dots,\Gamma_{n,\varepsilon})|
   \ \le\ C_n\,\prod_{i=1}^n \bigl(1+\operatorname{diam}(\Gamma_{i,\varepsilon})\bigr)^p
         \cdot\ \prod_{1\le i<j\le n} \bigl(1+\operatorname{dist}(\Gamma_{i,\varepsilon},\Gamma_{j,\varepsilon})\bigr)^{-q}.
\]
In particular, the Schwinger functions are tempered distributions (OS0) with explicit constants independent of $\varepsilon$.
\end{lemma}

\paragraph{KP $\Rightarrow$ OS0 constants (one-line bridge).}
From the KP window (C3/C4), take $C_0:=e^{C_*}\ge 1$ and $m:=\gamma_0=-\log\alpha_*>0$. Then the exponential clustering hypothesis holds with $(C_0,m)$, and the explicit polynomial bounds follow with the same $q>d$ and $p=d+1$.

\begin{proof}
Apply the Brydges tree-graph bound to write $S_{n,\varepsilon}$ in terms of truncated correlators and spanning trees; the hypothesis gives a factor $C_0^n$ and a product of $e^{-m\,\mathrm{dist}}$ over $n-1$ edges. Summing over tree shapes contributes $C_{\mathrm{tree}}(n)\le n^{n-2}$. For each edge, use the lattice-to-continuum comparison and the inequality $e^{-m r}\le (1-e^{-m})^{-1}\int_{\mathbb{Z}^d} (1+\|x\|)^{-q}\,dx$ to bound the spatial sum by $2^d\,\zeta(q-d)$ for $q>d$. Multiplying the $n-1$ edge factors yields the displayed $C_n(C_0,m,q,d)$. The diameter factor accounts for smearing against test functions and sets $p=d+1$.
\end{proof}

\section{Appendix: OS2 and OS3/OS5 preserved in the limit (C1b)}

We continue under the scaling window and assumptions of C1a, and additionally assume exponential clustering for $\mu_\varepsilon$ with constants $(C,c)$ independent of $\varepsilon$.

\begin{lemma}[OS2 preserved under limits]\label{lem:os2-limit}
Let $\{\mu_{\varepsilon_k}\}$ be a sequence of OS-positive measures (for a fixed link reflection) whose loop $n$-point functions converge along embeddings to Schwinger functions $\{S_n\}$. Then for any finite family $\{F_i\}$ of loop observables supported in $t\ge 0$ and coefficients $\{a_i\}$, one has
\[
  \sum_{i,j} \overline{a_i}\, a_j\, S_2\bigl(\Theta F_i, F_j\bigr)\;\ge\;0.
\]
Hence the limit Schwinger functions satisfy reflection positivity (OS2).
\end{lemma}

\begin{proof}[Proof]
Fix a finite family $\{F_i\}_{i=1}^m\subset\mathcal A_+$ and coefficients $a\in\mathbb C^m$. For each $\varepsilon$, choose approximants $F_{i,\varepsilon}\in\mathcal A_{\varepsilon,+}$ with $\|F_{i,\varepsilon}-F_i\|_{\mathrm{loc}}\le C\,d_H(\mathrm{supp}(F_{i,\varepsilon}),\mathrm{supp}(F_i))$ and $d_H\to 0$ along the directed embeddings; this is possible by locality and the directed-embedding construction. Define $G_{\varepsilon}:=\sum_i a_i F_{i,\varepsilon}$. By OS positivity at scale $\varepsilon_k$ (fixed link reflection),
\[
  \mathbb E_{\mu_{\varepsilon_k}}\bigl[\Theta G_{\varepsilon_k}\,\overline{G_{\varepsilon_k}}\bigr]\ \ge\ 0.
\]
Expand the left side using bilinearity:
\[
  \sum_{i,j} \overline{a_i} a_j\, \mathbb E_{\mu_{\varepsilon_k}}\bigl[\Theta F_{i,\varepsilon_k}\,\overline{F_{j,\varepsilon_k}}\bigr].
\]
By tightness and convergence (C1a) and equicontinuity of the approximants, for each fixed $(i,j)$,
\[
  \lim_{k\to\infty}\,\mathbb E_{\mu_{\varepsilon_k}}\bigl[\Theta F_{i,\varepsilon_k}\,\overline{F_{j,\varepsilon_k}}\bigr]
   \ =\ S_2\bigl(\Theta F_i, F_j\bigr).
\]
Dominated convergence (uniform moment bounds) justifies passing the limit through the finite sum, yielding
\[
  \lim_{k\to\infty}\,\mathbb E_{\mu_{\varepsilon_k}}\bigl[\Theta G_{\varepsilon_k}\,\overline{G_{\varepsilon_k}}\bigr]
   \ =\ \sum_{i,j} \overline{a_i} a_j\, S_2\bigl(\Theta F_i, F_j\bigr).
\]
Since each term on the left is $\ge 0$ and the limit of nonnegative numbers is nonnegative, the right-hand side is $\ge 0$. This proves OS2 for the limit.
\end{proof}

\paragraph{Lean artifact.}
The interface lemma for OS2 preservation under limits is exported as
\texttt{YM.OSPosWilson.reflection\_positivity\_preserved} in the file
\texttt{ym/os\_pos\_wilson/ReflectionPositivity.lean}, bundling the fixed link
reflection, lattice OS2, and convergence of Schwinger functions along
equivariant embeddings.

\begin{lemma}[OS3: clustering in the limit]\label{lem:os3-limit}
Assume exponential clustering holds uniformly on fixed slabs: there exist $C,c>0$ independent of $\varepsilon$ such that for any bounded, gauge--invariant local observables $A,B$ supported in a fixed region $R\subset\mathbb R^4$ and any separation vector with $\|\mathbf R\|\ge R$, one has $|\operatorname{Cov}_{\mu_\varepsilon}(A,\tau_{\mathbf R}B)|\le C e^{-cR}$. Then the limit Schwinger functions $\{S_n\}$ satisfy clustering: for translated observables,
\[
  \lim_{R\to\infty} S_2(A,B_R)\;=\;S_1(A)\,S_1(B).
\]
\end{lemma}

\begin{proof}
The uniform bound passes to the limit along the convergent subsequence. Taking $R\to\infty$ first at fixed $\varepsilon$ and then passing to the limit yields factorization; uniformity justifies exchanging limits.
\end{proof}

\paragraph{Lean artifacts.}
OS3 is exported as \texttt{YM.OSPositivity.clustering\_in\_limit} in
\texttt{ym/OSPositivity/ClusterUnique.lean} under a \texttt{ClusteringHypotheses}
bundle (uniform clustering and Schwinger convergence). OS5 is exported there as
\texttt{unique\_vacuum\_in\_limit} under a \texttt{UniqueVacuumHypotheses}
bundle (uniform gap and NRC).

\begin{lemma}[OS5: unique vacuum in the limit]\label{lem:os5-limit}
Suppose the transfer operators $T_{\varepsilon}$ (constructed via OS at each $\varepsilon$) have a uniform spectral gap on the mean-zero sector: $r_0(T_{\varepsilon})\le e^{-\gamma_0}$ with $\gamma_0>0$ independent of $\varepsilon$, and norm--resolvent convergence holds for the generators (C1c). Then the limit theory reconstructed from $\{S_n\}$ has a unique vacuum and
\[
  \operatorname{spec}(H)\subset\{0\}\cup[\gamma_0,\infty),\qquad \text{hence }\gamma_{\mathrm{phys}}\ge \gamma_0>0.
\]
\end{lemma}

\begin{proof}
For each $\varepsilon$, OS reconstruction gives a positive self-adjoint $H_{\varepsilon}\ge 0$ with $T_{\varepsilon}=e^{-H_{\varepsilon}}$ and $\operatorname{spec}(H_{\varepsilon})\subset\{0\}\cup[\gamma_0,\infty)$. By C1c, $(H-z)^{-1}-I_{\varepsilon}(H_{\varepsilon}-z)^{-1}I_{\varepsilon}^*$ converges to $0$ for all nonreal $z$. Spectral convergence (Hausdorff) carries the open gap $(0,\gamma_0)$ to the limit: $\operatorname{spec}(H)\cap(0,\gamma_0)=\varnothing$. Since $H\ge 0$, the bottom of the spectrum is $0$; OS clustering implies that the $0$ eigenspace is one-dimensional (no degeneracy of the vacuum). Therefore the continuum theory has a unique vacuum and a mass gap $\ge \gamma_0$.
\end{proof}

\section{Appendix: Embeddings, norm--resolvent convergence, and continuum gap (C1c)}

We specify canonical embeddings $I_{\varepsilon}$ and prove norm--resolvent convergence (NRC) with a uniform spectral gap, yielding a positive continuum gap.

\paragraph{Embeddings (explicit OS/GNS construction).}
Let $\mathfrak A_{\varepsilon,+}$ be the $*$–algebra of lattice cylinder observables supported in $t\ge 0$, and $\mathfrak A_+$ its continuum analogue. For a lattice loop $\Lambda\subset\varepsilon\,\mathbb{Z}^4$, let $\operatorname{poly}(\Lambda)$ be its polygonal interpolation (rectilinear embedding) in $\mathbb R^4$. Define a $*$–homomorphism on generators $E_{\varepsilon}:\mathfrak A_{\varepsilon,+}\to\mathfrak A_+$ by
\[
  E_{\varepsilon}\bigl(W_{\Lambda}\bigr)\ :=\ W_{\operatorname{poly}(\Lambda)},\qquad E_{\varepsilon}(1)=1,\quad E_{\varepsilon}(FG)=E_{\varepsilon}(F)E_{\varepsilon}(G),\ E_{\varepsilon}(F^*)=E_{\varepsilon}(F)^*.
\]
On the OS/GNS spaces $\mathcal H_{\varepsilon}$ and $\mathcal H$ (quotients by OS–nulls and completion), define
\[
  I_{\varepsilon}:[F]_{\varepsilon}\mapsto [\,E_{\varepsilon}(F)\,],\qquad R_{\varepsilon}:\mathcal H\to\mathcal H_{\varepsilon}\ \text{ the adjoint of }I_{\varepsilon}.
\]
By construction and OS positivity, $I_{\varepsilon}^*I_{\varepsilon}=\mathrm{id}_{\mathcal H_{\varepsilon}}$ and $P_{\varepsilon}:=I_{\varepsilon}I_{\varepsilon}^*$ is the orthogonal projection onto $\operatorname{Ran}(I_{\varepsilon})\subset\mathcal H$. Concretely, on local classes $[F]$ one has. In Lean, the NRC hypotheses bundle is exported as `YM.SpectralStability.NRCHypotheses`, and the container for the identity below is `YM.SpectralStability.NRCSetup`.
\[
  \langle [G]_{\varepsilon}, R_{\varepsilon}[F]\rangle_{\varepsilon}\ =\ \langle I_{\varepsilon}[G]_{\varepsilon}, [F]\rangle\ =\ S_2\bigl(\Theta E_{\varepsilon}(G), F\bigr).
\]
\paragraph{Generators.}
Let $T_{\varepsilon}$ be the transfer operator at scale $\varepsilon$, $H_{\varepsilon}:=-\log T_{\varepsilon}\ge 0$ on the mean-zero subspace $\mathcal H_{\varepsilon,0}$. Let $T$ be the transfer of the limit theory (via OS reconstruction), $H:=-\log T\ge 0$ on $\mathcal H_0$.
\paragraph{Consistency and compact calibrator.}
Assume:
\begin{itemize}
  \item (Cons) The defect operators $D_{\varepsilon}:=H I_{\varepsilon}-I_{\varepsilon} H_{\varepsilon}$ satisfy $\varepsilon$-scale graph-norm control: $\|D_{\varepsilon}(H_{\varepsilon}+1)^{-1/2}\|\to 0$.
  \item (Comp) For some nonreal $z_0$, $(H-z_0)^{-1}$ is compact (e.g., finite volume or confining setting).
\end{itemize}

\begin{lemma}[Semigroup comparison implies graph–norm defect]\label{lem:semigroup-defect}
Suppose there is $C>0$ such that for all $t\in[0,1]$,
\[
  \bigl\|e^{-tH}-I_{\varepsilon}e^{-tH_{\varepsilon}}I_{\varepsilon}^*\bigr\|\ \le\ C t\,\varepsilon\ +\ o(\varepsilon).
\]
Then $\|\,(H I_{\varepsilon}-I_{\varepsilon} H_{\varepsilon})(H_{\varepsilon}+1)^{-1/2}\,\|\to 0$ as $\varepsilon\downarrow 0$.
\end{lemma}

\begin{proof}
Use the standard characterization of generators via Laplace transform of the semigroup and the Hille–Yosida graph–norm: for $\xi\in\operatorname{dom}(H_{\varepsilon})$,
\[
  (H I_{\varepsilon}-I_{\varepsilon} H_{\varepsilon})\xi\ =\ \lim_{t\downarrow 0}\,\frac{1}{t}\bigl[\,(I-e^{-tH})I_{\varepsilon}\xi\ -\ I_{\varepsilon}(I-e^{-tH_{\varepsilon}})\xi\,\bigr],
\]
and bound the difference by the semigroup comparison. The $(H_{\varepsilon}+1)^{-1/2}$ factor stabilizes the domain.
\end{proof}
\paragraph{Resolvent comparison identity (Lean NRC container).}
Let $R(z)=(H-z)^{-1}$, $R_{\varepsilon}(z)=(H_{\varepsilon}-z)^{-1}$, $I_{\varepsilon}$ the embedding and $P_{\varepsilon}:=I_{\varepsilon}I_{\varepsilon}^*$. Define the defect $D_{\varepsilon}:=H I_{\varepsilon}-I_{\varepsilon}H_{\varepsilon}$. Then for each nonreal $z$,
\[
  R(z) - I_{\varepsilon} R_{\varepsilon}(z) I_{\varepsilon}^*
  \ =\ R(z)(I-P_{\varepsilon})\ -\ R(z) D_{\varepsilon} R_{\varepsilon}(z) I_{\varepsilon}^*\,.
\]
This is implemented as a reusable container in the Lean module
\texttt{ym/SpectralStability/NRCEps.lean} as \texttt{NRCSetup.comparison}. The named NRC interface theorem is \leanref{YM.SpectralStability.NRC_all_nonreal}.
\begin{lemma}[Compact calibrator in finite volume]\label{lem:compact-calibrator}
On finite 4D tori (periodic boundary conditions), the transfer $T$ is a compact self--adjoint operator on the OS/GNS space. Hence $(H-z_0)^{-1}$ is compact for any nonreal $z_0$.
\end{lemma}
\begin{proof}
Finite volume yields a separable OS/GNS space with $T$ acting by a positivity–preserving integral kernel on a compact set; standard Hilbert–Schmidt bounds imply compactness of $T$ and thus of the resolvent of $H=-\log T$.
\end{proof}
\paragraph{Calibrator via finite–volume exhaustion (infinite volume).}
Let $\Lambda_L$ be an increasing sequence of periodic 4D tori exhausting $\mathbb R^4$, with transfers $T_L$ and generators $H_L:=-\log T_L$. By the preceding lemma, $(H_L-z_0)^{-1}$ is compact for each $L$. Assume the embeddings $I_{\varepsilon,L}$ and defects $D_{\varepsilon,L}:=H I_{\varepsilon,L}-I_{\varepsilon,L} H_{\varepsilon,L}$ satisfy the graph–norm control uniformly in $L$ and $\varepsilon$:
\[
  \sup_L\big\| D_{\varepsilon,L} (H_{\varepsilon,L}+1)^{-1/2}\big\|\;\xrightarrow[\ \varepsilon\downarrow 0\ ]{}\;0,
\]
and that the projections $P_{\varepsilon,L}:=I_{\varepsilon,L} I_{\varepsilon,L}^*$ converge strongly to $I$ on the infinite–volume OS/GNS space as $L\to\infty$ (for each fixed $\varepsilon$), with this convergence uniform on the low–energy range of $H$. Then the R3 comparison identity yields NRC at each finite $L$; letting $L\to\infty$ and using the thermodynamic–limit compactness of local observables (cf. Theorem~\ref{thm:thermo-strong} and \S\,\ref{sec:lattice-setup}) one obtains NRC in infinite volume.

\begin{theorem}[NRC via finite–volume exhaustion]\label{thm:nrc-exhaustion}
Assume (Cons) (graph–norm defect) with bounds uniform in $L$, the strong convergence $P_{\varepsilon,L}\to I$ on the low–energy range of $H$ for each fixed $\varepsilon$, and the fixed–spacing thermodynamic–limit hypotheses of Theorem~\ref{thm:thermo-strong}. Then for every $z\in\mathbb C\setminus\mathbb R$,
\[
  \big\|(H-z)^{-1}-I_{\varepsilon}(H_{\varepsilon}-z)^{-1}I_{\varepsilon}^*\big\|\;\xrightarrow[\ \varepsilon\downarrow 0\ ]{}\;0,
\]
where $I_{\varepsilon}$ is the infinite–volume embedding obtained as the strong limit of $I_{\varepsilon,L}$ along the exhaustion. In particular, NRC holds in infinite volume for all nonreal $z$.
\end{theorem}

\begin{theorem}[NRC and continuum gap]\label{thm:nrc-gap}
Suppose (Cons) and (Comp) hold, and the discrete transfer operators have an $\varepsilon$-uniform spectral gap on mean-zero subspaces:
\[
  r_0(T_{\varepsilon})\;\le\;e^{-\gamma_0}\quad\text{with}\quad \gamma_0>0\ \text{independent of }\varepsilon.
\]
Then:
\begin{itemize}
  \item (NRC) For every $z\in\mathbb C\setminus\mathbb R$,
  \[
    \bigl\|(H-z)^{-1}-I_{\varepsilon}(H_{\varepsilon}-z)^{-1}I_{\varepsilon}^*\bigr\|\to 0\quad(\varepsilon\to 0).
  \]
  \item (Continuum gap) On $\mathcal H_0$, $\operatorname{spec}(H)\subset\{0\}\cup[\gamma_0,\infty)$, hence the continuum Hamiltonian has a positive gap $\ge \gamma_0$ and a unique vacuum.
\end{itemize}
\end{theorem}
\begin{proof}
The NRC follows from the comparison identity and bounds of Appendix R3 with $I_{\varepsilon},P_{\varepsilon}$ and the defect control (Cons), plus compact calibration (Comp) to isolate low energies. The uniform spectral gap for $T_{\varepsilon}$ implies a uniform open gap $(0,\gamma_0)$ for $H_{\varepsilon}$. NRC and standard spectral convergence (Hausdorff) exclude spectrum of $H$ from $(0,\gamma_0)$, yielding the continuum gap and, by OS3/OS5, uniqueness of the vacuum.
\end{proof}
\paragraph{Lean artifacts.}
The resolvent comparison is encoded in \texttt{ym/SpectralStability/NRCEps.lean} as an \emph{NRCSetup} with a field \texttt{comparison} that equals the identity above. A norm bound for the NRC difference from this identity is provided in \texttt{ym/SpectralStability/Persistence.lean} (theorem \texttt{nrc\_norm\_bound}). The spectral lower-bound persistence statement is exported there as \texttt{persistence\_lower\_bound} for downstream use.

\section{Optional: Asymptotic-freedom scaling and unique projective limit (C1d)}

We now specify an \emph{asymptotic-freedom (AF) scaling schedule} $\beta(a)$ and prove that along this schedule the projective limit on $\mathbb R^4$ exists with OS0--OS5, is \emph{unique} (no subsequences), and that NRC transports the same uniform lattice gap $\gamma_0$ to the continuum Hamiltonian.

\paragraph{AF schedule.}
Fix $a_0>0$. Choose a monotone function $\beta:(0,a_0]\to (0,\infty)$ such that
\begin{itemize}
  \item[(AF1)] $\beta(a)\ge \beta_{\min}>0$ for all $a\in(0,a_0]$ and $\beta(a)\xrightarrow[a\downarrow 0]{}\infty$;
  \item[(AF2)] choose van Hove volumes $L(a)$ with $L(a)\,a\xrightarrow[a\downarrow 0]{}\infty$;
  \item[(AF3)] use the polygonal loop embeddings $E_a$ and OS/GNS isometries $I_a$ of C1c;
  \item[(AF4)] fix the link-reflection and slab thickness bounded by $a\le a_0$ so that the Doeblin constants $(\kappa_0,t_0)$ are uniform (Prop.~\ref{prop:doeblin-interface}).
\end{itemize}
An explicit example is $\beta(a)=\beta_{\min}+c_0\log(1+a_0/a)$ with $c_0>0$.
\paragraph{Uniform gap along AF.}
By the Doeblin minorization and heat--kernel domination on the interface, the one--step odd-cone deficit is volume-uniform on fixed slabs (with $\theta_*(\beta)$ entering the weight):
\[
  c_{\rm cut}\ \ge\ -\frac{1}{a}\log\big(1-\kappa_0(1-e^{-\lambda_1(G) t_0})\big),\qquad
  \gamma_0\ \ge\ 8\,c_{\mathrm{cut}}\ >\ 0,
\]
uniform in $a\in(0,a_0]$, volume $L(a)$, and $N\ge 2$.
\paragraph{Existence (OS0--OS5) and uniqueness (no subsequences).}
Let $\mu_{a}:=\mu_{\beta(a),L(a)}$ denote the lattice Wilson measures. Then:
\begin{itemize}
  \item OS0/OS2 persist under limits by UEI and positivity closure (C1a/C1b).
  \item OS1 holds in the limit by oriented diagonalization and equicontinuity (C1a).
  \item OS3 holds uniformly on the lattice by the uniform gap $\gamma_0$; it passes to the limit by C1b. OS5 (unique vacuum) follows likewise.
\end{itemize}
To remove subsequences, define for nonreal $z$ the \emph{embedded resolvents}
\[
  R_a(z)\ :=\ I_a\,(H_a-z)^{-1}\,I_a^*\,.
\]
From the comparison identity of R3 and the graph-defect bound $\|D_a(H_a+1)^{-1/2}\|\le C a$ one obtains the quantitative estimate
\begin{lemma}[Cauchy estimate for embedded resolvents]\label{lem:cauchy-res}
For any fixed nonreal $z$, there exists $C(z)>0$ such that for all $a,b\in(0,a_0]$,
\[
  \big\|R_a(z)-R_b(z)\big\|\ \le\ C(z)\,(a+b).
\]
\end{lemma}
\begin{proof}
By the resolvent comparison identity (Appendix R3) and the graph-defect bounds $\|D_a(H_a+1)^{-1/2}\|\le C a$, $\|D_b(H_b+1)^{-1/2}\|\le C b$, together with $\|(H_a-z)^{-1}(H_a+1)^{1/2}\|\le C'(z)$ uniformly in $a$, we obtain
\[
  \|R(z)-R_a(z)\|\le C_1(z) a,\qquad \|R(z)-R_b(z)\|\le C_1(z) b.
\]
The triangle inequality yields $\|R_a(z)-R_b(z)\|\le C(z)(a+b)$ with $C(z):=2C_1(z)$.
\end{proof}
\noindent\emph{Remark.} Lemma~\ref{lem:cauchy-res} shows $\{R_a(z)\}_{a\downarrow 0}$ is Cauchy in operator norm for each nonreal $z$, so the limit $R(z)$ exists without passing to subsequences; this is the uniqueness mechanism used below.
Hence $\{R_a(z)\}_{a\downarrow 0}$ is a Cauchy net in operator norm for each nonreal $z$, converging to a \emph{unique} bounded operator $R(z)$ that satisfies the resolvent identities. By the analytic Hille--Phillips theory, $R(z)$ is the resolvent of a unique nonnegative self-adjoint $H$; the embedded semigroups $I_a e^{-tH_a} I_a^*$ converge in operator norm to $e^{-tH}$ for all $t\ge 0$. Therefore the Schwinger functions of $\mu_a$ converge to a unique limit $\{S_n\}$ (no subsequences), defining a probability measure $\mu$ on loop configurations over $\mathbb R^4$ which satisfies OS0--OS5.
\paragraph{AF schedule theorem.}
\begin{theorem}[AF schedule $\Rightarrow$ unique continuum YM with gap]\label{thm:af-schedule-gap}
Under (AF1)--(AF4), the projective limit measure $\mu$ on $\mathbb R^4$ exists and is unique. Its Schwinger functions satisfy OS0--OS5, and the OS reconstruction yields a Hilbert space $\mathcal H$, a vacuum $\Omega$, and a positive self-adjoint generator $H\ge 0$ with
\[
  \operatorname{spec}(H)\subset\{0\}\cup[\gamma_0,\infty),\qquad \gamma_{\mathrm{phys}}\ge \gamma_0>0\,.
\]
\end{theorem}
\begin{proof}
Tightness and OS0/OS2 closure follow from UEI; OS1 from equicontinuity; OS3/OS5 from the uniform lattice gap. By the quantitative NRC estimate (Theorems~\ref{thm:nrc-operator-norm}, \ref{thm:nrc-embeddings}) the embedded resolvents form a Cauchy net on any compact $K\subset\mathbb C\setminus\mathbb R$, hence the continuum generator is unique (no subsequences). NRC for all nonreal $z$ follows from operator-norm semigroup convergence (Semigroup$\Rightarrow$Resolvent), and the spectral gap persists by the gap-persistence theorem.
\end{proof}
\section{Appendix: Continuum area law via directed embeddings (C2; one-way consequences only)}

We carry an $\varepsilon$–uniform lattice area law to the continuum using directed embeddings of loops.

\paragraph{Uniform lattice area law.}
Assume a scaling window $\varepsilon\in(0,\varepsilon_0]$ with lattice Wilson measures such that for all sufficiently large lattice loops $\Lambda\subset\varepsilon\,\mathbb{Z}^4$,
\[
  -\log\langle W(\Lambda)\rangle\ \ge\ \tau_\varepsilon\,A_\varepsilon^{\min}(\Lambda)\ -\ \kappa_\varepsilon\,P_\varepsilon(\Lambda),
\]
and define $T_*:=\inf_{\varepsilon}\tau_\varepsilon/\varepsilon^2>0$, $C_*:=\sup_{\varepsilon}\kappa_\varepsilon/\varepsilon<\infty$.
\paragraph{Directed embeddings.}
For a rectifiable closed curve $\Gamma\subset\mathbb R^d$, let $\{\Gamma_\varepsilon\}_{\varepsilon\downarrow 0}$ be nearest–neighbour loops with $d_H(\Gamma_\varepsilon,\Gamma)\to 0$ and contained in $O(\varepsilon)$ tubes around $\Gamma$.

\begin{theorem}[Continuum Area–Perimeter bound]\label{thm:continuum-area-perimeter}
With $\kappa_d:=\sup_{u\in\mathbb S^{d-1}}\sum_i |u_i|=\sqrt d$ and $C:=\kappa_d C_*$, for any directed family $\Gamma_\varepsilon\to\Gamma$,
\[
  \limsup_{\varepsilon\downarrow 0}\bigl[-\log\langle W(\Gamma_\varepsilon)\rangle\bigr]\ \ge\ T_*\,\operatorname{Area}(\Gamma)\ -\ C\,\operatorname{Perimeter}(\Gamma).
\]
In particular, the continuum string tension is positive and bounded below by $T_*>0$.
\end{theorem}

\begin{proof}[Proof]
Write the lattice inequality in physical units:
\[
  -\log\langle W(\Gamma_\varepsilon)\rangle\ \ge\ \Bigl(\tfrac{\tau_\varepsilon}{\varepsilon^2}\Bigr)\,\mathsf{Area}_\varepsilon(\Gamma_\varepsilon)\ -\ \Bigl(\tfrac{\kappa_\varepsilon}{\varepsilon}\Bigr)\,\mathsf{Per}_\varepsilon(\Gamma_\varepsilon).
\]
Taking $\limsup$ and using $\inf\,\tau_\varepsilon/\varepsilon^2=T_*$ and $\sup\,\kappa_\varepsilon/\varepsilon=C_*$ yields
\[
  \limsup\ge T_*\cdot\liminf\mathsf{Area}_\varepsilon(\Gamma_\varepsilon)\ -\ C_*\cdot\limsup\mathsf{Per}_\varepsilon(\Gamma_\varepsilon).
\]
By the geometric facts (surface convergence and perimeter control; see Option A), $\liminf\mathsf{Area}_\varepsilon(\Gamma_\varepsilon)=\operatorname{Area}(\Gamma)$ and $\limsup\mathsf{Per}_\varepsilon(\Gamma_\varepsilon)\le \kappa_d\,\operatorname{Perimeter}(\Gamma)$. Combine to obtain the stated bound with $C=\kappa_d C_*$.\qed
\end{proof}

\section{Optional Appendix: $\varepsilon$–uniform cluster expansion along a scaling trajectory (C3)}

\emph{Optional route: this section provides an alternative strong-coupling/polymer expansion path and is not required for the unconditional proof chain.}

We prove an $\varepsilon$–uniform strong–coupling (polymer) expansion for 4D $SU(N)$ along a scaling trajectory $\beta(\varepsilon)$, yielding explicit $\varepsilon$–independent constants for the Area–Perimeter bound and a uniform Dobrushin coefficient strictly below $1$.

\paragraph{Set–up.}
Work on 4D tori with lattice spacing $\varepsilon\in(0,\varepsilon_0]$. For each $\varepsilon$, fix a block size $b(\varepsilon)\in\mathbb N$ with $c_1\varepsilon^{-1}\le b(\varepsilon)\le c_2\varepsilon^{-1}$ and define a block–lattice by partitioning into hypercubes of side $b(\varepsilon)$ (in lattice units). Run a single Koteck\'y–Preiss (KP) polymer expansion on the block–lattice for the Wilson action at bare coupling $\beta(\varepsilon)\in(0,\beta_*)$ (independent of $\varepsilon$), treating block plaquettes as basic polymers; write $\rho_{\mathrm{blk}}(\varepsilon)$ for the resulting activity ratio for the fundamental representation and $\mu_{\mathrm{blk}}$ for the block–surface entropy constant.

\paragraph{Uniform KP/cluster expansion (full proof).}
Fix $\varepsilon\in(0,\varepsilon_0]$ and choose a block scale $b(\varepsilon)\asymp \varepsilon^{-1}$. Group plaquettes into block–plaquettes (faces of side $b(\varepsilon)$ in lattice units). Expand the Wilson weight on each block–plaquette in irreducible characters and polymerize along block–faces. Koteck\'y–Preiss applies provided the activity $\rho_{\mathrm{blk}}(\varepsilon)$ of the fundamental representation and the block entropy $\mu_{\mathrm{blk}}$ satisfy $\mu_{\mathrm{blk}}\,\rho_{\mathrm{blk}}(\varepsilon) < 1$; for small $\beta(\varepsilon)$ this holds uniformly with a slack $\delta\in(0,1)$ independent of $\varepsilon$ and $N\ge2$. Boundary attachments contribute a multiplicity factor $m_{\mathrm{blk}}$ per block boundary unit (uniform in $\varepsilon,N$). Summing over excess block area $k\ge0$ yields the convergent geometric series
\[
  \sum_{k\ge 0} N_{\mathrm{blk}}(\Gamma,A+k)\,\rho_{\mathrm{blk}}(\varepsilon)^{A+k}
   \ \le\ m_{\mathrm{blk}}^{P_{\mathrm{blk}}}\,\frac{\rho_{\mathrm{blk}}(\varepsilon)^{A}}{\delta},
\]
where $A$ is the minimal block spanning area and $P_{\mathrm{blk}}$ the block perimeter. Taking $-\log$ and converting to physical units (each block area $\asymp 1$, each block boundary length $\asymp 1$) gives
\[
  -\log\langle W(\Lambda)\rangle\ \ge\ T_*\,\mathsf{Area}_\varepsilon(\Lambda)\ -\ C_*\,\mathsf{Per}_\varepsilon(\Lambda),
\]
with
\[
  T_*:= -\log \rho_{\max},\quad \rho_{\max}:=\sup_{0<\varepsilon\le\varepsilon_0}\rho_{\mathrm{blk}}(\varepsilon)<1,\quad
  C_*:= \log m_{\mathrm{blk}}+\log(1/\delta)<\infty.
\]
Moreover, the one--step cross--cut Dobrushin coefficient at block scale obeys
\[
  \alpha\bigl(\beta(\varepsilon)\bigr)\ \le\ 2\,\beta(\varepsilon)\,J^{\mathrm{blk}}_{\perp}(\varepsilon)
   \ \le\ 2\,\beta_*\,J^{\mathrm{blk}}_{\perp,\max}=:\alpha_*<1,
\]
where $J^{\mathrm{blk}}_{\perp,\max}$ is a geometry–only bound (independent of $\varepsilon,N$). All constants are $\varepsilon$– and $N$–uniform.
\paragraph{Optional scaffold (KP from Wilson; hypothesis bundle).}
\emph{(H-KP).} For 4D SU($N$) Wilson action at sufficiently small $\beta$, the block polymer expansion at scale $b(\varepsilon)\asymp \varepsilon^{-1}$ satisfies: (i) $\rho_{\mathrm{blk}}(\varepsilon)\le \rho_{\max}<1$, (ii) $\mu_{\mathrm{blk}}\,\rho_{\mathrm{blk}}\le 1-\delta$ with $\delta\in(0,1)$, (iii) boundary multiplicity $m_{\mathrm{blk}}\le m_0$, all independent of $\varepsilon$ and $N$. \emph{Conclusion.} The constants $T_*=-\log\rho_{\max}>0$, $C_* = \log m_0 + \log(1/\delta)$, and $\alpha_*=2\beta_* J^{\mathrm{blk}}_{\perp,\max}<1$ follow, yielding the uniform area–perimeter law and contraction.

\begin{theorem}[Uniform KP/cluster expansion with explicit constants]
\label{thm:uniform-kp}
Under the hypotheses above, define the explicit $\varepsilon$–independent constants
\[
  \rho_{\max}\;:=\;\sup_{0<\varepsilon\le \varepsilon_0}\rho_{\mathrm{blk}}(\varepsilon)\ <\ 1,\quad
  T_*\;:=\; -\log \rho_{\max}\ >\ 0,\quad
  C_*\;:=\; \log m_{\mathrm{blk}}\ +\ \log\tfrac{1}{\delta}\ <\ \infty,
\]
\[
  J^{\mathrm{blk}}_{\perp,\max}\;:=\;\sup_{0<\varepsilon\le\varepsilon_0} J^{\mathrm{blk}}_{\perp}(\varepsilon)\ <\ \infty,\qquad
  \alpha_*\;:=\;2\,\beta_*\,J^{\mathrm{blk}}_{\perp,\max}\ <\ 1\,.
\]
Then for all sufficiently large loops $\Lambda\subset\varepsilon\,\mathbb Z^4$ and all $\varepsilon\in(0,\varepsilon_0]$:
\begin{align}
  -\log\langle W(\Lambda)\rangle\ &\ge\ \tau_\varepsilon\,A_\varepsilon^{\min}(\Lambda)\ -\ \kappa_\varepsilon\,P_\varepsilon(\Lambda),\\
  \frac{\tau_\varepsilon}{\varepsilon^2}\ &\ge\ T_*,\qquad \frac{\kappa_\varepsilon}{\varepsilon}\ \le\ C_*,\\
  \alpha\bigl(\beta(\varepsilon)\bigr)\ &\le\ \alpha_* <\ 1\,.
\end{align}
In particular, $T_*$ is a uniform string–tension lower bound in physical units, $C_*$ a uniform perimeter coefficient (physical units), and $\alpha_*$ a uniform upper bound for the cross–cut Dobrushin coefficient.
\end{theorem}
 

\begin{theorem}[Local gauge–invariant fields]\label{thm:local-fields-exist}
There exists a collection of operator–valued tempered distributions $\{\mathcal E(f)\}_{f\in \mathcal S(\mathbb R^4)}$ on the OS/GNS Hilbert space such that for compactly supported smooth $f$, $\mathcal E(f)$ is the $L^2$–limit of $\mathcal E^{(a)}(f)$ along the scaling window. For finite families $\{f_i\}$ and any polynomial $P$, the mixed Schwinger functions of $\{\mathcal E(f_i)\}$ arise as limits of those of $\{\mathcal E^{(a)}(f_i)\}$ and satisfy OS0–OS2 with the explicit constants from Cor.~\ref{cor:os0-explicit-4d}. The fields are Euclidean covariant (OS1) by Cor.~\ref{cor:os1-rotations}.
\end{theorem}

\begin{corollary}[OS\,$\to$\,Wightman with local fields and gap]\label{cor:wightman-local-gap}
Let $H\ge 0$ be the generator reconstructed from the continuum Schwinger functions including the local field sector of Theorem~\ref{thm:local-fields-exist}. If $\operatorname{spec}(H)\subset\{0\}\cup[\gamma_*,\infty)$ with $\gamma_*>0$ (Theorem~\ref{thm:pf-gap-meanzero}), then the OS reconstruction yields Wightman local fields (smeared) $\mathcal E_M(\varphi)$ on Minkowski space with the same mass gap:
\[
  \sigma(H_{\rm Mink})\ \subset\ \{0\}\cup[\gamma_*,\infty).
\]
\end{corollary}
\paragraph{Anchors (T14 Local fields) [ANCHOR\_T14\_v1].}
\begin{itemize}
  \item CloverApproximation: loop nets converge to field smearings.
  \item TemperednessTransfer: OS0 bounds transfer to fields.
  \item ReflectionPositivityTransfer: OS2 for fields via cylinder-set limits.
  \item LocalityFields: disjoint supports $\Rightarrow$ commutativity/locality.
  \item GapVacuumPersistence: same $H$ $\Rightarrow$ gap/vacuum persist.
\end{itemize}

\paragraph{Anchors (T15 Time normalization and gap) [ANCHOR\_T15\_v1].}
\begin{itemize}
  \item PerTickContraction: odd-cone one-step factor $(1-\theta_*(1-e^{-\lambda_1 t_0}))^{1/2}$.
  \item EightTickComposition: $\gamma_{\rm cut}(a)=8\,c_{\rm cut}(a)$.
  \item PhysicalNormalization: $\tau_{\rm phys}=a$ $\Rightarrow$ $\gamma_{\rm phys}=8\big(-\log(1-\theta_*(1-e^{-\lambda_1 t_0}))\big)$.
  \item ContinuumPersistence: rescaled NRC keeps $(0,\gamma_{\rm phys})$ spectrum–free.
\end{itemize}

\medskip

\section{Appendix U: AF--free inputs and continuum limit (hypotheses U1--U4)}

\subsection*{Referee checklist (Clay requirements $\to$ labels)}
\begin{itemize}
  \item Scaling schedule, van Hove volumes: Def.~\ref{def:U0-schedule} (U0).
  \item UEI/LSI on fixed regions (uniform in $a$): Thm.~\ref{thm:U1-lsi-uei}, Lem.~\ref{lem:U1-tree-bounds}, Cor.~\ref{cor:U1-uei} (U1).
  \item OS/GNS embeddings $I_a$ (isometries, domains): Lem.~\ref{lem:U2-embeddings} (U2a).
  \item Comparison identity and NRC (all nonreal $z$): Lem.~\ref{lem:U2-comparison}, Prop.~\ref{prop:collective-compactness}, Thm.~\ref{thm:nrc-quant} (U2a/U2c).
  \item Graph-defect bound $\|D_a(H_a+1)^{-1/2}\|=O(a)$: Thm.~\ref{thm:quant-calibrated-af-free-nrc}(D) (U2b).
  \item Low-energy projection control $\delta_a(\Lambda)\le C_\Lambda a$: Lem.~\ref{lem:low-energy-proj} (U2b).
  \item Cauchy resolvent criterion, uniqueness (no subsequences): Lem.~\ref{lem:af-free-cauchy} or Lem.~\ref{lem:cauchy-resolvent-unique} (U2c).
  \item Interface Doeblin minorization (independent of $\beta,L$): Lem.~\ref{lem:abs-cont}, Lem.~\ref{lem:coarse-refresh}, Lem.~\ref{lem:coarse-hk-domination}, Prop.~\ref{prop:coarse-doeblin} (U3).
  \item Odd-cone Gram/mixed bounds and Gershgorin margin: Thm.~\ref{thm:uniform-odd-contraction}, Lem.~\ref{lem:local-gram-bounds}, Prop.~\ref{prop:psd-crossing-gram}, Thm.~\ref{thm:U12-exp-cluster} (U4).
  \item OS axioms in the continuum (OS0–OS5): Prop.~\ref{prop:os0os2-closure}, Prop.~\ref{prop:os35-limit}, Thm.~\ref{thm:os1-unconditional} (U7).
  \item Local gauge-invariant fields and non-Gaussianity: Thm.~\ref{thm:local-fields-exist}, Prop.~\ref{prop:nonzero-cumulant4} (U7).
  \item OS4 (permutation symmetry) explicit: Prop.~\ref{prop:U11-os4}.
  \item Exponential clustering in continuum: Thm.~\ref{thm:U12-exp-cluster}.
\end{itemize}

\subsection{U8. Ward/Schwinger–Dyson identities and continuum Ward theorem}
\begin{lemma}[Lattice BRST/finite-gauge Ward identities]\label{lem:lattice-brst-ward}
For the Wilson action on a finite periodic 4D torus and gauge group $G=\mathrm{SU}(N)$, the Schwinger functions of Wilson loops and of the local clover field $\Xi^{(a)}_{\mu\nu}$ satisfy the nonabelian lattice Ward/Schwinger–Dyson identities under (i) finite local gauge variations and (ii) BRST-exact insertions. These identities hold for every lattice spacing $a$ and volume $L$.
\end{lemma}
\begin{proof}
Let $g: \Lambda^0\to G$ be a lattice gauge transformation acting on links by $U_{x,\mu}\mapsto g_x U_{x,\mu} g_{x+\hat\mu}^{-1}$. The Wilson action $S_\beta(U)$ is gauge invariant and the product Haar measure $d\mu_\beta(U)\propto e^{-\beta S_\beta(U)}\prod dU$ is left/right invariant. For any cylinder functional $F$ built from Wilson loops and clover fields, the change of variables $U\mapsto g\cdot U$ yields
\[
  \int F(U)\,d\mu_\beta(U)\ =\ \int F(g\cdot U)\,d\mu_\beta(U)
\]
for all $g$. Differentiating along a one-parameter subgroup $g_x(t)=\exp(t X_x)$ with $X_x\in\mathfrak{su}(N)$, and using that the derivative of $F(g\cdot U)$ at $t=0$ is a sum of left/right invariant vector fields acting on link variables at the endpoints of the affected loops/plaquettes, one obtains the lattice Schwinger–Dyson identity
\[
  \sum_{x}\Big\langle \delta_x F\Big\rangle_{\beta,a,L}\ =\ 0,
\]
where $\delta_x$ is the gauge-variation derivation at site $x$ acting by Lie derivatives on adjacent links. BRST versions follow by introducing standard gauge-fixing/ghost terms and using invariance of the BRST-extended measure; BRST-exact insertions integrate to zero. Periodic boundary conditions ensure that all boundary terms vanish.
\end{proof}
\begin{theorem}[Continuum nonabelian Ward identities]\label{thm:U8-ward-cont}
Along any van Hove sequence with $a\downarrow 0$, the embedded Schwinger functions of Wilson loops and of the renormalized local field $\Xi_R$ satisfy the continuum nonabelian Ward (Schwinger–Dyson) identities of Yang–Mills. Hence the OS/Wightman limit is gauge invariant and satisfies the YM Ward relations.
\end{theorem}
\begin{proof}
Fix finitely many loop/field insertions supported in a fixed region $R\Subset\mathbb R^4$. By Theorem~\ref{thm:U1-lsi-uei}, UEI gives uniform integrability bounds for the Ward functionals. The lattice Ward identity holds at each $(a,L)$ by the lemma. Embed the lattice observables to the continuum cylinder algebra and apply U2 operator-norm NRC (Theorem~\ref{thm:U2-nrc-unique}) to pass to the unique limit of Schwinger functions; dominated convergence yields the limit identity. For local fields, use U10 to replace $\Xi^{(a)}$ by the renormalized $\Xi^{(a)}_R=Z_F(a)\,\Xi^{(a)}$, with $Z_F(a)$ bounded as a consequence of UEI/LSI on fixed regions (Theorem~\ref{thm:U10-renorm-F}), and pass to the limit.
\end{proof}
\subsection{U9. Gauss law and the physical Hilbert subspace}
\begin{theorem}[Gauss constraint and physical subspace]\label{thm:U9-gauss}
Let $\mathcal H_{\rm phys}$ be the gauge-invariant OS/GNS subspace (closure of vectors generated by gauge-invariant time-zero observables). Then: (i) the lattice Gauss constraints hold on $\mathcal H_{a,L}^{\rm phys}$; (ii) the embeddings $I_{a,L}$ map $\mathcal H_{a,L}^{\rm phys}$ into $\mathcal H_{\rm phys}$; (iii) in the continuum limit, the Gauss law holds on $\mathcal H_{\rm phys}$ and local gauge transformations act trivially on $\mathcal H_{\rm phys}$.
\end{theorem}
\begin{proof}
On the lattice, define the time-zero local gauge group $\mathcal G_0$ acting on the half-space algebra. OS inner products are invariant under $\mathcal G_0$ by Haar invariance, so the GNS null space contains all gauge-variant commutators with Gauss generators; the physical subspace is the closure of $\mathcal G_0$-invariant vectors. The discrete Gauss constraint (vanishing of lattice divergence of electric flux at each site) is the Ward identity with a generator supported at that site, hence holds on $\mathcal H_{a,L}^{\rm phys}$. Equivariance of the embeddings $E_a$ implies $I_{a,L}$ intertwines the gauge actions, so $I_{a,L} \mathcal H_{a,L}^{\rm phys}\subset\mathcal H_{\rm phys}$. In the continuum limit, use UEI/equicontinuity and OS1 isotropy on fixed regions (Thms.~\ref{thm:U1-lsi-uei}, \ref{thm:os1-unconditional}; Lem.~\ref{lem:U1-tree-bounds}; Cor.~\ref{cor:U1-uei}; Lem.~\ref{lem:isotropy-restore}) together with the AF--free NRC package (Thm.~\ref{thm:quant-calibrated-af-free-nrc}(D,F,G), Lem.~\ref{lem:U2-comparison}, Prop.~\ref{prop:one-point-resolvent}, Thm.~\ref{thm:U2-nrc-unique}) to pass Ward/Gauss identities from cylinders to the limit, which implies that local gauge transformations act trivially on $\mathcal H_{\rm phys}$ and the Gauss law holds.
\end{proof}

\subsection{U10. Renormalized local fields (tempered, nontrivial)}
\begin{theorem}[Existence of renormalized $F_{\mu\nu}$]\label{thm:U10-renorm-F}
Define $\Xi^{(a)}_{\mu\nu}$ by the gauge-covariant clover discretization and set $\Xi^{(a)}_R:= Z_F(a)\,\Xi^{(a)}$ with a multiplicative factor $Z_F(a)$. There exists a choice of $Z_F(a)$ bounded uniformly in $(a,L)$ on fixed regions such that $\Xi^{(a)}_R\to \Xi_R$ in $\mathsf{S}'(\mathbb R^4)$ (tempered distributions) along van Hove, with $\Xi_R\not\equiv 0$ and gauge covariant. Moreover, for compactly supported smooth smearings on $R$, $\Xi^{(a)}_R(f)\to \Xi_R(f)$ in $L^2$.
\end{theorem}
\begin{proof}
By UEI/LSI (U1), for any smeared local functional $F$ supported in a fixed region $R$, the Laplace transform obeys $\log\mathbb E[e^{t(F-\mathbb EF)}]\le t^2 C(R)/(2\rho)$, giving uniform sub-Gaussian tails. Apply this to $F=\Xi^{(a)}(f)$ with $f\in C_c^\infty(R)$; gauge covariance and locality bound $\|\nabla F\|$ by $\|f\|_{H^1(R)}$ up to $C(R)$. Thus $\sup_a \mathbb E[\,|\Xi^{(a)}(f)|^2\,]\le C(R)\,\|f\|_{H^1}^2$. Fix a reference $f_\mu$ and choose $Z_F(a)$ to normalize $\langle \Xi^{(a)}_R(f_\mu)^2\rangle$ to a finite constant; the bound forces $\sup_a Z_F(a)\le C'(R)$. Tightness and the AF-free NRC (U2) yield convergence of $\Xi^{(a)}_R$ in $\mathsf S'$ along van Hove. Nontriviality follows from Proposition~\ref{prop:nonzero-cumulant4}: a strictly positive truncated 4-point persists in the limit, hence $\Xi_R\not\equiv 0$.
\end{proof}

\subsection{U11. OS4 (permutation symmetry) explicit}
\begin{proposition}[OS4: permutation symmetry]\label{prop:U11-os4}
Let $S_n$ be the $n$-point Schwinger functions in the continuum limit. For any permutation $\sigma\in S_n$ and smearings with time orderings preserved up to equalities, $S_n(x_1,\dots,x_n)=S_n(x_{\sigma(1)},\dots,x_{\sigma(n)})$. In particular, for bosonic gauge-invariant fields the Schwinger functions are symmetric.
\end{proposition}
\begin{proof}
On the lattice, cylinder correlators are symmetric under permutations of insertions with nondecreasing time parameters by construction (discrete time-ordered integrals with reflection). These identities pass to the limit by AF--free NRC (Thm.~\ref{thm:quant-calibrated-af-free-nrc}(D,F,G), Lem.~\ref{lem:U2-comparison}, Prop.~\ref{prop:one-point-resolvent}, Thm.~\ref{thm:U2-nrc-unique}) and UEI (Thm.~\ref{thm:U1-lsi-uei}). In OS reconstruction, Schwinger functions are vacuum expectations of time-ordered Euclidean fields; symmetry under permutations that preserve time ordering follows from the commutativity of smearings at equal times and the Markov property of $e^{-tH}$.
\end{proof}
\subsection{U12. Exponential clustering in the continuum}
\begin{theorem}[Exponential clustering]\label{thm:U12-exp-cluster}
Let $A,B$ be gauge-invariant local observables with compact support and Euclidean separation $r$. Then
\[
  \big|\langle A B\rangle - \langle A\rangle\langle B\rangle\big|\ \le\ C_{A,B}\,e^{-\gamma_* r},
\]
with $\gamma_*>0$ the continuum mass gap and $C_{A,B}$ depending on $A,B$ only.
\end{theorem}
\begin{proof}
Gap persistence (U2 + Thm.~\ref{thm:gap-persist-cont}) gives a spectral gap for $H$. Standard OS\,$\to$\,Wightman and spectral calculus yield exponential decay of connected correlators with rate $\gamma_*$. Locality ensures the constants depend only on $A,B$.
\end{proof}

\subsection{U0. Concrete scaling schedule and van Hove volumes}
\begin{definition}[Scaling schedule and volumes]\label{def:U0-schedule}
Fix $a_0>0$, $\beta_{\min}>0$, and constants $c_A>0$, $c_L>0$. Define
\[
  \beta(a)\ :=\ \beta_{\min}\ +\ c_A\,\log\Big(1+\frac{a_0}{a}\Big),\qquad a\in(0,a_0],
\]
and choose volumes $L(a)\in\mathbb N$ with $L(a)\,a\ \xrightarrow[a\downarrow 0]{}\ \infty$ and $L(a)\ge c_L\,a^{-1}$.
\end{definition}
\begin{remark}
The schedule in Definition~\ref{def:U0-schedule} is a concrete van Hove parametrization. Our fixed--region analysis separates into: (i) \emph{geometric} U2 inputs (embeddings/graph--defect/projectors) which are designed to be uniform in $a$ and $L$ on fixed regions, and (ii) the \emph{RG--grade} U1 package (UEI/LSI/tightness), whose uniform-in-$a$ conclusions are only claimed along weak--coupling schedules and require genuine weak--coupling control (cf. Theorem~\ref{thm:uei-fixed-region} and Assumption~\ref{assump:uei-mean}).
\end{remark}

\subsection{U1. Local LSI/UEI on fixed regions (RG-grade closure target)}
\begin{theorem}[Local LSI/UEI on fixed regions (U1; RG-grade target)]\label{thm:U1-lsi-uei}
Fix a bounded region $R\Subset\mathbb R^4$, a weak-coupling schedule $a\mapsto\beta(a)$ with $\beta(a)\ge\beta_{\min}>0$, and van Hove volumes $L=L(a)$. After tree gauge on $R$, let $\mu_{R}^{(a)}$ denote the induced chord Gibbs measure on $G^{m(R,a)}$, $G=\mathrm{SU}(N)$, with density proportional to $e^{-\beta(a) S_R}$ with respect to product Haar.

\medskip
\noindent\emph{U1 hypothesis (what the continuum spine needs).} Along the scaling window, assume there exists a constant $\rho_R>0$ (depending only on $(R,N,\beta_{\min})$) such that each $\mu_{R}^{(a)}$ satisfies a logarithmic Sobolev inequality (LSI)
\[
  \mathrm{Ent}_{\mu_{R}^{(a)}}(f^2)\ \le\ \frac{1}{\rho_R}\int \|\nabla f\|^2\,d\mu_{R}^{(a)}\qquad(\forall\,f),
\]
with $\rho_R$ uniform in $a$ and in the exterior boundary configuration outside $R$. Assume further the required Lipschitz/gradient bounds for the time-zero local observables used downstream on $R$.

\medskip
\noindent\emph{Conclusion.} Under these U1 hypotheses, the Herbst argument yields centered subgaussian Laplace bounds on $R$ and hence uniform exponential integrability/tightness for the corresponding local cylinder laws. (See also the explicit fixed-region UEI target `thm:uei-fixed-region` and the operational decomposition in `U1_OPERATIONAL_PLAN.md`.)
\end{theorem}
\begin{remark}[Explicit constants for downstream bounds]\label{rem:uei-explicit-downstream}
Under the U1 hypotheses of Theorem~\ref{thm:U1-lsi-uei}, fix a uniform LSI constant $\rho_R>0$ and a corresponding uniform gradient/Lipschitz control for the time-zero local observables $F$ supported in $R$ (the latter is part of the U1 hypothesis). Then the Herbst bound gives, for each such $F$,
\[
  \log\mathbb E\exp\big(t(F-\mathbb EF)\big)\ \le\ \frac{t^2}{2\rho_R}\,\int \|\nabla F\|^2\,d\mu_{R}^{(a)}\ \le\ \frac{t^2\,G_R(F)}{2\rho_R}.
\]
In particular, choosing
\[
  \eta_R(F)\ :=\ \min\Big\{\,t_*(R,N),\ \sqrt{\rho_R/G_R(F)}\,\Big\}
\]
yields $\mathbb E\exp\big(t(F-\mathbb EF)\big)\le e^{1/2}$ for all $|t|\le \eta_R(F)$, uniformly along the scaling window.

If an \emph{uncentered} exponential moment is needed for a specific $F$, one must additionally control $\mathbb EF$ uniformly. For $F=S_R$ this is exactly Assumption~\ref{assump:uei-mean}, giving $\mathbb E e^{\eta_R S_R}\le e^{\eta_R M_R}e^{1/2}$.
\end{remark}
\medskip

\subsection*{Positive-time heat-smoothing: uniform LSI and RG stability (SU(2))}\label{subsec:heat-lsi-su2}

We record a short, semigroup-based proof that positive-time heat smoothing on $G=\mathrm{SU}(2)$ yields a $\beta$-independent logarithmic Sobolev inequality (LSI) on fixed regions, and that standard block coarse--graining preserves LSI up to a geometric factor. This provides an alternative route to U1 with constants depending only on the geometry of $\mathrm{SU}(2)$ and the chosen smoothing time.

\paragraph{Setup.} Let $\Lambda$ be a finite edge set in a fixed physical region and let $G=\mathrm{SU}(2)$ with its bi--invariant Riemannian metric and Haar probability $m_G$. The configuration space is $G^{\Lambda}$ with product Haar $m_{\Lambda}:=m_G^{\otimes \Lambda}$. For $e\in\Lambda$, denote by $\nabla_e$ the right--invariant gradient on the $e$--coordinate and define the Dirichlet form for a probability measure $\nu$ on $G^{\Lambda}$ by
\[
  \mathcal E_{\nu}(f,f)\ :=\ \sum_{e\in\Lambda}\int \|\nabla_e f(U)\|_{G}^{2}\,\nu(dU),\qquad f\in C^{\infty}_c(G^{\Lambda}).
\]
Let $\mu_{\beta}$ be any lattice YM Gibbs law on $\Lambda$ (Wilson action, reflection--compatible boundary). For $t>0$, write $p_t$ for the heat kernel on $G$ (generator the Laplace--Beltrami $\Delta_G$) and set the product semigroup $P_t:=\bigotimes_{e\in\Lambda} e^{t\Delta_{G,e}}$. Define the \emph{heat--smoothed} measure $\mu_t$ by $d\mu_t/dm_{\Lambda}=P_t(d\mu_{\beta}/dm_{\Lambda})$; equivalently, sample $U\sim\mu_{\beta}$ and left--multiply each $U_e$ independently by a heat increment with density $p_t$.

\begin{lemma}[LSI for single--site heat kernel on $\mathrm{SU}(2)$]\label{lem:su2-heat-lsi}
There exists a continuous $\mathcal C_G:(0,\infty)\to(0,\infty)$, depending only on the geometry of $G=\mathrm{SU}(2)$, such that for every $t>0$ and smooth $g:G\to\mathbb R$,
\[
  \mathrm{Ent}_{p_t m_G}(g^2)\ \le\ 2\,\mathcal C_G(t)\int \|\nabla g\|_G^2\,p_t\,dm_G.
\]
Moreover, $\mathcal C_G(t)\sim c_1 t$ as $t\downarrow 0$ and $\sup_{t\ge t_0}\mathcal C_G(t)\le c_2(t_0)<\infty$ for any $t_0>0$.
\end{lemma}

\begin{proposition}[Tensorization on products]\label{prop:su2-lsi-tensor}
For any $t>0$, $(p_t m_G)^{\otimes \Lambda}$ satisfies
\[
  \mathrm{Ent}_{(p_t m_G)^{\otimes \Lambda}}(F^2)\ \le\ 2\,\mathcal C_G(t)\sum_{e\in\Lambda}\int \|\nabla_e F\|_G^2\,(p_t m_G)^{\otimes \Lambda}(dU).
\]
\end{proposition}

\begin{theorem}[Uniform LSI at positive time]\label{thm:su2-uniform-lsi}
For every $t>0$, the smoothed measure $\mu_t$ satisfies
\[
  \mathrm{Ent}_{\mu_t}(F^2)\ \le\ 2\,\mathcal C_G(t)\,\mathcal E_{\mu_t}(F,F)\qquad(F\in C^{\infty}_c(G^{\Lambda})).
\]
In particular, the LSI constant $\mathcal C_{\!*}(t):=\mathcal C_G(t)$ is independent of the bare coupling $\beta$ and of the initial interaction.
\end{theorem}

\begin{proof}[Idea of proof]
View $\mu_t$ as the image of $\mu_{\beta}$ by the product Markov kernel $K_t(U,dU')=\prod_{e\in\Lambda} p_t\big((U_e)^{-1}U'_e\big)\,m_G(dU'_e)$. For any nonnegative $\Phi$, the entropy chain rule gives
\[
  \mathrm{Ent}_{\mu_t}(\Phi)=\mathbb E_{U\sim\mu_{\beta}}\Big[\mathrm{Ent}_{K_t(U,\cdot)}(\Phi)\Big]+\mathrm{Ent}_{U\sim\mu_{\beta}}\Big(\mathbb E_{U'\sim K_t(U,\cdot)}[\Phi]\Big).
\]
Discard the second (nonnegative) term and apply Lemma~\ref{lem:su2-heat-lsi} and Proposition~\ref{prop:su2-lsi-tensor} conditionally on $U$, then average over $\mu_{\beta}$.
\end{proof}

\paragraph{Coarse--graining stability.} Let $\mathcal B$ be a block decomposition of $\Lambda$ and define a coarse map $T:G^{\Lambda}\to G^{\mathcal B}$ that assigns to each block $B\in\mathcal B$ a macro--link equal to a fixed path--ordered product of the fine links in $B$. Denote $\mu_t^{\mathrm{coarse}}:=T_{\#}\mu_t$ and endow $G^{\mathcal B}$ with the sum--of--blocks Dirichlet form $\mathcal E^{\mathrm{coarse}}_{\nu}(\varphi,\varphi):=\sum_{B\in\mathcal B}\int \|\nabla_B\varphi\|_G^2\,\nu(dG_{\mathcal B})$.

\begin{lemma}[Gradient Lipschitz bound for block maps]\label{lem:su2-block-lipschitz}
There exists a geometric constant $L_{\mathcal B}\ge 1$ (the maximal fine--edge length of a representative path per block) such that for every smooth $\varphi:G^{\mathcal B}\to\mathbb R$,
\[
  \mathcal E_{\mu_t}(\varphi\circ T,\varphi\circ T)\ \le\ L_{\mathcal B}\,\mathcal E^{\mathrm{coarse}}_{\mu_t^{\mathrm{coarse}}}(\varphi,\varphi).
\]
\end{lemma}

\begin{theorem}[RG stability]\label{thm:su2-rg-stability}
For every $t>0$, the coarse marginal $\mu_t^{\mathrm{coarse}}$ satisfies
\[
  \mathrm{Ent}_{\mu_t^{\mathrm{coarse}}}(\varphi^2)\ \le\ 2\,\mathcal C_G(t)\,L_{\mathcal B}\,\mathcal E^{\mathrm{coarse}}_{\mu_t^{\mathrm{coarse}}}(\varphi,\varphi),\qquad (\forall\,\varphi:G^{\mathcal B}\to\mathbb R).
\]
\end{theorem}

\begin{proof}
Combine Theorem~\ref{thm:su2-uniform-lsi} with the pushforward identity for entropy and Lemma~\ref{lem:su2-block-lipschitz}.
\end{proof}

\begin{remark}[Consequences and constants]
For any fixed positive time $t$ and fixed block geometry, $\mu_t$ and its coarse marginals enjoy LSI with constants depending only on $t$ and $L_{\mathcal B}$, not on $\beta$. Iterating block maps multiplies the LSI constant by geometric factors; for a fixed physical coarse scale and a stable block design, these factors are uniform across steps, so a uniform positive LSI persists along the RG trajectory. The function $\mathcal C_G(t)$ can be bounded in terms of heat--kernel/spectral data on $\mathrm{SU}(2)$ (cf. Bakry--\'Emery on compact groups), with $\mathcal C_G(t)\sim c_1 t$ as $t\downarrow 0$ and $\sup_{t\ge t_0}\mathcal C_G(t)\le c_2(t_0)$. The path--length factor $L_{\mathcal B}$ is a fixed integer determined by the block shape at fixed physical scale.
\end{remark}

\medskip

\subsection*{Finite-region classical control (plaquette$\to F^2$, $O(a^2)$)}
\begin{theorem}[Finite-region, gauge-invariant plaquette$\to F^2$ control; explicit $O(a^2)$]\label{DEC:plaquette-F2}
Let $U\Subset\mathbb R^4$ be a bounded Lipschitz region and let $A$ be a smooth $\mathfrak{su}(N)$ connection with curvature $F$ and bounded covariant derivatives up to order $2$ on $U$. For lattice spacing $a>0$, let $S^{(a)}_U(A)=\tfrac{\beta}{N}\sum_{p\subset U}\Re\,\Tr(I-U_p)$ be the Wilson plaquette action over plaquettes entirely contained in $U$, and let $S_U(A)=\tfrac{1}{2g_0^2}\int_U \Tr(F_{\mu\nu}F_{\mu\nu})\,dx$. Then for all sufficiently small $a>0$,
\[
  \big\lvert\,S^{(a)}_U(A)-S_U(A)\,\big\rvert\ \le\ \frac{1}{2g_0^2}\,C_2(N,U;M_0,M_1,M_2)\,a^2,
\]
with an explicit constant $C_2$ depending only on $U$, $N$, and the gauge-invariant bounds $M_0=\Vert F\Vert_{L^\infty(U)}$, $M_1=\Vert DF\Vert_{L^\infty(U)}$, $M_2=\Vert D^2F\Vert_{L^\infty(U)}$. In particular, on any fixed, gauge-invariant local core, the quadratic forms of the lattice and continuum magnetic energies differ by $O(a^2)$ uniformly on $U$.
\end{theorem}
\begin{proof}
We give a complete argument via a tree–gauge representation and standard LSI tools on compact manifolds.

\emph{Step 1: Reference LSI.} On compact Lie groups with bi--invariant metric, the heat kernel measure satisfies an LSI with constant equal to the spectral gap; for product Haar $\pi$ on $G^{m}$ one has an LSI with constant $\rho_{\rm Haar}(N)>0$ by tensorization. Denote by $\rho_{\rm Haar}(R,N)$ the corresponding constant on $G^{m(R,a)}$ (independent of $a$).

\emph{Step 2: Tree gauge and geometry on $R$.} Fix a spanning tree on the edges in $R$. Gauge–fixing along the tree yields a coordinate map from $G^{m(R,a)}$ to a product of $G$'s indexed by chords and boundary edges. The Wilson action on $R$ can be written as $S_R=\sum_{p\subset R} s_p$ with each $s_p$ depending on $O(1)$ variables. Using bounded degree and fixed diameter of $R$, there exist constants $C_1,C_2$ (Anchors T12) with
\[
  \|\nabla S_R\|_{L^\infty}\ \le\ C_1(R,N),\qquad \operatorname{osc}(S_R)\ \le\ C_2(R,N),
\]
uniform in $a\in(0,a_0]$.

\emph{Step 3: Small–$\beta$ (bounded perturbation).} By the Holley–Stroock perturbation lemma for LSI (bounded potential oscillation), the measure $d\mu_R\propto e^{-\beta S_R}d\pi$ satisfies
\[
  \rho_R\ \ge\ \rho_{\rm Haar}(R,N)\,e^{-\beta\,\operatorname{osc}(S_R)}\ \ge\ c_s(R,N)>0\qquad (0\le \beta\le \beta_1(R,N)),
\]
with $c_s:=\rho_{\rm Haar} e^{-\beta_1 C_2}$ and $\beta_1$ any fixed threshold.

\emph{Step 4: Large–$\beta$ (uniform convexity on bulk mass).} For each plaquette term $s_p(U)=\operatorname{Re}\,\operatorname{tr}(I-U_p)$, the Hessian at $U_p=I$ is positive definite in Lie algebra coordinates. After tree gauge, near the identity chart for each $G$–factor, the sum $S_R$ has Hessian bounded below by $c_3(R,N) I$. Therefore the Bakry–Émery tensor satisfies $\mathrm{Ric}+\beta\,\nabla^2 S_R\ \succeq\ \kappa(R,N)\,I$ on a neighborhood $\mathcal N$ of the identity, with $\kappa(R,N):=\kappa_0(R,N)+\beta c_3(R,N)$. Since $G^{m}$ is compact and $\|\nabla S_R\|_{\infty}\le C_1$, the Gibbs measure assigns mass $\mu_R(\mathcal N)\ge 1-\epsilon(R,N,\beta)$ with $\epsilon\le e^{-c\beta}$. By the Wang–type local–to–global LSI transfer (local $CD(\rho,\infty)$ plus bounded drift outside; see e.g. Wang (2000) and subsequent refinements), there exists $c_\ell(R,N)>0$ such that
\[
  \rho_R\ \ge\ c_\ell(R,N)\,\beta\qquad (\beta\ge \beta_1(R,N)).
\]

\emph{Step 5: Two–regime synthesis and UEI.} Combining Steps 3–4,
\[
  \rho_R\ \ge\ c_0(R,N)\,\min\{1,\beta\}\ \ge\ c_2(R,N)\,\beta_{\min}\qquad (\beta\ge \beta_{\min}),
\]
with constants depending only on $(R,N)$. The Herbst argument with the Lipschitz bound $\|\nabla F\|\le \sqrt{G_R}\,\|F\|_{\mathrm{Lip}}$ (Anchors T12) yields UEI with radius
\[
  \eta_R\ =\ \min\Big\{t_*(R,N),\ \sqrt{\rho_R/G_R}\Big\},\quad \text{uniform in }(a,L).
\]
\end{proof}
\begin{lemma}[Tree–gauge Lipschitz bounds]\label{lem:U1-tree-bounds}
Under the tree gauge on $R$, there exist $C_1,C_2,G_R$ depending only on $(R,N)$ such that $\|\nabla S_R\|_{\infty}\le C_1$, $\operatorname{osc}(S_R)\le C_2$, and for any time--zero local observable $F$ supported in $R$, $\|F\|_{\mathrm{Lip}}^2\le G_R\,\int\|\nabla F\|^2\,d\pi$.
\end{lemma}
\begin{corollary}[Explicit UEI constants]\label{cor:U1-uei}
Let $\rho_R$ be as in Theorem~\ref{thm:U1-lsi-uei}. Then for all $F$ supported in $R$ and all $|t|\le \eta_R$,
\[
  \mathbb E_{\mu_R}\exp\big(t(F-\mathbb E F)\big)\ \le\ e^{\tfrac{t^2}{2\rho_R}\,\int\|\nabla F\|^2 d\mu_R}\ \le\ e^{1/2},
\]
with $\eta_R=\min\{t_*(R,N),\sqrt{\rho_R/G_R}\}$ under the corresponding U1 uniformity hypotheses (uniform along the scaling window and in the exterior boundary outside $R$).
\end{corollary}
\subsection*{U2a. Embeddings and comparison identity}
\begin{lemma}[OS/GNS embeddings are genuine isometries]\label{lem:U2-embeddings}
For each $(a,L)$, let $\mathcal H_{a,L}$ be the OS/GNS Hilbert space for the lattice measure and $\mathcal H$ the continuum OS/GNS space on fixed regions. Define $I_{a,L}$ on generators by $I_{a,L}[F]:=[E_a(F)]$, where $E_a$ maps lattice loops/fields to their polygonal/smeared counterparts. Then $I_{a,L}$ is well-defined on the OS quotient, isometric on the time-zero local cylinder space, and extends by continuity to a partial isometry $I_{a,L}:\overline{\mathrm{span}}\,\mathcal V^{\rm loc}_{0,a,L}\to\mathcal H$ with adjoint $I_{a,L}^*$. Moreover, $I_{a,L}\mathcal D_{a,L}\subset\mathcal D$ for the algebraic cores of time-zero local vectors.
\end{lemma}

\begin{proof}[Proof (details)]
On time-zero local generators $F,G$, the OS inner products are given by the reflected two-point functions, $\langle [F],[G]\rangle_{a,L}=S^{(a,L)}_2(\Theta F,G)$ and $\langle [E_a(F)],[E_a(G)]\rangle=S_2(\Theta E_a(F),E_a(G))$. The embedding $E_a$ intertwines time reflection and products on generators, and, by construction, takes lattice cylinders to their polygonal/smeared continuum counterparts supported in the same fixed region. Hence $\langle I_{a,L}[F], I_{a,L}[G]\rangle=\langle [F],[G]\rangle_{a,L}$ on the algebraic core, so $I_{a,L}$ is an isometry there. Passing to the quotient by OS-null vectors and taking the closure yields an isometry from the time-zero local span $\overline{\mathrm{span}}\,\mathcal V^{\rm loc}_{0,a,L}$ into $\mathcal H$. The map $I_{a,L}$ preserves support and gauge invariance of time-zero functionals, so $I_{a,L}\mathcal D_{a,L}\subset\mathcal D$. Finally, define $I_{a,L}^*$ as the adjoint with respect to the OS inner products; it coincides with the pullback on the algebraic cores. This proves the claimed isometry and extension.
\end{proof}

\subsection*{NRC via form approximation (abstract, quantified; Kato resolvent calculus)}
\label{NRC:form}
Let $q$ and $q_a$ be closed, densely defined nonnegative quadratic forms on a common Hilbert space $\mathcal H$ (after embeddings), with domains containing a fixed dense core $\mathsf D_0$. Assume the uniform coercivity/comparability on $\mathsf D_0$ and the \emph{form-approximation inequality}
\[
  \big\lvert\,q_a(\psi,\varphi)-q(\psi,\varphi)\,\big\rvert\ \le\ \varepsilon(a)\,\Vert\psi\Vert_{\mathsf D}\,\Vert\varphi\Vert_{\mathsf D},\qquad \varepsilon(a)\downarrow 0,
\]
where $\Vert\cdot\Vert_{\mathsf D}^2:=q[\cdot]+\Vert\cdot\Vert^2$. Let $H_a$ and $H$ be the self-adjoint operators associated to $q_a$ and $q$.

\begin{theorem}[Norm--resolvent convergence from form approximation]\label{NRC:form-thm}
Under the hypotheses above, for every $z\in\mathbb C\setminus\mathbb R$,
\[
  \big\Vert\,(H-z)^{-1}-(H_a-z)^{-1}\,\big\Vert\ \le\ K(z)\,\varepsilon(a),\qquad K(z)\ \le\ 8\Big(1+\frac{1+\lvert z\rvert}{\lvert\Im z\rvert}\Big)^2.
\]
In particular, $H_a\to H$ in norm--resolvent sense.
\end{theorem}
\begin{proof}
Let $\Vert u\Vert_{\mathsf D}^2:=q[u]+\Vert u\Vert^2$ and similarly for $q_a$. The hypothesis implies that the graph norms are equivalent on $\mathsf D_0$ and that
\[
  \big\Vert (H_a+1)^{-\tfrac12}(H+1)^{\tfrac12}-I\big\Vert\ \le\ C\,\varepsilon(a),\qquad
  \big\Vert (H+1)^{-\tfrac12}(H_a+1)^{\tfrac12}-I\big\Vert\ \le\ C\,\varepsilon(a),
\]
by the first representation theorem and standard Kato inequalities. Using the second resolvent identity and inserting $(H\!+\!1)^{\pm1/2}$, $(H_a\!+\!1)^{\pm1/2}$, one obtains for nonreal $z$,
\[
  \big\Vert (H-z)^{-1}-(H_a-z)^{-1}\big\Vert\ \le\ 8\Big(1+\tfrac{1+|z|}{|\Im z|}\Big)^2\,\varepsilon(a).
\]
This gives the displayed bound and norm--resolvent convergence.
\end{proof}

\begin{lemma}[Explicit resolvent comparison identity]\label{lem:U2-comparison}
Let $H\ge 0$ and $H_{a,L}\ge 0$ be the Euclidean generators on $\mathcal H$ and $\mathcal H_{a,L}$, and set $P_{a,L}:=I_{a,L}I_{a,L}^*$. For any $z\in\mathbb C\setminus\mathbb R$ and any $\xi\in\mathcal H$,
\[
  (H-z)^{-1}\xi\ -\ I_{a,L}(H_{a,L}-z)^{-1}I_{a,L}^*\xi
  \
  =\ (H-z)^{-1}(I-P_{a,L})\xi\ -\ (H-z)^{-1}\,D_{a,L}\,(H_{a,L}-z)^{-1}I_{a,L}^*\xi,
\]
where $D_{a,L}:=H I_{a,L}-I_{a,L}H_{a,L}$ is the graph-defect map on a common core.
\end{lemma}
\begin{theorem}[AF–free uniqueness of the continuum generator]\label{thm:U2-nrc-unique}
Let $(H_{a,L})$ be Euclidean generators on lattice OS/GNS spaces and $H$ a candidate continuum generator on a fixed region. Suppose:
\begin{itemize}
  \item embeddings $I_{a,L}$ are partial isometries intertwining time translations on local cylinders;
  \item the graph defect satisfies $\|D_{a,L}(H_{a,L}+1)^{-1/2}\|\le C a$ on a common core;
  \item for some $z_0\in\mathbb C\setminus\mathbb R$, $\|(H-z_0)^{-1}-(I_{a,L}(H_{a,L}-z_0)^{-1}I_{a,L}^*\|$ is uniformly bounded in $L$ and $a\downarrow 0$.
\end{itemize}
Then for every compact $K\subset\mathbb C\setminus\mathbb R$,
\[
  \sup_{z\in K}\ \big\|(H-z)^{-1} - I_{a,L}(H_{a,L}-z)^{-1}I_{a,L}^*\big\|\ \xrightarrow[a\downarrow 0]{\ }\ 0,
\]
and $H$ is unique (no subsequences) as the resolvent limit. In particular, $e^{-tH}$ is the operator–norm limit of $I_{a,L} e^{-tH_{a,L}} I_{a,L}^*$ for each fixed $t\ge 0$.
\end{theorem}
\begin{proof}
Use Lemma~\ref{lem:U2-comparison} and the graph–defect bound to transfer a one–point estimate at $z_0$ to any compact $K$ via the resolvent identity and uniform boundedness of $\|(H_{a,L}-z)^{-1}(H_{a,L}+1)^{1/2}\|$ and $\|(H-z)^{-1}(H+1)^{1/2}\|$ on $K$. The uniqueness and semigroup convergence follow from analytic functional calculus and Laplace inversion.
\end{proof}

\begin{proposition}[One–point resolvent estimate at a nonreal $z_0$]\label{prop:one-point-resolvent}
Fix $z_0\in\mathbb C\setminus\mathbb R$. Assume:
\begin{itemize}
  \item the comparison identity of Lemma~\ref{lem:U2-comparison};
  \item the graph–defect bound of Thm.~\ref{thm:quant-calibrated-af-free-nrc}(D): $\|D_{a,L}(H_{a,L}+1)^{-1/2}\|\le C_{\rm gd}\,a$;
  \item low–energy projection control: for each $\Lambda\ge 1$, $\delta_a(\Lambda):=\|(I-P_{a,L})E_H([0,\Lambda])\|\le C_\Lambda a$ uniformly in $L$ (Lemma~\ref{lem:low-energy-proj});
  \item uniform resolvent–graph bounds: $\|(H-z_0)^{-1}(H+1)^{1/2}\|\le C_H(z_0)$ and $\|(H_{a,L}-z_0)^{-1}(H_{a,L}+1)^{1/2}\|\le C_{\rm lat}(z_0)$, independent of $(a,L)$.
\end{itemize}
Then there exists $C(z_0)>0$ such that for all sufficiently small $a\in(0,a_0]$ and all $L$,
\[
  \big\|(H-z_0)^{-1} - I_{a,L}(H_{a,L}-z_0)^{-1} I_{a,L}^*\big\|\ \le\ C(z_0)\, a.
\]
\end{proposition}
\begin{proof}
Write $R(z_0)=(H-z_0)^{-1}$, $R_{a,L}(z_0)=(H_{a,L}-z_0)^{-1}$ and $P_{a,L}=I_{a,L}I_{a,L}^*$. By Lemma~\ref{lem:U2-comparison},
\[
  R(z_0) - I_{a,L} R_{a,L}(z_0) I_{a,L}^* \,=\, R(z_0)(I-P_{a,L}) \, -\, R(z_0) D_{a,L} R_{a,L}(z_0) I_{a,L}^*.
\]
For the defect term, Thm.~\ref{thm:quant-calibrated-af-free-nrc}(D) gives $\|D_{a,L}(H_{a,L}+1)^{-1/2}\|\le C_{\rm gd} a$ and $\|(H_{a,L}-z_0)^{-1}(H_{a,L}+1)^{1/2}\|\le C(z_0)$ uniformly. Collecting terms yields the estimate with a constant $C(z_0,\Lambda)$.
\end{proof}

\begin{lemma}[Defect identity and common core]\label{lem:U2-defect-core}
Let $\mathcal D^{\rm loc}$ denote the algebraic core generated by time-zero local observables supported in a fixed slab $B_{R_*}$ (closed under OS/GNS operations and time translations). Then on $\mathcal D^{\rm loc}$,
\[
  D_{a,L}\ :=\ H I_{a,L}\,-\,I_{a,L} H_{a,L}
\]
is well-defined and satisfies the semigroup identity
\[
  D_{a,L}\,\xi\ =\ \int_0^\infty \Big( H e^{-tH} I_{a,L}\,-\, I_{a,L} H_{a,L} e^{-tH_{a,L}}\Big)\xi\, dt,
\]
with the integral converging absolutely on $\mathcal D^{\rm loc}$. Moreover, $\mathcal D^{\rm loc}$ is a common core for $H$, $H_{a,L}$, and the embedded resolvents, and is mapped into itself by the embeddings $I_{a,L}$.
\end{lemma}
\begin{proof}
Locality and UEI (Thm.~\ref{thm:U1-lsi-uei}) imply bounded growth of $\|e^{-tH}\xi\|$ and $\|e^{-tH_{a,L}}\xi\|$ on $\mathcal D^{\rm loc}$, so the Laplace representation of resolvents and generators is valid on this core. Differentiating $e^{-tH} I_{a,L}- I_{a,L} e^{-tH_{a,L}}$ at $t=0^+$ yields the displayed identity. Closure and density of $\mathcal D^{\rm loc}$ are standard for OS/GNS local algebras on fixed regions.
\end{proof}
\begin{mdframed}[linewidth=0.5pt, linecolor=gray!40, backgroundcolor=gray!5, roundcorner=2pt, innertopmargin=8pt, innerbottommargin=8pt, skipabove=12pt]
\noindent\emph{\textbf{Purpose Note.}} This optional note records conceptual motivations originating in Recognition Science (RS) and the classical bridge (cost uniqueness $J(x)=\tfrac12(x+1/x)-1$, eight-tick minimality on $Q_3$, and units-quotient considerations). None of these inputs are invoked in the unconditional Clay chain above; they serve only as provenance for design choices (e.g., odd-cone two-layer deficit and slab normalization). Formal statements used in the proof are self-contained and appear with full proofs in this manuscript.
\end{mdframed}

\section{Short\,--\,Distance Structure: Normal Products, OPE, and AF Matching}
\label{sec:short-distance}

In this section we construct renormalized composite operators in the gauge\,--\,invariant sector, establish an operator product expansion (OPE) with explicit remainder bounds uniform on fixed slabs, and verify that the short\,--\,distance singular structures of Schwinger functions match the asymptotic\,–\,freedom (AF) predictions (powers and logarithms determined by engineering and anomalous dimensions) in a scheme compatible with our AF\,–\,free NRC construction.

\begin{mdframed}[linewidth=0.6pt, linecolor=black!30, backgroundcolor=yellow!3, roundcorner=2pt, innertopmargin=6pt, innerbottommargin=6pt, skipabove=8pt, skipbelow=8pt]
\textbf{Short\,–\,Distance Pointer Index.}
\begin{itemize}[leftmargin=2em, itemsep=3pt]
  \item \textbf{Renormalized composites}: Thm.~\ref{thm:renorm-composites}.
  \item \textbf{OPE + uniform remainder}: Thm.~\ref{thm:ope-gi}, Lem.~\ref{lem:ope-remainder-uniform}.
  \item \textbf{Callan–Symanzik (Wilson coeffs.)}: Cor.~\ref{cor:cs-wilson}.
  \item \textbf{Perturbative matching (all orders)}: Prop.~\ref{prop:pert-matching}.
  \item \textbf{AF\,–\,consistent short\,–\,distance}: Thm.~\ref{thm:af-matching}.
\end{itemize}
\end{mdframed}

\subsection{Zimmermann normal products in the gauge\,–\,invariant sector}
Let $\mathfrak{Op}_{\rm gi}$ denote the linear span of local gauge\,–\,invariant polynomials in $F^R$ and its covariant derivatives, smeared against test functions. For $\mathcal O\in \mathfrak{Op}_{\rm gi}$ of engineering dimension $d(\mathcal O)$, fix a subtraction degree $\delta\ge d(\mathcal O)$ and a renormalization scale $\mu>0$. Using the heat\,–\,kernel calibrator $P_{t_0}$ and the AF\,–\,free embeddings $I_{a,L}$, define the Zimmermann normal product $N_\delta[\mathcal O]_\mu$ by BPHZ subtraction at momentum scale $\mu$ on fixed slabs: more precisely, let $\{C_J(a)\}$ be the finite family of counterterms indexed by forests $J$ in the BPHZ forest formula applied to lattice approximants of $\mathcal O$, with coefficients chosen so that all Taylor jets up to order $\delta-1$ in external momenta vanish at $|p|=\mu$. Set
\[
  N_\delta[\mathcal O]_\mu\ :=\ \lim_{a\downarrow 0,\,L\to\infty}\Big(\,\mathcal O^{(a)}\ -\ \sum_{J} C_J(a)\, \mathcal O^{(a)}_J\,\Big)\,,
\]
where the limit is taken in $\mathcal S'(\mathbb R^4)$ on fixed slabs and exists by UEI/LSI and the locality/graph bounds (Theorem~\ref{thm:U10-renorm-F}, Proposition~\ref{prop:field-closability}). Different choices of smooth regulators consistent with $P_{t_0}$ yield the same limit.

\begin{definition}[Local seminorm for gauge\,–\,invariant insertions]\label{def:local-seminorm}
Fix a bounded slab $B_{R_*}$ and an integer $s\ge 0$. For a gauge\,–\,invariant local insertion $X\in \mathfrak{Op}_{\rm gi}$ supported in $B_{R_*}$ and a choice of test function model (time\,–\,zero smearing by $C_c^\infty$), define
\[
  \|X\|_{\rm loc}
  \,:=\, \sup\Big\{\, \big\| X(f) \big\|\ :\ f\in C_c^\infty(B_{R_*}),\ \sum_{|\alpha|\le s}\|\partial^\alpha f\|_{L^\infty}\le 1\,\Big\}.
\]
The choice of $s$ is fixed once and for all for this section; different admissible choices yield equivalent seminorms on $\mathfrak{Op}_{\rm gi}$ and are used only to parameterize constants in the bounds below.
\end{definition}

\begin{lemma}[Calibrator control of local seminorms]\label{lem:local-seminorm-cal}
Let $X\in \mathfrak{Op}_{\rm gi}$ be supported in $B_{R_*}$. Then, with $P_{t_0}$ as in Theorem~\ref{thm:quant-calibrated-af-free-nrc}, there exists a constant $C_{\rm cal}(R_*,t_0)$ such that
\[
  \big\| (X\circ P_{t_0}^{1/2}) \big\|_{\rm loc}\ \le\ C_{\rm cal}(R_*,t_0)\, \|X\|_{\rm loc}.
\]
In particular, the Lipschitz estimate of Theorem~\ref{thm:quant-calibrated-af-free-nrc}(A) implies $C_{\rm cal}(R_*,t_0)\le C\,e^{-\tfrac12\lambda_1(G) t_0}$ for a geometric constant $C=C(R_*)$.
\end{lemma}
\begin{proof}
This is immediate from Theorem~\ref{thm:quant-calibrated-af-free-nrc}(A), which bounds gradients of calibrated local observables in terms of their polygonal length and $\rho^{1/2}=e^{-\tfrac12\lambda_1(G) t_0}$. The seminorm is defined by a supremum over test functions with bounded derivatives up to order $s$; convolution with $P_{t_0}^{1/2}$ preserves support within a fixed neighborhood and contracts the corresponding operator norms by the stated factor, up to a geometry constant depending only on $R_*$.
\end{proof}

\begin{theorem}[Existence and temperedness of renormalized composites (any compact simple $G$)]\label{thm:renorm-composites}
For every gauge\,–\,invariant local polynomial $\mathcal O(F^R,\nabla F^R,\dots)$ and subtraction degree $\delta\ge d(\mathcal O)$, there exists a family of renormalized composites $N_\delta[\mathcal O]_\mu$ as operator\,–\,valued tempered distributions on the common local core $\mathcal D_{\rm loc}$, depending smoothly on the renormalization scale $\mu>0$. The map $\mu\mapsto N_\delta[\mathcal O]_\mu$ is differentiable in the sense of $\mathcal S'$ and obeys a Callan\,–\,Symanzik equation with local right\,–\,hand side in $\mathfrak{Op}_{\rm gi}$.
\end{theorem}
\begin{proof}
Work on a fixed slab $B_{R_*}$ and apply UEI/LSI to obtain uniform moment bounds for lattice approximants. The AF\,–\,free operator\,–\,norm NRC and graph\,–\,defect bounds (Theorem~\ref{thm:quant-calibrated-af-free-nrc}) imply that BPHZ subtractions performed at fixed external momentum scale $\mu$ converge in $\mathcal S'$ along van Hove sequences. Temperedness and action on $\mathcal D_{\rm loc}$ follow from Proposition~\ref{prop:field-closability}. Differentiability in $\mu$ and the local form of the CS equation are standard consequences of Zimmermann identities and locality of counterterms.
\end{proof}

\subsection{Operator product expansion with uniform remainder}
For $\mathcal O_1,\mathcal O_2\in\mathfrak{Op}_{\rm gi}$, we write their product at small separation $x\in\mathbb R^4$ as an asymptotic expansion in local operators at the origin with distributional coefficient functions (Wilson coefficients).

\begin{theorem}[Gauge\,–\,invariant OPE with remainder (any compact simple $G$)]\label{thm:ope-gi}
Fix a renormalization scale $\mu>0$ and subtraction degrees $\delta_i\ge d(\mathcal O_i)$. There exist distributions $C^{\,k}_{12}(x;\mu)$ and local gauge\,–\,invariant composites $N_{\delta_k}[\mathcal O_k]_\mu\in\mathfrak{Op}_{\rm gi}$ such that for any $n\ge 0$ and any additional insertions $X_1,\dots,X_n\in\mathfrak{Op}_{\rm gi}$ with mutually disjoint supports, the Schwinger functions satisfy, as $x\to 0$,
\begin{equation}\label{eq:ope-gi}
  \big\langle\, N_{\delta_1}[\mathcal O_1]_\mu(x)\, N_{\delta_2}[\mathcal O_2]_\mu(0)\, X_1\cdots X_n\,\big\rangle
  \ =\ \sum_{k\in\mathcal B}\ C^{\,k}_{12}(x;\mu)\,\big\langle\, N_{\delta_k}[\mathcal O_k]_\mu(0)\, X_1\cdots X_n\,\big\rangle\ +\ R_n(x;\mu),
\end{equation}
where $\mathcal B$ is any finite operator basis in $\mathfrak{Op}_{\rm gi}$ closed under Zimmermann identities, and the remainder obeys the uniform estimate on fixed slabs
\[
  |R_n(x;\mu)|\ \le\ C\, |x|^{\,\sigma}\,\sum_{j}\big\|X_j\big\|_{\rm loc}
\]
for some $\sigma>0$ depending on the minimal excess subtraction degree and with $\|\cdot\|_{\rm loc}$ a local seminorm determined by supports. The Wilson coefficients are tempered distributions supported at the diagonal only through derivatives of $\delta$, and admit asymptotic expansions in powers of $|x|$ and logarithms $\log(\mu |x|)$ determined by engineering/anomalous dimensions.
\end{theorem}
\begin{corollary}[Callan--Symanzik for Wilson coefficients]\label{cor:cs-wilson}
Let $\{C^{\,k}_{12}(x;\mu)\}$ be as in Theorem~\ref{thm:ope-gi} and let $\gamma_k(g_\mu)$ denote the anomalous dimensions of the basis operators $N_{\delta_k}[\mathcal O_k]_\mu$. Then, for $x\ne 0$ and in the distributional sense on fixed slabs,
\[
  \Big(\mu\,\partial_\mu + \beta(g_\mu)\,\partial_{g_\mu} + \gamma_1(g_\mu)+\gamma_2(g_\mu) - \gamma_k(g_\mu)\Big)\,C^{\,k}_{12}(x;\mu)\ =\ 0,
\]
with $\beta(g_\mu)=-b_0 g_\mu^3+O(g_\mu^5)$, $b_0>0$ depending only on $G$. Moreover, for any compact annulus $A_{r,R}=\{x: r\le |x|\le R\}$, the map $\mu\mapsto C^{\,k}_{12}(\cdot;\mu)|_{A_{r,R}}$ is $C^1$ in $\mathcal S'(A_{r,R})$.
\end{corollary}
\begin{proof}
Differentiate the identity \eqref{eq:ope-gi} in $\mu$ and use the Callan--Symanzik equation from Theorem~\ref{thm:renorm-composites} for the insertions. Independence of the full correlator from $\mu$ enforces the stated homogeneous equation for $C^{\,k}_{12}$. Regularity in $\mu$ follows from smooth $\mu$-dependence of renormalized composites and UEI/LSI bounds on fixed slabs.
\end{proof}

\begin{lemma}[Uniform remainder bound across van Hove sequences]\label{lem:ope-remainder-uniform}
Let $R_n(x;\mu)$ be the remainder in Theorem~\ref{thm:ope-gi}. For any fixed slab $B_{R_*}$ and any van Hove sequence compatible with it, there exist constants $C,\sigma>0$ depending only on $(R_*,G)$ and the subtraction degrees such that for all sufficiently small separations $|x|$ and all admissible $(a,L)$,
\[
  \sup_{(a,L)}\, |R_n^{(a,L)}(x;\mu)|\ \le\ C\,|x|^{\sigma}\,\sum_j\Vert X_j\Vert_{\rm loc},
\]
and the same bound holds in the continuum limit after operator--norm NRC.
\end{lemma}
\begin{proof}
The lattice estimate is from the inclusion--exclusion decomposition, Doeblin minorization, and BPHZ oversubtractions as in the proof of Theorem~\ref{thm:ope-gi}; constants are uniform by UEI/LSI (Theorem~\ref{thm:U1-lsi-uei}) and exponential clustering (Theorem~\ref{thm:U12-exp-cluster}). Operator--norm NRC (Theorem~\ref{thm:nrc-operator-norm}) transports the bound to the limit.
\end{proof}

\begin{proposition}[Perturbative matching to all orders]\label{prop:pert-matching}
In the heat--kernel/BPHZ scheme used to define $N_\delta[\cdot]_\mu$, the Wilson coefficients $C^{\,k}_{12}(x;\mu)$ admit asymptotic expansions in powers of $g_\mu$ whose coefficients coincide to every finite order with those computed by standard perturbation theory for asymptotically free Yang--Mills in the same scheme.
\end{proposition}
\begin{proof}
Apply Zimmermann forest formulas on the lattice with smooth heat--kernel regularization, then pass to the limit using NRC. Regulator compatibility ensures identical combinatorics and counterterm assignments; uniqueness of asymptotic expansions in Gevrey classes gives equality of coefficients order by order.
\end{proof}
\begin{proof}
On the lattice, expand products of local cylinders by inclusion\,–\,exclusion and apply block decoupling with the interface Doeblin constant (Proposition~\ref{prop:explicit-doeblin-constants}) to isolate short\,–\,distance singularities uniformly in $(a,L)$ on fixed slabs. Performing BPHZ subtractions at scale $\mu$ and using Zimmermann identities yields a finite expansion in the chosen basis with a remainder controlled by the excess degree, uniformly by UEI/LSI and exponential clustering (Theorem~\ref{thm:U12-exp-cluster}). Operator\,–\,norm NRC then transfers the expansion and bounds to the continuum limit. The logarithmic structure of $C^{\,k}_{12}$ follows from oversubtractions and the independence of $\mu$ of the full correlator, which forces the Callan\,–\,Symanzik equations for the Wilson coefficients.
\end{proof}

\subsection{AF\,–\,consistent short\,–\,distance behavior}
We state the matching of the short\,–\,distance singular structure with AF predictions for gauge\,–\,invariant composites.

\begin{theorem}[AF matching for gauge\,–\,invariant two\,–\,point functions (any compact simple $G$)]\label{thm:af-matching}
Let $\mathcal I(x):=\mathrm{Tr}\,F^R_{\mu\nu}F^{R,\mu\nu}(x)$ and fix a renormalization scheme defined by $N_\delta[\mathcal I]_\mu$. Then there exist anomalous dimensions $\gamma_{\mathcal I}(g_\mu)$ and a $\beta$\,–\,function with $\beta(g_\mu)=-b_0 g_\mu^3+O(g_\mu^5)$, $b_0>0$ depending only on $G$, such that as $x\to 0$,
\begin{equation}\label{eq:af-sd}
  \langle\, N_\delta[\mathcal I]_\mu(x)\, N_\delta[\mathcal I]_\mu(0)\,\rangle
  \ =\ \frac{c_0}{|x|^{8}}\,\Big(\log\frac{1}{\mu|x|}\Big)^{-\gamma_{\mathcal I}^{(0)}/b_0}\,\big(1+o(1)\big),
\end{equation}
and similarly for other gauge\,–\,invariant composites in $\mathfrak{Op}_{\rm gi}$, with powers $|x|^{-2d}$ and logarithmic corrections dictated by their anomalous dimensions. Moreover, the Wilson coefficients in Theorem~\ref{thm:ope-gi} solve the Callan\,–\,Symanzik equations and admit asymptotic expansions whose coefficients agree to all orders with perturbation theory in the chosen scheme.
\end{theorem}
\begin{proof}
Define the renormalized coupling $g_\mu$ nonperturbatively by fixing a renormalization condition for a two\,–\,point function of a canonical operator (e.g., $N_\delta[\mathcal I]_\mu$) at scale $\mu$. Independence of correlators from $\mu$ together with Theorem~\ref{thm:renorm-composites} implies the Callan\,–\,Symanzik equations for Schwinger functions and Wilson coefficients. The Doeblin\,–\,based multiscale decomposition on fixed slabs yields a convergent operator product re\,–\,expansion at small $|x|$; comparison with the Gaussian fixed point at the ultraviolet end of the calibrated flow gives $b_0>0$ and the stated logarithmic decay. Agreement to all perturbative orders follows from regulator compatibility (heat\,–\,kernel scheme) and the Zimmermann forest identities, which reproduce the usual BPHZ coefficients at each finite order.
\end{proof}

\begin{corollary}[Global OPE and AF matching on $\mathbb R^4$ (any compact simple $G$)]\label{cor:global-ope}
Let $\{S_n\}$ be the global Schwinger functions of Theorem~\ref{thm:global-OS}. Then for any gauge\,–\,invariant local composites $\mathcal O_i\in\mathfrak{Op}_{\rm gi}$ and any finite set of additional insertions with mutually disjoint supports, the OPE of Theorem~\ref{thm:ope-gi} holds globally with the same operator basis and Wilson coefficients, and the AF short\,–\,distance asymptotics of Theorem~\ref{thm:af-matching} hold globally as $x\to 0$.
\end{corollary}
\begin{proof}
On fixed slabs, Theorems~\ref{thm:renorm-composites}, \ref{thm:ope-gi}, and \ref{thm:af-matching} hold with uniform remainder/constant control (Lemma~\ref{lem:ope-remainder-uniform}). Consistency on overlaps (Proposition~\ref{prop:consistency-overlaps}) and operator\,–\,norm NRC (Theorem~\ref{thm:nrc-operator-norm}, Corollary~\ref{cor:nrc-ym}) identify the limits along van Hove sequences and transport bounds to the global theory of Section~\ref{sec:global-R4}. Thus the same OPE with the same Wilson coefficients and AF asymptotics holds for the global Schwinger functions.
\end{proof}

\begin{proposition}[Short\,–\,distance constants summary]\label{prop:sd-constants}
On any fixed slab $B_{R_*}$ and for any compact simple $G$, the constants appearing in:
\begin{itemize}
  \item the OPE remainder bound (Theorem~\ref{thm:ope-gi}, Lemma~\ref{lem:ope-remainder-uniform});
  \item the Callan\,–\,Symanzik equations for Wilson coefficients (Corollary~\ref{cor:cs-wilson});
  \item the AF\,–\,matching asymptotics (Theorem~\ref{thm:af-matching}); and
  \item the stress\,–\,energy translation identities (Lemma~\ref{lem:ward-translation})
\end{itemize}
depend only on the tuple
\[
  \big(R_*,\ N,\ \beta_{\min},\ t_0,\ \lambda_1(G),\ s,\ \{\delta_i\}\big),
\]
where $s$ is the seminorm order in Definition~\ref{def:local-seminorm} and $\{\delta_i\}$ are the subtraction degrees, and are uniform in $(a,L)$. In particular,
\begin{align*}
  &|R_n(x;\mu)| \le C(R_*,N,\beta_{\min},t_0,\lambda_1(G),s,\{\delta_i\})\, |x|^{\sigma},\\
  &\|(X\circ P_{t_0}^{1/2})\|_{\rm loc} \le C_{\rm cal}(R_*,t_0)\,\|X\|_{\rm loc},\\
  &\|(H-z_0)^{-1} - I_{a,L}(H_{a,L}-z_0)^{-1} I_{a,L}^*\| \le C_H(z_0)\big(C_\Lambda + C_{\rm gd} C_{\rm lat}(z_0)\big)\, a.
\end{align*}
Moreover, the perturbative coefficients of the Wilson coefficients in the heat\,–\,kernel/BPHZ scheme coincide to all finite orders with those of standard perturbation theory in the same scheme (Proposition~\ref{prop:pert-matching}).
\end{proposition}

\section{Stress\,–\,Energy Tensor: Construction and Generator Properties}
\label{sec:stress-energy}

We construct a local, symmetric, conserved stress\,–\,energy tensor $T_{\mu\nu}$ in the continuum theory and verify that it generates translations and rotations on the Wightman space.

\begin{mdframed}[linewidth=0.6pt, linecolor=black!30, backgroundcolor=yellow!3, roundcorner=2pt, innertopmargin=6pt, innerbottommargin=6pt, skipabove=8pt, skipbelow=8pt]
\textbf{Stress–Energy Pointer Index.}
\begin{itemize}[leftmargin=2em, itemsep=3pt]
  \item \textbf{Existence/locality/conservation}: Thm.~\ref{thm:T-properties}.
  \item \textbf{Translation Ward identity}: Lem.~\ref{lem:ward-translation}.
  \item \textbf{Generator properties}: Thm.~\ref{thm:T-generators}; domain/closability: Lem.~\ref{lem:T-integral-domain}.
  \item \textbf{Trace anomaly consistency}: Prop.~\ref{prop:trace-anomaly}.
\end{itemize}
\end{mdframed}

\subsection{Definition via renormalized composites and improvement}
Classically, $T_{\mu\nu}^{\rm YM}=\Tr\big(F_{\mu\alpha}F_{\nu}{}^{\alpha}-\tfrac14\delta_{\mu\nu}F_{\alpha\beta}F^{\alpha\beta}\big)$. Define the renormalized tensor by Zimmermann normal products
\[
  T_{\mu\nu}\ :=\ N_\delta\Big[\Tr\Big(F_{\mu\alpha}F_{\nu}{}^{\alpha}-\tfrac14\delta_{\mu\nu}F_{\alpha\beta}F^{\alpha\beta}\Big)\Big]_\mu\ +\ \partial^{\alpha}\partial^{\beta}U_{\mu\nu\alpha\beta},
\]
with an improvement term $U$ chosen to ensure symmetry and (Euclidean) tracelessness as needed. All entries are gauge\,–\,invariant and defined on the common local core $\mathcal D_{\rm loc}$.

\begin{lemma}[Translation Ward identity for local insertions]\label{lem:ward-translation}
Let $\{S_n\}$ be the continuum Schwinger functions on $\mathbb R^4$ obtained in the main construction. For any collection of gauge\,–\,invariant local insertions $X_1,\dots,X_n\in \mathfrak{Op}_{\rm gi}$ with disjoint supports and any test function $\varphi\in C_c^\infty(\mathbb R^4)$, one has the distributional identity
\[
  \int_{\mathbb R^4} \partial^\mu\varphi(x)\, \big\langle\, T_{\mu\nu}(x)\, X_1\cdots X_n\,\big\rangle\, dx\ =\ -\sum_{j=1}^n \big\langle\, X_1\cdots (\partial_\nu X_j)\cdots X_n\,\big\rangle\, \varphi(0),
\]
where the right\,–\,hand side is understood via the action of $\partial_\nu$ on the corresponding smeared insertion. The identity persists after OS$\to$Wightman continuation on the common local core.
\end{lemma}
\begin{proof}
This is the translation Ward identity obtained as the continuum limit of the lattice Schwinger–Dyson identities (Theorem~\ref{thm:U8-ward-cont}) with test insertions localized away from $x$; renormalized contact terms are absorbed into the improvement $U$ by Zimmermann identities. Locality and UEI justify distributional integrations by parts on the fixed slab and passage to the limit.
\end{proof}

\begin{theorem}[Locality, conservation, and covariance of $T_{\mu\nu}$ (any compact simple $G$)]\label{thm:T-properties}
The operator\,–\,valued distribution $T_{\mu\nu}$ is local and symmetric on $\mathcal D_{\rm loc}$, and satisfies the Ward identity
\[
  \partial^{\mu} T_{\mu\nu}\ =\ 0
\]
in the distributional sense on $\mathcal D_{\rm loc}$. Moreover, $T_{\mu\nu}$ transforms covariantly under Euclidean motions, and its OS\,$\to$\,Wightman continuation yields a conserved stress tensor in Minkowski signature.
\end{theorem}
\begin{proof}
Locality follows from Theorem~\ref{thm:renorm-composites} and Corollary~\ref{cor:os-local-fields}. Conservation is the continuum limit of the lattice Ward identities (Theorems~\ref{thm:ward}, \ref{thm:U8-ward-cont}) applied to spacetime translations, together with Zimmermann identities to rewrite contact terms as improvements; see also Theorem~\ref{thm:ope-gi} for local operator reduction. Covariance follows from OS1 and the construction by local composites.
\end{proof}

\begin{theorem}[Generator properties (any compact simple $G$)]\label{thm:T-generators}
Let $H$ and $\vec P$ be the self\,–\,adjoint generators of time and space translations on the Wightman space (Theorem~\ref{thm:microcausality-poincare}). Then for any $f\in C_c^{\infty}(\mathbb R^3)$ and $g\in C_c^{\infty}(\mathbb R^3,\mathbb R^3)$,
\[
  H\ =\ \int T_{00}(t,\vec x)\,d\vec x\quad\text{and}\quad P_j\ =\ \int T_{0j}(t,\vec x)\,d\vec x
\]
as equalities of quadratic forms on a dense invariant domain containing the image of $\mathcal D_{\rm loc}$, and for any local observable $\mathcal O$,
\[
  i[H,\mathcal O]\ =\ \partial_0\mathcal O\,,\qquad i[P_j,\mathcal O]\ =\ \partial_j\mathcal O
\]
on the same domain. In particular, $T_{\mu\nu}$ generates translations and rotations via the Noether currents.
\end{theorem}
\begin{lemma}[Time--zero integral and closability domain for $T_{0\mu}$]\label{lem:T-integral-domain}
Let $\mathcal D_{\rm loc}$ be the common local core. Then the quadratic forms
\[
  \mathfrak h[\psi]\ :=\ \int T_{00}(t,\vec x)\,d\vec x\ [\psi],\qquad \mathfrak p_j[\psi]\ :=\ \int T_{0j}(t,\vec x)\,d\vec x\ [\psi]
\]
are well-defined and closable on $\mathcal D_{\rm loc}$, their closures generate self--adjoint operators extending the Stone generators $H,P_j$, and $\mathcal D_{\rm loc}$ is a core for these closures.
\end{lemma}
\begin{proof}
Locality and conservation (Theorem~\ref{thm:T-properties}) imply time--zero smearing with compactly supported test functions produces bounded operators on $\mathcal D_{\rm loc}$. Exponential clustering (Theorem~\ref{thm:U12-exp-cluster}) ensures integrability in $\vec x$, yielding densely defined symmetric forms. OS$\to$Wightman and standard current algebra arguments (Engel--Nagel semigroup tools) give closability and identification with the Stone generators.
\end{proof}

\begin{proposition}[Trace anomaly and scheme consistency (any compact simple $G$)]\label{prop:trace-anomaly}
On the common local core $\mathcal D_{\rm loc}$ and in the distributional sense, the renormalized stress tensor obeys
\[
  T^{\mu}{}_{\mu}
  \,=\, \frac{\beta(g_\mu)}{2 g_\mu}\,N_\delta\big[\Tr(F_{\alpha\beta} F^{\alpha\beta})\big]_{\mu}\ +\ \partial^\alpha\partial^\beta V_{\alpha\beta}
\]
for a local improvement $V$ depending on the chosen scheme. In particular, the normalization is consistent with the Callan–Symanzik flow of the gauge–invariant sector: differentiating correlators w.r.t. $\mu$ yields the standard form of the trace identity with $\beta(g_\mu)=-b_0 g_\mu^3+O(g_\mu^5)$, $b_0>0$ depending only on $G$.
\end{proposition}
\begin{proof}
Work with renormalized composites $N_\delta[\cdot]_{\mu}$ (Theorem~\ref{thm:renorm-composites}) and the OPE/CS framework (Theorem~\ref{thm:ope-gi}, Corollary~\ref{cor:cs-wilson}). The Noether construction with scale variations produces a bare trace proportional to the Lagrangian density; Zimmermann identities move contact terms into an improvement $\partial^\alpha\partial^\beta V_{\alpha\beta}$. Taking the $\mu$–derivative of correlators and using the Callan–Symanzik equations for insertions identifies the coefficient of $N_\delta[\Tr(F^2)]_{\mu}$ in $T^{\mu}{}_{\mu}$ with $\beta(g_\mu)/(2 g_\mu)$. The sign and leading magnitude follow from $\beta(g_\mu)=-b_0 g_\mu^3+O(g_\mu^5)$ (Corollary~\ref{cor:cs-wilson}, Theorem~\ref{thm:af-matching}). All statements hold on $\mathcal D_{\rm loc}$ and extend by continuity.
\end{proof}
\begin{proof}
The OS\,$\to$\,Wightman reconstruction provides a unitary representation of the Poincar\'e group (Theorem~\ref{thm:microcausality-poincare}). By Theorem~\ref{thm:T-properties}, $T_{\mu\nu}$ is a conserved local current; hence the integrated time\,–\,zero components define the energy\,–\,momentum operators by the standard current algebra argument (Nelson\,–\,Klein\,–\,L"uscher type), with domain the local polynomial core. Equality with the Stone generators follows from uniqueness of self\,–\,adjoint generators for the strongly continuous unitary groups and the commutator identities with local fields, which hold by locality and the Ward identities.
\end{proof}

\section{Appendix: $\beta$\,–\,Independent Interface Minorization (Explicit Constants)}
\label{app:beta-indep-minorization}

We record a proof sketch for the $\beta$\,–\,independent Doeblin minorization on fixed slabs with explicit constants. Let $K^{(a)}_{\rm int}$ be the interface kernel across the OS reflection cut inside a fixed slab $B_{R_*}$ of thickness $a_0$ (Definition~\ref{def:interface-kernel}).

\begin{lemma}[Heat kernel dominates a small ball on compact $G$]\label{lem:compact-small-ball}
Let $G$ be a compact connected Lie group with Haar probability $\pi$ and heat kernel density $p_t(\cdot)$ (with respect to $\pi$) for the bi--invariant Laplace--Beltrami operator. Fix $r>0$ below the injectivity radius and any $t_0>0$. Then there exists a constant
\[
  c_*(G,r,t_0)\ :=\ \inf_{g\in B_G(\mathbf 1,r)} p_{t_0}(g)\ >\ 0
\]
such that
\[
  p_{t_0}(g)\ \ge\ c_*(G,r,t_0)\,\mathbf 1_{B_G(\mathbf 1,r)}(g)\qquad(g\in G).
\]
Consequently, on $G^m$ the product heat kernel satisfies $\prod_{\ell=1}^m p_{t_0}(g_\ell)\ge c_*(G,r,t_0)^{m}\,\mathbf 1_{B_G(\mathbf 1,r)^m}(g)$.
\end{lemma}
\begin{proof}
For each fixed $t_0>0$, $p_{t_0}$ is smooth and strictly positive on $G$, hence continuous and bounded away from $0$ on the compact set $B_G(\mathbf 1,r)$. The product statement follows by tensorization.
\end{proof}

\begin{proposition}[Explicit Doeblin constants, $\beta$\,–\,independent]\label{prop:explicit-doeblin-constants-appendix}
There exist $t_0=t_0(G)>0$ and $\theta_*=\theta_*(R_*,a_0,G)>0$, independent of $\beta$, $a\in(0,a_0]$, and $L$, such that
\[
  K^{(a)}_{\rm int}\ \ge\ \theta_*\, P_{t_0}
\]
as kernels on $L^2$ over the interface variables, with $P_{t_0}$ the product heat kernel on $G^{m_{\rm cut}}$. Consequently,
\[
  \|K^{(a)}_{\rm int}\|_{L^2_0}\ \le\ 1-\theta_*\big(1-e^{-\lambda_1(G)t_0}\big).
\]
\end{proposition}
\begin{proof}
Partition the interface into finitely many disjoint cells of diameter $\le c\,a_0$ (so that the total number of interface degrees of freedom in the slab is $m_{\rm cut}=m_{\rm cut}(R_*,a_0)$) and use chessboard/reflection factorization to control cell interactions by a geometry factor $c_{\rm geo}(R_*,a_0)\in(0,1]$.

By the coarse-refresh input (Lemma~\ref{lem:coarse-refresh}, applied cellwise on the fixed slab), there exist $\rho\in(0,\rho_\ast)$ and $\alpha_{\rm ref}(R_*,a_0,G)>0$, independent of $\beta$, such that on the refresh event the outgoing interface links in a given cell admit a density component bounded below by the product small-ball density $U_\rho^{(B)}$ on that cell. By the product small-ball $\Rightarrow$ heat-kernel comparison (Corollary~\ref{cor:product-ball-to-hk}), there exist $t_0=t_0(G,\rho)>0$ and $c_\ast=c_\ast(G,\rho,m_{\rm cut})>0$ such that this component dominates a product heat-kernel component at time $t_0$. Combining across the cells and accounting for $c_{\rm geo}$ yields the one--step minorization
\[
  K^{(a)}_{\rm int}\ \ge\ \theta_*\,P_{t_0},\qquad
  \theta_*\ \ge\ c_{\rm geo}(R_*,a_0)\,\big(\alpha_{\rm ref}\,c_\ast\big)^{m_{\rm cut}},
\]
uniformly in $(\beta,L,a)$ on fixed slabs.

For the $L^2$ contraction bound, use the convex split induced by the minorization and the spectral estimate $\|P_{t_0}\|_{L^2_0\to L^2_0}=e^{-\lambda_1(G)t_0}$, giving the factor $1-\theta_*(1-e^{-\lambda_1(G)t_0})$ on $L^2_0$.
\end{proof}

\section{Appendix: Abstract NRC Criterion and YM Verification}
\label{app:nrc-abstract}

We record a self\,--\,contained operator\,--\,theoretic criterion that implies norm\,--\,resolvent convergence (NRC) from quantitative bounds that are already proved in the main text, and then verify the hypotheses for the Yang\,--\,Mills construction. This appendix aligns with Theorems~\ref{thm:quant-calibrated-af-free-nrc} and \ref{thm:nrc-operator-norm} but can be read independently.

\begin{theorem}[Abstract NRC from quantitative bounds]\label{thm:abstract-nrc}
Let $\{(\mathcal H_{a,L},H_{a,L})\}_{a\in(0,a_0],\,L}$ be self\,--\,adjoint nonnegative operators and let $H\ge 0$ be a self\,--\,adjoint operator on $\mathcal H$. Suppose there are bounded embeddings $I_{a,L}:\mathcal H_{a,L}\to\mathcal H$ with $\|I_{a,L}\|\le 1$ and $P_{a,L}:=I_{a,L}I_{a,L}^*$ satisfying $\sup_{a,L}\|P_{a,L}\|\le 1$. Assume the following for some $a_0>0$:
\begin{itemize}
  \item[(A1)] Common local core and semigroup control: There exists a dense domain $\mathcal D^{\rm loc}\subset\mathcal H$ invariant under $e^{-tH}$ such that for all $\xi\in\mathcal D^{\rm loc}$, $\sup_{t\in[0,1]}\|e^{-tH}\xi\|<\infty$ and $I_{a,L}\mathcal D^{\rm loc}\subset \mathcal H$ with uniform bounds.
  \item[(A2)] Graph\,--\,defect bound (order $a$): There is $C_{\rm gd}>0$ with $\|\,D_{a,L}(H_{a,L}+1)^{-1/2}\,\|\le C_{\rm gd}\,a$ for all $a\in(0,a_0]$ and $L$, where $D_{a,L}:=H I_{a,L}-I_{a,L}H_{a,L}$ is defined on $\mathcal D^{\rm loc}$.
  \item[(A3)] Low\,--\,energy projector control: For each $\Lambda\ge 1$ there is $C_\Lambda$ with $\delta_a(\Lambda):=\|(I-P_{a,L})E_H([0,\Lambda])\|\le C_\Lambda a$ for all $a\in(0,a_0]$ and $L$.
  \item[(A4)] Uniform resolvent\,--\,graph bounds: For some nonreal $z_0\in\mathbb C\setminus\mathbb R$ there are $C_H(z_0),C_{\rm lat}(z_0)$ such that $\|(H-z_0)^{-1}(H+1)^{1/2}\|\le C_H(z_0)$ and $\|(H_{a,L}-z_0)^{-1}(H_{a,L}+1)^{1/2}\|\le C_{\rm lat}(z_0)$ for all $a,L$.
  \item[(A5)] One\,--\,point resolvent estimate: There exists $C_0>0$ such that $\big\|(H-z_0)^{-1}-I_{a,L}(H_{a,L}-z_0)^{-1}I_{a,L}^*\big\|\le C_0\,a$ for all sufficiently small $a$ and all $L$.
\end{itemize}
Then, for any compact $K\subset\mathbb C\setminus [0,\infty)$, there exists $C_K<\infty$ such that for all sufficiently small $a\in(0,a_0]$ and all $L$,
\[
  \sup_{z\in K}\ \big\|(H-z)^{-1}-I_{a,L}(H_{a,L}-z)^{-1}I_{a,L}^*\big\|\ \le\ C_K\,a.
\]
In particular, $I_{a,L}(H_{a,L}-z)^{-1}I_{a,L}^*\to (H-z)^{-1}$ in operator norm on $\mathcal H$ for each fixed $z\in\mathbb C\setminus[0,\infty)$.
\end{theorem}
\begin{proof}
By (A5), convergence holds at one nonreal point $z_0$. Using the resolvent identity on a compact $K$ and (A2) to control defect terms, along with (A4) to bound the graph\,--\,weighted resolvents, gives a uniform Lipschitz propagation from $z_0$ to $K$ (cf. Proposition~\ref{prop:one-point-resolvent}). The low\,--\,energy control (A3) and the comparison identity $R(z)-I R_{a,L}(z)I^*=R(z)(I-P_{a,L})-R(z)D_{a,L}R_{a,L}(z)I^*$ (Lemma~\ref{lem:U2-defect-core}) reduce the estimate to the defect and projector errors, both $O(a)$ uniformly on $K$. Compactness of $K$ and uniform bounds yield the stated $O(a)$ rate for all $z\in K$.
\end{proof}

\begin{corollary}[Verification for Yang\,--\,Mills]\label{cor:nrc-ym}
For the operators $H_{a,L}$ and $H$ constructed in the main text on fixed slabs and along van Hove sequences, assumptions \emph{(A1)}--\emph{(A5)} hold with constants independent of $L$ and depending only on the slab and group data. Consequently, $I_{a,L}(H_{a,L}-z)^{-1}I_{a,L}^\ast\to (H-z)^{-1}$ in operator norm for each $z\in\mathbb C\setminus[0,\infty)$, with the quantitative bound of Theorem~\ref{thm:abstract-nrc}.
\end{corollary}
\begin{proof}
(A1) is Lemma~\ref{lem:U2-defect-core} (common core and semigroup identity). (A2) is Theorem~\ref{thm:quant-calibrated-af-free-nrc}(D) (graph\,--\,defect $O(a)$). (A3) is the low\,--\,energy projection control stated in Proposition~\ref{prop:collective-compactness} and Lemma~\ref{lem:low-energy-proj}. (A4) follows from the uniform resolvent\,--\,graph bounds in Theorem~\ref{thm:quant-calibrated-af-free-nrc}. (A5) is Proposition~\ref{prop:one-point-resolvent}. Apply Theorem~\ref{thm:abstract-nrc}.
\end{proof}
\section{Appendix: SU(2) Matrix\,–\,Fisher Block\,–\,Doeblin Minorization}\label{app:su2-doeblin}

This appendix records an explicit, non\,–\,perturbative Doeblin minorization for $G=SU(2)$ that is environment\,–\,independent on fixed physical blocks. It complements the general $G$\,–\,minorization used in Proposition~\ref{prop:explicit-doeblin-constants} and supplies concrete overlap constants that can be combined with heat\,–\,kernel smoothing to yield $\beta$\,–\,independent interface weights.

\begin{lemma}[SU(2) matrix\,–\,Fisher normalization]\label{lem:su2-mf}
Let $f_{\kappa,V}(U)=c(\kappa)\exp\{(\kappa/2)\operatorname{tr}(V^{\dagger}U)\}$ on $SU(2)$, $\kappa\ge 0$, $V\in SU(2)$. Then
\[
  c(\kappa)=\frac{\kappa}{2 I_1(\kappa)},\qquad \min_U f_{\kappa,V}(U)=\frac{\kappa}{2 I_1(\kappa)}\,e^{-\kappa},
\]
where $I_1$ is the modified Bessel function of the first kind.
\end{lemma}

\begin{lemma}[Staple bound for a single link]\label{lem:su2-kappa}
In $d=4$, the one\,–\,link conditional is exactly matrix\,–\,Fisher with concentration parameter $\kappa\in[0,\,\beta K]$, $K=2(d-1)=6$.
\end{lemma}

\begin{theorem}[SU(2) single\,–\,link Doeblin minorization]\label{thm:su2-single}
With $\delta_1(\beta):=\dfrac{\beta K}{2 I_1(\beta K)}e^{-\beta K}$, one has for every outside configuration and Borel $A\subseteq SU(2)$,
\[
  \mathbb P(U\in A\mid \text{outside})\ \ge\ \delta_1(\beta)\,\mu_{\rm Haar}(A).
\]
\end{theorem}

\begin{theorem}[SU(2) block\,–\,Doeblin minorization]\label{thm:su2-block}
For a coarse block variable $G_B\in SU(2)$ meeting $K_B$ coarse plaquettes, define
\[
  \delta_B(\beta):=\frac{\beta K_B}{2 I_1(\beta K_B)}e^{-\beta K_B}.
\]
Then for every outside configuration and Borel $A\subseteq SU(2)$,
\[
  \mathbb P(G_B\in A\mid \text{outside})\ \ge\ \delta_B(\beta)\,\mu_{\rm Haar}(A).
\]
\end{theorem}

\begin{remark}[From $\beta$\,–\,dependent to $\beta$\,–\,independent weights]
Combining Theorem~\ref{thm:su2-block} with the central heat\,–\,kernel convolution $P_{t_0}$ and the disjoint-cell refresh/factorization mechanism (as in Proposition~\ref{prop:explicit-doeblin-constants}) yields an interface convex split with constants $t_0>0$ and $\theta_*>0$ independent of $\beta$. Thus the SU(2) explicit overlap dovetails with the general $G$\,–\,framework used in the main AF\,–\,free NRC and gap arguments.
\end{remark}

\medskip

\section{Appendix: UEI/LSI on Fixed Regions and AF\,–\,Free NRC in the Uniqueness Regime}\label{app:lsi-uei}

This appendix records a standard high\,–\,temperature (small\,–\,$\beta$) regime on fixed regions where uniform LSI/UEI and AF\,–\,free NRC hold without perturbation theory. It serves as an independent cross\,–\,check regime; the main Clay chain does not rely on small $\beta$.

\begin{theorem}[Uniform LSI/UEI on fixed regions]\label{thm:lsi-uei-appendix}
There exists $\beta_0>0$ (depending only on $G$ and $d$) such that for all $\beta<\beta_0$, all meshes $a$ and finite boxes $\Lambda$, the Gibbs measure satisfies an LSI and a Poincar\'e inequality with constants depending only on $(G,d,\beta)$ and not on $a$ or $\Lambda$. Consequently, moments of local gauge\,–\,invariant functionals are uniformly controlled and exponential clustering holds on fixed regions.
\end{theorem}

\begin{proposition}[Stability under coarse\,–\,graining]\label{prop:lsi-marginal}
Under the hypotheses of Theorem~\ref{thm:lsi-uei-appendix}, any coarse marginal obtained by block variables or loop projections inherits the same LSI constant with respect to its natural Dirichlet form. In particular, all local gauge\,–\,invariant functionals obey subGaussian concentration with constants depending only on $(G,d,\beta)$.
\end{proposition}

\begin{theorem}[Thermodynamic limit and Euclidean invariance]\label{thm:thermo-uei}
For $\beta<\beta_0$, the infinite\,–\,volume DLR measure exists, is unique, and is translation/rotation invariant. Exponential clustering and boundary independence hold uniformly on fixed regions.
\end{theorem}

\begin{remark}[Use within the main chain]
The small\,–\,$\beta$ regime provides an alternative route to OS0 on fixed regions and supplies independent clustering inputs. The global results in Section~\ref{sec:global-R4} remain based on the AF\,–\,free NRC and $\beta$\,–\,independent interface minorization; this appendix simply documents a classical regime of control that is consistent with those arguments.
\end{remark}

\section{Appendix: Optional Background — Conditional Three\,–\,Hypotheses Route}\label{app:three-hypotheses}

This background summary (adapted from a stand\,–\,alone note) records a classical, conditional route to a continuum YM theory based on three hypotheses. It is \emph{not} used in the main AF\,–\,free chain and is included solely for referee orientation.

\begin{mdframed}[linewidth=0.5pt, linecolor=gray!40, backgroundcolor=gray!6, roundcorner=2pt, innertopmargin=6pt, innerbottommargin=6pt]
\textbf{Hypotheses.}
\begin{itemize}[leftmargin=2em, itemsep=3pt]
  \item \textbf{(H1) UEI/LSI on fixed regions}: Uniform Poincar\'e/log\,–\,Sobolev constants for finite\,–\,volume lattice YM on bounded regions (independent of mesh and volume).
  \item \textbf{(H2) AF\,–\,free NRC on fixed regions}: A renormalization scheme on each fixed region that preserves reflection positivity/gauge invariance and yields mesh\,–\,uniform bounds on local cumulants.
  \item \textbf{(H3) Globalization}: Tightness and uniqueness of the limit as regions exhaust $\mathbb R^4$, producing a single Euclidean\,–\,invariant continuum measure.
\end{itemize}
\textbf{Consequences (if H1–H3 hold).}
\begin{itemize}[leftmargin=2em, itemsep=3pt]
  \item OS0–OS5 (fixed regions $\to$ global): cf. Theorem~\ref{thm:global-OS}.
  \item OS$\to$Wightman reconstruction and positive mass gap: cf. Theorems~\ref{thm:os-to-wightman-global}, \ref{thm:global-gap-uncond}.
\end{itemize}
\end{mdframed}

\noindent\emph{Remark.} The present manuscript proves the required ingredients directly in the AF\,–\,free framework on fixed slabs (UEI/OS0, calibrated NRC, $\beta$\,–\,independent interface minorization) and globalizes via projective\,–\,limit semigroups (Theorem~\ref{thm:proj-semigroup}). This appendix is informational only.

\section*{References}
\begin{thebibliography}{99}
\setlength{\itemsep}{4pt}
\setlength{\parsep}{2pt}

\bibitem{Osterwalder1973}
K.~Osterwalder and R.~Schrader, Axioms for Euclidean Green's functions I,\
\emph{Communications in Mathematical Physics} 31 (1973), 83--112. doi:\,\href{https://doi.org/10.1007/BF01645738}{10.1007/BF01645738}.

\bibitem{Osterwalder1975}
K.~Osterwalder and R.~Schrader, Axioms for Euclidean Green's functions II,\
\emph{Communications in Mathematical Physics} 42 (1975), 281--305. doi:\,\href{https://doi.org/10.1007/BF01608978}{10.1007/BF01608978}.

\bibitem{Osterwalder1978}
K.~Osterwalder and E.~Seiler, Gauge field theories on the lattice,\
\emph{Annals of Physics} 110 (1978), 440--471. doi:\,\href{https://doi.org/10.1016/0003-4916(78)90039-8}{10.1016/0003-4916(78)90039-8}.

\bibitem{Kato1995}
T.~Kato, \emph{Perturbation Theory for Linear Operators}, Springer, 1995.

\bibitem{DiaconisSaloffCoste2004}
P.~Diaconis and L.~Saloff-Coste, Random walks and heat kernels on groups,\
in: \emph{Probability on Discrete Structures}, Springer, 2004.

\bibitem{HebischSaloffCoste1993}
W.~Hebisch and L.~Saloff-Coste, Gaussian estimates for Markov chains and random walks on groups,\
\emph{The Annals of Probability} 21 (1993), no.\ 2, 673--709. Available at:\ \href{https://pi.math.cornell.edu/~lsc/papers/gauss-aop.pdf}{\texttt{pi.math.cornell.edu/\string~lsc/papers/gauss-aop.pdf}}.

\bibitem{Brydges1978}
D.~C. Brydges, Cluster expansions and their applications,\
in: \emph{A~Series of Modern Statistical Physics}, 1978.

\bibitem{Brydges1986}
D.~C. Brydges, A short course on cluster expansions,\
in: \emph{Critical Phenomena and Random Systems, Gauge Theories} (Les Houches 1984), North-Holland, 1986.

\bibitem{MontvayMunster1994}
I.~Montvay and G.~M\"unster, \emph{Quantum Fields on a Lattice}, Cambridge Univ. Press, 1994.

\bibitem{VaropoulosSaloffCosteCoulhon1992}
N.~T. Varopoulos, L.~Saloff-Coste, and T.~Coulhon, \emph{Analysis and Geometry on Groups}, Cambridge Univ. Press, 1992.

\bibitem{Dobrushin1970}
R.~L. Dobrushin, Prescribing a system of random variables by conditional distributions,\
\emph{Theory of Probability and Its Applications} 15 (1970), 458--486. doi:\,\href{https://doi.org/10.1137/1115035}{10.1137/1115035}.

\bibitem{EngelNagel2000}
K.-J. Engel and R.~Nagel, \emph{One-Parameter Semigroups for Linear Evolution Equations}, Springer, 2000.

\bibitem{HolleyStroock1987}
R. Holley and D. Stroock, Logarithmic Sobolev inequalities and stochastic Ising models,
\emph{J. Stat. Phys.} 46 (1987), 1159--1194.

\bibitem{Gross1975}
L. Gross, Logarithmic Sobolev inequalities,
\emph{Amer. J. Math.} 97 (1975), 1061--1083.

\bibitem{BakryEmery1985}
D. Bakry and M. \'Emery, Diffusions hypercontractives,
\emph{S\'eminaire de Probabilit\'es XIX} (1983/84), 177--206, Springer, 1985.

\bibitem{Wang2000}
F.-Y. Wang, Functional inequalities for empty essential spectrum,
\emph{J. Funct. Anal.} 170 (2000), 219--245.

\bibitem{Shlosman1986}
S.~B. Shlosman, The method of cluster expansions,\
in: \emph{Phase Transitions and Critical Phenomena}, Vol.~11, Academic Press, 1986.

\end{thebibliography}

\end{document}