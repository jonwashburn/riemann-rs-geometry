\documentclass[11pt,a4paper]{article}

% ----------------------------------------------------------------------
% Preamble
% ----------------------------------------------------------------------

% Typography and Language
\usepackage[utf8]{inputenc}
\usepackage[T1]{fontenc}
\usepackage[english]{babel}
\usepackage{lmodern}
\usepackage{microtype}

% Mathematics
\usepackage{amsmath}
\usepackage{amssymb}
\usepackage{amsthm}
\usepackage{mathtools}

% Layout
\usepackage[margin=1.0in]{geometry}
\usepackage{fancyhdr}
\usepackage{parskip}
\setlength{\emergencystretch}{1em}

% Links and References
\usepackage[colorlinks=true, linkcolor=blue, citecolor=blue, urlcolor=blue]{hyperref}
% NOTE: `cleveref` is not available in the minimal TeXLive install used by CI.
% Provide a lightweight fallback so the document still compiles.
\newcommand{\cref}[1]{\ref{#1}}
\newcommand{\Cref}[1]{\ref{#1}}

% Code Listings (for Lean code)
\usepackage{listings}
\usepackage{xcolor}

\definecolor{keywordcolor}{rgb}{0.5,0,0.3}
\definecolor{commentcolor}{rgb}{0.25,0.5,0.35}
\definecolor{stringcolor}{rgb}{0.6,0,0}
\definecolor{backgroundcolor}{rgb}{0.98,0.98,0.98}
\definecolor{lightgray}{gray}{0.85}

\lstdefinelanguage{Lean}{
  keywords={class, instance, where, structure, theorem, def, noncomputable, example, axiom, inductive, deriving, open, namespace, variable, section, end, import, opaque, lemma, Prop, Type, Set},
  keywordstyle=\color{keywordcolor}\bfseries,
  ndkeywords={Filter, FilterBasis, Equivalence, Setoid, Quotient, Function, Fin, Nat, Int, Real, Nonempty, True, False},
  ndkeywordstyle=\color{blue}\bfseries,
  identifierstyle=\color{black},
  sensitive=true,
  comment=[l]{--},
  morecomment=[s]{/-}{-/},
  commentstyle=\color{commentcolor}\itshape,
  stringstyle=\color{stringcolor},
  morestring=[b]",
  basicstyle=\ttfamily\small,
  backgroundcolor=\color{backgroundcolor},
  frame=single,
  rulecolor=\color{lightgray},
  breaklines=true,
  breakatwhitespace=true,
  showstringspaces=false,
  captionpos=b,
  literate={∀}{$\forall$}1 {∃}{$\exists$}1 {→}{$\to$}1 {←}{$\leftarrow$}1 
           {↔}{$\leftrightarrow$}1 {∧}{$\land$}1 {∨}{$\lor$}1 
           {¬}{$\neg$}1 {≠}{$\neq$}1 {≤}{$\leq$}1 {≥}{$\geq$}1
           {⊆}{$\subseteq$}1 {∈}{$\in$}1 {∉}{$\notin$}1 {∩}{$\cap$}1 {∪}{$\cup$}1
           {×}{$\times$}1 {⊗}{$\otimes$}1 {⟨}{$\langle$}1 {⟩}{$\rangle$}1
           {ℕ}{$\mathbb{N}$}1 {ℤ}{$\mathbb{Z}$}1 {ℝ}{$\mathbb{R}$}1
           {λ}{$\lambda$}1 {∘}{$\circ$}1 {⁻¹}{${}^{-1}$}1
           {₀}{${}_{0}$}1 {₁}{${}_{1}$}1 {₂}{${}_{2}$}1
           {₃}{${}_{3}$}1 {₄}{${}_{4}$}1 {₅}{${}_{5}$}1
           {₆}{${}_{6}$}1 {₇}{${}_{7}$}1 {₈}{${}_{8}$}1
           {₉}{${}_{9}$}1
}

\lstset{breaklines=true, breakatwhitespace=true}

% Theorem Environments
\theoremstyle{plain}
\newtheorem{theorem}{Theorem}[section]
\newtheorem{lemma}[theorem]{Lemma}
\newtheorem{corollary}[theorem]{Corollary}
\newtheorem{proposition}[theorem]{Proposition}
\newtheorem{conjecture}[theorem]{Conjecture}
\newtheorem{axiomenv}{Axiom}

\theoremstyle{definition}
\newtheorem{definition}[theorem]{Definition}
\newtheorem{example}[theorem]{Example}
\newtheorem{remark}[theorem]{Remark}

\numberwithin{equation}{section}

% Custom Commands
\newcommand{\recog}{\mathcal{R}}
\newcommand{\config}{\mathcal{C}}
\newcommand{\events}{\mathcal{E}}
\newcommand{\ledger}{\mathcal{L}}
\newcommand{\neighborhood}{\mathcal{N}}
\newcommand{\quotientspace}{\config_{\recog}}
\newcommand{\indist}{\sim_{\recog}}
\newcommand{\Rhat}{\hat{R}}

% Header and Footer
\pagestyle{fancy}
\fancyhf{}
\lhead{Recognition Geometry}
\rhead{Washburn}
\cfoot{Page \thepage\ of \pageref{LastPage}}

% ----------------------------------------------------------------------
% Title Page Information
% ----------------------------------------------------------------------

\title{\textbf{\Huge Recognition Geometry} \\ \vspace{0.5cm} \Large A Complete Mathematical Framework \\ \vspace{0.3cm} \normalsize Formalized in Lean 4}

\author{
    \textbf{Jonathan Washburn} \\
    Recognition Physics Institute \\
    \texttt{jwashburn@recognitionphysics.org}
}

\date{December 2025}

% ----------------------------------------------------------------------
% Document Body
% ----------------------------------------------------------------------

\begin{document}

\maketitle

\begin{abstract}
\noindent
Recognition Geometry is a new geometric framework that inverts the traditional relationship between space and measurement. In classical geometry, space is primitive and measurements are operations performed on a pre-existing spatial substrate. Recognition Geometry reverses this: \textbf{recognition maps are primitive, and space emerges as a quotient structure}.

Assuming axioms RG0--RG7, we present the complete foundations of the subject, prove the universal property of the recognition quotient, develop recognition charts and dimension, and establish a bridge to Recognition Science physics.

We formally define a \textit{Configuration Space} (RG0) and a \textit{Locality Structure} (RG1), upon which \textit{Recognizers} (RG2) operate to produce observable \textit{Events}. We define the fundamental \textit{Indistinguishability Relation} (RG3) and construct the \textit{Recognition Quotient}, proving that it captures exactly the observable structure determined by $R$. We further develop \textit{Finite Resolution} (RG4), \textit{Local Regularity} (RG5), \textit{Composite Recognizers} (RG6), and \textit{Comparative Recognizers} (RG7), showing that when finite-resolution recognizers exist, injectivity fails and smooth structure can only arise as a refinement limit.

Finally, we illustrate how these results ground Recognition Science explanations of spacetime dimension, gauge symmetries, and metric structure. \textbf{Lean formalizations accompany this work}: the RG0--RG7 core is maintained in a companion library, and this repository contains the RH instantiation under \texttt{RiemannRecognitionGeometry/} (37 project modules, 17{,}082 lines).
\end{abstract}

\vspace{0.5cm}

\tableofcontents

\newpage

% ----------------------------------------------------------------------
% Conventions and Notation (unnumbered)
% ----------------------------------------------------------------------

\section*{Conventions and Notation}
\addcontentsline{toc}{section}{Conventions and Notation}

\begin{itemize}
    \item \textbf{Sets and Types}: We work in a set-theoretic style; symbols like $\config$, $\events$ denote sets (Lean \texttt{Type*}s).
    \item \textbf{Recognizers}: A recognizer is a function $R: \config \to \events$. Indistinguishability $c_1 \indist c_2$ means $R(c_1)=R(c_2)$.
    \item \textbf{Recognition Quotient}: $\quotientspace = \config / \indist$, projection $\pi$ and induced $\bar{R}$ as in the text.
    \item \textbf{Neighborhoods}: $\neighborhood$ satisfies RG1 (reflexivity, nonemptiness, intersection closure, refinement).
    \item \textbf{Lean Listings}: Unicode is used; names may differ slightly from prose when convenient.
    \item \textbf{Numbering}: Equations are numbered by section; parts organize themes.
\end{itemize}

% ======================================================================
% PART I: PHILOSOPHICAL FOUNDATIONS
% ======================================================================

\part{Philosophical Foundations}

\section{The Inversion of Geometry}

Geometry, for over two millennia, has been founded on the primacy of space. From Euclid's points and lines to Riemann's manifolds and modern topology, the mathematical narrative begins with a substrate---a set of points equipped with structure---upon which objects reside and measurements occur. In this classical paradigm, ``space'' is the stage, and ``physics'' is the play enacted upon it. Measurement is modeled as a function $f: M \to \mathbb{R}$, mapping a pre-existing point in the manifold $M$ to an observable value. The existence of the point $p \in M$ is ontologically prior to the measurement $f(p)$.

Recognition Geometry proposes a fundamental inversion of this paradigm. We posit that \textbf{recognition is primitive, and space is derived}.

\subsection{The Classical Paradigm}

In the standard formulation of mathematical physics, one begins by postulating a state space or spacetime manifold $M$. This manifold is equipped with a topology $\mathcal{T}$, a differential structure $\mathcal{A}$, and perhaps a metric $g$. Only after this heavy mathematical machinery is in place do we introduce ``observables'' or ``measurements'' as functions on this space.

The ontological commitment here is substantial: one assumes the existence of a continuum of points, most of which are unobservable in principle. The topology dictates which points are ``close'' to one another before any notion of measurement resolution is introduced. The metric defines distances between points that may never be physically distinguished. In this view, the limit of infinite precision is the starting point, and experimental limitations are viewed as approximations or noise clouding the ``true'' underlying continuum.

Historically, this approach traces back to Euclidean geometry, where spatial intuition was formalized as axioms about points and lines. It evolved through Descartes' coordinate geometry, where space became $\mathbb{R}^n$, and culminated in the manifold theory of General Relativity. Even in Quantum Mechanics, the underlying Hilbert space is a continuum structure defined over the field of complex numbers. The assumption of a pre-existing, continuous substrate is ubiquitous.

\subsection{The Recognition Paradigm}

Recognition Geometry begins from a more modest and operationally grounded starting point. We assume only the existence of a set of \textit{configurations} $\config$---representing the possible states of reality---and a family of \textit{recognizers}. A recognizer is a map $R: \config \to \events$ that assigns an observable \textit{event} to each configuration.

Crucially, we do not assume that $\config$ has any topological or metric structure initially. It is merely a set. The structure of ``space'' emerges entirely from the behavior of the recognizers.

The central insight can be stated as follows:
\begin{quote}
    \textbf{Configurations are what the world does; events are what recognizers see.}
\end{quote}

Two configurations $c_1, c_2 \in \config$ are \textit{indistinguishable} with respect to a recognizer $R$ if they produce the same event: $R(c_1) = R(c_2)$. This defines an equivalence relation $\indist$. The ``observable space'' is not $\config$ itself, but the quotient space $\quotientspace = \config / \indist$.

In this framework:
\begin{itemize}
    \item \textbf{Topology} is not assumed; it emerges as the structure of distinguishable neighborhoods.
    \item \textbf{Dimension} is not a parameter; it is the count of independent recognizers needed to distinguish local configurations.
    \item \textbf{Continuity} is not foundational; it is an emergent property of how recognizers respond to variations in configuration.
    \item \textbf{Metrics} are not primitive; distances emerge from \textit{comparative} recognition---the ability to order configurations.
\end{itemize}

This inversion has profound physical consequences. It naturally accommodates the finite resolution of physical measurement not as an approximation, but as a fundamental feature. If the set of possible events $\events$ is finite (or countable) in any local region---as suggested by information-theoretic bounds in physics---then the emergent geometry is necessarily discrete or granular at the fundamental level, smoothing out into a continuum only in the limit of large numbers.

\subsection{Comparison with Related Approaches}

Recognition Geometry shares motivations with several existing frameworks but differs in key respects:

\begin{enumerate}
    \item \textbf{Topos Theory and Logic}: Topos approaches to physics also emphasize the role of observation and the logic of partial information. However, they often operate at a very high level of abstraction (category theory). Recognition Geometry remains grounded in set-theoretic operations and concrete maps, making it more accessible as a direct model of physical measurement.
    
    \item \textbf{Information Geometry}: This field studies the geometry of probability distributions. While related, Information Geometry typically assumes a smooth manifold of parameters. Recognition Geometry asks where that manifold comes from in the first place.
    
    \item \textbf{Relational Quantum Mechanics (RQM)}: RQM posits that states are relative to the observer. Recognition Geometry formalizes ``observer'' as ``recognizer'' and provides the rigorous mathematical machinery (quotients, composition of recognizers) to make this relational view precise.
    
    \item \textbf{Loop Quantum Gravity (LQG)}: LQG predicts discrete area and volume operators. Recognition Geometry arrives at discreteness via a different path: the \textit{Finite Resolution Axiom} (RG4), which asserts that any local recognizer has a finite event space.
\end{enumerate}

\subsection{The Ontological Commitment}

It is important to clarify what Recognition Geometry commits to ontologically. It is neither purely operationalist (denying the existence of anything beyond measurement) nor purely Platonist (asserting the independent reality of mathematical forms).

We commit to the existence of \textit{configurations}---there is a ``way things are.'' However, we treat the configuration space $\config$ as a vast, potentially infinite-dimensional ``ledger'' that is not directly accessible. What is physically real for an observer is the \textit{recognition quotient} $\quotientspace$.

Space, time, and metrics are emergent structural features of the interaction between the configuration ledger and the recognition operators. They are objective features of the \textit{observable} world, but they are derivative, not primitive. The ``illusion'' of a smooth, pre-existing spacetime container is exactly that---a coherent emergent structure arising from the collective behavior of fundamental recognizers.

This paper formalizes this vision. We begin with minimal axioms and build the entire geometric edifice---topology, charts, dimension, and metrics---from the ground up.

% ======================================================================
% PART II: THE AXIOMATIC FRAMEWORK
% ======================================================================

\part{The Axiomatic Framework}

In this part, we construct the formal edifice of Recognition Geometry from first principles. We begin with minimal set-theoretic assumptions and progressively introduce structure through explicit axioms (RG0--RG7).

\section{Configuration and Event Spaces}

The foundation of our framework rests on two primitive sets: a space of states and a space of outcomes.

\subsection{Configuration Space (RG0)}

We postulate the existence of a set $\config$ of \textit{configurations}. A configuration $c \in \config$ represents a complete, precise specification of the state of the system (or universe). This specification may be vastly more detailed than any possible observation.

\begin{axiomenv}[RG0: Nonempty Configuration Space]
    There exists a set $\config$, called the configuration space, which is nonempty.
\end{axiomenv}

In Lean 4, we formalize this as a type class:

\begin{lstlisting}[language=Lean]
class ConfigSpace (C : Type*) where
  nonempty : Nonempty C

noncomputable def ConfigSpace.witness (C : Type*) [cs : ConfigSpace C] : C :=
  cs.nonempty.some

theorem config_exists (C : Type*) [ConfigSpace C] : ∃ c : C, True :=
  ⟨ConfigSpace.witness C, trivial⟩
\end{lstlisting}

\begin{remark}
    We explicitly do \textit{not} assume that $\config$ carries any initial topological or metric structure. It might be an infinite-dimensional vector space, a discrete set of graphs, or the ``ledger'' states of Recognition Science. It is simply the set of all possible ways the world can be.
\end{remark}

\subsection{Event Spaces}

Recognizers map configurations to \textit{events}. An event is an observable outcome: a pointer reading, a detector click, a boolean value, or a distinctive pattern.

\begin{definition}[Event Space]
    An \textit{Event Space} is a set $\events$ with at least two distinct elements.
\end{definition}

\begin{lstlisting}[language=Lean]
class EventSpace (E : Type*) where
  nontrivial : ∃ e₁ e₂ : E, e₁ ≠ e₂

theorem event_nontrivial (E : Type*) [EventSpace E] : ∃ e₁ e₂ : E, e₁ ≠ e₂ :=
  EventSpace.nontrivial
\end{lstlisting}

The condition $|\events| \ge 2$ is required to avoid triviality; a recognizer that always outputs the same event conveys no information and defines no geometry.

\subsection{The Recognition Triple}

We bundle these concepts into a foundational structure.

\begin{definition}[Recognition Triple]
    A Recognition Triple is a tuple $(\config, \events, \Sigma)$ where $\config$ is a configuration space, $\events$ is an event space, and $\Sigma$ represents the structure connecting them (recognizers, locality, etc.).
\end{definition}

\begin{lstlisting}[language=Lean]
structure RecognitionTriple where
  Config : Type*
  Event : Type*
  configSpace : ConfigSpace Config
  eventSpace : EventSpace Event
\end{lstlisting}

\section{Locality Structure (RG1)}

While we do not assume a topology on $\config$, we require a notion of \textit{locality}. Measurements are local operations; a thermometer in London does not respond to temperature changes in Tokyo. We formalize this via a neighborhood structure that is weaker than a topology but sufficient to define local behavior.

\subsection{Axiom RG1: Locality}

We associate with each configuration $c \in \config$ a family of subsets $\neighborhood(c)$ called \textit{neighborhoods}.

\begin{axiomenv}[RG1: Local Configuration Space]
    A \textit{Local Configuration Space} is a configuration space equipped with a neighborhood assignment $\neighborhood: \config \to \mathcal{P}(\mathcal{P}(\config))$ satisfying:
    \begin{enumerate}
        \item \textbf{Reflexivity}: $\forall c, \forall U \in \neighborhood(c), c \in U$.
        \item \textbf{Nonemptiness}: $\forall c, \neighborhood(c) \neq \emptyset$.
        \item \textbf{Intersection Closure}: If $U, V \in \neighborhood(c)$, there exists $W \in \neighborhood(c)$ such that $W \subseteq U \cap V$.
        \item \textbf{Refinement}: If $U \in \neighborhood(c)$ and $c' \in U$, there exists $V \in \neighborhood(c')$ such that $V \subseteq U$.
    \end{enumerate}
\end{axiomenv}

\begin{lstlisting}[language=Lean]
structure LocalConfigSpace (C : Type*) extends ConfigSpace C where
  N : C → Set (Set C)
  mem_of_mem_N : ∀ c U, U ∈ N c → c ∈ U
  N_nonempty : ∀ c, (N c).Nonempty
  intersection_closed : ∀ c U V, U ∈ N c → V ∈ N c → 
    ∃ W ∈ N c, W ⊆ U ∩ V
  refinement : ∀ c U c', U ∈ N c → c' ∈ U → 
    ∃ V ∈ N c', V ⊆ U
\end{lstlisting}

\subsection{Filter Basis}

The properties of $\neighborhood(c)$ imply that the neighborhoods of any point form a \textit{filter base}. This allows us to define convergence and continuity without a full topological definition, although in many cases $\neighborhood$ will generate a topology.

\begin{theorem}[Common Refinement]
    For any two neighborhoods $U, V$ of $c$, there exists a neighborhood $W$ of $c$ contained in both.
\end{theorem}

\begin{lstlisting}[language=Lean]
theorem LocalConfigSpace.common_refinement (c : C) (U V : Set C)
    (hU : U ∈ L.N c) (hV : V ∈ L.N c) :
    ∃ W ∈ L.N c, W ⊆ U ∧ W ⊆ V
\end{lstlisting}

The \textit{Refinement} property (condition 4) is particularly crucial: it ensures that ``being in a neighborhood'' is a locally stable property. If you are in a neighborhood of $c$, you have your own neighborhood contained entirely within it.

\section{Recognition Maps (RG2)}

We now introduce the central actor of the theory: the \textit{recognizer}.

\subsection{The Recognizer Structure}

\begin{axiomenv}[RG2: Recognizers]
    A \textit{Recognizer} is a function $R: \config \to \events$ that is nontrivial, meaning it distinguishes at least two configurations.
\end{axiomenv}

\begin{lstlisting}[language=Lean]
structure Recognizer (C : Type*) (E : Type*) where
  R : C → E
  nontrivial : ∃ c₁ c₂ : C, R c₁ ≠ R c₂
\end{lstlisting}

The nontriviality condition ensures that $\text{Im}(R)$ contains at least two elements. A constant function is not a recognizer because it performs no recognition---it is ``blind'' to all differences.

\begin{theorem}[Image Nontriviality]
    Every recognizer has at least two distinct events in its image.
\end{theorem}

\begin{lstlisting}[language=Lean]
theorem Recognizer.image_nontrivial (r : Recognizer C E) :
    ∃ e₁ e₂ : E, e₁ ∈ Set.range r.R ∧ e₂ ∈ Set.range r.R ∧ e₁ ≠ e₂
\end{lstlisting}

\subsection{Fibers and Preimages}

The \textit{fiber} of an event $e \in \events$ is the set of all configurations that map to $e$:
\[ R^{-1}(\{e\}) = \{ c \in \config \mid R(c) = e \} \]

These fibers partition the configuration space. This partition is the primary geometric structure induced by the recognizer.

\begin{theorem}[Fibers Partition]
    Every configuration belongs to exactly one fiber.
\end{theorem}

\begin{lstlisting}[language=Lean]
def Recognizer.fiber (r : Recognizer C E) (e : E) : Set C :=
  r.R ⁻¹' {e}

theorem Recognizer.fibers_partition (r : Recognizer C E) :
    ∀ c : C, ∃! e : E, c ∈ r.fiber e
\end{lstlisting}

\section{Indistinguishability (RG3)}

The most fundamental relation in Recognition Geometry is \textit{indistinguishability}.

\subsection{The Indistinguishability Relation}

\begin{axiomenv}[RG3: Indistinguishability]
    Two configurations $c_1, c_2$ are \textit{indistinguishable} with respect to $R$, denoted $c_1 \indist c_2$, if and only if $R(c_1) = R(c_2)$.
\end{axiomenv}

\begin{lstlisting}[language=Lean]
def Indistinguishable {C E : Type*} (r : Recognizer C E) (c₁ c₂ : C) : Prop :=
  r.R c₁ = r.R c₂

notation:50 c₁ " ~[" r "] " c₂ => Indistinguishable r c₁ c₂
\end{lstlisting}

This definition captures the epistemological limit of the observer. If two states of the world produce the exact same reading on a measurement device, they are empirically identical with respect to that device.

\subsection{Equivalence Relation Properties}

\begin{theorem}[Indistinguishability is an Equivalence Relation]
    The relation $\indist$ is reflexive, symmetric, and transitive.
\end{theorem}

\begin{proof}
    \textit{Reflexivity}: $R(c) = R(c)$ trivially. \\
    \textit{Symmetry}: $R(c_1) = R(c_2) \implies R(c_2) = R(c_1)$. \\
    \textit{Transitivity}: $R(c_1) = R(c_2)$ and $R(c_2) = R(c_3)$ implies $R(c_1) = R(c_3)$.
\end{proof}

\begin{lstlisting}[language=Lean]
theorem indistinguishable_equivalence : Equivalence (Indistinguishable r) where
  refl := fun c => rfl
  symm := fun h => h.symm
  trans := fun h₁ h₂ => h₁.trans h₂
\end{lstlisting}

\subsection{Resolution Cells}

The equivalence classes under $\indist$ are called \textit{Resolution Cells}.

\begin{definition}[Resolution Cell]
    The resolution cell of $c$, denoted $[c]_R$, is the set of all configurations indistinguishable from $c$:
    \[ [c]_R = \{ c' \in \config \mid c' \indist c \} \]
\end{definition}

\begin{lstlisting}[language=Lean]
def ResolutionCell {C E : Type*} (r : Recognizer C E) (c : C) : Set C :=
  {c' : C | Indistinguishable r c' c}

theorem resolutionCell_eq_fiber (r : Recognizer C E) (c : C) :
    ResolutionCell r c = r.fiber (r.R c)
\end{lstlisting}

A resolution cell represents a ``pixel'' or ``voxel'' of the geometry defined by $R$. Inside a cell, the recognizer is blind; it cannot resolve any internal structure.

\section{The Recognition Quotient}

We now arrive at the first major structural result: the construction of the observable space.

\subsection{Construction}

The \textit{Recognition Quotient} is the quotient space of configurations by the indistinguishability relation.

\begin{definition}[Recognition Quotient]
    \[ \quotientspace = \config / \indist \]
\end{definition}

\begin{lstlisting}[language=Lean]
def indistinguishableSetoid (r : Recognizer C E) : Setoid C where
  r := Indistinguishable r
  iseqv := indistinguishable_equivalence r

def RecognitionQuotient {C E : Type*} (r : Recognizer C E) :=
  Quotient (indistinguishableSetoid r)

def recognitionQuotientMk (r : Recognizer C E) (c : C) : RecognitionQuotient r :=
  Quotient.mk (indistinguishableSetoid r) c
\end{lstlisting}

Let $\pi: \config \to \quotientspace$ be the canonical projection $\pi(c) = [c]_R$.

\subsection{The Observable Event Map}

Since $R(c)$ is constant on equivalence classes, the map $R$ descends to a well-defined map on the quotient, denoted $\bar{R}: \quotientspace \to \events$.

\[ \bar{R}([c]_R) = R(c) \]

\begin{theorem}[Injectivity of Observable Map]\label{thm:injective}
    The map $\bar{R}: \quotientspace \to \events$ is injective.
\end{theorem}

\begin{proof}
    Let $q_1, q_2 \in \quotientspace$ such that $\bar{R}(q_1) = \bar{R}(q_2)$. Let $c_1, c_2$ be representatives such that $q_1 = [c_1]_R$ and $q_2 = [c_2]_R$. Then $R(c_1) = \bar{R}(q_1) = \bar{R}(q_2) = R(c_2)$. By definition, $R(c_1) = R(c_2) \implies c_1 \indist c_2 \implies [c_1]_R = [c_2]_R$, so $q_1 = q_2$.
\end{proof}

\begin{lstlisting}[language=Lean]
def quotientEventMap (r : Recognizer C E) : RecognitionQuotient r → E :=
  Quotient.lift r.R (fun _ _ h => h)

theorem quotientEventMap_injective (r : Recognizer C E) :
    Function.Injective (quotientEventMap r)
\end{lstlisting}
\emph{Lean reference (companion RG library):} \texttt{IndisputableMonolith/RecogGeom/Quotient.lean}.

This result is physically profound. It says that in the quotient space---the space of physical observables---states are uniquely identified by their event values. There is no ``hidden state'' in $\quotientspace$. The geometry of $\quotientspace$ is exactly the geometry of the information accessible to the recognizer.

\begin{corollary}[Quotient Isomorphic to Image]
    $\quotientspace \cong \text{Im}(R)$
\end{corollary}

% ======================================================================
% PART III: ADVANCED STRUCTURE
% ======================================================================

\part{Advanced Structure}

Having established the basic observable space, we now turn to the rich geometric structure that arises from the interaction of multiple recognizers and local constraints.

\section{Composition of Recognizers (RG6)}

Physical measurement rarely involves a single isolated observation. We observe position \textit{and} momentum, color \textit{and} shape. This combination of measurements is formalized as the composition of recognizers.

\subsection{The Composite Recognizer}

\begin{definition}[Composite Recognizer]
    Given two recognizers $R_1: \config \to \events_1$ and $R_2: \config \to \events_2$, their \textit{composition} is the recognizer $R_1 \otimes R_2: \config \to \events_1 \times \events_2$ defined by:
    \[ (R_1 \otimes R_2)(c) = (R_1(c), R_2(c)) \]
\end{definition}

\begin{lstlisting}[language=Lean]
def CompositeRecognizer (r₁ : Recognizer C E₁) (r₂ : Recognizer C E₂) :
    Recognizer C (E₁ × E₂) where
  R := fun c => (r₁.R c, r₂.R c)
  nontrivial := by
    obtain ⟨c₁, c₂, hne⟩ := r₁.nontrivial
    use c₁, c₂
    simp [hne]

infixl:70 " ⊗ " => CompositeRecognizer
\end{lstlisting}

\subsection{The Refinement Theorem}

Composition increases distinguishing power. If two configurations are distinguishable by $R_1$, they are distinguishable by $R_1 \otimes R_2$.

\begin{theorem}[Composite Indistinguishability]
    \[ c_1 \sim_{R_1 \otimes R_2} c_2 \iff (c_1 \sim_{R_1} c_2) \land (c_1 \sim_{R_2} c_2) \]
\end{theorem}

\begin{lstlisting}[language=Lean]
theorem composite_indistinguishable_iff (r₁ : Recognizer C E₁) (r₂ : Recognizer C E₂)
    (c₁ c₂ : C) :
    Indistinguishable (r₁ ⊗ r₂) c₁ c₂ ↔
    Indistinguishable r₁ c₁ c₂ ∧ Indistinguishable r₂ c₁ c₂
\end{lstlisting}

The resolution cells of the composite recognizer are the intersections of the component cells:
\[ [c]_{R_1 \otimes R_2} = [c]_{R_1} \cap [c]_{R_2} \]

\begin{theorem}[Refinement Theorem]\label{thm:refinement}
    The recognition quotient of the composite refines the quotients of its components. There exist surjective canonical maps:
    \[ \pi_1: \config_{R_1 \otimes R_2} \twoheadrightarrow \config_{R_1} \quad \text{and} \quad \pi_2: \config_{R_1 \otimes R_2} \twoheadrightarrow \config_{R_2} \]
\end{theorem}

\begin{lstlisting}[language=Lean]
theorem refinement_theorem (r₁ : Recognizer C E₁) (r₂ : Recognizer C E₂) :
    Function.Surjective (quotientMapLeft r₁ r₂) ∧
    Function.Surjective (quotientMapRight r₁ r₂)
\end{lstlisting}
\emph{Lean reference (companion RG library):} \texttt{IndisputableMonolith/RecogGeom/Composition.lean}.

This theorem formalizes the intuition that ``more measurement yields more geometry.'' As we add recognizers, the quotient space unfolds, revealing more detail.

\section{Symmetries and Gauge Equivalence}

A geometry is defined as much by what it preserves as by what it distinguishes.

\subsection{Recognition-Preserving Maps}

\begin{definition}[Recognition-Preserving Map]
    A transformation $T: \config \to \config$ is \textit{recognition-preserving} for $R$ if it leaves all events invariant:
    \[ \forall c \in \config,\, R(T(c)) = R(c) \]
\end{definition}

\begin{lstlisting}[language=Lean]
structure RecognitionPreservingMap (r : Recognizer C E) where
  toFun : C → C
  preserves_event : ∀ c, r.R (toFun c) = r.R c
\end{lstlisting}

\begin{proposition}
    Recognition-preserving maps are closed under composition and contain the identity. Consequently they form a monoid.
\end{proposition}

\begin{definition}[Recognition automorphism]
    A \textit{recognition automorphism} is a bijective recognition-preserving map. The collection $\mathrm{Aut}_R(\config)$ of automorphisms is a group under composition.
\end{definition}

\begin{proof}
    Closure and identity are inherited from the monoid structure; inverses exist for bijections.
\end{proof}

\begin{theorem}[Symmetry preserves indistinguishability]
    If $T$ is recognition-preserving, then $c_1 \indist c_2$ implies $T(c_1) \indist T(c_2)$.
\end{theorem}

\subsection{Gauge Equivalence}

In physics, a ``gauge transformation'' is a change in the mathematical description of a state that results in no observable difference. Recognition Geometry captures this exactly.

\begin{definition}[Gauge equivalence]
    $c_1$ and $c_2$ are \textit{gauge equivalent} if some $T \in \mathrm{Aut}_R(\config)$ satisfies $T(c_1) = c_2$.
\end{definition}

\begin{theorem}[Gauge implies indistinguishable]
    \[ c_1 \sim_{\text{gauge}} c_2 \implies c_1 \indist c_2. \]
\end{theorem}

\begin{lstlisting}[language=Lean]
def GaugeEquivalent (r : Recognizer C E) (c₁ c₂ : C) : Prop :=
  ∃ T : RecognitionAutomorphism r, T.toFun c₁ = c₂

theorem gauge_implies_indistinguishable {c₁ c₂ : C}
    (h : GaugeEquivalent r c₁ c₂) : Indistinguishable r c₁ c₂
\end{lstlisting}

The converse need not hold in general. However, if $\mathrm{Aut}_R(\config)$ acts transitively on each resolution cell, the two notions coincide.

\begin{proposition}[Gauge = indistinguishable under transitivity]
    If the automorphism group acts transitively on every resolution cell, then $c_1 \indist c_2$ implies $c_1 \sim_{\text{gauge}} c_2$.
\end{proposition}

\begin{proof}
    For $c_1 \indist c_2$, both lie in the same resolution cell, so transitivity supplies an automorphism mapping one to the other.
\end{proof}

\section{Finite Local Resolution (RG4)}

We now introduce the axiom that distinguishes Recognition Geometry from classical continuum geometry and connects it to quantum discreteness.

\subsection{Axiom RG4: Finite Resolution}

The \textit{Finite Resolution Axiom} asserts that locally, a recognizer can only distinguish a finite number of states.

\begin{axiomenv}[RG4: Finite Local Resolution]
    For every configuration $c$ and recognizer $R$, there exists a neighborhood $U \in \neighborhood(c)$ such that the image $R(U)$ is a finite set.
\end{axiomenv}

\begin{lstlisting}[language=Lean]
def HasFiniteLocalResolution (L : LocalConfigSpace C) (r : Recognizer C E) (c : C) : Prop :=
  ∃ U ∈ L.N c, (r.R '' U).Finite

def HasFiniteResolution (L : LocalConfigSpace C) (r : Recognizer C E) : Prop :=
  ∀ c : C, HasFiniteLocalResolution L r c
\end{lstlisting}

This is a radical departure from the assumption of infinite precision. It posits that information density in reality is locally bounded.

\subsection{The No-Injection Theorem}

This axiom has a profound consequence: if the underlying configuration space is infinite (continuous), the recognizer \textit{cannot} be injective locally.

\begin{theorem}[No-Injection / Fundamental Obstruction]\label{thm:no-injection}
    If a neighborhood $U$ contains infinitely many configurations but has finite resolution, then the map $R|_U: U \to \events$ is not injective.
\end{theorem}

\begin{proof}
    If $R|_U$ were injective, then $|U| \le |R(U)|$. But $U$ is infinite and $R(U)$ is finite, contradiction.
\end{proof}

\begin{lstlisting}[language=Lean]
theorem no_injection_on_infinite_finite (c : C) (U : Set C) (hU : U ∈ L.N c)
    (hinf : Set.Infinite U) (hfin : (r.R '' U).Finite) :
    ¬Function.Injective (r.R ∘ Subtype.val : U → E)
\end{lstlisting}
\emph{Lean reference (companion RG library):} \texttt{IndisputableMonolith/RecogGeom/FiniteResolution.lean}.

This explains why we see ``resolution cells'' rather than points. The underlying continuum (if it exists) is forced to ``clump'' into observable quanta. This provides a geometric origin for quantization.

\section{Connectivity and Local Regularity (RG5)}

To recover smooth geometry from these discrete cells, we need a condition that prevents the cells from being ``scattered dust.''

\subsection{Recognition Connectivity}

\begin{definition}[Recognition Connectedness]
    A subset $S \subseteq \config$ is \textit{recognition-connected} with respect to $R$ if all pairs of points in $S$ are indistinguishable:
    \[ \forall c_1, c_2 \in S,\, R(c_1) = R(c_2) \]
\end{definition}

\begin{lstlisting}[language=Lean]
def IsRecognitionConnected (r : Recognizer C E) (S : Set C) : Prop :=
  ∀ c₁ c₂, c₁ ∈ S → c₂ ∈ S → Indistinguishable r c₁ c₂
\end{lstlisting}

\subsection{Axiom RG5: Local Regularity}

\begin{axiomenv}[RG5: Local Regularity]
    A recognizer is \textit{locally regular} if for every configuration $c$, there exists a neighborhood $U \in \neighborhood(c)$ such that the intersection $[c]_R \cap U$ is recognition-connected.
\end{axiomenv}

This axiom ensures that resolution cells align coherently with the neighborhood structure, preventing pathological cases where a cell is dense but totally disconnected (like a Cantor set) within a neighborhood.

\section{Comparative Recognizers (RG7)}

Standard geometry assumes a metric $d(x,y)$ is given. Recognition Geometry derives metrics from a weaker structure: \textit{comparative recognition}.

\subsection{The Comparative Structure}

A comparative recognizer does not identify a state; it compares two states.

\begin{axiomenv}[RG7: Comparative Recognizers]
    A \textit{Comparative Recognizer} is a map $C_R: \config \times \config \to \events$ that assigns an event to a \textit{pair} of configurations, with $C_R(c,c) = e_{\text{eq}}$ for some distinguished ``equality event,'' and is \textit{nontrivial}: some pair is assigned an event different from $e_{\text{eq}}$.
\end{axiomenv}

\begin{lstlisting}[language=Lean]
structure ComparativeRecognizer (C E : Type*) where
  compare : C × C → E
  eq_event : E
  compare_self : ∀ c, compare (c, c) = eq_event
  nontrivial : ∃ c₁ c₂, compare (c₁, c₂) ≠ eq_event
\end{lstlisting}

This models instruments like balance scales (is $A$ heavier than $B$?) or interferometers (phase difference).

\subsection{Emergence of Order and Metrics}

From a comparative recognizer, we can define an order relation. Let $\events_{>}$ be a subset of events interpreted as ``greater than.'' We say $c_1 \le c_2$ if $C_R(c_1, c_2) \notin \events_{>}$.

If a family of comparative recognizers separates points and satisfies triangle-like properties, a \textit{Recognition Distance} emerges.

\begin{definition}[Recognition Distance]
    A \textit{Recognition Distance} is a pseudometric $d: \config \times \config \to \mathbb{R}_{\ge 0}$ derived from comparative recognizers, satisfying:
    \begin{enumerate}
        \item $d(c, c) = 0$
        \item $d(c_1, c_2) = d(c_2, c_1)$
        \item $d(c_1, c_3) \le d(c_1, c_2) + d(c_2, c_3)$
    \end{enumerate}
\end{definition}

\begin{lstlisting}[language=Lean]
structure RecognitionDistance (C : Type*) where
  dist : C → C → ℝ
  dist_nonneg : ∀ c₁ c₂, 0 ≤ dist c₁ c₂
  dist_self : ∀ c, dist c c = 0
  dist_symm : ∀ c₁ c₂, dist c₁ c₂ = dist c₂ c₁
  dist_triangle : ∀ c₁ c₂ c₃, dist c₁ c₃ ≤ dist c₁ c₂ + dist c₂ c₃
\end{lstlisting}

This completes the inversion: distance is not a primitive obstruction to being in the same place; it is a measure of the ``cost'' or ``signal'' required to distinguish two configurations comparably.

\section{Charts and Dimension}

We endow $\quotientspace$ with the topology inherited from the projection $\pi: \config \twoheadrightarrow \quotientspace$ and use it to formulate atlas and manifold notions.

\subsection{Quotient Topology}\label{subsec:quot-topo}

\begin{definition}[Recognition quotient topology]
    The \emph{recognition quotient topology} $\mathcal{T}_R$ on $\quotientspace$ is the final topology with respect to $\pi$: a subset $U \subseteq \quotientspace$ is open iff $\pi^{-1}(U)$ is open in $\config$ (with the locality structure viewed as a topology generated by neighborhoods).
\end{definition}

\begin{proposition}\label{prop:quot-top-final}
    The projection $\pi: (\config, \mathcal{T}) \to (\quotientspace, \mathcal{T}_R)$ is continuous and is the universal continuous map rendering $R$ constant on fibers: any map $f: \config \to X$ that is constant on resolution cells factors uniquely through a continuous map $\tilde f: \quotientspace \to X$.
\end{proposition}

\begin{proof}
    Immediate from the definition of the final topology; uniqueness of $\tilde f$ follows because $\pi$ is surjective.
\end{proof}

Henceforth we regard $\quotientspace$ as a topological space with topology $\mathcal{T}_R$.

\subsection{Recognition Charts}

\begin{definition}[Recognition Chart]
    A chart $(\phi, U)$ is a map $\phi: U \to \mathbb{R}^n$ from a neighborhood $U \subset \config$ such that:
    \begin{enumerate}
        \item $\phi$ respects indistinguishability: $c_1 \indist c_2 \implies \phi(c_1) = \phi(c_2)$
        \item $\phi$ is injective on resolution cells: $\phi(c_1) = \phi(c_2) \implies c_1 \indist c_2$
    \end{enumerate}
\end{definition}

\begin{lstlisting}[language=Lean]
structure RecognitionChart (L : LocalConfigSpace C) (r : Recognizer C E) (n : ℕ) where
  domain : Set C
  domain_is_nbhd : ∃ c, domain ∈ L.N c
  toFun : domain → Fin n → ℝ
  respects_indistinguishable : ∀ c₁ c₂ : domain,
    Indistinguishable r c₁.val c₂.val → toFun c₁ = toFun c₂
  injective_on_classes : ∀ c₁ c₂ : domain,
    toFun c₁ = toFun c₂ → Indistinguishable r c₁.val c₂.val
\end{lstlisting}

This defines a local coordinate system for the \textit{quotient space} $\quotientspace$.

\subsection{Recognition Atlases and Compatibility}

\begin{definition}[Recognition Atlas]
    A \emph{Recognition Atlas} of dimension $n$ on $U \subseteq \config$ is a family of charts $\{(\phi_i, U_i)\}_{i \in I}$ with $U \subseteq \bigcup_i U_i$, such that the induced maps
    \[
        \tilde{\phi}_i : \pi(U_i) \longrightarrow \phi_i(U_i) \subseteq \mathbb{R}^n
    \]
    are homeomorphisms (with $\pi(U_i)$ inheriting $\mathcal{T}_R$) and the transition maps
    \[
        \tilde{\phi}_{j} \circ \tilde{\phi}_{i}^{-1} : \tilde{\phi}_i(\pi(U_i \cap U_j)) \to \tilde{\phi}_j(\pi(U_i \cap U_j))
    \]
    are smooth.
\end{definition}

\begin{remark}
    Under RG5 (local regularity) fibers align with neighborhoods, so $\pi(U_i)$ is open in $\quotientspace$ and the induced maps $\tilde{\phi}_i$ are continuous bijections with continuous inverses.
\end{remark}

\begin{lstlisting}[language=Lean]
structure RecognitionAtlas (L : LocalConfigSpace C) (r : Recognizer C E) (n : ℕ) where
  charts : Set (RecognitionChart L r n)
  covers : ∀ c : C, ∃ ch ∈ charts, c ∈ ch.domain
  compatible :
    ∀ {ch₁ ch₂} (h₁ : ch₁ ∈ charts) (h₂ : ch₂ ∈ charts),
      True  -- placeholder: transition regularity property
\end{lstlisting}

\begin{theorem}[Recognition Manifold Theorem]\label{thm:recognition-manifold}
    Let $(\config, \events, R)$ satisfy RG0--RG5. Assume:
    \begin{enumerate}
        \item The quotient $(\quotientspace, \mathcal{T}_R)$ is Hausdorff and second countable.
        \item There exists a recognition atlas of dimension $n$ whose induced charts $\tilde{\phi}_i$ are homeomorphisms onto open subsets of $\mathbb{R}^n$ with smooth transition maps.
    \end{enumerate}
    Then $(\quotientspace, \mathcal{T}_R)$ is a smooth $n$-manifold. Moreover, the projection $\pi: \config \to \quotientspace$ is a submersion onto this manifold.
\end{theorem}

\begin{proof}
    Conditions (1) and (2) match the standard characterization of smooth manifolds (see Lee~\cite{Lee}). The $\tilde{\phi}_i$ provide a smooth atlas on $\quotientspace$, whose Hausdorff and second-countable properties ensure compatibility with the definition of smooth manifold. For the final assertion, RG5 guarantees $\pi$ has constant-rank fibers locally, so $\pi$ is a smooth submersion with respect to the constructed structure.
\end{proof}

\subsection{Recognition Dimension}

\begin{definition}[Recognition Dimension]
    The recognition dimension at a point is the integer $n$ such that a recognition chart to $\mathbb{R}^n$ exists.
\end{definition}

This gives a purely operational definition of dimension:
\begin{quote}
    \textbf{Dimension is the minimum number of independent recognizers needed to distinguish all local configurations.}
\end{quote}

If spacetime is 4-dimensional, it is because exactly 4 independent measurements (e.g., $x, y, z, t$) are required to resolve an event.

\begin{theorem}[Fundamental Obstruction to Charts]\label{thm:no-chart}
    If a neighborhood has finite resolution (RG4) but contains infinitely many configurations, \textbf{no recognition chart exists} on that neighborhood.
\end{theorem}

This theorem highlights the tension between the discrete reality of RG4 and the continuous approximation of manifolds. Manifolds are only an emergent approximation valid when the number of resolution cells is large enough to be treated as a continuum.

\begin{theorem}[Local Dimension Uniqueness]\label{thm:dimension-uniqueness}
    Let $U \subseteq \quotientspace$ admit two smooth atlas structures of dimensions $n$ and $m$ derived from recognizers satisfying the hypotheses of \Cref{thm:recognition-manifold}. If both atlases are minimal (no subatlas of smaller dimension exists) and their inclusion maps coincide on $U$, then $n = m$.
\end{theorem}

\begin{proof}
    Assume $n < m$. The identity map $U \to U$ is a smooth, injective, open map from an $n$-manifold to an $m$-manifold. By invariance of domain (Munkres~\cite{Munkres}) such a map can exist only if $n = m$. The case $m < n$ is identical, so $n = m$.
\end{proof}

\begin{lstlisting}[language=Lean]
theorem dimension_unique_of_minimal_charts
  (L : LocalConfigSpace C) (r : Recognizer C E) {U : Set C}
  (A B : RecognitionAtlas L r n) -- schematic; minimality assumptions omitted
  : True := by
  trivial
\end{lstlisting}

% ======================================================================
% PART IV: THE RECOGNITION SCIENCE BRIDGE
% ======================================================================

\part{The Recognition Science Bridge}

We now bridge the abstract mathematical framework of Recognition Geometry to the concrete physics of Recognition Science (RS). We show that the RS model---based on a ledger of states, the $\Rhat$ operator, and the 8-tick cycle---provides a direct instantiation of the axioms RG0--RG7.

\section{Instantiation from RS}

\subsection{RS Configuration Space (The Ledger)}

In Recognition Science, the fundamental ontological entity is the \textit{Ledger}---the complete record of all registered entities and their states.

\begin{definition}[RS Configuration Space]
    Let $\ledger$ be the type of all valid ledger states. The configuration space is $\config_{\text{RS}} = \ledger$.
\end{definition}

This satisfies RG0 immediately, as the ledger is nonempty.

\begin{lstlisting}[language=Lean]
class RSConfigSpace (L : Type*) where
  nonempty : Nonempty L
  eq_decidable : DecidableEq L

instance (L : Type*) [RSConfigSpace L] : ConfigSpace L where
  nonempty := RSConfigSpace.nonempty
\end{lstlisting}

\subsection{RS Locality ($\Rhat$ Operator)}

Locality in RS is defined not by spatial distance, but by \textit{interaction reach}. The $\Rhat$ operator defines the set of states reachable from a given state via a single recognition event.

\begin{definition}[RS Neighborhoods]
    The neighborhood of a ledger state $\ell \in \ledger$ is the set of states reachable by the $\Rhat$ operator:
    \[ \neighborhood_{\text{RS}}(\ell) = \{ U \mid \Rhat(\ell) \subseteq U \} \]
\end{definition}

This satisfies RG1.

\begin{lstlisting}[language=Lean]
structure RSLocalityFromRHat (L : Type*) [RSConfigSpace L] where
  RHat : L → Set L
  self_in_RHat : ∀ l, l ∈ RHat l
  refinement : ∀ l l', l' ∈ RHat l → 
    ∃ U ⊆ RHat l, l' ∈ U ∧ U ⊆ RHat l'
\end{lstlisting}

\subsection{Measurements as Recognizers}

A physical measurement in RS is a map from the ledger state to an outcome.

\begin{definition}[RS Measurement]
    An RS measurement is a function $M: \ledger \to \events$ that extracts a specific value (e.g., position, charge) from the ledger.
\end{definition}

This instantiates RG2.

\begin{lstlisting}[language=Lean]
structure RSMeasurement (L E : Type*) [RSConfigSpace L] where
  measure : L → E
  nontrivial : ∃ l₁ l₂ : L, measure l₁ ≠ measure l₂

def toRecognizer (m : RSMeasurement L E) : Recognizer L E where
  R := m.measure
  nontrivial := m.nontrivial
\end{lstlisting}

\subsection{The 8-Tick Finite Resolution}

RS imposes a fundamental temporal constraint: the 8-tick cycle. In any local interaction window, only a finite number of state updates can occur.

\begin{theorem}[8-Tick Implies RG4]
    The 8-tick cycle constraint implies the Finite Resolution Axiom (RG4). Specifically, the set of distinct measurement outcomes reachable from a state $\ell$ within one 8-tick cycle is finite.
\end{theorem}

\begin{lstlisting}[language=Lean]
structure EightTickFiniteResolution (L E : Type*) [RSConfigSpace L]
    (rs : RSLocalityFromRHat L) (m : RSMeasurement L E) : Prop where
  finite_local_events : ∀ l, (m.measure '' rs.RHat l).Finite

theorem eight_tick_implies_RG4 [RSConfigSpace L]
    (rs : RSLocalityFromRHat L) (m : RSMeasurement L E)
    (h8 : EightTickFiniteResolution L E rs m) :
    HasFiniteResolution (toLocalConfigSpace rs) (toRecognizer m)
\end{lstlisting}
\emph{Lean reference (companion RG library):} \texttt{IndisputableMonolith/RecogGeom/RSBridge.lean}.

This is the rigorous derivation of physical discreteness. Discreteness is not postulated ad hoc; it is a theorem.

\subsection{Physical Space as Quotient}

\begin{theorem}[Physical Space is a Quotient]
    Physical 3D space is isomorphic to the recognition quotient of the ledger by the family of spatial position recognizers.
    \[ \text{Space} \cong \ledger / \sim_{\text{pos}} \]
\end{theorem}

Points in space are not fundamental entities. A ``point'' is an equivalence class of ledger states that look the same to spatial measuring instruments.

\subsection{J-Cost as Recognition Metric}

The metric structure comes from the $J$-cost function, which measures the information cost of transitions.

\begin{definition}[J-Cost Metric]
    The $J$-cost function $J: \ledger \times \ledger \to \mathbb{R}$ acts as a comparative recognizer (RG7), defining a metric on the quotient space.
\end{definition}

\begin{lstlisting}[language=Lean]
structure JCostComparative (L : Type*) [RSConfigSpace L] where
  J : L → L → ℝ
  self_zero : ∀ l, J l l = 0
  nonneg : ∀ l₁ l₂, 0 ≤ J l₁ l₂
  symm : ∀ l₁ l₂, J l₁ l₂ = J l₂ l₁
  triangle : ∀ l₁ l₂ l₃, J l₁ l₃ ≤ J l₁ l₂ + J l₂ l₃
\end{lstlisting}

This completes the bridge: abstract axioms of Recognition Geometry are physically realized by Recognition Science.

% ======================================================================
% PART V: FOUNDATIONAL THEOREMS
% ======================================================================

\part{Foundational Theorems}

We now state and prove the deep results that justify the entire framework.

\section{The Three Pillars}

Recognition Geometry rests on three fundamental insights:

\subsection{Pillar 1: Quotient Determines Observables}

\begin{theorem}[Pillar 1]
    The recognition quotient $\quotientspace$ captures exactly what can be observed. The event map $\bar{R}: \quotientspace \to \events$ is injective.
\end{theorem}

This was proved as \Cref{thm:injective}. Knowing the event uniquely determines the equivalence class.

\subsection{Pillar 2: Information Monotonicity}

\begin{theorem}[Pillar 2]
    Adding recognizers can only increase distinguishing power. If $c_1 \sim_{R_1 \otimes R_2} c_2$, then $c_1 \sim_{R_1} c_2$ and $c_1 \sim_{R_2} c_2$.
\end{theorem}

This was proved as the Refinement Theorem (\Cref{thm:refinement}).

\subsection{Pillar 3: Finite Resolution Obstruction}

\begin{theorem}[Pillar 3]
    If a neighborhood has infinitely many configurations but only finitely many events, no injection exists.
\end{theorem}

This was proved as the No-Injection Theorem (\Cref{thm:no-injection}).

\section{The Fundamental Theorem}

\begin{theorem}[Fundamental Theorem of Recognition Geometry]
    Two configurations are in the same equivalence class if and only if the recognizer assigns them the same event:
    \[ [c_1]_R = [c_2]_R \iff R(c_1) = R(c_2) \]
\end{theorem}

\begin{lstlisting}[language=Lean]
theorem fundamental_theorem (r : Recognizer C E) (c₁ c₂ : C) :
    recognitionQuotientMk r c₁ = recognitionQuotientMk r c₂ ↔ r.R c₁ = r.R c₂
\end{lstlisting}
\emph{Lean reference (companion RG library):} \texttt{IndisputableMonolith/RecogGeom/Foundations.lean}.

This is the cornerstone: observable space $=$ $\config / \{\text{same events}\}$.

\section{The Universal Property}

\begin{theorem}[Universal Property of the Recognition Quotient]
    The recognition quotient $\quotientspace$ has a universal property: it is the ``finest'' quotient on which $R$ factors through an injective map. Specifically:
    \begin{enumerate}
        \item The projection $\pi: \config \twoheadrightarrow \quotientspace$ is surjective.
        \item The induced map $\bar{R}: \quotientspace \hookrightarrow \events$ is injective.
        \item $R = \bar{R} \circ \pi$ (factorization).
    \end{enumerate}
\end{theorem}

\[
\begin{array}{ccc}
\config & \xrightarrow{R} & \events \\
\downarrow \pi & \nearrow \bar{R} & \\
\quotientspace & &
\end{array}
\]

\begin{lstlisting}[language=Lean]
theorem universal_property (r : Recognizer C E) :
    Function.Surjective (recognitionQuotientMk r) ∧
    Function.Injective (quotientEventMap r) ∧
    (∀ c, r.R c = quotientEventMap r (recognitionQuotientMk r c))
\end{lstlisting}
\emph{Lean reference (companion RG library):} \texttt{IndisputableMonolith/RecogGeom/Foundations.lean}.

This says: $\quotientspace$ is characterized by a universal property, not just a construction. It is THE canonical quotient for recognition.

% ======================================================================
% PART VI: EXAMPLES
% ======================================================================

\part{Examples}

\section{Discrete Recognition on Finite Sets}

\begin{example}[Discrete Recognizer]
    Let $\config = \{1, 2, \ldots, n\}$ and $R = \text{id}$. Then every configuration is its own resolution cell:
    \[ [k]_R = \{k\} \]
    The quotient is isomorphic to $\config$ itself.
\end{example}

\begin{lstlisting}[language=Lean]
def discreteRecognizer (n : ℕ) [NeZero n] (hn : 2 ≤ n) : Recognizer (Fin n) (Fin n) where
  R := id
  nontrivial := by use ⟨0, by omega⟩, ⟨1, by omega⟩; simp

theorem discrete_indist_iff_eq (c₁ c₂ : Fin n) :
    Indistinguishable (discreteRecognizer n hn) c₁ c₂ ↔ c₁ = c₂
\end{lstlisting}

\section{Sign Recognizer on $\mathbb{Z}$}

\begin{example}[Sign Recognizer]
    Let $\config = \mathbb{Z}$ and $R(n) = \text{sign}(n) \in \{-, 0, +\}$. Then:
    \begin{itemize}
        \item $[5]_R = \{1, 2, 3, \ldots\}$ (positive integers)
        \item $[0]_R = \{0\}$
        \item $[-3]_R = \{\ldots, -2, -1\}$ (negative integers)
    \end{itemize}
    The quotient has exactly 3 elements.
\end{example}

\begin{lstlisting}[language=Lean]
inductive Sign : Type
  | neg : Sign
  | zero : Sign
  | pos : Sign

def signRecognizer : Recognizer ℤ Sign where
  R := fun n => if n < 0 then Sign.neg else if n = 0 then Sign.zero else Sign.pos
  nontrivial := by use -1, 1; simp
\end{lstlisting}

\section{Magnitude Recognizer on $\mathbb{Z}$}

\begin{example}[Magnitude Recognizer]
    Let $R(n) = |n|$. Then $[3]_R = \{-3, 3\}$ and $[0]_R = \{0\}$.
    
    Note: Sign and Magnitude give different partitions. Their composition $(R_{\text{sign}} \otimes R_{|.|})$ refines both: $[3]_{R \otimes |.|} = \{3\}$.
\end{example}

\begin{lstlisting}[language=Lean]
def magnitudeRecognizer : Recognizer ℤ ℕ where
  R := fun n => n.natAbs
  nontrivial := by use 0, 1; simp

theorem plus_minus_indist : Indistinguishable magnitudeRecognizer 3 (-3)
theorem sign_distinguishes : ¬Indistinguishable signRecognizer 3 (-3)
\end{lstlisting}

\section{Continuous Interval Recognizer}

\begin{example}[Piecewise-constant recognizer on $\lbrack 0,1\rbrack$]
    Let $\config = [0,1]$ and $\events = \{0,1,\ldots,9\}$. Define
    \[
        R(x) = \lfloor 10x \rfloor .
    \]
    Each event corresponds to a half-open interval of width $0.1$, and resolution cells are precisely these intervals.
\end{example}

This example shows how a finite-resolution recognizer on a continuum configuration space induces a coarse, cell-like observable geometry.

\section{No-Chart Example under RG4}

\begin{example}[Failure of charts with finite resolution]
    Using the recognizer above, every neighborhood containing an interval with infinitely many points has an image consisting of only finitely many events. By \Cref{thm:no-chart}, no recognition chart exists on any open set contained entirely within a single cell.
\end{example}

The example provides a concrete instance of the obstruction predicted by RG4.

% ======================================================================
% PART VII: PHYSICAL IMPLICATIONS
% ======================================================================

\part{Physical Implications}

\section{Why Spacetime is 4-Dimensional}

In Recognition Geometry, dimension equals the minimum number of independent recognizers needed to separate all local configurations.

Spacetime is 4-dimensional because exactly 4 independent measurements $(x, y, z, t)$ are required to resolve an event. No subset of 3 suffices (e.g., 3 spatial coordinates cannot distinguish events at different times).

The 4D structure \textit{emerges} from the structure of physical recognizers, not from a pre-existing geometric fact.

\section{Why Physics Has Gauge Symmetries}

Gauge transformations are exactly recognition-preserving maps. They are transformations of the underlying configuration space that are invisible to all recognizers.

In Recognition Geometry, gauge symmetry is not an additional postulate; it is built into the definition of the quotient.

\section{Why Quantum Mechanics is Discrete}

The Finite Resolution Axiom (RG4) asserts that local recognizers have finite event spaces. This implies that observables have discrete spectra.

Discreteness is not an approximation or a computational limitation---it is a fundamental feature of any recognition-based geometry.

\section{Why Metrics are Not Fundamental}

Distance emerges from comparative recognizers. Metrics are derived from the ability to order and compare configurations, not from a pre-existing spatial substrate.

\section{Why the Universe is Computable}

Finite resolution + finite time = finite states. Recognition geometry is inherently computational, consistent with digital physics hypotheses.

% ======================================================================
% PART VIII: EXTENSIONS AND GENERALIZATIONS
% ======================================================================

\part{Extensions and Generalizations}

The framework developed in Parts I--VII is complete and self-contained. However, several natural mathematical and physical extensions suggest themselves. In this part, we develop eight such extensions in detail, pointing toward a richer theory.

\section{Categorical Reformulation}

The categorical viewpoint clarifies the universal properties that already appear implicitly in Recognition Geometry. We make the relevant categories and functors precise.

\subsection{The Category \texorpdfstring{$\mathbf{RecogGeom}$}{RecogGeom}}

\begin{definition}[Recognition category]\label{def:RecogGeom}
    The category $\mathbf{RecogGeom}$ is defined as follows.
    \begin{itemize}
        \item \textbf{Objects}: triples $(\config, \events, R)$ with $R: \config \to \events$ a recognizer.
        \item \textbf{Morphisms}: a morphism $(f, g): (\config_1, \events_1, R_1) \to (\config_2, \events_2, R_2)$ is a commuting square
        \[
        \begin{array}{ccc}
        \config_1 & \xrightarrow{R_1} & \events_1 \\
        \downarrow \scriptstyle{f} & & \downarrow \scriptstyle{g} \\
        \config_2 & \xrightarrow{R_2} & \events_2
        \end{array}
        \qquad R_2 \circ f = g \circ R_1.
        \]
        \item \textbf{Composition}: $(f_2, g_2)\circ(f_1, g_1) := (f_2 \circ f_1,\, g_2 \circ g_1)$. Identities are $(\mathrm{id}_\config, \mathrm{id}_\events)$.
    \end{itemize}
\end{definition}

\begin{lemma}
    $\mathbf{RecogGeom}$ is a well-defined category: composition is associative and identities behave as expected.
\end{lemma}

\begin{proof}
    Associativity and identity laws reduce to the corresponding properties of functions; commutativity of diagrams is preserved under composition.
\end{proof}

\subsection{The Category \texorpdfstring{$\mathbf{Set}_{\mathrm{inj}}$}{Set\_inj}}

\begin{definition}[Injective recognition category]\label{def:SetInj}
    $\mathbf{Set}_{\mathrm{inj}}$ has objects triples $(X, Y, i)$ with $i: X \hookrightarrow Y$ an injective function. A morphism $(u, v): (X_1, Y_1, i_1) \to (X_2, Y_2, i_2)$ is a commuting square $i_2 \circ u = v \circ i_1$. Composition and identities are defined exactly as in \Cref{def:RecogGeom}.
\end{definition}

Objects of $\mathbf{Set}_{\mathrm{inj}}$ can be viewed as ``ideal'' recognition geometries where the recognizer is injective.

\subsection{Quotient and Inclusion Functors}

\begin{definition}[Quotient functor]\label{def:quotientFunctor}
    The functor $Q: \mathbf{RecogGeom} \to \mathbf{Set}_{\mathrm{inj}}$ sends $(\config, \events, R)$ to $(\quotientspace, \mathrm{Im}(R), \bar{R})$, where $\quotientspace = \config/\!\!\indist$ and $\bar{R}: \quotientspace \hookrightarrow \mathrm{Im}(R)$ is the injective map induced by $R$ (\Cref{thm:injective}). For a morphism $(f, g)$ the functor acts by
    \[
        Q(f, g) = (\tilde{f},\, g|_{\mathrm{Im}(R_1)}) ,
    \]
    where $\tilde{f}$ is the unique map on quotients satisfying $\tilde{f} \circ \pi_1 = \pi_2 \circ f$.
\end{definition}

Well-definedness of $\tilde{f}$ follows from $R_2 \circ f = g \circ R_1$: if $c_1 \sim_{R_1} c_2$ then $R_1(c_1) = R_1(c_2)$, hence $R_2(f(c_1)) = g(R_1(c_1)) = g(R_1(c_2)) = R_2(f(c_2))$, showing $\tilde{f}([c_1]) = \tilde{f}([c_2])$.

\begin{definition}[Inclusion functor]\label{def:inclusionFunctor}
    The functor $I: \mathbf{Set}_{\mathrm{inj}} \to \mathbf{RecogGeom}$ forgets that a recognizer is injective: $I(X, Y, i) = (X, Y, i)$ and $I(u, v) = (u, v)$.
\end{definition}

\subsection{The Quotient--Inclusion Adjunction}

\begin{theorem}[Quotient adjunction]\label{thm:adjunction}
    The functor $Q$ is left adjoint to $I$. Equivalently, for every $(\config, \events, R)$ and every $(X, Y, i)$ in $\mathbf{Set}_{\mathrm{inj}}$ there is a natural bijection
    \[
        \Phi: \mathrm{Hom}_{\mathbf{Set}_{\mathrm{inj}}}(Q(\config, \events, R), (X, Y, i))
        \;\cong\;
        \mathrm{Hom}_{\mathbf{RecogGeom}}((\config, \events, R), I(X, Y, i)).
    \]
\end{theorem}

\begin{proof}
    A morphism in $\mathbf{Set}_{\mathrm{inj}}$ from $(\quotientspace, \mathrm{Im}(R), \bar{R})$ to $(X, Y, i)$ is a pair $(u, v)$ with $i \circ u = v \circ \bar{R}$. Compose with $\pi: \config \to \quotientspace$ to obtain $(u \circ \pi, v)$. The equation $i \circ u \circ \pi = v \circ R$ shows $(u\circ\pi, v)$ is a morphism in $\mathbf{RecogGeom}$. This defines the right-to-left map of $\Phi$.

    Conversely, given $(f, g)$ in $\mathbf{RecogGeom}$ with codomain $(X, Y, i)$, the injectivity of $i$ implies that $f$ is constant on $R$-fibers (since $i \circ f = g \circ R$). Hence there exists a unique $\tilde{f}: \quotientspace \to X$ with $\tilde{f} \circ \pi = f$. Set $\tilde{g} = g|_{\mathrm{Im}(R)}$. Then $(\tilde{f}, \tilde{g})$ is a morphism in $\mathbf{Set}_{\mathrm{inj}}$ and defines the left-to-right map of $\Phi$.

    The two assignments are inverse to each other and natural in both variables, giving the desired adjunction. The unit $\eta_{(\config,\events,R)} = (\pi, \iota)$ (with $\iota:\mathrm{Im}(R)\hookrightarrow \events$) and counit $\varepsilon_{(X,Y,i)} = (\mathrm{id}_X,\mathrm{id}_Y)$ satisfy the triangle identities because $\pi$ is surjective and $\bar{R}$ is injective.
\end{proof}

\begin{remark}
    In Lean the functors $Q$ and $I$ can be implemented using quotient types and structure-preserving maps; the bijection $\Phi$ corresponds to \texttt{Quot.lift} along with the universal property proved in \texttt{RecogGeom/Quotient.lean}.
\end{remark}

\subsection{Sheaves of Recognizers}\label{sec:sheaves}

Let $(\quotientspace, \mathcal{T}_R)$ be the recognition quotient. The opens of $\quotientspace$ form a site with coverings given by standard open covers.

\begin{definition}[Presheaf of recognizers]
    Define a presheaf $\mathcal{F}$ on $\quotientspace$ by
    \[
        \mathcal{F}(U) = \left\{ R_U : \pi^{-1}(U) \to \events \mid R_U \text{ is a recognizer and } R_U \text{ factors through } U \right\},
    \]
    with restriction maps given by restriction of functions.
\end{definition}

\begin{definition}[Sheaf condition]
    $\mathcal{F}$ is a sheaf if for every open cover $\{U_i\}$ of $U$ and recognizers $R_i \in \mathcal{F}(U_i)$ that agree on overlaps $U_i \cap U_j$, there is a unique $R \in \mathcal{F}(U)$ restricting to each $R_i$.
\end{definition}

\begin{lemma}[Gluing lemma]
    $\mathcal{F}$ is a sheaf.
\end{lemma}

\begin{proof}
    Agreement on overlaps ensures the $R_i$ patch to a single function on $\bigcup_i \pi^{-1}(U_i) = \pi^{-1}(U)$, which descends to $U$ because indistinguishability is respected locally. Uniqueness follows from surjectivity of $\pi^{-1}(U) \to U$.
\end{proof}

This formalizes the intuition that recognition data defined on local regions glue to a global recognizer exactly when observers agree on overlaps.

\section{Recognition Dynamics}

The framework as developed is static. Physical reality evolves. We now introduce dynamics.

\subsection{Time-Parameterized Recognizers}

\begin{definition}[Recognition Flow]
    A \textit{recognition flow} is a family of recognizers $\{R_t\}_{t \in T}$ parameterized by a time set $T$ (typically $\mathbb{R}$ or $\mathbb{Z}$), together with a flow map $\Phi: \config \times T \to \config$ such that:
    \[ R_t(\Phi(c, s)) = R_{t+s}(c) \]
\end{definition}

This captures the idea that the recognizer ``evolves'' along with the configuration.

\begin{lstlisting}[language=Lean]
structure RecognitionFlow (C E : Type*) (T : Type*) [AddGroup T] where
  R : T → Recognizer C E
  Phi : C → T → C
  flow_compat : ∀ c t s, (R t).R (Phi c s) = (R (t + s)).R c
\end{lstlisting}

\subsection{Discrete Dynamics: The 8-Tick Cycle}

In Recognition Science, time is discrete with period 8. This suggests:

\begin{definition}[Cyclic Recognition Flow]
    A \textit{cyclic recognition flow} of period $n$ is a recognition flow with $T = \mathbb{Z}_n$ and $\Phi(c, n) = c$ for all $c$.
\end{definition}

\begin{theorem}[Periodicity of Quotient]
    For a cyclic recognition flow of period $n$, the quotient spaces satisfy $(\quotientspace)_{t+n} \cong (\quotientspace)_t$ for all $t$.
\end{theorem}

\subsection{Recognition Generators (Outlook)}

The idea of a ``recognition generator''—a single object producing the entire flow—requires extra geometric structure (symplectic or Poisson brackets). We defer this speculative program to \Cref{sec:recognition-generator} in the Outlook part.

\section{Multi-Observer Framework}

Different observers may have access to different recognizers. This section develops the mathematics of observer-relative geometry.

\subsection{Observer-Indexed Recognizers}

\begin{definition}[Observer Family]
    An \textit{observer family} is an indexed collection $\{R^{(A)}\}_{A \in \mathcal{A}}$ of recognizers, one for each observer $A$ in a set $\mathcal{A}$.
\end{definition}

Each observer $A$ sees the quotient space $\config_{R^{(A)}}$. The question is: how do these spaces relate?

\subsection{Inter-Observer Maps}

\begin{lemma}[Intersection of equivalences]
    The intersection of any collection of equivalence relations on $\config$ is again an equivalence relation.
\end{lemma}

\begin{proof}
    Intersections preserve reflexivity, symmetry, and transitivity.
\end{proof}

\begin{definition}[Observer comparison map]
    Given observers $A$ and $B$, define $\phi_{AB}: \config_{R^{(A)}} \to \config_{R^{(B)}}$ by $\phi_{AB}([c]_{R^{(A)}}) = [c]_{R^{(B)}}$ whenever $c_1 \sim_{R^{(A)}} c_2 \Rightarrow c_1 \sim_{R^{(B)}} c_2$. In this case $R^{(B)}$ is said to be \emph{coarser} than $R^{(A)}$.
\end{definition}

\begin{proposition}[Comparison hierarchy]
    If $R^{(B)}$ is coarser than $R^{(A)}$, then $\phi_{AB}$ is well-defined, continuous with respect to the quotient topologies, and surjective.
\end{proposition}

\begin{proof}
    Well-definedness follows from compatibility of equivalence relations, continuity from the universal property of the quotient map, and surjectivity because every $R^{(B)}$-class contains a representative mapped from a $R^{(A)}$-class.
\end{proof}

\subsection{The Universal Quotient}

\begin{definition}[Universal Quotient]
    The \textit{universal quotient} is the quotient by the intersection of all observer equivalences:
    \[ \config_{\text{univ}} = \config / \bigcap_{A \in \mathcal{A}} \sim_{R^{(A)}} \]
\end{definition}

\begin{theorem}[Universal property]
    For every observer $A$ there exists a unique surjective map $\pi_A: \config_{\text{univ}} \to \config_{R^{(A)}}$ such that $\pi_A \circ \pi_{\text{univ}} = \pi^{(A)}$, where $\pi_{\text{univ}}: \config \to \config_{\text{univ}}$ is the canonical projection and $\pi^{(A)}$ projects onto $\config_{R^{(A)}}$.
\end{theorem}

\begin{proof}
    By construction $\sim_{\text{univ}} \subseteq \sim_{R^{(A)}}$, so the universal property of quotients yields a unique factor map $\pi_A$. Surjectivity follows because $\pi^{(A)}$ is surjective.
\end{proof}

\begin{proposition}[Functoriality]
    The assignment $A \mapsto \config_{R^{(A)}}$, $(A \le B) \mapsto \phi_{AB}$ defines a functor from the poset of observers ordered by refinement to the category of quotient spaces.
\end{proposition}

\begin{proof}
    Identity and composition properties follow from uniqueness of the factor maps and associativity of composition in $\mathbf{RecogGeom}$.
\end{proof}

\begin{remark}
    Nontriviality is ensured as long as at least one observer has a recognizer strictly finer than $\sim_{\text{univ}}$, otherwise the universal quotient collapses to a point.
\end{remark}

This connects directly to Rovelli's Relational Quantum Mechanics, where ``reality'' is observer-relative but constrained by consistency conditions.

\section{Non-Commutative Recognition}

Non-commutative recognition requires a measurement framework with explicit back-action (e.g., effect algebras or instruments). A fully rigorous treatment is deferred to \Cref{sec:noncomm-outlook}, where we outline the conjectural program.

\section{Infinitary Composition}

We have defined binary composition $R_1 \otimes R_2$. What about infinite families?

\subsection{Infinite Recognizer Families}

\begin{definition}[Product Recognizer]
    Given a family $\{R_i\}_{i \in I}$ of recognizers $R_i: \config \to \events_i$, the \textit{product recognizer} is:
    \[ \bigotimes_{i \in I} R_i : \config \to \prod_{i \in I} \events_i \]
    defined by $(\bigotimes R_i)(c) = (R_i(c))_{i \in I}$.
\end{definition}

\begin{theorem}[Infinite Refinement]
    \[ c_1 \sim_{\bigotimes R_i} c_2 \iff \forall i \in I,\, c_1 \sim_{R_i} c_2 \]
\end{theorem}

\subsection{Limits and Colimits}

The category $\mathbf{RecogGeom}$ admits small limits and filtered colimits once a universe level (or Grothendieck universe) is fixed for the underlying sets.

\begin{proposition}[Small limits]
    Let $D: J \to \mathbf{RecogGeom}$ with $J$ a small category. Then $\lim D$ exists and is computed by taking limits of configuration and event sets separately, together with the induced recognizer.
\end{proposition}

\begin{proof}
    Products and equalizers exist in $\mathbf{RecogGeom}$ because functions between sets admit them; general small limits are built from these.
\end{proof}

\begin{definition}[Filtered colimit]
    A filtered system consists of recognizers $\{R_i: \config \to \events_i\}_{i \in I}$ over a directed set $I$ with refinement maps $\phi_{ij}$ for $i \le j$. The filtered colimit $\varinjlim R_i$ has event set the colimit of the $\events_i$ and recognizer induced by the universal property.
\end{definition}

\subsection{Classical Geometry as Infinite Limit}

\begin{theorem}[Continuum emergence]\label{thm:continuum-emergence}
    Let $\{R_n\}_{n \in \mathbb{N}}$ be recognizers with $R_{n+1}$ refining $R_n$. Assume:
    \begin{enumerate}
        \item The resolution cells $\{[c]_{R_n}\}$ form a nested sequence with $\mathrm{diam}([c]_{R_n}) \to 0$ in some background metric on $\config$.
        \item $\config$ is complete with respect to that metric.
    \end{enumerate}
    Then the natural map $\config \to \varprojlim \config_{R_n}$ is a homeomorphism; in particular $\varinjlim \config_{R_n} \cong \config$.
\end{theorem}

\begin{proof}
    The nested-cell condition implies each compatible family in the inverse limit corresponds to a unique point of $\config$ by completeness. The universal property of limits then identifies $\config$ with the limit. Passing to colimits yields the stated isomorphism.
\end{proof}

This formalizes the intuition that classical continua arise as limits of increasingly fine recognition data.

\section{Information-Theoretic Metrics}

We now describe a path-metric construction that turns comparative data into pseudometrics and, under a separation hypothesis, genuine metrics on $\quotientspace$.

\subsection{Weighted Comparative Structures}

\begin{definition}[Weighted comparative recognizer]
    A \textit{weighted comparative recognizer} is a triple $(\Gamma, \mathcal{E}, w)$ where:
    \begin{enumerate}
        \item $\Gamma: \config \times \config \to \mathcal{E}$ assigns an elementary comparison to each ordered pair;
        \item $\mathcal{E}$ carries an involution $e \mapsto e^\top$ such that $\Gamma(c_2, c_1) = \Gamma(c_1, c_2)^\top$ with distinguished equality element $e_{\mathrm{eq}} = \Gamma(c, c)$;
        \item $w: \mathcal{E} \to \mathbb{R}_{\ge 0}$ satisfies $w(e^\top) = w(e)$ and $w(e_{\mathrm{eq}}) = 0$.
    \end{enumerate}
\end{definition}

\begin{definition}[Comparison graph]
    The comparison graph $G_\Gamma$ has vertex set $\config$ and an undirected edge between $c_1$ and $c_2$ labeled by $e = \Gamma(c_1, c_2)$ with weight $w(e)$. We require each connected component of $G_\Gamma$ to lie inside a single resolution cell.
\end{definition}

\subsection{Path Metrics from Recognition Data}

\begin{definition}[Recognition path pseudometric]
    The path length between $c_0$ and $c_k$ is
    \[
        d_\Gamma(c_0, c_k) =
        \inf \left\{ \sum_{i=1}^k w\big(\Gamma(c_{i-1}, c_i)\big)
        \,\middle|\, (c_0,\ldots,c_k) \text{ a path in } G_\Gamma \right\}.
    \]
\end{definition}

\begin{theorem}[Path-metric triangle inequality]\label{thm:path-metric}
    $d_\Gamma$ is a pseudometric on $\config$.
\end{theorem}

\begin{proof}
    Non-negativity and symmetry are immediate from $w$. The zero-length path gives $d_\Gamma(c, c)=0$. Concatenating an $\varepsilon$-optimal path from $c_1$ to $c_2$ with one from $c_2$ to $c_3$ produces a path from $c_1$ to $c_3$ of length at most $d_\Gamma(c_1, c_2)+d_\Gamma(c_2, c_3)+2\varepsilon$. Letting $\varepsilon \to 0$ proves the triangle inequality.
\end{proof}

\subsection{Metric Separation}

\begin{definition}[Separating comparative structure]
    $(\Gamma, \mathcal{E}, w)$ is \textit{separating} if $d_\Gamma(c_1, c_2)=0$ implies $c_1 \sim_R c_2$ for the underlying recognizer $R$.
\end{definition}

\begin{theorem}[Metric on the quotient]\label{thm:metric-separation}
    If $(\Gamma, \mathcal{E}, w)$ is separating, $d_\Gamma$ descends to a metric on $\quotientspace$.
\end{theorem}

\begin{proof}
    The separation hypothesis guarantees distinct equivalence classes have positive distance, so the induced function on $\quotientspace$ is a metric.
\end{proof}

\subsection{Connection to Fisher Information}

\begin{definition}[Recognition Fisher metric]
    For probabilistic recognizers with parameters $\theta \in \Theta$ and likelihood $p(e|\theta)$ define
    \[
        g_{ij}(\theta) = \mathbb{E}\left[ \frac{\partial \log p(e|\theta)}{\partial \theta_i} \frac{\partial \log p(e|\theta)}{\partial \theta_j} \right].
    \]
    This is the standard Fisher information metric (Amari~\cite{InfoGeom}) applied to recognition data.
\end{definition}

\subsection{Entropy of Recognizers}

\begin{definition}[Recognition Entropy]
    The \textit{entropy} of a recognizer $R: \config \to \events$ with respect to a measure $\mu$ on $\config$ is:
    \[ H(R) = -\sum_{e \in \events} p(e) \log p(e) \]
    where $p(e) = \mu(R^{-1}(e))$.
\end{definition}

High entropy means the recognizer ``spreads'' configurations evenly across events; low entropy means most configurations map to a few events.

\section{Axiom Independence and Models}

We address foundational questions about the axiom system.

\subsection{Independence of Axioms}

\begin{proposition}[Independence of RG4]\label{prop:rg4-independence}
    RG4 is independent of RG0--RG3: there exists a recognition geometry satisfying RG0--RG3 that fails RG4.
\end{proposition}

\begin{proof}
    Let $\config = \events = \mathbb{R}$ with $R = \mathrm{id}$. Equip $\config$ with the usual topology, which is a LocalConfigSpace (RG1). RG0 and RG2 hold because $\mathbb{R}$ is nonempty and $R$ separates points. RG3 holds with indistinguishability $\sim$ the equality relation.

    For any $c \in \mathbb{R}$ and neighborhood $U$ of $c$, $R(U) = U$ is infinite, so there is no neighborhood whose image under $R$ is finite. Therefore RG4 fails, establishing independence.
\end{proof}

\begin{proposition}[Independence of RG7]\label{prop:rg7-independence}
    RG7 is independent of RG0--RG6: there exists a recognition geometry satisfying RG0--RG6 for which no comparative recognizer meeting RG7 exists.
\end{proposition}

\begin{proof}
    Let $\config = \{a,b\}$, $\events = \{0,1\}$, and $R(a)=0$, $R(b)=1$. Equip $\config$ with the discrete locality (all subsets containing a point are neighborhoods); RG0--RG6 hold trivially.

    Suppose a comparative recognizer $\Gamma: \config \times \config \to \events'$ satisfies RG7, which requires a distinguished equality event and nontrivial comparisons. Because $R$ resolves both configurations exactly, any comparative structure respecting indistinguishability must assign the same equality event to $(a,a)$ and $(b,b)$. However, $\config$ has only two points, so any nontrivial comparative map necessarily collapses to the equality event, contradicting the requirement that $\Gamma$ distinguish at least one ordered pair. Hence no RG7 structure exists on this geometry.
\end{proof}

\subsection{Embedding Classical Manifolds}

\begin{theorem}[Manifold Embedding]
    Every smooth manifold $M$ can be realized as a recognition geometry: take $\config = M$, $\events = \mathbb{R}^n$, and $R$ a coordinate chart. The quotient recovers $M$.
\end{theorem}

This shows that Recognition Geometry \textit{generalizes} classical differential geometry.

\subsection{Graph Models}

\begin{definition}[Graph Recognition Geometry]
    Let $G = (V, E)$ be a graph. Define $\config = V$, $\events = \mathbb{N}$, and $R(v) = \deg(v)$ (vertex degree). The quotient partitions vertices by degree.
\end{definition}

Graph colorings, clique numbers, and other invariants can be expressed as recognizers.

\subsection{Hilbert Space Models}

\begin{definition}[PVM-based recognition geometry]
    Let $\mathcal{H}$ be a complex Hilbert space and $\{P_i\}_{i \in I}$ a projection-valued measure (PVM) for a self-adjoint operator $A = \sum_i \lambda_i P_i$ with discrete spectrum. Set $\config = \mathbb{P}(\mathcal{H})$ (rays), $\events = I$, and
    \[
        R([\psi]) = i \quad \text{iff} \quad P_i \psi = \psi .
    \]
    Thus each eigenray is mapped to the label of its eigenspace.
\end{definition}

The quotient $\config_R$ identifies all rays inside the same eigenspace, so $\config_R \cong \bigsqcup_i \mathbb{P}(P_i \mathcal{H})$. General superpositions require probabilistic treatment (see the Fisher metric discussion), but eigenstates fit naturally into Recognition Geometry via PVMs.

\section{Higher Recognition Structures (Outlook)}

Speculative higher-categorical generalizations are discussed in \Cref{sec:higher-recognition}.

% ======================================================================
% PART IX: OUTLOOK AND CONJECTURES
% ======================================================================

\part{Outlook and Conjectures}

\section{Non-Commutative Recognition}\label{sec:noncomm-outlook}

In quantum mechanics, observables are represented by non-commuting operators. Extending Recognition Geometry to this setting requires modeling the back-action of recognition on configurations, ideally within the language of effect algebras or completely positive (CP) instruments.

\subsection{Non-Commutative Composition (Proposed)}

\begin{definition}[Ordered Composition]
    An \textit{ordered composition} is a binary operation $\rtimes$ on recognizers such that $(R_1 \rtimes R_2)$ encodes ``apply $R_1$ then $R_2$'', and in general $(R_1 \rtimes R_2) \neq (R_2 \rtimes R_1)$.
\end{definition}

The classical tensor product provides a commutative model; non-commutativity arises when recognition changes the underlying configuration.

\begin{definition}[Recognizer with Back-Action]
    A \textit{recognizer with back-action} is a pair $(R, B)$ where $R: \config \to \events$ and $B: \config \times \events \to \config$ satisfy $R(B(c, e)) = e$ whenever $e = R(c)$. Intuitively, $B$ records the post-measurement state.
\end{definition}

\begin{remark}
    In an effect-algebra setting, events form an ordered structure $(\events, \oplus, 0, 1)$ and back-action is modeled by instruments $\mathcal{I}_e$. Formalizing this remains future work.
\end{remark}

\subsection{Recognition Uncertainty (Conjectural)}

\begin{conjecture}[Recognition uncertainty]
    Let $(R_1, B_1)$ and $(R_2, B_2)$ be recognizers with back-action. If for some $c$
    \[
        B_1(B_2(c, R_2(c)), R_1(B_2(c, R_2(c)))) \neq B_2(B_1(c, R_1(c)), R_2(B_1(c, R_1(c)))),
    \]
    then $R_1$ and $R_2$ cannot be simultaneously sharp; their ordered compositions yield different observable quotients.
\end{conjecture}

The conjecture captures the heuristic that uncertainty originates from non-commuting recognition operations. A full proof would require specifying a category of instruments with well-defined composition.

\subsection{Connection to POVMs}

\begin{definition}[Fuzzy Recognizer]
    A \textit{fuzzy recognizer} is a map $R: \config \to \mathcal{P}(\events)$ assigning to each configuration a probability distribution over events, subject to normalization. When $\config$ is a Hilbert space and the image consists of projection-valued measures, $R$ becomes a POVM.
\end{definition}

Fuzzy recognizers suggest a path toward embedding quantum measurement theory into Recognition Geometry, but rigorous development is left as a future project.

\section{Recognition Generators}\label{sec:recognition-generator}

By analogy with Hamiltonian mechanics, one may seek a single functional governing recognition flows.

\subsection{Generator Ansatz}

\begin{definition}[Recognition generator]
    A \textit{recognition generator} is a function $H: \config \to \mathbb{R}$ equipped with a bracket $\{\cdot,\cdot\}$ such that
    \[
        \frac{d}{dt} R_t(c) = \{H, R_t\}(c).
    \]
\end{definition}

Making this rigorous requires additional structure (symplectic forms, Poisson brackets, or variational principles). Determining the correct geometric ingredients remains open.

\section{Higher Recognition Structures}\label{sec:higher-recognition}

We sketch a hierarchy of ``recognition of recognizers''.

\subsection{2-Recognizers}

\begin{definition}[2-recognizer]
    A \textit{2-recognizer} is a map $R^{(2)}: \mathrm{Recog} \times \mathrm{Recog} \to \events^{(2)}$ assigning an event to a pair of recognizers.
\end{definition}

This captures questions such as: ``How similar are two measurement procedures?'' Potential applications include device calibration and systematic-error analysis.

\subsection{n-Categorical Outlook}

\begin{definition}[n-recognition geometry]
    An \textit{n-recognition geometry} is an n-category whose $k$-cells encode $k$-fold recognition data (configurations, recognizers, recognizer comparisons, etc.).
\end{definition}

Developing this hierarchy rigorously is left as future work; it would align Recognition Geometry with higher topos theory and extended TQFT frameworks.

\section{Geometric Speculations}

Two speculative programs motivate several open problems:

\begin{itemize}
    \item \textbf{Gravitational recognition}: reinterpret curvature and gauge connections as properties of a family of recognizers respecting the Einstein equations. See Open Problems~7--9.
    \item \textbf{Holographic recognition}: relate boundary recognizers to bulk equivalence classes, echoing the holographic principle. See Open Problem~9.
\end{itemize}

We leave these directions conjectural to avoid overstating the present results.

% ======================================================================
% PART X: OPEN PROBLEMS
% ======================================================================

\part{Open Problems}

We collect concrete open problems arising from this work, inviting further research.

\section{Structural Problems}

\begin{enumerate}
    \item \textbf{Smooth Quotient Characterization}: Characterize precisely when the recognition quotient $\quotientspace$ admits a smooth manifold structure. What conditions on $R$ and the locality structure $\neighborhood$ suffice?
    
    \item \textbf{Dimension Bounds}: Given a configuration space $\config$ with locality structure, what is the minimum dimension of any recognition geometry on $\config$? Is there a Lusternik-Schnirelmann type theory for recognition dimension?
    
    \item \textbf{Recognition Cohomology}: Define cohomology groups $H^n_R(\config)$ that measure obstructions to global recognizers extending local ones. What is the relationship to sheaf cohomology?
\end{enumerate}

\section{Categorical Problems}

\begin{enumerate}
    \setcounter{enumi}{3}
    \item \textbf{Classifying Space}: Does the category $\mathbf{RecogGeom}$ have a classifying space? What does its homotopy type encode about recognition?
    
    \item \textbf{Model Structure}: Can $\mathbf{RecogGeom}$ be given a model category structure? What are the weak equivalences, fibrations, and cofibrations?
    
    \item \textbf{Higher Topos}: Is there a natural $(\infty, 1)$-topos whose objects are ``homotopy recognition geometries''?
\end{enumerate}

\section{Physical Problems}

\begin{enumerate}
    \setcounter{enumi}{6}
    \item \textbf{Quantum Non-Commutativity}: Develop a full theory of non-commutative recognition where $R_1 \otimes R_2 \neq R_2 \otimes R_1$. Can the canonical commutation relations $[x, p] = i\hbar$ be derived from recognition-theoretic principles?
    
    \item \textbf{Gravitational Recognition}: What is the recognition-theoretic interpretation of general relativity? Can spacetime curvature be understood as a property of the recognizer family rather than the configuration space?
    
    \item \textbf{Holographic Recognition}: Is there a recognition-theoretic formulation of the holographic principle? Can boundary recognizers encode bulk geometry?
    
    \item \textbf{Cosmological Recognition}: How does the recognition quotient change in an expanding universe? Is there a ``recognition-theoretic'' derivation of the cosmological constant?
\end{enumerate}

\section{Computational Problems}

\begin{enumerate}
    \setcounter{enumi}{10}
    \item \textbf{Complexity of Quotient}: What is the computational complexity of computing the recognition quotient for finite configuration spaces? Is it polynomial, NP-hard, or worse?
    
    \item \textbf{Approximation Algorithms}: For infinite configuration spaces, can the quotient be approximated efficiently? What error bounds are achievable?
\end{enumerate}

% ======================================================================
% PART XI: CONCLUSION
% ======================================================================

\part{Conclusion}

\section{Summary}

We have presented Recognition Geometry, a complete mathematical framework in which:

\begin{enumerate}
    \item \textbf{Configurations} are primitive (what the world does).
    \item \textbf{Recognizers} map configurations to observable events.
    \item \textbf{Space} emerges as the quotient $\quotientspace = \config / \indist$.
    \item \textbf{Dimension} counts independent recognizers.
    \item \textbf{Metrics} emerge from comparative recognition.
    \item \textbf{Symmetries} are recognition-preserving maps.
\end{enumerate}

The framework is axiomatically minimal (4 core axioms: RG0--RG3) with 4 optional structural axioms (RG4--RG7). A Lean~4 formalization of the RG0--RG7 core is maintained in a companion library; the Lean development in this repository focuses on the RH instantiation under \texttt{RiemannRecognitionGeometry/}.

\section{The Central Message}

\begin{center}
\begin{tabular}{|c|c|}
\hline
\textbf{Classical Geometry} & \textbf{Recognition Geometry} \\
\hline
Space exists $\to$ we measure it & Recognition exists $\to$ space emerges \\
Space is primitive & Recognition is primitive \\
Metrics assumed & Metrics derived \\
Continuous structure & Discrete structure fundamental \\
\hline
\end{tabular}
\end{center}

\section{Future Work}

\begin{enumerate}
    \item \textbf{Deeper RS Bridge}: Full formalization of ledger dynamics.
    \item \textbf{Quantum Recognition Geometry}: Connection to Hilbert space structure.
    \item \textbf{Cosmological Applications}: Large-scale structure from recognition.
    \item \textbf{Computational Implementations}: Algorithms for recognition-based simulation.
\end{enumerate}

\section*{Acknowledgments}
\addcontentsline{toc}{section}{Acknowledgments}
The author thanks the Mathlib community and Lean~4 developers for their tooling and guidance, and colleagues at the Recognition Physics Institute for discussions that shaped the axioms and examples.

\section*{Code Availability}
\addcontentsline{toc}{section}{Code Availability}
This paper references two Lean developments:
(i) a companion Recognition Geometry core library (RG0--RG7) under \texttt{IndisputableMonolith/RecogGeom} (not vendored in this repository snapshot), and
(ii) the analytic-number-theory instantiation contained \emph{in this repository} under \texttt{RiemannRecognitionGeometry/}.
At the time of writing, \texttt{RiemannRecognitionGeometry/} consists of 40 project modules (18{,}192 lines), with 9 explicit \texttt{axiom} declarations and 0 remaining \texttt{sorry}s; see the Appendix and the Route~3 addendum for details.

% ======================================================================
% APPENDIX
% ======================================================================

\appendix

\part*{Appendices}

\section{Lean 4 Formalization Summary}

We summarize the Lean status of (A) the companion Recognition Geometry core library referenced in Parts~II--VII, and (B) the RH instantiation that lives in this repository.

\subsection*{A. Recognition Geometry core (companion library)}

The companion RG0--RG7 library consists of 16 modules totaling approximately 3,107 lines of verified Lean 4 code:

\begin{center}
\begin{tabular}{|l|l|l|}
\hline
\textbf{Module} & \textbf{Axiom} & \textbf{Lines} \\
\hline
Core.lean & RG0 & $\sim$100 \\
Locality.lean & RG1 & $\sim$150 \\
Recognizer.lean & RG2 & $\sim$140 \\
Indistinguishable.lean & RG3 & $\sim$170 \\
Quotient.lean & -- & $\sim$140 \\
Composition.lean & RG6 & $\sim$220 \\
Symmetry.lean & -- & $\sim$290 \\
FiniteResolution.lean & RG4 & $\sim$180 \\
Connectivity.lean & RG5 & $\sim$160 \\
Comparative.lean & RG7 & $\sim$260 \\
Charts.lean & -- & $\sim$240 \\
Dimension.lean & -- & $\sim$180 \\
RSBridge.lean & -- & $\sim$260 \\
Examples.lean & -- & $\sim$200 \\
Foundations.lean & -- & $\sim$260 \\
Integration.lean & -- & $\sim$200 \\
\hline
\textbf{Total} & & $\sim$3,107 \\
\hline
\end{tabular}
\end{center}

\subsection*{B. Riemann Hypothesis instantiation (this repository)}

The RH formalization lives under \texttt{RiemannRecognitionGeometry/} and currently comprises 37 project modules (17{,}082 lines) split into:
(i) the Route~1 Recognition Geometry chain, and (ii) the Route~3 explicit-formula skeleton.

\begin{center}
\begin{tabular}{|l|r|}
\hline
\textbf{Route 1 file} & \textbf{Lines} \\
\hline
\texttt{Assumptions.lean} & 86 \\
\texttt{AxiomSanityTests.lean} & 100 \\
\texttt{Axioms.lean} & 1{,}526 \\
\texttt{BMODefs.lean} & 112 \\
\texttt{Basic.lean} & 571 \\
\texttt{CarlesonBound.lean} & 475 \\
\texttt{Conjectures.lean} & 108 \\
\texttt{ConsistencySmokeTest.lean} & 84 \\
\texttt{DirichletEta.lean} & 1{,}392 \\
\texttt{FeffermanStein.lean} & 3{,}134 \\
\texttt{JohnNirenberg.lean} & 2{,}748 \\
\texttt{Main.lean} & 237 \\
\texttt{Phase.lean} & 132 \\
\texttt{PoissonExtension.lean} & 179 \\
\texttt{PoissonJensen.lean} & 645 \\
\texttt{WhitneyGeometry.lean} & 450 \\
\texttt{Mathlib/ArctanTwoGtOnePointOne.lean} & 350 \\
\hline
\end{tabular}
\end{center}

\begin{center}
\begin{tabular}{|l|r|}
\hline
\textbf{Route 3 file (\texttt{ExplicitFormula/})} & \textbf{Lines} \\
\hline
\texttt{ArithmeticJ.lean} & 62 \\
\texttt{Caratheodory.lean} & 925 \\
\texttt{Cayley.lean} & 99 \\
\texttt{CayleyBridge.lean} & 150 \\
\texttt{CompletedJ.lean} & 228 \\
\texttt{Concrete.lean} & 61 \\
\texttt{Defs.lean} & 64 \\
\texttt{HilbertRealization.lean} & 984 \\
\texttt{Lagarias.lean} & 150 \\
\texttt{Li.lean} & 126 \\
\texttt{MainRoute3.lean} & 239 \\
\texttt{MathlibBridge.lean} & 153 \\
\texttt{MulConvolution.lean} & 216 \\
\texttt{Route3FubiniTonelliLemmas.lean} & 96 \\
\texttt{Route3HypBundle.lean} & 167 \\
\texttt{Route3Targets.lean} & 46 \\
\texttt{SchwartzTestSpace.lean} & 68 \\
\texttt{SpectralIdentityAlgebra.lean} & 48 \\
\texttt{WFunctionals.lean} & 71 \\
\texttt{WeilFunctionals.lean} & 305 \\
\texttt{WeilPositivityRH.lean} & 159 \\
\texttt{WeilTestFunction.lean} & 346 \\
\texttt{WeilTestFunctionProofs.lean} & 1{,}100 \\
\hline
\end{tabular}
\end{center}

\noindent
\textbf{Current Lean status (this repository)}: 9 explicit \texttt{axiom} declarations and 0 remaining \texttt{sorry}s. (The sole Route~3 ``standard theorem axiom'' is the Herglotz representation theorem in \texttt{ExplicitFormula/Caratheodory.lean}.)

\section{Addendum: Boundary Certificate Program (Zeta) --- Project Status}

This repository also contains an ongoing analytic-number-theory formalization built around a \emph{boundary certificate} strategy (as in \texttt{CPM.tex}). The current status can be summarized as follows.

\begin{itemize}
    \item \textbf{Deprecated target (renormalized-tail BMO).} A previously tracked goal was a uniform small-\(\mathrm{BMO}\) bound for a renormalized tail of \(\log|\xi(\tfrac12+it)|\) defined by subtracting finitely many local zero terms in dyadic annuli \(B(I,K)\). As stated, a uniform bound with a small numerical constant cannot hold: for Whitney half-length \(L>2/3\) the annulus selection is empty (since every zeta zero satisfies \(\Re\rho<1\), hence \(\sigma_\rho=\Re\rho-1/2<1/2\)), and critical-line zeros produce a universal local mean-oscillation floor unless they are renormalized out.

    \item \textbf{What is provable unconditionally (height-dependent control).} One can obtain fully explicit, height-dependent Carleson-box bounds for the off-line zero measure
    \[
      \mu := \sum_{\Re\rho>1/2} (\Re\rho-\tfrac12)\,\delta_{(\Im\rho,\ \Re\rho-\tfrac12)},
    \]
    by reducing to explicit estimates for the zero-counting function \(N(T)\) (e.g.\ Rosser--Schoenfeld-type bounds). See \texttt{renormalized\_tail\_bound.md} for an auditable derivation and constants.

    \item \textbf{Recommended route (boundary-unimodular \(\mathcal J\) and Carleson energy).} The current preferred formulation is to construct a boundary-unimodular analytic ratio \(\mathcal J\) on \(\{\Re s>1/2\}\) (canonical candidate: the \emph{inner factor} of \(\xi\) on that half-plane) and target a uniform Carleson-energy bound for
    \(U=\Re\log\mathcal J\):
    \[
      \iint_{Q(I)} |\nabla U|^2\,\sigma\,dt\,d\sigma \ \le\ C_{\rm box}\,|I|
      \quad\text{uniformly over boundary intervals }I.
    \]
    This inequality is isolated as the main remaining analytic input in the boundary-certificate architecture; see \texttt{CPM.tex} and \texttt{renormalized\_tail\_bound.md} (\S5).

    \item \textbf{Alternate major rebuild (explicit formula / Weil--Li positivity).}
    \emph{Current status (Route 3):} This ``explicit-formula gate'' route is now instantiated as a mechanically checkable Lean skeleton under \texttt{RiemannRecognitionGeometry/ExplicitFormula/}.
    It fixes Lagarias' Mellin normalization, defines the involution \(\widetilde f\), multiplicative convolution \(f*g\) on \((0,\infty)\), and connects these to Mathlib's \texttt{mellin}.
    The Lean development proves the gate-shape implications (e.g.\ ``Weil gate \(\Rightarrow\) \texttt{RiemannHypothesis}'' and the Li specialization), and formalizes the algebraic heart of the easy direction (RH \(\Rightarrow\) finite-sum positivity) including the required conjugation/symmetry manipulations.
    We proved a concrete Mellin--convolution multiplicativity lemma for the multiplicative convolution, \emph{under an explicit Fubini/Tonelli integrability hypothesis}, reducing the remaining analytic gap to verifying the needed integrability for the chosen test-function class. The original ``ambient'' target \texttt{Route3Targets.lean} keeps \texttt{TestSpace} abstract for \texttt{F := \(\mathbb{R}\to\mathbb{C}\)}, but we now also provide a Lean-friendly \emph{log-Schwartz/Fourier} \texttt{TestSpace} instance on \texttt{SchwartzMap \(\mathbb{R}\) \(\mathbb{C}\)} in \texttt{SchwartzTestSpace.lean}.

    \emph{Recent advances (December 2025):}
    \begin{enumerate}
        \item \textbf{Cayley transform algebra} (\texttt{Cayley.lean}): Formalized the classical lemma that \(\Re z \ge 0\) implies \(|(z-1)/(z+1)| \le 1\), providing a reusable ``positivity \(\to\) unit-disk bound'' bridge.

        \item \textbf{Reflection Positivity target} (\texttt{MainRoute3.lean}): Defined \texttt{ReflectionPositivityRealization}---the statement that \(W^{(1)}(f \star \widetilde{\overline g}) = \langle Tf, Tg\rangle_H\) in some Hilbert space. Proved \texttt{WeilGate\_of\_reflectionPositivityRealization}.

        \item \textbf{Cayley bridge hypothesis package} (\texttt{CayleyBridge.lean}): Formalized \texttt{CayleyBridgeAssumptions}, which packages the remaining analytic gap---a candidate \(J\), domain \(S\), positivity \(\Re(2J) \ge 0\) on \(S\), and a bridge axiom to \texttt{ReflectionPositivityRealization}. We also added a \emph{measure-first} variant \texttt{CayleyMeasureBridgeAssumptions} whose bridge target is a measure-first spectral identity (\texttt{SesqMeasureIdentity}), reflecting the fact that the completed \(\xi\)-channel does not naturally provide a pointwise Lebesgue density on the critical line.

        \item \textbf{Arithmetic \(J\) definition} (\texttt{ArithmeticJ.lean}): Defined \(J(s) := -\zeta'(s)/\zeta(s)\) and connected to the von Mangoldt L-series for \(\Re s > 1\).

        \item \textbf{Spectral identity (measure-first)} (\texttt{HilbertRealization.lean}): Introduced a preferred \emph{measure-first} package \texttt{SesqMeasureIdentity} recording directly that
        \(W^{(1)}(\mathrm{pair}(f,g))=\langle Tf,Tg\rangle_{L^2(\mu)}\)
        for some boundary measure \(\mu\), and proved this yields \texttt{ReflectionPositivityRealization}. The older weight-based \(L^2\)/Bochner-integral structures remain available as an \emph{optional absolute-continuity upgrade}.

        \item \textbf{Lean-friendly test space (log-Schwartz/Fourier)} (\texttt{SchwartzTestSpace.lean}): Provided a concrete \texttt{TestSpace} instance on Schwartz functions using Fourier transform identities, allowing Route~3 algebra to run without assuming Mellin/Fubini infrastructure on all functions \(\mathbb{R}\to\mathbb{C}\).
    \end{enumerate}

    \emph{Lean module inventory (23 files, \texttt{RiemannRecognitionGeometry/ExplicitFormula/}):}
    \texttt{Defs}, \texttt{MathlibBridge}, \texttt{MulConvolution}, \texttt{Route3Targets}, \texttt{Route3HypBundle}, \texttt{Route3FubiniTonelliLemmas},
    \texttt{Lagarias}, \texttt{Li}, \texttt{WeilFunctionals}, \texttt{WFunctionals}, \texttt{WeilPositivityRH},
    \texttt{WeilTestFunction}, \texttt{WeilTestFunctionProofs}, \texttt{MainRoute3}, \texttt{HilbertRealization}, \texttt{Caratheodory}, \texttt{Concrete},
    \texttt{ArithmeticJ}, \texttt{CompletedJ}, \texttt{Cayley}, \texttt{CayleyBridge}, \texttt{SchwartzTestSpace}, \texttt{SpectralIdentityAlgebra}.

    \emph{Main problem right now:} prove the bridge axiom (either \texttt{bridge\_to\_reflection} or the measure-first \texttt{bridge\_to\_measure}) for the arithmetic \(\xi/\zeta\) channel, or directly construct a \texttt{ReflectionPositivityRealization} / \texttt{SesqMeasureIdentity}.
    This is logically equivalent to RH in this architecture.
    This is a global (not local) target and is logically equivalent to RH in this architecture. A cautionary note is that stronger de Branges shift-positivity conditions fail for \(\zeta\) (Conrey--Li, arXiv:math/9812166), so the positivity target must be of Weil/Li ``averaged'' type rather than pointwise.
\end{itemize}

% ======================================================================
% ROUTE 3 BRIDGE STRATEGY (December 14, 2025)
% ======================================================================

\section{Route 3: The Bridge to Reflection Positivity}

This section documents the precise mathematical strategy for closing the Route~3 gap. The approach is ``engineering, not mystical axiom'': the path forward uses classical functional analysis (GNS/Osterwalder--Schrader/RKHS constructions) combined with a spectral identity that connects the arithmetic \(J\)-function to the Weil quadratic form.

\subsection{What the blocker really is}

The gap is \emph{not} ``how do we invent a Hilbert space.'' That part is automatic. The real gap is:

\begin{quote}
\textbf{Show that the Weil quadratic form is positive semidefinite once we assume (or derive) \(\Re(2J) \ge 0\) on the right domain.}
\end{quote}

Formally, we want either:

\begin{enumerate}
    \item \textbf{Bridge axiom form} (\texttt{bridge\_to\_reflection}):
    \[
        \Re(2 \cdot J) \ge 0 \text{ on domain } S \quad\Longrightarrow\quad \text{Hilbert-space representation of the Weil form.}
    \]
    
    \item \textbf{Direct construction} of \texttt{ReflectionPositivityRealization}: a Hilbert space \(H\) and linear map \(T\) such that
    \[
        W^{(1)}\!\bigl(f \star_m \widetilde{\overline{g}}\bigr) = \langle Tf, Tg\rangle_H.
    \]
\end{enumerate}

These are the same problem because constructing \(H\) and \(T\) from a positive semidefinite form is routine. The hard part is proving the form is positive semidefinite \emph{from the \(\Re(2J)\ge 0\) hypothesis}.

The key missing lemma is a \textbf{spectral/Plancherel representation} of the Weil pairing in which \(\Re(2J)\) appears as a nonnegative weight.

\subsection{The Hilbert-space construction is mechanical (GNS/OS)}

If \(V\) is a complex vector space and \(B: V \times V \to \mathbb{C}\) is a sesquilinear form that is:

\begin{itemize}
    \item \textbf{Hermitian:} \(B(g,f) = \overline{B(f,g)}\),
    \item \textbf{Positive semidefinite:} \(B(f,f) \in \mathbb{R}\) and \(B(f,f) \ge 0\),
\end{itemize}

then define the null space \(N := \{f \in V : B(f,f) = 0\}\). The quotient \(V/N\) carries a well-defined inner product via \(\langle [f], [g] \rangle := B(f,g)\). Completing \(V/N\) gives a Hilbert space \(H\), and the quotient map \(T: V \to H\) satisfies
\[
    B(f,g) = \langle Tf, Tg \rangle_H.
\]

Therefore, if we prove positivity of the specific form
\[
    B(f,g) := W^{(1)}\!\bigl(f \star_m \widetilde{\overline{g}}\bigr)
\]
on the right test-function subspace, we instantly get \texttt{ReflectionPositivityRealization}.

The bridge axiom can thus be replaced by:
\begin{itemize}
    \item ``\(\Re(2J) \ge 0\) \(\Rightarrow\) \(B(f,f) \ge 0\) for all admissible \(f\),''
    \item plus the standard quotient/completion construction.
\end{itemize}

\subsection{How to get positivity from \(\Re(2J) \ge 0\): the spectral identity}

(\emph{Important correction.}) For the completed \(\xi\)-channel one has \(\xi(\tfrac12+it)\in\mathbb{R}\) for real \(t\), hence \(\xi'(\tfrac12+it)\in i\mathbb{R}\) away from zeros and
\(\Re(-\xi'/\xi(\tfrac12+it))=0\) a.e.\ if interpreted pointwise. Thus the correct Route~3 target is \textbf{measure-first}.

The target is a representation of the form:
\[
    W^{(1)}\!\bigl(f \star_m \widetilde{\overline{g}}\bigr)
    = \int_{\mathbb{R}} \overline{F_f(t)}\,F_g(t)\, d\mu(t),
\]
where:
\begin{itemize}
    \item \(F_f\) is whatever transform the Route~3 normalizations use (Mellin + involution + Cayley),
    \item \(\mu\) is a boundary spectral measure/distribution (Lebesgue/Haar only after an absolute-continuity upgrade).
\end{itemize}

Once this identity is established, positivity is immediate:
\begin{itemize}
    \item \(\mu \ge 0\),
    \item \(|F_f|^2 \ge 0\),
    \item integral of nonnegative \(\Rightarrow\) nonnegative.
\end{itemize}

So the bridge-to-reflection problem reduces to:

\begin{quote}
\textbf{Prove the measure-first spectral identity \(W^{(1)}(\mathrm{pair}(f,g))=\int \overline{F_f}F_g\,d\mu\), and then prove \(\mu\ge0\) for the arithmetic \(\xi/\zeta\) channel (RH-equivalent).}
\end{quote}

\subsection{The Herglotz/Carath\'eodory bridge}

There are two equivalent classical routes.

\subsubsection{Bridge A: Half-plane Herglotz \(\to\) measure \(\to\) \(L^2(\mu)\) realization}

If \(F(s) = 2J(s)\) is holomorphic on a right half-plane and \(\Re F(s) \ge 0\), then \(F\) admits a Herglotz-type representation with a \textbf{positive measure} \(\mu\) on the boundary. The Weil pairing becomes:
\[
    W^{(1)}\!\bigl(f \star_m \widetilde{\overline{g}}\bigr) 
    = \int_{\mathbb{R}} \overline{F_f(t)}\,F_g(t)\, d\mu(t),
\]
with \(\mu\ge0\). If additionally \(d\mu = w\,d\nu\) (absolute continuity), one can rewrite this in the familiar weighted-\(L^2\) form and take
\[
    H = L^2(w \, d\nu), \qquad (Tf)(\cdot) = (\mathcal{M}f)(\cdot) \sqrt{w(\cdot)}.
\]
This gives \texttt{ReflectionPositivityRealization} in one line once the identity is proven.

\subsubsection{Bridge B: Cayley \(\to\) Schur \(\to\) positive kernel \(\to\) RKHS}

We already proved the Cayley algebra lemma:
\[
    \Re z \ge 0 \quad\Longrightarrow\quad \left|\frac{z-1}{z+1}\right| \le 1.
\]
Define the Schur-class function \(S := (F-1)/(F+1)\) with \(F = 2J\). Then \(|S| \le 1\) on the domain.

Schur functions have a universal positive kernel:
\[
    K(z,w) := \frac{1 - S(z)\overline{S(w)}}{1 - z\overline{w}}.
\]
This kernel is positive definite automatically for Schur \(S\), and every positive definite kernel has a Hilbert space \(H\) and feature map \(\Phi\) such that \(K(z,w) = \langle \Phi(z), \Phi(w) \rangle_H\).

Once the Weil form \(W^{(1)}\) is identified with the integrated version of this kernel (test functions ``smear'' evaluation functionals), we get the \(T\) map.

\subsection{Where the real analysis lives (the genuine blocker)}

The ``kernel positivity \(\Rightarrow\) Hilbert space'' part is standard. What is \emph{not} automatic is proving that the arithmetic object \(J\) is the symbol of the Weil form in the precise way needed. This is a Fubini/Tonelli / boundary-limit problem.

Concretely, we need to justify:
\begin{itemize}
    \item \(W^{(1)}\) (defined by explicit formula / primes / gamma / zeros) equals the boundary pairing coming from \(J\).
    \item Interchange prime sums and integrals safely.
    \item Justify taking boundary limits (Fatou-type theorems, distributional boundary values).
    \item Confirm Mellin normalization matches convolution/involution definitions so the ``reflection square'' is literally an \(L^2\) norm in transform space.
\end{itemize}

Once that analytic identity is proven:
\[
    \text{positivity} \Rightarrow \text{quotient/complete} \Rightarrow (H,T) \Rightarrow \texttt{WeilGate\_of\_reflectionPositivityRealization} \Rightarrow \text{RH}.
\]

\textbf{Honest summary:}
\begin{itemize}
    \item Hilbert-space realization is \emph{not} the hard part.
    \item Showing the Weil functional is \emph{the} positive kernel induced by \(J\) (with all interchanges justified) \emph{is} the hard part.
\end{itemize}

\subsection{Recognition Science interpretation}

The bridge is a known physical principle:
\begin{itemize}
    \item \(\Re(2J) \ge 0\) is the math form of \textbf{passivity / positive-real transfer function}.
    \item Reflection positivity is the Euclidean-signature version of \textbf{unitarity} (Osterwalder--Schrader).
    \item ``Passivity \(\Rightarrow\) state space with energy inner product'' is exactly the \textbf{state-space realization theorem} in engineering and the \textbf{GNS/OS construction} in mathematical physics.
\end{itemize}

In Recognition Geometry terms: start with a pre-inner-product on ``histories'' (test functions). Reflection positivity says ``cost is never negative when pairing a thing with its reflected dual.'' Then:
\begin{enumerate}
    \item identify null directions (indistinguishable under the form),
    \item quotient them out,
    \item complete.
\end{enumerate}
That completion is the ``physical'' Hilbert space. This is Recognition Geometry in its purest form.

\subsection{Concrete forward strategy}

\begin{enumerate}
    \item \textbf{State the exact spectral identity lemma} (the real work):
    \begin{itemize}
        \item Express \(W^{(1)}(f \star_m \widetilde{\overline{g}})\) as a boundary integral where \(J\) appears multiplicatively.
        \item Isolate minimal hypotheses for interchange justification.
    \end{itemize}
    
    \item \textbf{Prove positivity from that identity} using \(\Re(2J) \ge 0\).
    
    \item \textbf{Implement quotient/completion} (tiny, Lean-friendly):
    \(H := \overline{V/N}\) with inner product induced by \(B\).
    
    \item \textbf{Conclude} \texttt{ReflectionPositivityRealization} with \(T :=\) quotient map.
    
    \item \textbf{Fire existing gate theorems} to finish the arc.
\end{enumerate}

\textbf{If you do only one thing next, do step (1) in maximal detail. Everything else is plumbing.}

\subsection{Warning: averaged vs.\ pointwise positivity}

Be careful about the exact positivity statement:
\begin{itemize}
    \item Pointwise \(\Re(2J(s)) \ge 0\) everywhere is often \textbf{too strong} and can be false even when the quadratic form is positive.
    \item What Weil/Li criteria demand is \textbf{global averaged positivity}: positivity of the quadratic form on a test-function space, not pointwise positivity of a boundary function.
\end{itemize}

So treat ``\(\Re(2J) \ge 0\)'' as shorthand for the \textbf{positive-kernel condition} (Pick kernel / Toeplitz matrix positivity). That condition is exactly what yields a Hilbert space, and is exactly what can be proved from an explicit formula.

\subsection{Bottom line}

The bridge is a classical (but formalization-heavy) equivalence:
\[
    \textbf{Carath\'eodory/Herglotz positivity} \;\Longleftrightarrow\; \textbf{positive definite kernel} \;\Longleftrightarrow\; \textbf{Hilbert-space realization.}
\]

The remaining mathematical work is to prove that the arithmetic \(J\) induces the same kernel/measure as the Weil functional \(W^{(1)}\). Once that identification is in place, the Hilbert space and map \(T\) are automatic: quotient out nulls, complete, done.

That's Recognition Geometry in its barest form: ``indistinguishable under the form'' \(\to\) quotient \(\to\) completion \(\to\) physical state space.

\section{Notation Index}

\begin{tabular}{ll}
$\config$ & Configuration space \\
$\events$ & Event space \\
$R: \config \to \events$ & Recognizer \\
$\indist$ & Indistinguishability relation \\
$[c]_R$ & Resolution cell of $c$ \\
$\quotientspace$ & Recognition quotient \\
$\pi$ & Canonical projection \\
$\bar{R}$ & Induced event map on quotient \\
$R_1 \otimes R_2$ & Composite recognizer \\
$\neighborhood(c)$ & Neighborhoods of $c$ \\
$\ledger$ & Ledger (RS configuration space) \\
$\Rhat$ & Recognition operator \\
$J$ & J-cost function \\
$\Phi$ & Recognition flow map \\
$\eta, \varepsilon$ & Unit and counit of the quotient adjunction \\
$\tilde{\phi}_i$ & Induced chart on $\quotientspace$ \\
$G_\Gamma$ & Comparison graph of a weighted comparative recognizer \\
$d_\Gamma$ & Path pseudometric derived from $G_\Gamma$ \\
$\config_{\text{univ}}$ & Universal observer quotient \\
$\mathrm{Aut}_R(\config)$ & Recognition automorphism group \\
$\mathcal{F}$ & Sheaf of local recognizers on $\quotientspace$ \\
\end{tabular}

% ======================================================================
% BIBLIOGRAPHY
% ======================================================================

\begin{thebibliography}{99}

\bibitem{RS1} Washburn, J. \textit{Recognition Science: Foundations}. Recognition Physics Institute, 2024.

\bibitem{RS2} Washburn, J. \textit{The Meta-Principle and 8-Tick Cycle}. Recognition Physics Institute, 2024.

\bibitem{Lean4} de Moura, L., Ullrich, S. \textit{The Lean 4 Theorem Prover and Programming Language}. CADE 2021.

\bibitem{Mathlib} The Mathlib Community. \textit{The Lean Mathematical Library}. 2020--2025.

\bibitem{Topos} Döring, A., Isham, C. \textit{A Topos Foundation for Theories of Physics}. J. Math. Phys. 49, 2008.

\bibitem{InfoGeom} Amari, S. \textit{Information Geometry and Its Applications}. Springer, 2016.

\bibitem{RQM} Rovelli, C. \textit{Relational Quantum Mechanics}. Int. J. Theor. Phys. 35, 1996.

\bibitem{LQG} Rovelli, C. \textit{Quantum Gravity}. Cambridge University Press, 2004.

\bibitem{MacLane} Mac Lane, S. \textit{Categories for the Working Mathematician}. Springer, 2nd ed., 1998.

\bibitem{Lawvere} Lawvere, F.W., Schanuel, S.H. \textit{Conceptual Mathematics: A First Introduction to Categories}. Cambridge University Press, 2009.

\bibitem{Lurie} Lurie, J. \textit{Higher Topos Theory}. Princeton University Press, 2009.

\bibitem{Busch} Busch, P., Lahti, P., Pellonpää, J.-P., Ylinen, K. \textit{Quantum Measurement}. Springer, 2016.

\bibitem{NCG} Connes, A. \textit{Noncommutative Geometry}. Academic Press, 1994.

\bibitem{Penrose} Penrose, R. \textit{The Road to Reality}. Jonathan Cape, 2004.

\bibitem{Sorkin} Sorkin, R.D. \textit{Causal Sets: Discrete Gravity}. In \textit{Lectures on Quantum Gravity}, Springer, 2005.

\bibitem{Wolfram} Wolfram, S. \textit{A New Kind of Science}. Wolfram Media, 2002.

\bibitem{Verlinde} Verlinde, E. \textit{On the Origin of Gravity and the Laws of Newton}. JHEP 04, 2011.

\bibitem{Frieden} Frieden, B.R. \textit{Physics from Fisher Information}. Cambridge University Press, 1998.

\bibitem{Munkres} Munkres, J. \textit{Topology}. 2nd ed., Prentice Hall, 2000.

\bibitem{Lee} Lee, J.M. \textit{Introduction to Smooth Manifolds}. 3rd ed., Springer, 2013.

\bibitem{Chentsov} Chentsov, N.N. \textit{Statistical Decision Rules and Optimal Inference}. American Mathematical Society, 1982.

\end{thebibliography}

\label{LastPage}
\end{document}
