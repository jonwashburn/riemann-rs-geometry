\documentclass[11pt]{article}

\usepackage[margin=1in]{geometry}
\usepackage{amsmath,amssymb,amsthm}
\usepackage{mathtools}
\usepackage{microtype}
\usepackage{hyperref}

\title{Recognition Science, Prime Numbers, and the Riemann Hypothesis:\\
A Standalone Roadmap of What We Know, What We Built, and What Still Blocks Us}
\author{Jonathan Washburn (project notes compiled into a paper draft by an automated assistant)}
\date{December 24, 2025}

\newtheorem{theorem}{Theorem}
\newtheorem{conjecture}{Conjecture}
\newtheorem{definition}{Definition}
\newtheorem{remark}{Remark}

\begin{document}
\maketitle

\begin{abstract}
This note is a standalone ``state-of-the-art'' writeup for a specific research codebase
(\texttt{riemann-geometry-rs}) and a specific guiding narrative (``Recognition Science'').
We assume, as a working hypothesis, that Recognition Science (RS) is the correct architecture of
reality, and we explain what that hypothesis \emph{suggests} about prime numbers and the Riemann
Hypothesis (RH). We then state our concrete formal status in Lean and the final remaining blockers.

The main practical message is simple: the current Lean development already reduces the Connes
Route--3$'$ convergence bottleneck to a short list of explicit analytic estimates, and the next
classical theorem worth developing/formalizing is a quantitative spectral perturbation lemma
(Davis--Kahan / min--max), because it turns a ``gap versus perturbation'' inequality into the
missing approximation step required by the Connes--Consani--Moscovici (CCM) strategy.
\end{abstract}

\tableofcontents

\section{How to read this note (non-mathematician friendly)}

There are two different ``RH stories'' in this repository:
\begin{itemize}
  \item \textbf{Recognition Geometry / boundary-certificate route} (call it \emph{Route~1}):
  RH is reduced to a Carleson-energy/Hardy-space control statement about a boundary ratio.
  This route currently uses a small number of explicit \texttt{axiom} declarations for analytic
  boundary-limit infrastructure.

  \item \textbf{Connes Route--3$'$ (CCM determinant approximants)}:
  RH is reduced to building entire approximants $F_n(t)$ with (i) zeros on the real axis, and
  (ii) locally uniform convergence to Riemann's $\Xi(t)$ on the strip $|\Im t|<\tfrac12$.
  The core Hurwitz step is formalized in Lean.
\end{itemize}

If you are not a mathematician and you want a concrete decision, use this rule:
\begin{quote}
\textbf{Stop after ``reduction.''} It is worth finishing the reduction (turn RH into a short list
of explicit inequalities). It is \emph{not} worth grinding more formal machinery beyond that
unless you have a clear, sourced path to the missing inequalities.
\end{quote}

This note is written to make that reduction explicit and auditable.

\section{Recognition Science (RS): the primitives we will use here}

\subsection{The working hypothesis}

\begin{remark}[Assumption of this note]
We assume RS is the accurate architecture of reality. This is not a claim of scientific proof
inside this note; it is a framing assumption to organize the discussion.
\end{remark}

\subsection{Core RS ideas (as used in this paper)}

The repository document \texttt{Recognition-Science-Full-Theory.txt} (RSFT) summarizes RS as
deriving physical structure from a single ``Meta-Principle'' and a small collection of derived
structures:
\begin{itemize}
  \item A \textbf{ledger} (double-entry conservation constraint).
  \item A \textbf{recognition operator} $\widehat{R}$ replacing the Hamiltonian in the fundamental
  update rule.
  \item A \textbf{unique convex cost} $J(x)=\tfrac12(x+x^{-1})-1$ on $\mathbb{R}_{>0}$.
  \item A \textbf{fixed-point scale} $\varphi$ (golden ratio) and an \textbf{eight-tick cycle}
  as a minimal ledger-compatible periodic walk.
\end{itemize}

These primitives matter for RH because they force a very specific style of argument:
\begin{quote}
\textbf{RS philosophy:} ``Hard facts are those that follow from conservation + convexity +
stability (spectral gap), plus a normalization fixed by a units-quotient.''
\end{quote}

As we will see, that aligns unusually well with the standard analytic-number-theory ``explicit
formula'' worldview: primes and zeros appear as two sides of a conserved trace identity.

\section{What RS teaches us about prime numbers}

This section is intentionally \emph{conceptual}: it translates RS motifs into classical number
theory objects.

\subsection{Primes as recognition events and ``ledger constraints''}

In classical analytic number theory, primes are encoded by the von Mangoldt function
$\Lambda(n)$ via
\[
  -\frac{\zeta'(s)}{\zeta(s)} = \sum_{n\ge 1} \frac{\Lambda(n)}{n^s},\qquad \Re s>1.
\]
This is literally a \emph{logarithmic accounting identity}: the Euler product on the right
encodes prime powers as ``atomic events'' and forces multiplicativity as a conservation law.

Under the RS framing, this is exactly the kind of object one expects:
\begin{itemize}
  \item a conserved count of discrete events,
  \item expressed as a generating function,
  \item whose logarithmic derivative is the most stable observable.
\end{itemize}

\subsection{Eight-phase / eight-beat structure and sieve factors}

RSFT records a ``PrimeSieveFactor'' bridge with the claim:
\begin{quote}
``Eight-beat cancellation selects square-free patterns; prime-sieve density factor
$P=\varphi^{-1/2}\cdot 6/\pi^2$.'' \quad (RSFT, \texttt{BRIDGE;PrimeSieveFactor})
\end{quote}

Classically, $6/\pi^2$ is the density of square-free integers (probability that a random integer
has no repeated prime factor), since
\[
  \mathbb{P}(n\ \text{square-free}) = \prod_{p} \left(1-\frac{1}{p^2}\right)
  = \frac{1}{\zeta(2)} = \frac{6}{\pi^2}.
\]

From a purely number-theoretic perspective, the extra factor $\varphi^{-1/2}$ is not a standard
constant; it is an RS-specific modulation tied to the RS scale recursion and eight-tick
structure. The important point for this note is not whether this modulation is correct, but
what kind of \emph{mechanism} it implies:
\begin{quote}
\textbf{Mechanism implied by RSFT:} prime-adjacent sieve weights should be expressible as a
low-period (\emph{mod~8}) kernel enforcing cancellation/neutrality constraints, with a global
scale weight determined by $\varphi$.
\end{quote}

This is consistent with other RSFT ``kernel'' constructions (e.g. \texttt{@GOLDBACH\_MOD8}) that
explicitly use mod-8 gates and fourth-moment bounds.

\subsection{What this suggests about primes in practice}

If RS is correct, a plausible ``prime story'' is:
\begin{enumerate}
  \item \textbf{Local periodic kernels} (small modulus gates) enforce a neutrality constraint
  reminiscent of ledger balance.
  \item \textbf{Global scale selection} (via $\varphi$) fixes which coarse-graining schedules are
  stable and which densities survive in the limit.
  \item \textbf{Explicit formula identities} become the formal expression of conservation:
  the prime side and the zero side are two ways to compute the same invariant.
\end{enumerate}

This connects directly to RH, because RH is (among other things) a sharp constraint on how zeros
can conspire to produce large deviations in prime counting.

\section{RH as a stability statement in the RS worldview}

\subsection{Classical statement}

Let $\zeta(s)$ be the Riemann zeta function. RH states:
\begin{quote}
All nontrivial zeros of $\zeta(s)$ have real part $\Re s=\tfrac12$.
\end{quote}

Equivalently, the completed $\xi$-function has no zeros off the real axis in the spectral
variable $t$ when one writes $\Xi(t) := \xi(\tfrac12 + it)$.

\subsection{RS interpretation: robustness and spectral gaps}

RSFT repeatedly ties ``robustness'' to spectral gaps, explicitly via statements like
\texttt{SigmaGraphRobustness} involving $\lambda_2$ (the Laplacian spectral gap).
This is the same mathematical pattern that appears in Connes/CCM: the missing step (M2) is an
approximation of a ground state by an explicit kernel, and the standard route to that is
``gap $\Rightarrow$ stability under perturbation.''

So, under RS, RH should look like:
\begin{quote}
\textbf{RH as stability:} a certain canonical structure (a ``ground state'' or ``inner factor'')
is uniquely stable under admissible perturbations, and that stability forces zeros to lie on a
symmetry axis.
\end{quote}

This is not mystical; it is exactly how Hurwitz-type arguments operate:
zero-free approximants converging uniformly preserve zero-freeness off the axis.

\section{Our current Lean architecture for RH (what is actually formalized)}

This section describes the current state of \texttt{riemann-geometry-rs} (this repository).

\subsection{The Connes Route--3$'$ skeleton (CCM $\Rightarrow$ Hurwitz gate $\Rightarrow$ RH)}

The repository contains a typed ``Hurwitz gate'':
\begin{itemize}
  \item \texttt{RiemannRecognitionGeometry/ExplicitFormula/HurwitzGate.lean}:
  a theorem \texttt{hurwitz\_zeroFree\_of\_tendstoLocallyUniformlyOn} formalizing the standard
  Hurwitz-style nonvanishing principle for locally uniform limits.
\end{itemize}

It also contains the final bridge:
\begin{itemize}
  \item \texttt{RiemannRecognitionGeometry/ExplicitFormula/ConnesHurwitzBridge.lean}:
  packages the Hurwitz assumptions for $\Xi$ as \texttt{ConnesHurwitzAssumptions} and proves
  \texttt{riemannHypothesis\_of\_connesHurwitz}.
\end{itemize}

Thus, the only missing work is to build (and prove properties of) the approximants.

\subsection{Where we are on the CCM approximants}

The CCM surface lives in:
\begin{itemize}
  \item \texttt{RiemannRecognitionGeometry/ExplicitFormula/ConnesApproximantsCCM.lean}.
\end{itemize}

Current status:
\begin{itemize}
  \item \textbf{We have an explicit toy closed-form model} for \texttt{CCM.F} via the CCM formula-level
  expression \texttt{CCM.Formula.F\_lamN} (with placeholder coefficients).
  \item \textbf{Play A is implemented}: a bridge lemma
  \texttt{CCM.tendstoXi\_of\_exists\_intermediate} reducing \texttt{CCM.tendstoXi} to
  \emph{(i)} locally uniform convergence of an intermediate family $G_n$ and
  \emph{(ii)} compactwise uniform closeness $\sup_{z\in K}|F_n(z)-G_n(z)|\to 0$ for each compact $K$.
\end{itemize}

This is the key payoff of the engineering work: it turns the convergence problem into a short,
checklistable list of analytic inequalities.

\section{The last remaining blockers (what actually stops an unconditional RH proof)}

We separate ``engineering blockers'' from ``math blockers.''

\subsection{Engineering blockers (Lean/plumbing)}

These are not the true roadblocks:
\begin{itemize}
  \item The Route--3$'$ skeleton compiles and the Hurwitz gate is proved.
  \item The convergence glue (Play A) is proved.
\end{itemize}

In other words: the formal pipeline exists. The remaining work is mathematical.

\subsection{Math blockers (the real bottlenecks)}

\paragraph{Blocker 1: define the genuine CCM approximants.}
The toy closed-form \texttt{CCM.F} is not the genuine determinant of the CCM operator.
To be faithful to CCM, one must define the truncated operator and its regularized determinant
or an equivalent closed form with coefficients coming from the normalized ground state.

\paragraph{Blocker 2: prove ``all zeros are real'' for the genuine approximants.}
This is expected to follow from self-adjointness (or Hermitian matrix structure) plus a
determinant identity. The finite-dimensional spectral theorem exists in Mathlib, but the CCM
rank-one determinant identity needs to be instantiated cleanly.

\paragraph{Blocker 3 (main): prove locally uniform convergence \texttt{CCM.tendstoXi}.}
This is the current central bottleneck. Thanks to Play A, it reduces to:
\begin{itemize}
  \item constructing a suitable intermediate family $G_n$,
  \item proving explicit uniform error bounds on compact sets in the strip.
\end{itemize}

CCM's own narrative identifies two missing analytic steps (often summarized as M1/M2):
\begin{itemize}
  \item M1: a uniqueness/simple-even statement about the relevant ground state,
  \item M2: a quantitative approximation statement $k_\lambda \approx c_\lambda\,\xi_\lambda$
  on a controlled window, with error $\varepsilon(\lambda)\to 0$.
\end{itemize}

\paragraph{Blocker 4: the ``gap vs perturbation'' inequality.}
To get M2 unconditionally in a robust way, one needs:
\begin{itemize}
  \item a lower bound $g(\lambda)$ on the spectral gap separating the ground eigenvalue, and
  \item an upper bound $\delta(\lambda)$ on the perturbation size,
\end{itemize}
and then show $\delta(\lambda)/g(\lambda)\to 0$ along the regime $\lambda\to\infty$ (or the
cofinal sequence you choose).

This is the exact RS-style robustness story in classical analytic clothing.

\section{What resolves the blockers (the realistic plan)}

\subsection{Resolution plan: finish the reduction, then stop unless the estimates are sourced}

The best ``non-mathematician'' strategy is:
\begin{enumerate}
  \item \textbf{Finish the reduction fully:}
  write down the smallest explicit list of inequalities that imply \texttt{CCM.tendstoXi},
  using the Play A lemma.
  \item \textbf{Try to source each inequality:}
  does CCM prove it? does it follow from a known theorem? is it actually new?
  \item \textbf{If any one inequality is unsourced/new:}
  stop and record it as the genuine hole.
\end{enumerate}

This produces a durable research artifact even if RH is not resolved.

\subsection{The specific classical theorem to develop next: Davis--Kahan / min--max perturbation}

If you want exactly one ``doable, classical'' theorem to formalize next, it is this:

\begin{theorem}[Davis--Kahan style eigenvector stability (informal)]
Let $A$ and $A+E$ be Hermitian (self-adjoint) operators/matrices. Assume the smallest eigenvalue
of $A$ is simple and separated from the rest of the spectrum by a gap $g>0$. Then the angle
between the ground eigenspaces of $A$ and $A+E$ is bounded by a constant multiple of $\|E\|/g$.
\end{theorem}

Why this matters here:
\begin{itemize}
  \item It turns the entire M2 problem into the single inequality $\|E\|/g \to 0$.
  \item It matches the RSFT ``robustness = spectral gap'' design pattern.
  \item It is a classical, well-documented theorem suitable for Lean formalization.
\end{itemize}

After this theorem is in Lean, the remaining work is no longer ``proof assistant work'': it is
deriving $g(\lambda)$ and $\|E(\lambda)\|$ for the specific CCM truncations.

\section{Actionable next steps (what to do now)}

\subsection*{If you want to continue (recommended bounded scope)}
\begin{enumerate}
  \item In \texttt{ConnesApproximantsCCM.lean}, write the concrete intermediate family $G_n$
  you actually want (even as a placeholder definition).
  \item State the compactwise estimate that would imply
  \texttt{TendstoUniformlyCloseOn CCM.F G atTop K}.
  \item Stop at the first missing analytic estimate and write it as a single lemma in Lean and
  as a single named inequality in prose.
\end{enumerate}

\subsection*{If you want to call it (also defensible)}
\begin{enumerate}
  \item Freeze the repository in a green build state.
  \item Keep this note plus \texttt{recognition-geometry-dec-18.tex} as the audit trail showing
  the reduction and the exact missing estimate list.
  \item Treat the remaining estimate list as a research question to hand to a specialist.
\end{enumerate}

\section*{Acknowledgments}
This draft is a writeup of an evolving codebase and internal project notes. It is meant to be
useful as a roadmap, not as a formal mathematical publication.

\begin{thebibliography}{9}

\bibitem{IK}
H. Iwaniec and E. Kowalski,
\emph{Analytic Number Theory},
AMS Colloquium Publications, 2004.

\bibitem{ConnesNCG}
A. Connes,
\emph{Noncommutative Geometry},
Academic Press, 1994.

\bibitem{DavisKahan}
C. Davis and W. M. Kahan,
``The rotation of eigenvectors by a perturbation,''
\emph{SIAM J. Numer. Anal.} 7(1), 1970.

\bibitem{FeffermanStein}
C. Fefferman and E. M. Stein,
``$H^p$ spaces of several variables,''
\emph{Acta Math.} 129 (1972), 137--193.

\end{thebibliography}

\end{document}


