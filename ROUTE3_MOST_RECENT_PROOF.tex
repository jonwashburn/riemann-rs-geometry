\documentclass[11pt]{article}

\usepackage[utf8]{inputenc}
\usepackage[T1]{fontenc}
\usepackage{lmodern}
\usepackage{microtype}
\usepackage{geometry}
\usepackage{amsmath,amssymb,amsthm,mathtools}
\usepackage{hyperref}

\geometry{margin=1in}

\title{Route 3 (Explicit Formula): Most Recent Proof Sketch}
\author{Jonathan Washburn}
\date{December 2025}

\begin{document}
\maketitle

\section{Executive summary}

Yes --- there is a real path forward, and it is much more \emph{engineering} than a \emph{mystical leap}.

Right now the gap is \emph{not} ``how do we invent a Hilbert space.'' That part is basically automatic.
The real gap is:

\begin{quote}
Show that the Weil quadratic form you already defined admits a \emph{measure-first spectral representation}
and that the representing object is \emph{positive} for the arithmetic \(\xi/\zeta\) channel.
\end{quote}

After that, the Hilbert-space realization drops out by a standard construction (GNS / OS / RKHS flavor).

\subsection*{Critical correction (important)}
For the standard completed Riemann--Lagarias \(\xi\)-function, one has \(\xi(\tfrac12+it)\in\mathbb{R}\) for real \(t\),
hence \(\xi'(\tfrac12+it)\in i\mathbb{R}\) away from zeros and therefore
\[
\Re\!\left(-\frac{\xi'}{\xi}\Big(\tfrac12+it\Big)\right)=0
\quad \text{for a.e.\ } t \text{ with }\xi(\tfrac12+it)\neq 0.
\]
So the ``weight'' written as a \emph{Lebesgue density}
\(\;w(t)=\Re(2\cdot J(\tfrac12+it))=\Re(-\xi'/\xi(\tfrac12+it))\;\)
is \emph{trivial} a.e.\ if interpreted pointwise. The correct Route~3 target is therefore
\textbf{measure-first}: a boundary \emph{measure/distribution} \(\mu\), with absolute continuity
\(d\mu=w(t)\,dt\) as an optional upgrade (not the default).

\section{What the blocker really is}

You want either:

\begin{enumerate}
  \item \textbf{\texttt{bridge\_to\_reflection}:}
  \[
    \Re(2\cdot J) \ge 0 \ \text{(on a domain)}\ \Longrightarrow\ \text{existence of a Hilbert-space representation of the Weil form.}
  \]

  \item \textbf{Direct construction} of \texttt{ReflectionPositivityRealization} (equivalently, a measure-first identity):
  a Hilbert space \(H\) and linear map \(T\) such that
  \[
    W^{(1)}\!\bigl(f \star_m \widetilde{\overline{g}}\bigr) \,=\, \langle Tf, Tg\rangle_H.
  \]
\end{enumerate}

These are basically the same problem because:

\begin{itemize}
  \item Constructing \(H\) and \(T\) from a \emph{positive semidefinite} form is routine.
  \item The hard part is proving the form is positive semidefinite \emph{from the \(\Re(2\cdot J)\ge 0\) hypothesis} (and showing you are pairing the right objects).
\end{itemize}

So the key missing lemma is a ``spectral/Plancherel representation'' of your Weil pairing in which \(\Re(2\cdot J)\) appears as a nonnegative weight.

\section{The Hilbert-space construction is mechanical (GNS/OS/RKHS)}

Here is the blunt functional-analysis fact you can turn into a Lean lemma.

Let \(V\) be a complex vector space and let
\[
  B: V\times V \to \mathbb{C}
\]
be a sesquilinear form that is:

\begin{itemize}
  \item \textbf{Hermitian:} \(B(g,f) = \overline{B(f,g)}\).
  \item \textbf{Positive semidefinite:} \(B(f,f)\in\mathbb{R}\) and \(B(f,f)\ge 0\) for all \(f\).
\end{itemize}

Define the null space \(N := \{f\in V : B(f,f)=0\}\). Then:

\begin{itemize}
  \item The quotient \(V/N\) carries a well-defined inner product via
  \(\langle [f],[g]\rangle := B(f,g)\).
  \item Completing \(V/N\) gives a Hilbert space \(H\).
  \item The quotient map \(T:V\to H\) satisfies
  \[
    B(f,g) = \langle Tf, Tg\rangle_H.
  \]
\end{itemize}

Therefore, if you can prove positivity of the specific form
\[
  B(f,g) := W^{(1)}\!\bigl(f \star_m \widetilde{\overline{g}}\bigr)
\]
(on the right test-function subspace), you instantly get your \texttt{ReflectionPositivityRealization}.

In other words, the bridge hypothesis can be replaced by:

\begin{itemize}
  \item ``\(\Re(2\cdot J)\ge 0\) \(\Rightarrow\) \(B(f,f)\ge 0\) for all admissible \(f\)''
  \item plus the standard quotient/completion construction above.
\end{itemize}

\section{How to get positivity from \(\Re(2\cdot J)\ge 0\)}

This is the real target.

The correct target is \textbf{measure-first}. You want a representation of the following schematic shape:
\[
  W^{(1)}\!\bigl(f \star_m \widetilde{\overline{g}}\bigr)
  \,=\,\int_{\mathbb{R}} \overline{F_f(t)}\,F_g(t)\, d\mu(t),
\]
for some boundary measure (or distribution) \(\mu\). Positivity is then immediate if \(\mu\ge 0\):
\[
  W^{(1)}\!\bigl(f \star_m \widetilde{\overline{f}}\bigr) = \int |F_f(t)|^2\, d\mu(t)\ \ge\ 0.
\]

Here:

\begin{itemize}
  \item \(F_f\) is whatever transform your Route 3 normalization uses (Mellin + your involution conventions + the Cayley reparameterization).
  \item \(\mu\) is the boundary spectral measure/distribution (Lebesgue/Haar \emph{only if} you can prove absolute continuity).
\end{itemize}

Once you have \emph{that exact identity} with \(\mu\ge 0\), positivity is immediate:

\begin{itemize}
  \item \(\mu\ge 0\),
  \item \(|F_f|^2\ge 0\),
  \item integral of nonnegative \(\Rightarrow\) nonnegative.
\end{itemize}

So the bridge-to-reflection problem reduces to:

\begin{quote}
Prove the spectral identity that expresses the Weil form as an \(L^2(\mu)\) pairing (measure-first),
and then prove \(\mu\ge 0\) for the arithmetic \(\xi/\zeta\) channel (this is RH-equivalent).
\end{quote}

\subsection*{Optional upgrade (density form)}
If one can further show \(\mu \ll dt\) with Radon--Nikodym derivative \(w(t)\), then
\[
  \int \overline{F_f(t)}\,F_g(t)\, d\mu(t) \;=\;
  \int \overline{F_f(t)}\,F_g(t)\, w(t)\,dt.
\]
This is \emph{not} the default in the $\xi$-channel because the naive pointwise density
\(\Re(-\xi'/\xi(\tfrac12+it))\) is \(0\) a.e.\ away from zeros; the boundary object is naturally singular/measure-valued.

\section{A direct bridge: Herglotz / Carath\'eodory kernel}

You already have a Cayley-transform package, which is good: classical complex analysis exactly matches the ``positive real part'' condition.

\subsection{Unit disk side (Carath\'eodory)}

Let \(F(z) := 2\cdot J(z)\) be analytic on \(|z|<1\) with \(\Re F(z)\ge 0\). Then \(F\) is a \emph{Carath\'eodory function}.

A standard theorem says the kernel
\[
  K_F(z,w) := \frac{F(z) + \overline{F(w)}}{1 - z\,\overline{w}}
\]
is \textbf{positive definite}.

Positive definite kernel \(\Rightarrow\) there exists a Hilbert space \(H\) and a map \(v(z)\in H\) such that
\[
  K_F(z,w) = \langle v(z), v(w)\rangle_H.
\]

This already forces a Hilbert-space realization from \(\Re F\ge 0\). The remaining job is to connect your Weil pairing to the kernel pairing induced by \(K_F\), i.e. to show your admissible test functions smear evaluation functionals in exactly the right way.

\subsection{Half-plane side (Nevanlinna/Herglotz)}

Same story with different formulas: analytic functions with \(\Re F\ge 0\) on a half-plane admit a Herglotz/Nevanlinna representation with a positive measure on the boundary. That measure gives a canonical Hilbert space (typically \(L^2(\mu)\)).

Either way, the conceptual bridge is identical:
\[
\text{positive real part} \Rightarrow \text{positive kernel/measure} \Rightarrow \text{Hilbert space} \Rightarrow \text{inner-product representation.}
\]

\section{Where the real analysis lives (the genuine blocker)}

``Kernel positivity \(\Rightarrow\) Hilbert space'' is standard and clean.

What is \emph{not} automatic is proving your arithmetic object \(J\) is the symbol of the Weil form in the precise way needed. This is a genuine Fubini/Tonelli / boundary-limit problem.

Concretely, you need to justify at least one of these equivalences:

\begin{itemize}
  \item Your \(W^{(1)}\) (defined by explicit formula / primes / gamma factors / zeros) equals the boundary pairing coming from \(J\).
  \item You can safely interchange prime sums / zero sums with integrals.
  \item You can justify boundary limits of analytic functions (Fatou-type theorems, distributional boundary values).
  \item Your Mellin normalization matches the convolution/involution definitions so that the ``reflection square'' is literally an \(L^2\) norm in transform space.
\end{itemize}

Once that analytic identity is proven, the remainder is a straight shot:
\[
\text{positivity} \Rightarrow \text{quotient/complete} \Rightarrow (H,T) \Rightarrow \texttt{WeilGate\_of\_reflectionPositivityRealization} \Rightarrow \text{RH (via the Route 3 gate).}
\]

\section{A Recognition Science / systems viewpoint that is actually useful}

The bridge is a known engineering/physics equivalence in disguise:

\begin{itemize}
  \item \(\Re(2\cdot J)\ge 0\) is the math form of \emph{passivity / positive-real transfer function}.
  \item Reflection positivity is the Euclidean-signature version of \emph{unitarity} (Osterwalder--Schrader).
  \item ``Passivity \(\Rightarrow\) a state space with an energy inner product'' is exactly a \emph{state-space realization theorem} in systems theory and a \emph{GNS/OS construction} in mathematical physics.
\end{itemize}

So from a Recognition/physics viewpoint, the missing bridge is:

\begin{quote}
If the ``response'' function is passive (positive real part), then there exists a Hilbert space of states whose inner product reproduces the observed quadratic form.
\end{quote}

That is not philosophy; it is a standard equivalence that can be made fully formal.

\section{Concrete forward strategy}

The shortest path that respects the current Route 3 structure:

\begin{enumerate}
  \item \textbf{State the exact spectral identity lemma} you need (this is the real work):
  express \(W^{(1)}\!\bigl(f \star_m \widetilde{\overline g}\bigr)\) as a boundary integral/pairing where \(J\) appears multiplicatively, and isolate minimal hypotheses to justify interchanges.

  \item \textbf{Prove positivity from that identity} using \(\Re(2\cdot J)\ge 0\).

  \item \textbf{Implement the quotient/completion construction} as a small Lean-friendly lemma:
  \(H := \overline{V/N}\) with the induced inner product.

  \item \textbf{Conclude} \texttt{ReflectionPositivityRealization} by taking \(T\) to be the quotient map (followed by inclusion into the completion).

  \item \textbf{Fire the existing gate theorems} to finish the arc.
\end{enumerate}

If you do only one thing next, do step (1) in maximal detail. Everything else is plumbing.

\section{Lean formalization status (Track 2 skeleton)}

The Route 3 \emph{implication skeleton} for Track 2 is now formalized in Lean.

\subsection{Current gaps (classical analysis input)}

The formalization has \textbf{one standard-theorem axiom and zero sorrys} in \texttt{Caratheodory.lean}:
\begin{enumerate}
  \item \textbf{Axiom}: \texttt{herglotz\_representation} --- the Herglotz representation theorem (1911).
    This is the classical analysis input: every function holomorphic on the unit disk with $\Re F \ge 0$
    admits a representation $F(z) = \int_{|\zeta|=1} \frac{\zeta+z}{\zeta-z}\, d\mu(\zeta) + i\cdot\mathrm{Im}(F(0))$
    for a positive finite measure $\mu$ on the unit circle.  Formalizing this would require Poisson integral
    theory, the Riesz representation theorem for positive functionals on $C(\text{circle})$, and Fatou's theorem.
  
  \item \textbf{Kernel decomposition (COMPLETED)}: The calculation showing $K_F = \int K_{H_\zeta}\, d\mu$
    and using Fubini to swap finite sums with integrals has been fully formalized using Mathlib's
    \texttt{integral\_conj}, \texttt{integral\_add}, \texttt{integral\_div}, \texttt{integral\_finset\_sum},
    and \texttt{integral\_re} lemmas.  Integrability is established via boundedness on the unit circle support.
\end{enumerate}

\subsection{Completed components}

\begin{itemize}
  \item The \emph{mechanical} Hilbert-space construction (quotient/completion / GNS) is implemented in
  \texttt{RiemannRecognitionGeometry/ExplicitFormula/HilbertRealization.lean}.

  \item \textbf{Measure-first spectral identity wiring:} the preferred Route~3 intermediate target
  \texttt{SesqMeasureIdentity} (an \(L^2(\mu)\) representation with no pointwise weight) is implemented in
  \texttt{RiemannRecognitionGeometry/ExplicitFormula/HilbertRealization.lean}.
  The concrete Route~3 pipeline \(\texttt{AssumptionsMeasure} \to \texttt{RiemannHypothesis}\) is implemented (Lean) as:
  \begin{quote}
    \texttt{Route3HypBundle.Route3.RH}\(\mu\)\texttt{ : AssumptionsMeasure -> RiemannHypothesis}
  \end{quote}
  in \texttt{RiemannRecognitionGeometry/ExplicitFormula/Route3HypBundle.lean}.

  \item The Route 3 reduction ``\emph{spectral identity $\Rightarrow$ reflection positivity $\Rightarrow$ Weil gate $\Rightarrow$ RH}''
  is formalized as the theorem
  \begin{quote}
    \texttt{Route3HypBundle.Route3.RH : Assumptions -> RiemannHypothesis}
  \end{quote}
  in \texttt{RiemannRecognitionGeometry/ExplicitFormula/Route3HypBundle.lean}.

  \item \textbf{Lean-friendly test-space layer (log-Schwartz/Fourier):}
  a concrete \texttt{TestSpace} instance on \(\mathcal{S}(\mathbb{R};\mathbb{C})\) (Schwartz functions)
  is provided in \texttt{RiemannRecognitionGeometry/ExplicitFormula/SchwartzTestSpace.lean}, using Fourier
  transform identities to discharge the convolution/involution axioms.

  \item The algebraic foundations for Carath\'eodory kernels are complete:
  \begin{itemize}
    \item Szeg\H{o} kernel positivity (\texttt{szegoKernel\_positive\_definite})
    \item Herglotz kernel positivity (\texttt{caratheodoryKernel\_herglotz\_positive})
    \item Point evaluation kernel identity (\texttt{caratheodoryKernel\_herglotz\_eq\_eval})
  \end{itemize}
  
  \item Prime series convergence (\texttt{summable\_primeTerm}) with exponential decay bounds.
  
  \item Convolution theorems for Weil test functions on the critical strip.
\end{itemize}

\subsection{Remaining mathematical work}

The Track 2 formalization is essentially complete with \textbf{one axiom and zero sorrys}.

To construct a concrete \texttt{Route3.Assumptions} or \texttt{Route3.AssumptionsMeasure} instance for the arithmetic \(\xi/\zeta\) channel:
\begin{enumerate}
  \item Prove the \textbf{measure-first} spectral identity connecting the Weil form to an $L^2(\mu)$ pairing.
  \item Justify Fubini/Tonelli for the intended test-function class (Schwartz with exponential decay).
  \item \textbf{(Standard-theorem axiom)} The Herglotz representation theorem.  This is a classical result from 1911
    that is not currently in Mathlib.  Proving it from scratch would require formalizing:
    \begin{itemize}
      \item Poisson integral representation for positive harmonic functions on the disk
      \item Riesz representation theorem for positive linear functionals on $C(\text{circle})$
      \item Fatou's theorem for boundary values (or weak-$*$ compactness via Banach-Alaoglu)
    \end{itemize}
\end{enumerate}

\textbf{Current build status}: \texttt{lake build} succeeds with no errors.  All measure-theoretic
calculations (Fubini, \texttt{integral\_conj}, integrability bounds) are complete.

\subsection{Summary of Track 2 formalization}

\begin{center}
\begin{tabular}{|l|c|l|}
\hline
\textbf{Metric} & \textbf{Count} & \textbf{Notes} \\
\hline
Sorrys & 0 & All proofs complete \\
\hline
Axioms (Track 2) & 1 & \texttt{herglotz\_representation} \\
\hline
Build Status & \checkmark & \texttt{lake build} succeeds \\
\hline
\end{tabular}
\end{center}

The single axiom represents the \textbf{Herglotz Representation Theorem} (1911), a classical
result that appears in every complex analysis textbook (Rudin, Ahlfors, Conway, etc.).
It is not a conjecture or unproven claim---it is established mathematics that predates
formalization by over a century.  Treating it as an axiom is standard practice when
the underlying infrastructure (Poisson kernels, Riesz representation for positive
functionals, weak-$*$ compactness) is not available in the proof assistant's library.

\section{Bottom line}

There is a path, and it is structurally clean:

\begin{itemize}
  \item The correct Route~3 finish line is \textbf{measure-first}: produce a spectral measure \(\mu\) and show \(\mu\ge 0\).
  \item Positive kernel/measure \(\Rightarrow\) Hilbert-space realization is standard (GNS/RKHS/OS).
  \item The only real gap is proving that the arithmetic \(\xi/\zeta\) channel yields the required positive measure / reflection positivity statement.
\end{itemize}

That is the blocker, and it is answerable in a conventional, formalizable way.

\end{document}
