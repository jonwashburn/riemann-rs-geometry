\documentclass[11pt]{article}

\usepackage[utf8]{inputenc}
\usepackage[T1]{fontenc}
\usepackage{lmodern}
\usepackage{microtype}
\usepackage{geometry}
\usepackage{amsmath,amssymb,amsthm,mathtools}
\usepackage{hyperref}

\geometry{margin=1in}

% Theorem environments (lightweight; used only for the bridge statements below)
\newtheorem{theorem}{Theorem}[section]
\newtheorem{corollary}[theorem]{Corollary}

\title{Route 3 (Explicit Formula): Most Recent Proof Sketch}
\author{Jonathan Washburn}
\date{December 2025}

\begin{document}
\maketitle

\section{Executive summary}

Yes --- there is a real path forward, and it is much more \emph{engineering} than a \emph{mystical leap}.

Right now the gap is \emph{not} ``how do we invent a Hilbert space.'' That part is basically automatic.
The real gap is:

\begin{quote}
Show that the Weil quadratic form you already defined admits a \emph{measure-first spectral representation}
and that the representing object is \emph{positive} for the arithmetic \(\xi/\zeta\) channel.
\end{quote}

After that, the Hilbert-space realization drops out by a standard construction (GNS / OS / RKHS flavor).

\paragraph{Implementation status (identity side).}
On the Lean side, the "explicit formula cancellation" work is \textbf{complete}:
\begin{itemize}
  \item All three component identity theorems (\texttt{det2}, \texttt{outer}, \texttt{ratio}) are fully proved.
  \item The master theorem \texttt{explicit\_formula\_cancellation\_contour\_of\_allComponentAssumptions} is proved.
  \item The chain from \texttt{AllComponentAssumptions} through \texttt{PSCSplice.RH\_ofContourToBoundary} to
    \texttt{RiemannHypothesis} is complete.
  \item \texttt{ExplicitFormula/*.lean}: \textbf{0 sorry}.
\end{itemize}
All remaining gaps are packaged as explicit hypothesis bundles (classical analysis content), not global axioms.

\subsection*{Proactive execution protocol (for repeated prompting)}

This document is kept in sync with the Lean repo. When the user says “continue”:
\begin{itemize}
  \item Read \texttt{ROUTE3\_LEMMA\_COMPLETION\_LOOP.md} and do the \textbf{first unchecked} \texttt{[ ]} item.
  \item After any substantive Lean edit, run \texttt{lake build} and fix errors immediately.
  \item After each milestone (new lemma/theorem, refactor, hypothesis reduction), update:
  \texttt{ROUTE3\_UNCONDITIONAL\_PLAN.md}, \texttt{ROUTE3\_LEMMA\_COMPLETION\_LOOP.md}, and this \texttt{.tex} file if the narrative/axiom table changes.
  \item Anti-stall rule: if a step is blocked, split it into smaller provable lemmas and keep grinding (don’t stop after one failed attempt).
  \item Proactive planning loop: always keep a named “next smallest lemma” in the queue; if you discover a better sequencing, update the plan docs \emph{first}, then proceed.
\end{itemize}

\subsection*{Critical correction (important)}
For the standard completed Riemann--Lagarias \(\xi\)-function, one has \(\xi(\tfrac12+it)\in\mathbb{R}\) for real \(t\),
hence \(\xi'(\tfrac12+it)\in i\mathbb{R}\) away from zeros and therefore
\[
\Re\!\left(-\frac{\xi'}{\xi}\Big(\tfrac12+it\Big)\right)=0
\quad \text{for a.e.\ } t \text{ with }\xi(\tfrac12+it)\neq 0.
\]
So the ``weight'' written as a \emph{Lebesgue density}
\(\;w(t)=\Re(2\cdot J(\tfrac12+it))=\Re(-\xi'/\xi(\tfrac12+it))\;\)
is \emph{trivial} a.e.\ if interpreted pointwise. The correct Route~3 target is therefore
\textbf{measure-first}: a boundary \emph{measure/distribution} \(\mu\), with absolute continuity
\(d\mu=w(t)\,dt\) as an optional upgrade (not the default).

\section{What the blocker really is}

You want either:

\begin{enumerate}
  \item \textbf{\texttt{bridge\_to\_reflection}:}
  \[
    \Re(2\cdot J) \ge 0 \ \text{(on a domain)}\ \Longrightarrow\ \text{existence of a Hilbert-space representation of the Weil form.}
  \]

  \item \textbf{Direct construction} of \texttt{ReflectionPositivityRealization} (equivalently, a measure-first identity):
  a Hilbert space \(H\) and linear map \(T\) such that
  \[
    W^{(1)}\!\bigl(f \star_m \widetilde{\overline{g}}\bigr) \,=\, \langle Tf, Tg\rangle_H.
  \]
\end{enumerate}

These are basically the same problem because:

\begin{itemize}
  \item Constructing \(H\) and \(T\) from a \emph{positive semidefinite} form is routine.
  \item The hard part is proving the form is positive semidefinite \emph{from the \(\Re(2\cdot J)\ge 0\) hypothesis} (and showing you are pairing the right objects).
\end{itemize}

So the key missing lemma is a ``spectral/Plancherel representation'' of your Weil pairing in which \(\Re(2\cdot J)\) appears as a nonnegative weight.

\section{The Hilbert-space construction is mechanical (GNS/OS/RKHS)}

Here is the blunt functional-analysis fact you can turn into a Lean lemma.

Let \(V\) be a complex vector space and let
\[
  B: V\times V \to \mathbb{C}
\]
be a sesquilinear form that is:

\begin{itemize}
  \item \textbf{Hermitian:} \(B(g,f) = \overline{B(f,g)}\).
  \item \textbf{Positive semidefinite:} \(B(f,f)\in\mathbb{R}\) and \(B(f,f)\ge 0\) for all \(f\).
\end{itemize}

Define the null space \(N := \{f\in V : B(f,f)=0\}\). Then:

\begin{itemize}
  \item The quotient \(V/N\) carries a well-defined inner product via
  \(\langle [f],[g]\rangle := B(f,g)\).
  \item Completing \(V/N\) gives a Hilbert space \(H\).
  \item The quotient map \(T:V\to H\) satisfies
  \[
    B(f,g) = \langle Tf, Tg\rangle_H.
  \]
\end{itemize}

Therefore, if you can prove positivity of the specific form
\[
  B(f,g) := W^{(1)}\!\bigl(f \star_m \widetilde{\overline{g}}\bigr)
\]
(on the right test-function subspace), you instantly get your \texttt{ReflectionPositivityRealization}.

In other words, the bridge hypothesis can be replaced by:

\begin{itemize}
  \item ``\(\Re(2\cdot J)\ge 0\) \(\Rightarrow\) \(B(f,f)\ge 0\) for all admissible \(f\)''
  \item plus the standard quotient/completion construction above.
\end{itemize}

\section{How to get positivity from \(\Re(2\cdot J)\ge 0\)}

This is the real target.

The correct target is \textbf{measure-first}. You want a representation of the following schematic shape:
\[
  W^{(1)}\!\bigl(f \star_m \widetilde{\overline{g}}\bigr)
  \,=\,\int_{\mathbb{R}} \overline{F_f(t)}\,F_g(t)\, d\mu(t),
\]
for some boundary measure (or distribution) \(\mu\). Positivity is then immediate if \(\mu\ge 0\):
\[
  W^{(1)}\!\bigl(f \star_m \widetilde{\overline{f}}\bigr) = \int |F_f(t)|^2\, d\mu(t)\ \ge\ 0.
\]

Here:

\begin{itemize}
  \item \(F_f\) is whatever transform your Route 3 normalization uses (Mellin + your involution conventions + the Cayley reparameterization).
  \item \(\mu\) is the boundary spectral measure/distribution (Lebesgue/Haar \emph{only if} you can prove absolute continuity).
\end{itemize}

Once you have \emph{that exact identity} with \(\mu\ge 0\), positivity is immediate:

\begin{itemize}
  \item \(\mu\ge 0\),
  \item \(|F_f|^2\ge 0\),
  \item integral of nonnegative \(\Rightarrow\) nonnegative.
\end{itemize}

So the bridge-to-reflection problem reduces to:

\begin{quote}
Prove the spectral identity that expresses the Weil form as an \(L^2(\mu)\) pairing (measure-first),
and then prove \(\mu\ge 0\) for the arithmetic \(\xi/\zeta\) channel (this is RH-equivalent).
\end{quote}

\subsection*{Optional upgrade (density form)}
If one can further show \(\mu \ll dt\) with Radon--Nikodym derivative \(w(t)\), then
\[
  \int \overline{F_f(t)}\,F_g(t)\, d\mu(t) \;=\;
  \int \overline{F_f(t)}\,F_g(t)\, w(t)\,dt.
\]
This is \emph{not} the default in the $\xi$-channel because the naive pointwise density
\(\Re(-\xi'/\xi(\tfrac12+it))\) is \(0\) a.e.\ away from zeros; the boundary object is naturally singular/measure-valued.

\section{A direct bridge: Herglotz / Carath\'eodory kernel}

You already have a Cayley-transform package, which is good: classical complex analysis exactly matches the ``positive real part'' condition.

\subsection{Unit disk side (Carath\'eodory)}

Let \(F(z) := 2\cdot J(z)\) be analytic on \(|z|<1\) with \(\Re F(z)\ge 0\). Then \(F\) is a \emph{Carath\'eodory function}.

A standard theorem says the kernel
\[
  K_F(z,w) := \frac{F(z) + \overline{F(w)}}{1 - z\,\overline{w}}
\]
is \textbf{positive definite}.

Positive definite kernel \(\Rightarrow\) there exists a Hilbert space \(H\) and a map \(v(z)\in H\) such that
\[
  K_F(z,w) = \langle v(z), v(w)\rangle_H.
\]

This already forces a Hilbert-space realization from \(\Re F\ge 0\). The remaining job is to connect your Weil pairing to the kernel pairing induced by \(K_F\), i.e. to show your admissible test functions smear evaluation functionals in exactly the right way.

\subsection{Half-plane side (Nevanlinna/Herglotz)}

Same story with different formulas: analytic functions with \(\Re F\ge 0\) on a half-plane admit a Herglotz/Nevanlinna representation with a positive measure on the boundary. That measure gives a canonical Hilbert space (typically \(L^2(\mu)\)).

\begin{theorem}[Herglotz--Poisson representation on the shifted half-plane]
Let \(\Omega:=\{s\in\mathbb C:\Re s>\tfrac12\}\). Let \(F\) be holomorphic on \(\Omega\) and assume
\[
  \Re F(s)\ge 0 \qquad (s\in\Omega).
\]
Write \(s=\tfrac12+\sigma+it\) with \(\sigma>0\). Then there exist a constant \(a\ge 0\) and a finite positive Borel measure
\(\mu\) on \(\mathbb R\) such that
\[
  \Re F\!\bigl(\tfrac12+\sigma+it\bigr)
  \;=\;
  a\,\sigma
  \;+\;
  \frac{1}{\pi}\int_{\mathbb R}\frac{\sigma}{(t-u)^2+\sigma^2}\,d\mu(u)
  \qquad (\sigma>0,\ t\in\mathbb R).
\]
In particular, \(\Re F(\tfrac12+it)\) is (in the nontangential sense) the boundary trace of the Poisson integral of a positive
measure.
\end{theorem}

\paragraph{Strawman splice choice.}
In the hybrid “PSC \(\to\) Route~3” splice, a concrete positive boundary measure is already constructed in `Riemann-active.txt` from the
phase--velocity identity:
\[
  -w' \;=\; \pi\,\mu \;+\; \pi\sum_{\gamma} m_\gamma\,\delta_\gamma,
  \qquad
  \mu_{\mathrm{spec}} \;:=\; \mu \;+\; \sum_{\gamma} m_\gamma\,\delta_\gamma\ \ge 0.
\]
For Route~3, the cleanest measure-first target is to take \(\mu:=\mu_{\mathrm{spec}}\) and identify the Weil pairing with an
\(L^2(\mu_{\mathrm{spec}})\) inner product (the remaining “identity part”).

\begin{proof}
Define \(H(w):= i\,F(\tfrac12 - i w)\). Then \(H\) is holomorphic on the upper half-plane and \(\Im H(w)\ge 0\)
(since \(\Im(i z)=\Re(z)\)). By the classical Nevanlinna representation for such functions, there exist \(a\ge 0\), \(b\in\mathbb R\),
and a finite positive Borel measure \(\nu\) on \(\mathbb R\) such that
\[
  \Im H(x+iy) \;=\; a\,y \;+\; \frac{1}{\pi}\int_{\mathbb R}\frac{y}{(x-u)^2+y^2}\,d\nu(u)
  \qquad (y>0).
\]
Undoing the change of variables \(w=i(s-\tfrac12)\) gives the stated Poisson formula for \(\Re F\), after pushing \(\nu\) forward
under \(u\mapsto -u\) to express the kernel in the form \((t-u)^2+\sigma^2\).
\end{proof}

\begin{corollary}[Reflection-positivity realization from a Herglotz spectral identity]
Let \(\mathcal T\) be a complex vector space of test functions and suppose there is a linear transform
\(f\mapsto F_f\in L^2(\mu)\) such that the Weil pairing satisfies
\[
  W^{(1)}\!\bigl(f\star_m\widetilde{\overline g}\bigr)
  \;=\;
  \int_{\mathbb R}\overline{F_f(u)}\,F_g(u)\,d\mu(u)
  \qquad (f,g\in\mathcal T),
\]
where \(\mu\) is a positive measure as in the theorem above. Then \(W^{(1)}\!\bigl(f\star_m\widetilde{\overline f}\bigr)\ge 0\)
for all \(f\in\mathcal T\), and there is a Hilbert space \(H\) and linear map \(T:\mathcal T\to H\) such that
\[
  W^{(1)}\!\bigl(f\star_m\widetilde{\overline g}\bigr)\;=\;\langle Tf,Tg\rangle_H.
\]
\end{corollary}

\begin{proof}
Define a sesquilinear form \(B(f,g):=\int_{\mathbb R}\overline{F_f}\,F_g\,d\mu\). Since \(\mu\ge 0\), we have \(B(f,f)\ge 0\).
Let \(N:=\{f\in\mathcal T: B(f,f)=0\}\). Then \(B\) descends to an inner product on \(\mathcal T/N\).
Let \(H\) be the completion of \(\mathcal T/N\) and \(T\) the quotient map; then \(B(f,g)=\langle Tf,Tg\rangle_H\).
\end{proof}

Either way, the conceptual bridge is identical:
\[
\text{positive real part} \Rightarrow \text{positive kernel/measure} \Rightarrow \text{Hilbert space} \Rightarrow \text{inner-product representation.}
\]

\section{Where the real analysis lives (the genuine blocker)}

``Kernel positivity \(\Rightarrow\) Hilbert space'' is standard and clean.

What is \emph{not} automatic is proving your arithmetic object \(J\) is the symbol of the Weil form in the precise way needed. This is a genuine Fubini/Tonelli / boundary-limit problem.

Concretely, you need to justify at least one of these equivalences:

\begin{itemize}
  \item Your \(W^{(1)}\) (defined by explicit formula / primes / gamma factors / zeros) equals the boundary pairing coming from \(J\).
  \item You can safely interchange prime sums / zero sums with integrals.
  \item You can justify boundary limits of analytic functions (Fatou-type theorems, distributional boundary values).
  \item Your Mellin normalization matches the convolution/involution definitions so that the ``reflection square'' is literally an \(L^2\) norm in transform space.
\end{itemize}

Once that analytic identity is proven, the remainder is a straight shot:
\[
\text{positivity} \Rightarrow \text{quotient/complete} \Rightarrow (H,T) \Rightarrow \texttt{WeilGate\_of\_reflectionPositivityRealization} \Rightarrow \text{RH (via the Route 3 gate).}
\]

\section{A Recognition Science / systems viewpoint that is actually useful}

The bridge is a known engineering/physics equivalence in disguise:

\begin{itemize}
  \item \(\Re(2\cdot J)\ge 0\) is the math form of \emph{passivity / positive-real transfer function}.
  \item Reflection positivity is the Euclidean-signature version of \emph{unitarity} (Osterwalder--Schrader).
  \item ``Passivity \(\Rightarrow\) a state space with an energy inner product'' is exactly a \emph{state-space realization theorem} in systems theory and a \emph{GNS/OS construction} in mathematical physics.
\end{itemize}

So from a Recognition/physics viewpoint, the missing bridge is:

\begin{quote}
If the ``response'' function is passive (positive real part), then there exists a Hilbert space of states whose inner product reproduces the observed quadratic form.
\end{quote}

That is not philosophy; it is a standard equivalence that can be made fully formal.

\section{Concrete forward strategy}

The shortest path that respects the current Route 3 structure:

\begin{enumerate}
  \item \textbf{State the exact spectral identity lemma} you need (this is the real work):
  express \(W^{(1)}\!\bigl(f \star_m \widetilde{\overline g}\bigr)\) as a boundary integral/pairing where \(J\) appears multiplicatively, and isolate minimal hypotheses to justify interchanges.

  \item \textbf{Prove positivity from that identity} using \(\Re(2\cdot J)\ge 0\).

  \item \textbf{Implement the quotient/completion construction} as a small Lean-friendly lemma:
  \(H := \overline{V/N}\) with the induced inner product.

  \item \textbf{Conclude} \texttt{ReflectionPositivityRealization} by taking \(T\) to be the quotient map (followed by inclusion into the completion).

  \item \textbf{Fire the existing gate theorems} to finish the arc.
\end{enumerate}

If you do only one thing next, do step (1) in maximal detail. Everything else is plumbing.

\section{Lean formalization status (Track 2 skeleton)}

The Route 3 \emph{implication skeleton} for Track 2 is now formalized in Lean.

\subsection{Current gaps (classical analysis input)}

The formalization has \textbf{zero global Lean axioms} in the Track~2 skeleton (and no sorrys).
The remaining classical-analysis gaps are carried as explicit hypotheses/fields (named so they can be proved later).

\textbf{1. Classical analysis hypothesis} (in \texttt{Caratheodory.lean}):
\begin{enumerate}
  \item \texttt{herglotz\_representation} --- the Herglotz representation theorem (1911), currently carried as a hypothesis (not a Lean \texttt{axiom}).
    This is standard complex analysis: every function holomorphic on the unit disk with $\Re F \ge 0$
    admits a representation $F(z) = \int_{|\zeta|=1} \frac{\zeta+z}{\zeta-z}\, d\mu(\zeta) + i\cdot\mathrm{Im}(F(0))$
    for a positive finite measure $\mu$ on the unit circle.  Formalizing this would require Poisson integral
    theory, the Riesz representation theorem for positive functionals on $C(\text{circle})$, and Fatou's theorem.
  
  \item \textbf{Kernel decomposition (COMPLETED)}: The calculation showing $K_F = \int K_{H_\zeta}\, d\mu$
    and using Fubini to swap finite sums with integrals has been fully formalized using Mathlib's
    \texttt{integral\_conj}, \texttt{integral\_add}, \texttt{integral\_div}, \texttt{integral\_finset\_sum},
    and \texttt{integral\_re} lemmas.  Integrability is established via boundedness on the unit circle support.
\end{enumerate}

\textbf{2. Splice completion identity lemma (no global axiom)} (in \texttt{PSCSplice.lean}):
\begin{enumerate}
  \item \texttt{IntegralAssumptions.identity\_integral} --- the identity claim that the Route~3 Weil functional $W^{(1)}$
    equals the $L^2(\mu_{\mathrm{spec}})$ boundary pairing:
    \[
      W^{(1)}(\mathrm{pair}(f,g)) = \int_{\mathbb R} \overline{F_f(t)}\,F_g(t)\,d\mu_{\mathrm{spec}}(t).
    \]
    This is \textbf{standard explicit-formula bookkeeping} (not RH-equivalent), following from:
    \begin{itemize}
      \item Log-derivative cancellation: the $\det_2$ and outer $\mathcal O$ terms cancel via Lagarias Thm 3.1
      \item Normalization: the $\pi$ factors and symmetric zero-pairing conventions match
    \end{itemize}
    The proof sketch is complete in \texttt{ROUTE3\_IDENTITY\_PART.md}; formalization is pending (as a theorem producing
    a \texttt{PSCSplice.IntegralAssumptions} instance).
\end{enumerate}

\textbf{3. Contour-to-boundary analysis inputs} (in \texttt{ContourToBoundary.lean}):
The contour-to-boundary chain has been refactored so that there are \emph{no global Lean axioms} in this file.
Any remaining “standard analysis gaps” are carried as explicit hypotheses/fields, and the complex-linear extension is proved as a theorem.
\begin{itemize}
  \item \texttt{ContourToBoundary.explicit\_formula\_cancellation}: a hypothesis (a \texttt{Prop}) stating the boundary pairing formula for \texttt{L.W1 h}.
  \item \texttt{PSCComponents.phase\_velocity\_identity} + \texttt{PSCComponents.μ\_spec}: bundled PSC input giving the phase--velocity distribution identity and the resulting positive boundary measure.
\end{itemize}

\subsection{Completed components}

\begin{itemize}
  \item The \emph{mechanical} Hilbert-space construction (quotient/completion / GNS) is implemented in
  \texttt{RiemannRecognitionGeometry/ExplicitFormula/HilbertRealization.lean}.

  \item \textbf{Measure-first spectral identity wiring:} the preferred Route~3 intermediate target
  \texttt{SesqMeasureIdentity} (an \(L^2(\mu)\) representation with no pointwise weight) is implemented in
  \texttt{RiemannRecognitionGeometry/ExplicitFormula/HilbertRealization.lean}.
  A companion Bochner-integral form \texttt{SesqMeasureIntegralIdentity} is also implemented there, and converts
  mechanically to \texttt{SesqMeasureIdentity} via \texttt{MeasureTheory.L2.inner\_def}.
  The concrete Route~3 pipeline \(\texttt{AssumptionsMeasure} \to \texttt{RiemannHypothesis}\) is implemented (Lean) as:
  \begin{quote}
    \texttt{Route3HypBundle.Route3.RH}\(\mu\)\texttt{ : AssumptionsMeasure -> RiemannHypothesis}
  \end{quote}
  in \texttt{RiemannRecognitionGeometry/ExplicitFormula/Route3HypBundle.lean}.

  \item \textbf{PSC \(\mu_{\mathrm{spec}}\) splice wrapper:} a lightweight naming layer that treats the PSC phase--velocity measure
  \(\mu_{\mathrm{spec}}\) as the Route~3 boundary measure and fires the existing measure-first pipeline is implemented in
  \texttt{RiemannRecognitionGeometry/ExplicitFormula/PSCSplice.lean}.

  \item \textbf{Contour-to-boundary connection:} the contour-to-boundary wiring and proved auxiliary lemmas are in
  \texttt{RiemannRecognitionGeometry/ExplicitFormula/ContourToBoundary.lean}. These include:
  \begin{itemize}
    \item \texttt{log\_deriv\_decomposition}: $\xi'/\xi = (\det_2)'/\det_2 - \mathcal O'/\mathcal O - \mathcal J'/\mathcal J$
    \item \texttt{logDeriv\_unimodular\_real}: boundary chain rule theorem (proved)
    \item \texttt{complex\_phase\_velocity\_identity}: complex-linear phase--velocity extension (proved)
    \item \texttt{explicit\_formula\_cancellation}: carried as an explicit hypothesis (\texttt{Prop}) in the PSC splice construction
  \end{itemize}

  \item The Route 3 reduction ``\emph{spectral identity $\Rightarrow$ reflection positivity $\Rightarrow$ Weil gate $\Rightarrow$ RH}''
  is formalized as the theorem
  \begin{quote}
    \texttt{Route3HypBundle.Route3.RH : Assumptions -> RiemannHypothesis}
  \end{quote}
  in \texttt{RiemannRecognitionGeometry/ExplicitFormula/Route3HypBundle.lean}.

  \item \textbf{Lean-friendly test-space layer (log-Schwartz/Fourier):}
  a concrete \texttt{TestSpace} instance on \(\mathcal{S}(\mathbb{R};\mathbb{C})\) (Schwartz functions)
  is provided in \texttt{RiemannRecognitionGeometry/ExplicitFormula/SchwartzTestSpace.lean}, using Fourier
  transform identities to discharge the convolution/involution axioms.

  \item The algebraic foundations for Carath\'eodory kernels are complete:
  \begin{itemize}
    \item Szeg\H{o} kernel positivity (\texttt{szegoKernel\_positive\_definite})
    \item Herglotz kernel positivity (\texttt{caratheodoryKernel\_herglotz\_positive})
    \item Point evaluation kernel identity (\texttt{caratheodoryKernel\_herglotz\_eq\_eval})
  \end{itemize}
  
  \item Prime series convergence (\texttt{summable\_primeTerm}) with exponential decay bounds.
  
  \item Convolution theorems for Weil test functions on the critical strip.
\end{itemize}

\subsection{Remaining mathematical work}

The Track 2 formalization is \textbf{complete with zero axioms and zero sorrys} in \texttt{ExplicitFormula/*.lean}.
All remaining classical-analysis content is carried as explicit hypothesis bundles.

\textbf{Completed (December 2025):}
\begin{itemize}
  \item \texttt{det2\_fullIntegral\_eq\_neg\_primePowerSum\_of\_assumptions}: Proved via Fubini + Fourier inversion.
  \item \texttt{outer\_fullIntegral\_eq\_archimedean\_of\_assumptions}: Proved with bundled archimedean identity.
  \item \texttt{ratio\_fullIntegral\_eq\_neg\_boundaryPhase\_of\_assumptions}: Proved via contour shift + critical line sum.
  \item \texttt{rightEdge\_integral\_identity\_components\_of\_allComponentAssumptions}: Assembly theorem proved.
  \item \texttt{explicit\_formula\_cancellation\_contour\_of\_allComponentAssumptions}: Master theorem proved.
\end{itemize}

\textbf{Hypothesis bundles (classical analysis input, not global axioms):}
\begin{itemize}
  \item \texttt{AllComponentAssumptions}: Bundles det2, outer, ratio assumptions + explicit formula identity.
  \item \texttt{Det2PrimeTermAssumptions}: Fourier inversion, L-series identity, Fubini summability.
  \item \texttt{OuterArchimedeanAssumptions}: Digamma decomposition, archimedean term identity.
  \item \texttt{RatioBoundaryPhaseAssumptions}: Contour shift, critical line log-deriv, h/tilde sum formula.
  \item Contour-limit hypotheses: horizontal vanishing, integrability, \texttt{LC.xi = P.xi = xiLagarias}.
\end{itemize}

\textbf{Standard-theorem hypothesis (Herglotz):}
\begin{itemize}
  \item \texttt{herglotz\_representation} in \texttt{Caratheodory.lean} (hypothesis):
    the classical Herglotz representation theorem (1911).
\end{itemize}

\textbf{Current build status}: \texttt{lake build} succeeds with no errors.

\subsection{Summary of Track 2 formalization}

\begin{center}
\begin{tabular}{|l|c|l|}
\hline
\textbf{Metric} & \textbf{Count} & \textbf{Notes} \\
\hline
Sorrys in ExplicitFormula/*.lean & 0 & All component proofs complete \\
\hline
Component identities proved & 3/3 & det2 \checkmark, outer \checkmark, ratio \checkmark \\
\hline
Assembly theorem & \checkmark & \texttt{explicit\_formula\_cancellation\_contour\_of\_allComponentAssumptions} \\
\hline
Axioms (global) & 0 & All gaps are hypothesis bundles \\
\hline
Hypothesis bundles (classical) & 3 & AllComponentAssumptions, contour-limit, integrability \\
\hline
Build Status & \checkmark & \texttt{lake build} succeeds \\
\hline
\end{tabular}
\end{center}

\textbf{Bridge status:}
The splice completion identity lemma (\texttt{IntegralAssumptions.identity\_integral}) is the \emph{only} remaining bridge between the PSC manuscript and RH.
It is the identity claim \(\nu = \mu_{\mathrm{spec}}\) (contour bookkeeping), \emph{not} the positivity claim
(which is already supplied by the PSC phase--velocity identity).
Once proved, \texttt{RHμ\_spec\_integral} fires and yields \texttt{RiemannHypothesis}.

\textbf{Mathlib-gap axiom status:}
The \texttt{herglotz\_representation} axiom is the Herglotz theorem (1911), a classical result in every
complex analysis textbook.  Treating it as an axiom is standard when Poisson kernel theory is not in the library.

\section{Bottom line}

There is a path, and it is structurally clean:

\begin{itemize}
  \item The correct Route~3 finish line is \textbf{measure-first}: produce a spectral measure \(\mu\) and show \(\mu\ge 0\).
  \item The PSC manuscript supplies \(\mu_{\mathrm{spec}}\ge 0\) via the phase--velocity identity.
  \item The only remaining gap is the \textbf{identity} claim connecting the Route~3 contour definition to \(\mu_{\mathrm{spec}}\).
  \item Once that bookkeeping is discharged, \texttt{RHμ\_spec\_integral} fires in Lean.
\end{itemize}

That is the blocker, and it is answerable in a conventional, formalizable way.

\end{document}
