\documentclass[11pt]{article}

\usepackage[utf8]{inputenc}
\usepackage[T1]{fontenc}
\usepackage{lmodern}
\usepackage{microtype}
\usepackage{geometry}
\usepackage{amsmath,amssymb,amsthm,mathtools}
\usepackage{hyperref}

\geometry{margin=1in}

\title{Route 3 (Explicit Formula): Most Recent Proof Sketch}
\author{Jonathan Washburn}
\date{December 2025}

\begin{document}
\maketitle

\section{Executive summary}

Yes --- there is a real path forward, and it is much more \emph{engineering} than \emph{mystical axiom}.

Right now the gap is \emph{not} ``how do we invent a Hilbert space.'' That part is basically automatic.
The real gap is:

\begin{quote}
Show that the Weil quadratic form you already defined is actually a positive semidefinite Hermitian form once you assume (or derive) \(\Re(2\cdot J) \ge 0\) on the right domain.
\end{quote}

After that, the Hilbert-space realization drops out by a standard construction (GNS / OS / RKHS flavor).

\section{What the blocker really is}

You want either:

\begin{enumerate}
  \item \textbf{\texttt{bridge\_to\_reflection}:}
  \[
    \Re(2\cdot J) \ge 0 \ \text{(on a domain)}\ \Longrightarrow\ \text{existence of a Hilbert-space representation of the Weil form.}
  \]

  \item \textbf{Direct construction} of \texttt{ReflectionPositivityRealization}:
  a Hilbert space \(H\) and linear map \(T\) such that
  \[
    W^{(1)}\!\bigl(f \star_m \widetilde{\overline{g}}\bigr) \,=\, \langle Tf, Tg\rangle_H.
  \]
\end{enumerate}

These are basically the same problem because:

\begin{itemize}
  \item Constructing \(H\) and \(T\) from a \emph{positive semidefinite} form is routine.
  \item The hard part is proving the form is positive semidefinite \emph{from the \(\Re(2\cdot J)\ge 0\) hypothesis} (and showing you are pairing the right objects).
\end{itemize}

So the key missing lemma is a ``spectral/Plancherel representation'' of your Weil pairing in which \(\Re(2\cdot J)\) appears as a nonnegative weight.

\section{The Hilbert-space construction is mechanical (GNS/OS/RKHS)}

Here is the blunt functional-analysis fact you can turn into a Lean lemma.

Let \(V\) be a complex vector space and let
\[
  B: V\times V \to \mathbb{C}
\]
be a sesquilinear form that is:

\begin{itemize}
  \item \textbf{Hermitian:} \(B(g,f) = \overline{B(f,g)}\).
  \item \textbf{Positive semidefinite:} \(B(f,f)\in\mathbb{R}\) and \(B(f,f)\ge 0\) for all \(f\).
\end{itemize}

Define the null space \(N := \{f\in V : B(f,f)=0\}\). Then:

\begin{itemize}
  \item The quotient \(V/N\) carries a well-defined inner product via
  \(\langle [f],[g]\rangle := B(f,g)\).
  \item Completing \(V/N\) gives a Hilbert space \(H\).
  \item The quotient map \(T:V\to H\) satisfies
  \[
    B(f,g) = \langle Tf, Tg\rangle_H.
  \]
\end{itemize}

Therefore, if you can prove positivity of the specific form
\[
  B(f,g) := W^{(1)}\!\bigl(f \star_m \widetilde{\overline{g}}\bigr)
\]
(on the right test-function subspace), you instantly get your \texttt{ReflectionPositivityRealization}.

In other words, the bridge axiom can be replaced by:

\begin{itemize}
  \item ``\(\Re(2\cdot J)\ge 0\) \(\Rightarrow\) \(B(f,f)\ge 0\) for all admissible \(f\)''
  \item plus the standard quotient/completion construction above.
\end{itemize}

\section{How to get positivity from \(\Re(2\cdot J)\ge 0\)}

This is the real target.

You want a representation of the following schematic shape:
\[
  W^{(1)}\!\bigl(f \star_m \widetilde{\overline{f}}\bigr)
  \,=\,\int \Re\bigl(2\cdot J(\text{boundary point})\bigr)\,\bigl|F_f(\text{boundary point})\bigr|^2\, d\nu.
\]

Here:

\begin{itemize}
  \item \(F_f\) is whatever transform your Route 3 normalization uses (Mellin + your involution conventions + the Cayley reparameterization).
  \item \(\nu\) is the natural boundary measure (Lebesgue on the critical-line parameter, or Haar on the unit circle after Cayley).
\end{itemize}

Once you have \emph{that exact identity}, positivity is immediate:

\begin{itemize}
  \item \(\Re(2\cdot J)\ge 0\),
  \item \(|F_f|^2\ge 0\),
  \item integral of nonnegative \(\Rightarrow\) nonnegative.
\end{itemize}

So the bridge-to-reflection problem reduces to:

\begin{quote}
Prove the spectral identity that expresses the Weil form on ``reflection squares'' as an \(L^2\) norm with weight \(\Re(2\cdot J)\) (or as an integral against a positive measure generated by \(J\)).
\end{quote}

\section{A direct bridge: Herglotz / Carath\'eodory kernel}

You already have a Cayley-transform package, which is good: classical complex analysis exactly matches the ``positive real part'' condition.

\subsection{Unit disk side (Carath\'eodory)}

Let \(F(z) := 2\cdot J(z)\) be analytic on \(|z|<1\) with \(\Re F(z)\ge 0\). Then \(F\) is a \emph{Carath\'eodory function}.

A standard theorem says the kernel
\[
  K_F(z,w) := \frac{F(z) + \overline{F(w)}}{1 - z\,\overline{w}}
\]
is \textbf{positive definite}.

Positive definite kernel \(\Rightarrow\) there exists a Hilbert space \(H\) and a map \(v(z)\in H\) such that
\[
  K_F(z,w) = \langle v(z), v(w)\rangle_H.
\]

This already forces a Hilbert-space realization from \(\Re F\ge 0\). The remaining job is to connect your Weil pairing to the kernel pairing induced by \(K_F\), i.e. to show your admissible test functions smear evaluation functionals in exactly the right way.

\subsection{Half-plane side (Nevanlinna/Herglotz)}

Same story with different formulas: analytic functions with \(\Re F\ge 0\) on a half-plane admit a Herglotz/Nevanlinna representation with a positive measure on the boundary. That measure gives a canonical Hilbert space (typically \(L^2(\mu)\)).

Either way, the conceptual bridge is identical:
\[
\text{positive real part} \Rightarrow \text{positive kernel/measure} \Rightarrow \text{Hilbert space} \Rightarrow \text{inner-product representation.}
\]

\section{Where the real analysis lives (the genuine blocker)}

``Kernel positivity \(\Rightarrow\) Hilbert space'' is standard and clean.

What is \emph{not} automatic is proving your arithmetic object \(J\) is the symbol of the Weil form in the precise way needed. This is a genuine Fubini/Tonelli / boundary-limit problem.

Concretely, you need to justify at least one of these equivalences:

\begin{itemize}
  \item Your \(W^{(1)}\) (defined by explicit formula / primes / gamma factors / zeros) equals the boundary pairing coming from \(J\).
  \item You can safely interchange prime sums / zero sums with integrals.
  \item You can justify boundary limits of analytic functions (Fatou-type theorems, distributional boundary values).
  \item Your Mellin normalization matches the convolution/involution definitions so that the ``reflection square'' is literally an \(L^2\) norm in transform space.
\end{itemize}

Once that analytic identity is proven, the remainder is a straight shot:
\[
\text{positivity} \Rightarrow \text{quotient/complete} \Rightarrow (H,T) \Rightarrow \texttt{WeilGate\_of\_reflectionPositivityRealization} \Rightarrow \text{RH (via the Route 3 gate).}
\]

\section{A Recognition Science / systems viewpoint that is actually useful}

The bridge is a known engineering/physics equivalence in disguise:

\begin{itemize}
  \item \(\Re(2\cdot J)\ge 0\) is the math form of \emph{passivity / positive-real transfer function}.
  \item Reflection positivity is the Euclidean-signature version of \emph{unitarity} (Osterwalder--Schrader).
  \item ``Passivity \(\Rightarrow\) a state space with an energy inner product'' is exactly a \emph{state-space realization theorem} in systems theory and a \emph{GNS/OS construction} in mathematical physics.
\end{itemize}

So from a Recognition/physics viewpoint, the missing bridge is:

\begin{quote}
If the ``response'' function is passive (positive real part), then there exists a Hilbert space of states whose inner product reproduces the observed quadratic form.
\end{quote}

That is not philosophy; it is a standard equivalence that can be made fully formal.

\section{Concrete forward strategy}

The shortest path that respects the current Route 3 structure:

\begin{enumerate}
  \item \textbf{State the exact spectral identity lemma} you need (this is the real work):
  express \(W^{(1)}\!\bigl(f \star_m \widetilde{\overline g}\bigr)\) as a boundary integral/pairing where \(J\) appears multiplicatively, and isolate minimal hypotheses to justify interchanges.

  \item \textbf{Prove positivity from that identity} using \(\Re(2\cdot J)\ge 0\).

  \item \textbf{Implement the quotient/completion construction} as a small Lean-friendly lemma:
  \(H := \overline{V/N}\) with the induced inner product.

  \item \textbf{Conclude} \texttt{ReflectionPositivityRealization} by taking \(T\) to be the quotient map (followed by inclusion into the completion).

  \item \textbf{Fire the existing gate theorems} to finish the arc.
\end{enumerate}

If you do only one thing next, do step (1) in maximal detail. Everything else is plumbing.

\section{Bottom line}

There is a path, and it is structurally clean:

\begin{itemize}
  \item \(\Re(2\cdot J)\ge 0\) is exactly the condition that forces a positive kernel/measure.
  \item Positive kernel/measure \(\Rightarrow\) Hilbert-space realization is standard (GNS/RKHS/OS).
  \item The only real gap is proving that your Weil form is \emph{the} quadratic form induced by that kernel/measure with your exact Mellin/Cayley conventions, with the measure-theory justifications done carefully.
\end{itemize}

That is the blocker, and it is answerable in a conventional, formalizable way.

\end{document}
